\chapter{Definitionen und Notationen}

Die Definitionen dieses Kapitels sind größtenteils aus~\cite{Vogler2014EIO}
übernommen. Hierbei handelt es sich um die Grundlagen der Automaten mit denen
hier gearbeitet werden soll. Es wurde jedoch angepasst, dass für die
Parallelkomposition die Inputaktionen der EIOs nicht disjunkt sein müssen. Dies
wäre eine unnötige Einschränkung. Die Inputs der zu komponierenden EIOs werden
als Inputs der Parallelkomposition übernommen.

\section{Error-IO-Transitionssystem}
Die hier betrachteten EIOs sind Systeme, deren Übergänge mit Inputs und Outputs
beschriftet sind. Ebenfalls zulässig ist eine Kantenbeschriftung mit einem
$\tau$, dass dann eine interne, unbeobachtbare Aktion darstellt, die also keine Interaktion mit
der Umwelt darstellt, sondern meist dafür steht, dass die Inputs und Outputs
dieses Übergangs verborgen wurden.

\begin{Def}[Error-IO-Transitionssystem]
  Ein Error-IO-Transitionssystem (EIO) ist
  als Tupel $S=(Q,I,O,\delta, q_0, E)$ definiert, mit:
  \begin{itemize}
    \item $Q$ $-$ die Menge der Zustände
    \item $I,O$ $-$ die disjunkte Mengen der (sichtbaren) Input- und Outputaktionen
    \item $\delta\subseteq Q\times (I\cup U\cup\{\tau\})\times Q$ $-$ die
      Übergangsrelation
    \item $q_0\in Q$ $-$ der Startzustand
    \item $E\subseteq Q$ $-$ die Menge der Error-Zustände
  \end{itemize}
\end{Def}

Die Handlungen von $S$ sind $\Sigma = I\cup U$ und die Signatur
$Sig(S)=(I,O)$.\\
Um in graphischen Veranschaulichungen Inputs und Outputs zu kennzeichnet wird
folgende Notation verwendet: $x?$ für den Input $x$ und $x!$ für den Output
$x$.\\
Um die Komponenten einzelner Automaten, diesen zuordnen zu können werden für
die Komponenten die gleichen Indizes wie für ihre zugehörigen Automaten
verwendet.\\
Die Elemente der Übergangsrelation $\delta$ werden wir wie folgt notieren:
\begin{itemize}
  \item $p\overset{a}{\rightarrow} q$ für $(p,a,q)\in\delta$
  \item $p\overset{a}{\rightarrow}$ für $\exists p: (p,a,q)\in\delta$
  \item diese Notation kann analog auf Wörter $w\in(\Sigma\cup\{\tau\})^*$
    erweitert werden
  \item $w|_B$ steht für die Zeichenfolge, die aus $w$ entsteht durch löschen
    aller Zeichen, die nicht in $B\subseteq\Sigma$ enthalten sind, d.h.\ es
    bezeichnet die Projektion von $w$ auf die Menge $B$
  \item $p\overset{w}{\Rightarrow} q$ für $w\in\Sigma^*$ mit $\exists
    w'\in(\Sigma\cup\{\tau\})^*:w'|_{\Sigma}=w\wedge p\overset{w'}{\rightarrow}
    q$
  \item $p\overset{w}{\Rightarrow}$ für $\exists q:p\overset{w}{\Rightarrow} q$
\end{itemize}
Die Sprache von $S$ ist
$L(S)=\{w\in\Sigma^*\mid q_0\overset{w}{\Rightarrow}\}$.

\section{Parallelkomposition}
Zwei EIOs sind komponierbar, wenn ihre Outputaktionsmengen disjunkt sind. Die
Error-Zustände der Parallelkomposition setzten sich aus den Error-Zuständen der
beiden zusammengesetzten Komponenten (geerbte Errors) und den Outputs zusammen, die von der
anderen Komponente nicht angenommen werden können (neue Errors).

\begin{Def}[Parallelkomposition]
  Zwei EIOs $S_1, S_2$ sind komponierbar falls
  $O_1\cap O_2=\emptyset$ gilt. Die Parallelkomposition sind dann als
  $S_1\|S_2=(Q,I,O,\delta ,q_0,E)$ mit:
  \begin{itemize}
    \item $Q=Q_1\times Q_2$
    \item $I=(I_1\backslash O_2)\cup(I_2\backslash O_1)$
    \item $O=O_1\cup O_2$
    \item $q_0=(q_{01},q_{02})$
    \item $\begin{aligned}[t]
    \delta =&\{((q_1,q_2),\alpha ,(p_1,q_2))\mid (q_1,\alpha ,p_1)\in\delta
      _1,\alpha\in(\Sigma _1\cup\{\tau\})\backslash Synch(S_1,S_2)\}\cup\\
      &\{((q_1,q_2),\alpha ,(q_1,p_2))\mid (q_2,\alpha ,p_2)\in\delta
      _2,\alpha\in(\Sigma _2\cup\{\tau\})\backslash Synch(S_1,S_2)\}\cup\\
      &\{((q_1,q_2),\alpha ,(p_1,p_2))\mid (q_1,\alpha ,p_1)\in\delta
      _1, (q_2,\alpha ,p_2)\in\delta _2, \alpha\in Synch(S_1,S_2)\}
  \end{aligned}$
    \item $\begin{aligned}[t]
    E=&(Q_1\times E_2)\cup (E_1\times Q_2)\cup\\
      &\{(q_1,q_2)\mid \exists a\in O_1\cap I_2: q_1\overset{a}{\rightarrow}\wedge
  q_2\overset{a}{\not{\hspace{-0.1cm}\rightarrow}}\}\cup\\
  &\{(q_1,q_2)\mid \exists a\in I_1\cap O_2:
q_1\overset{a}{\not{\hspace{-0.1cm}\rightarrow}}\wedge
q_2\overset{a}{\rightarrow}\}
  \end{aligned}$
  \end{itemize}
  definiert. Dabei werden die synchronisierten Handlungen $Synch(S_1,
  S_2)=(I_1\cap O_2)\cup(I_2\cap O_1)$ nicht versteckt.
\end{Def}

\begin{Def}[Parallelkomposition auf Traces]
  Gegeben zwei EIOs $S_1, S_2,
  w_1\in\Sigma _1, w_2\in\Sigma _2, W_1\subseteq\Sigma _1^*, W_2\subseteq\Sigma
  _2^*$, definieren wir:
  \begin{itemize}
    \item $w_1\| w_2=\{w\in (\Sigma _1\cup\Sigma _2)^*\mid w|_{\Sigma _1}=w_1\wedge
      w|_{\Sigma _2}=w_2\}$
    \item $W_1\| W_2=\bigcup\{w_1\| w_2\mid w_1\in W_1\wedge w_2\in W_2\}$
  \end{itemize}
\end{Def}

Die Semantik der späteren Kapitel basiert darauf die jeweiligen Zustände, die
zu Problemen führen mit ihren Traces zu betrachten. Um dies besser umsetzten zu
können definieren wir eine prune-Funktion, die alle Outputs von dem Endzustand
eines Traces entfernt. Zusätzlich wird auch noch eine Funktion definiert, die
die Traces beliebig fortsetzt.

\begin{Def}[Pruning und Fortsetzungs Funktionen]
  Für einen EIO $S$ definieren wir:
  \begin{itemize}
    \item $prune:\Sigma ^*\rightarrow\Sigma ^*, w\mapsto u$, mit $w=uv,
      u=\varepsilon\vee u\in\Sigma ^*\cdot I$ und $v\in O^*$
    \item $cont:\Sigma ^*\rightarrow\mathfrak{P}(\Sigma ^*),
      w\mapsto\{wu\mid u\in\Sigma ^*\}$
    \item $cont:\mathfrak{P}(\Sigma ^*)\rightarrow\mathfrak{P}(\Sigma ^*),
      L\mapsto\bigcup\{cont(w)\mid w\in L\}$
  \end{itemize}
\end{Def}

Für zwei komponierbare EIOs $S_1, S_2$ ist ein Ausführungsweg ihrer
Parallelkomposition $S_1\| S_2$ eine Transitionsfolge der Form $(q_1,q_2)
\overset{w}{\Rightarrow}$ für ein $w\in\Sigma_{12}^*$. So ein Ausführungsweg
kann auf Wege von $S_1$ und $S_2$ projiziert werden. Diese projizierten
Ausführungswege erfüllen $q_i \overset{w_i}{\Rightarrow} p_i$ mit $w|_{\Sigma
_i}=w_i$ für $i=1,2$. Umgekehrt sind zwei Ausführungswege von $S_1$ und $S_2$,
die wie oben aufgebaut sind, Projektionen von genau einem Ausführungsweg
$S_1\|S_2$, der ebenfalls wie oben aufgebaut ist. Daraus folgt die Behauptung
des folgenden Lemmas.

\begin{lem}[Sprache der Parallelkomposition]
  \label{LemmaSprache}
  Für zwei komponierbare EIOs $S_1$ und $S_2$ gilt: $L_{12} = L(S_1\|S_2) =
  L(S_1)\|L(S_2)$.
\end{lem}
