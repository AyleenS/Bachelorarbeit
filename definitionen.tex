\chapter{Definitionen und Notationen}

Die Definitionen dieses Kapitels sind größtenteils aus~\cite{Vogler2014EIO}
übernommen, mit den in der Einleitung erwähnten Abänderungen. In diesen
Definitionen werden die Grundlagen der Transitionssysteme, mit denen hier
gearbeitet werden soll, behandelt.

\section{Error-IO-Transitionssystem}
Die hier betrachteten \EIO{}s sind Systeme, deren Transitionen mit Inputs und
Outputs beschriftet sind. Jede Transition ist dabei mit einem Input oder einem
Output versehen. Ebenfalls zulässig ist eine Transitionsbeschriftung mit
$\tau$, einer \emph{internen}, unbeobachtbaren \emph{Aktion}. Diese interne
Aktion lässt also keine Interaktion mit der Umwelt, d.h.\ mit anderen Systemen,
zu. In~\cite{Vogler2014EIO} entsteht das $\tau$ in vielen Fällen durch das
Verbergen der Inputs und Outputs, die in einer Komposition synchronisiert
werden. Hier werden diese Aktionen hingegen nicht verborgen. Jedoch wird im
weiteren Verlauf noch das Hiding betrachten, in dem Outputs durch interne
Aktionen ersetzt werden.

\begin{Def}[Error-IO-Transitionssystem]
  Ein \emph{Error-IO-Transitionssystem \linebreak (\EIO{})} ist
  ein Tupel $S=(Q,I,O,\delta, q_0, E)$, mit den Komponenten:
  \begin{itemize}
    \item $Q$ $-$ die Menge der Zustände,
    \item $I,O$ $-$ die disjunkten Mengen der (sichtbaren) Input- und
      Output-Aktionen,
    \item $\delta\subseteq Q\times (I\cup O\cup\{\tau\})\times Q$ $-$ die
      Transitionsrelation,
    \item $q_0\in Q$ $-$ der Startzustand,
    \item $E\subseteq Q$ $-$ die Menge der Error-Zustände.
  \end{itemize}
\end{Def}

Die \emph{Aktionsmenge} eines \EIO{}s $S$ ist $\Sigma = I\cup O$ und die
\emph{Signatur} $Sig(S)=(I,O)$.\\
Um in graphischen Veranschaulichungen Inputs und Outputs zu unterscheiden, wird
folgende Notation verwendet: $x?$ für den Input $x$ und $x!$ für den Output
$x$. Falls ein $x$ ohne $?$ oder $!$ verwendet wird, steht dies für eine
Aktion, bei der nicht festgelegt ist, ob sie ein Input oder ein Output ist.\\
Um die Komponenten der entsprechenden Transitionssystem zuzuordnen, werden für
die Komponenten die gleichen Indizes wie für ihr zugehöriges System verwendet,
z.B.\ wird $I_1$ für die Inputmenge des Transitionssystems $S_1$ geschrieben. Diese
Notation wird später analog für die Sprachen, Traces und Zustandsmengen eines
Systems verwendet.\\
Die Elemente der Transitionsrelation $\delta$ werden wie folgt notieren:
\begin{itemize}
  \item $p\overset{\alpha}{\rightarrow} q$ für $(p,\alpha ,q)\in\delta$,
  \item $p\overset{\alpha}{\rightarrow}$ für $\exists q \in Q: (p,\alpha ,q)\in\delta$,
  \item $p\overset{w}{\rightarrow} q$ für $p \overset{\alpha _1}{\rightarrow}
    p_1 \overset{\alpha _2}{\rightarrow} p_2\dots \overset{\alpha
    _n}{\rightarrow} q$ mit $w\in (\Sigma\cup\{\tau\})^*, w=\alpha _1\alpha
    _2\dots \alpha _n$,
  \item $p\overset{w}{\rightarrow}$ für $p \overset{\alpha _1}{\rightarrow}
    \overset{\alpha _2}{\rightarrow} \dots \overset{\alpha _n}{\rightarrow}$
    mit $w\in (\Sigma\cup\{\tau\})^*, w=\alpha _1\alpha _2\dots \alpha _n$,
  \item $w|_B$ steht für die Zeichenfolge, die aus $w$ entsteht durch Löschen
    aller Zeichen, die nicht in $B\subseteq\Sigma$ enthalten sind, d.h.\ es
    bezeichnet die Projektion von $w$ auf die Menge $B$,
  \item $p\overset{w}{\Rightarrow} q$ für $w\in\Sigma^*$ mit $\exists
    w'\in(\Sigma\cup\{\tau\})^*:w'|_{\Sigma}=w\wedge p\overset{w'}{\rightarrow}
    q$,
  \item $p\overset{w}{\Rightarrow}$ für $\exists q:p\overset{w}{\Rightarrow}
    q$.
\end{itemize}
Die \emph{Sprache} von $S$ ist
$L(S)=\{w\in\Sigma^*\mid q_0\overset{w}{\Rightarrow}\}$.

\section{Parallelkomposition}
Zwei \EIO{}s sind komponierbar, wenn ihre Output-Mengen disjunkt sind. Die
Error-Zustän\-de der Parallelkomposition setzen sich aus den Errors der
beiden zusammengesetzten Komponenten (geerbte Errors) und den Zuständen, die
Outputs aus der Menge der synchronisierten Aktionen besitzen, für die im zu
komponierenden System jedoch kein passender Input vorhanden ist (neue Errors),
zusammen.\\
In der folgenden Definition muss eine Veränderung
gegenüber~\cite{Vogler2014EIO} an der Menge der synchronisierten
Aktionen vorgenommen werden. Da nicht mehr $I_1\cap I_2 =\emptyset$ gelten
muss, werden die gemeinsamen Inputs synchronisiert. Somit handelt es sich in
der Parallelkomposition bei synchronisierten Aktionen nicht mehr nur um
Outputs, wie in~\cite{Vogler2014EIO}, sondern im Fall von $I_1\cap I_2$ auch um
Inputs. Falls es bei Inputs aus $I_1\cap I_2$ zu einem fehlenden Input für die
Synchronisation kommt, ist die Transition für die Parallelkomposition nicht
ausführbar, jedoch handelt es sich auch nicht um einen neuen Error, da es
zwischen den beiden Systemen dadurch nicht zu einem Kommunikationsfehler
kommt. Die beiden Transitionssysteme können über die beiden Inputs nicht
miteinander kommunizieren, sondern nur mit anderen Systemen.

\begin{Def}[Parallelkomposition]
\label{DefParallelkomposition}
  Zwei \EIO{}s $S_1, S_2$ sind \emph{komponierbar}, falls
  $O_1\cap O_2=\emptyset$ gilt. Die \emph{Parallelkomposition} der \EIO{}s
  $S_1$ und $S_2$ ist
  $S_{12}:=S_1\|S_2=(Q,I,O,\delta ,q_0,E)$ mit den Komponenten:
  \begin{itemize}
    \item $Q=Q_1\times Q_2$,
    \item $I=(I_1\backslash O_2)\cup(I_2\backslash O_1)$,
    \item $O=O_1\cup O_2$,
    \item $q_0=(q_{01},q_{02})$,
    \item $\begin{aligned}[t]
    \delta =&\{((q_1,q_2),\alpha ,(p_1,q_2))\mid (q_1,\alpha ,p_1)\in\delta
      _1,\alpha\in(\Sigma _1\cup\{\tau\})\backslash Synch(S_1,S_2)\}\\
      &\cup\{((q_1,q_2),\alpha ,(q_1,p_2))\mid (q_2,\alpha ,p_2)\in\delta
      _2,\alpha\in(\Sigma _2\cup\{\tau\})\backslash Synch(S_1,S_2)\}\\
      &\cup\{((q_1,q_2),\alpha ,(p_1,p_2))\mid (q_1,\alpha ,p_1)\in\delta
      _1, (q_2,\alpha ,p_2)\in\delta _2, \alpha\in Synch(S_1,S_2)\},
  \end{aligned}$
    \item $\begin{aligned}[t]
        E=&(Q_1\times E_2)\cup (E_1\times Q_2)
        &&\phantom{neue}\mathrm{geerbte}~\mathrm{Errors}\\
        &\left.\begin{aligned}
        &\cup\left\{(q_1,q_2)\mid \exists a\in O_1\cap I_2: q_1\overset{a}{\rightarrow}\wedge
      q_2\overset{a}{\not{\hspace{-0.1cm}\rightarrow}}\right\}\\
      &\cup\left\{(q_1,q_2)\mid \exists a\in I_1\cap O_2:
q_1\overset{a}{\not{\hspace{-0.1cm}\rightarrow}}\wedge
q_2\overset{a}{\rightarrow}\right\}
\end{aligned}\hspace{1cm}\right\}
      &&\phantom{neue}\mathrm{neue}~\mathrm{Errors}.\\
  \end{aligned}$
  \end{itemize}
  Dabei werden die \emph{synchronisierten Aktionen} $Synch(S_1,
  S_2)=(I_1\cap O_2)\cup(O_1\cap I_2)\cup (I_1\cap I_2)$ nicht versteckt,
  sondern als Outputs bzw.\ im Fall von $I_1\cap I_2$ als Inputs der
  Komposition beibehalten.\\
  $S_1$ wird \emph{Partner} von $S_2$ genannt, wenn ihre
  Parallelkomposition geschlossen ist, d.h.\ wenn sie duale Signaturen
  $Sig(S_1)=(I,O)$ und $Sig(S_2)=(O,I)$ haben.\\
  Die oben definierte Notation $S_{12}=S_1\|S_2$ wird auch für andere
  Indizierungen der Systeme analog angewendet, so gilt also allgemein
  $S_{ij}:=S_i\|S_j$ für $i,j\in\mathbb{N}$.
\end{Def}

Die Parallelkomposition kann nicht nur für Transitionssysteme betrachtet
werden, wie bisher in dieser Arbeit, sondern auch über Aktionsfolgen.
\emph{Traces} sind die möglichen Wege des Systems, mit ihrer
Transitionsbeschriftung. Diese Transitionsbeschriftung besteht aus Inputs und
Outputs, mit denen die Folge ab dem Startzustand $q_0$ beschriftet ist. Somit
kann ein Trace auch als das Wort aufgefasst werden, dass verarbeitet wird
während des Ablaufs des Systems.

\begin{Def}[Parallelkomposition auf Traces]
  Gegeben zwei \EIO{}s $S_1$ und $S_2$, sowie $w_1\in\Sigma _1, w_2\in\Sigma
  _2, W_1\subseteq\Sigma _1^*, W_2\subseteq\Sigma _2^*$:
  \begin{itemize}
    \item $w_1\| w_2:=\{w\in (\Sigma _1\cup\Sigma _2)^*\mid w|_{\Sigma _1}=w_1\wedge
      w|_{\Sigma _2}=w_2\}$,
    \item $W_1\| W_2:=\bigcup\hspace{1pt}\{w_1\| w_2\mid w_1\in W_1\wedge w_2\in W_2\}$.
  \end{itemize}
\end{Def}

Die Semantik der späteren Kapitel basiert darauf die jeweiligen Zustände, die
zu Problemen führen, mit den Traces zu betrachten, mit denen man diese
Zustände erreicht. Um dies besser umsetzen zu können, wird eine
$\prune{}$-Funktion definiert, die alle Outputs am Ende eines Traces entfernt.
Zusätzlich werden Funktionen definiert, die die Traces beliebig fortsetzen.

\begin{Def}[Pruning- und Fortsetzungs-Funktion]
  Für ein \EIO{} $S$ wird definiert:
  \begin{itemize}
    \item $\prune{}:\Sigma ^*\rightarrow\Sigma ^*, w\mapsto u$, mit $w=uv,
      u=\varepsilon\vee u\in\Sigma ^*\cdot I$ und $v\in O^*$,
    \item $\cont{}:\Sigma ^*\rightarrow\mathfrak{P}(\Sigma ^*),
      w\mapsto\{wu\mid u\in\Sigma ^*\}$,
    \item $\cont{}:\mathfrak{P}(\Sigma ^*)\rightarrow\mathfrak{P}(\Sigma ^*),
      L\mapsto\bigcup\hspace{1pt}\{\cont{}(w)\mid w\in L\}$.
  \end{itemize}
\end{Def}

Für zwei komponierbare \EIO{}s $S_1$ und $S_2$ ist ein Ablauf ihrer
Parallelkomposition $S_{12}$ eine Transitionsfolge der Form $(p_1,p_2)
\overset{w}{\Rightarrow} (q_1,q_2)$ für ein $w\in\Sigma_{12}^*$. So ein Ablauf
kann auf Abläufe von $S_1$ und $S_2$ projiziert werden. Diese Projektionen
erfüllen $p_i \overset{w_i}{\Rightarrow} q_i$ mit $w|_{\Sigma
_i}=w_i$ für $i=1,2$. Umgekehrt sind zwei Abläufe von $S_1$ und $S_2$ der Form
$p_i \overset{w_i}{\Rightarrow} q_i$ mit $w| _{\Sigma _i}= w_i$ für $i=1,2$,
Projektionen von einem Ablauf in $S_{12}$ der Form $(p_1,p_2)
\overset{w}{\Rightarrow} (q_1.q_2)$. Es ist dafür nötig, dass die Abläufe der
beiden Systeme $S_1$, $S_2$ und die Systeme selbst komponierbar sind. Das $w$ wurde so
gewählt, dass die Projektion auf die einzelnen Alphabete $\Sigma _1$, $\Sigma
_2$ die jeweiligen Wörter $w_1$, $w_2$ ergibt. Falls keine internen Aktionen
zugelassen wären, würde sogar nur genau ein Ablauf möglich sein in $S_{12}$. Da
jedoch auch interne Aktionen zulässig sind, sind mehrere Abläufe möglich, da
nicht klar ist, wann ein $\tau$ in einem Trace ausgeführt wird. Daraus ergibt
sich das folgende Lemma.

\begin{lem}[Sprache der Parallelkomposition]
\label{LemmaSprache}
  Für zwei komponierbare \EIO{}s $S_1$ und $S_2$ gilt: \[L_{12} := L(S_{12}) =
  L_1\|L_2.\]
\end{lem}

\section{Hiding}

Hiding wurde in dem hier verwendeten Kontext bereits in~\cite{Chilton2013} auf
Traces betrachtet. Da hier die Betrachtungsweise von Transitionssystemen aus
startet, wird auch Hiding aus der Sicht dieser Systeme definieren, wie
in~\cite{Schlosser2012BA}. Eine ähnliche Betrachtung für Hiding bei LTS mit
Inputs und Outputs wurde auch bereits in~\cite{Lynch1996} umgesetzt. Dort
werden nur Output-Aktionen internalisiert, jedoch gibt es eine Menge an
internen Aktionen und nicht nur eine. Das Hiding wird durch einen
Internalisierungsoperator umgesetzt. Es sollen dadurch Aktionen versteckt
werden können, d.h.\ durch $\tau$s ersetzt werden. In~\cite{Chilton2013} ist es
in der Definition des Hidings möglich Outputs und Inputs zu verstecken. Durch
das Verstecken von Outputs sind diese nach außen nicht mehr sichtbar. Werden
jedoch Inputs versteckt sind alle Traces, die diesen Input benötigen, nicht
mehr ausführbar. Sie sind dann ab dem versteckten Input nicht mehr
weiterführbar. Es handelt sich also um echte Einschränkungen des Systems. Die
Transitionen werden durch das Hiding von Inputs verboten, ähnlich wie bei der Anwendung
von Restriktionen in CSS, siehe dazu~\cite{Milner1989}. Diese Art der
Einschränkung der Transitionssysteme sollen hier jedoch nicht behandelt werden.
Somit wird in der folgenden Definition nur die Internalisierung von Outputs
erlaubt, entsprechend Quelle~\cite{Schlosser2012BA}.

\begin{Def}[Internalisierungsoperator]
  Für ein \EIO{} $S=(Q,I,O,\delta ,q_0.E)$ ist $S/X$, mit
  dem \emph{Internalisierungsoperator} $\cdot /\cdot$,
  definiert als $S'=(Q,I,O',\delta ', q_0,E)$ mit:
  \begin{itemize}
    \item $\tau \notin X$,
    \item $X\subseteq O$,
    \item $O'=O\backslash X$,
    \item $\delta '=(\delta\cup\{(q,\tau ,q')\mid (q,x,q')\in\delta
      ,x\in X\})\backslash \{(q,x,q')\mid x\in X\}$.
  \end{itemize}
\end{Def}

Die Menge hinter dem Internalisierungsoperator ist in dieser Definition auf
Outputs beschränkt. Diese Einschränkung wurde vorgenommen, um die weitere
Betrachtung zu erleichtern. Jedoch kann es sinnvoll sein die Möglichkeit zu
haben dort weitere Aktionen aufnehmen zu können. Dies wird jedoch nicht mehr
Teil dieser Arbeit sein.\\
In~\cite{Vogler2014EIO} wird die Parallelkomposition nur mit Verbergen der
synchronisierten Aktionen betrachtet, die durch die Synchronisation von einem
Input mit einem Output entstehen. Diese Parallelkomposition wird nun mit dem
Internalisierungsoperator durch Hiding der synchronisierten Aktionen, die in
der Parallelkomposition zu Outputs werden, nachbildet. Da in dieser Arbeit
die Inputmengen der Systeme, die komponiert werden, nicht disjunkt sein müssen,
ergeben sich auch Inputs aus der Synchronisation von Aktionen. Diese können
jedoch mit der hier verwendeten Definition des Internalisierungsoperators nicht verborgen
werden. Dies wäre auch nicht sinnvoll, da diese Synchronisation von Inputs
keine Kommunikation zwischen den Systemen ist, sondern nur eine Zusammenfügung,
damit die Parallelkomposition über diesen Input mit weiteren Systemen
kommunizieren kann. Somit ergibt sich die folgende Definition, mit der die
Parallelkomposition aus~\cite{Vogler2014EIO} nachgebildet werden kann.

\begin{Def}[Parallelkomposition mit Internalisierung]
\label{defIntParal}
  Seinen $S_1$ und $S_2$ komponierbare \EIO{}s, dann ist
  $S_1|S_2=S_{12}/(Synch(S_1,S_2)\cap O_{12})$.
\end{Def}
