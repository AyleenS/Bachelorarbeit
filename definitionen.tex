\chapter{Definitionen und Notationen}

Die Definitionen dieses Kapitels sind größtenteils aus~\cite{Vogler2014EIO}
übernommen. Hierbei handelt es sich um die Grundlagen der Transitionssysteme mit denen
hier gearbeitet werden soll. Jedoch wurde angepasst, dass für die
Parallelkomposition die Inputaktionen der Error-IO-Transitionssysteme (EIOs) nicht disjunkt sein müssen. Dies
wäre eine unnötige Einschränkung. Die nicht synchronisierten Inputs der zu komponierenden EIOs werden
als Inputs der Parallelkomposition übernommen. Zusätzlich verzichten wir hier
auf das verbergen der synchronisierten Handlungen.

\section{Error-IO-Transitionssystem}
Die hier betrachteten EIOs sind Systeme, deren Übergänge mit Inputs und Outputs
beschriftet sind. Jeder Übergang ist dabei mit einem Input oder einem Output
versehen. Ebenfalls zulässig ist eine Kantenbeschriftung mit $\tau$, einer
internen, unbeobachtbaren Aktion. Diese interne
Aktion lässt also keine Interaktion mit
der Umwelt zu. In vielen Fällen entstehen sie durch das Verbergen der Inputs und Outputs
dieses Übergangs, da diese in einer Komposition synchronisiert
wurden.

\begin{Def}[Error-IO-Transitionssystem]
  Ein Error-IO-Transitionssystem (EIO) ist
  ein Tupel $S=(Q,I,O,\delta, q_0, E)$, mit den Komponenten:
  \begin{itemize}
    \item $Q$ $-$ die Menge der Zustände,
    \item $I,O$ $-$ die disjunkten Mengen der (sichtbaren) Input- und
      Outputaktionen,
    \item $\delta\subseteq Q\times (I\cup U\cup\{\tau\})\times Q$ $-$ die
      Übergangsrelation,
    \item $q_0\in Q$ $-$ der Startzustand,
    \item $E\subseteq Q$ $-$ die Menge der Error-Zustände.
  \end{itemize}
\end{Def}

Die Handlungsmenge eines EIOs $S$ ist $\Sigma = I\cup U$ und die Signatur
$Sig(S)=(I,O)$.\\
Um in graphischen Veranschaulichungen Inputs und Outputs zu unterscheiden wird
folgende Notation verwendet: $x?$ für den Input $x$ und $x!$ für den Output
$x$. Falls ein $x$ ohne $?$ oder $!$ verwendet wird, steht dies für eine
Handlung, bei der nicht festgelegt ist, ob sie ein Input oder ein Output ist.\\
Um die Komponenten den entsprechenden Automaten zuzuordnen, werden für
die Komponenten die gleichen Indizes wie für ihre zugehörigen Automaten
verwendet, z.B.\ für die Inputmenge des Automaten $S_1$ schreiben wir $I_1$.
Diese Notation verwenden wir später analog für die Sprachen, die einem
Automaten zugeordnet sind.\\
Die Elemente der Übergangsrelation $\delta$ werden wir wie folgt notieren:
\begin{itemize}
  \item $p\overset{a}{\rightarrow} q$ für $(p,a,q)\in\delta$,
  \item $p\overset{a}{\rightarrow}$ für $\exists q: (p,a,q)\in\delta$,
  \item $p\overset{w}{\rightarrow} q$ für $p \overset{a_1}{\rightarrow} p_1
    \overset{a_2}{\rightarrow} p_2\dots \overset{a_n}{\rightarrow} q$ mit $w\in
    (\Sigma\cup\{\tau\})^*, w=a_1a_2\dots a_n$,
  \item $p\overset{w}{\rightarrow}$ für $p \overset{a_1}{\rightarrow}
    \overset{a_2}{\rightarrow} \dots \overset{a_n}{\rightarrow}$ mit $w\in
    (\Sigma\cup\{\tau\})^*, w=a_1a_2\dots a_n$,
  \item $w|_B$ steht für die Zeichenfolge, die aus $w$ entsteht durch löschen
    aller Zeichen, die nicht in $B\subseteq\Sigma$ enthalten sind, d.h.\ es
    bezeichnet die Projektion von $w$ auf die Menge $B$,
  \item $p\overset{w}{\Rightarrow} q$ für $w\in\Sigma^*$ mit $\exists
    w'\in(\Sigma\cup\{\tau\})^*:w'|_{\Sigma}=w\wedge p\overset{w'}{\rightarrow}
    q$,
  \item $p\overset{w}{\Rightarrow}$ für $\exists q:p\overset{w}{\Rightarrow}
    q$.
\end{itemize}
Die Sprache von $S$ ist
$L(S)=\{w\in\Sigma^*\mid q_0\overset{w}{\Rightarrow}\}$.

\section{Parallelkomposition}
Zwei EIOs sind komponierbar, wenn ihre Outputaktionsmengen disjunkt sind. Die
Error-Zustände der Parallelkomposition setzten sich aus den Error-Zuständen der
beiden zusammengesetzten Komponenten (geerbte Errors) und den Outputs zusammen, die von der
anderen Komponente nicht als Inputs angenommen werden können (neue Errors).

\begin{Def}[Parallelkomposition]
  Zwei EIOs $S_1, S_2$ sind komponierbar, falls
  $O_1\cap O_2=\emptyset$ gilt. Die Parallelkomposition ist
  $S_1\|S_2=(Q,I,O,\delta ,q_0,E)$ mit den Komponenten:
  \begin{itemize}
    \item $Q=Q_1\times Q_2$,
    \item $I=(I_1\backslash O_2)\cup(I_2\backslash O_1)$,
    \item $O=O_1\cup O_2$,
    \item $q_0=(q_{01},q_{02})$,
    \item $\begin{aligned}[t]
    \delta =&\{((q_1,q_2),\alpha ,(p_1,q_2))\mid (q_1,\alpha ,p_1)\in\delta
      _1,\alpha\in(\Sigma _1\cup\{\tau\})\backslash Synch(S_1,S_2)\}\\
      &\cup\{((q_1,q_2),\alpha ,(q_1,p_2))\mid (q_2,\alpha ,p_2)\in\delta
      _2,\alpha\in(\Sigma _2\cup\{\tau\})\backslash Synch(S_1,S_2)\}\\
      &\cup\{((q_1,q_2),\alpha ,(p_1,p_2))\mid (q_1,\alpha ,p_1)\in\delta
      _1, (q_2,\alpha ,p_2)\in\delta _2, \alpha\in Synch(S_1,S_2)\},
  \end{aligned}$
    \item $\begin{aligned}[t]
        E=&(Q_1\times E_2)\cup (E_1\times Q_2)
        &&\phantom{neue}\mathrm{geerbte}~\mathrm{Errors}\\
        &\left.\begin{aligned}
        &\cup\{(q_1,q_2)\mid \exists a\in O_1\cap I_2: q_1\overset{a}{\rightarrow}\wedge
      q_2\overset{a}{\not{\hspace{-0.1cm}\rightarrow}}\}\\
      &\cup\{(q_1,q_2)\mid \exists a\in I_1\cap O_2:
q_1\overset{a}{\not{\hspace{-0.1cm}\rightarrow}}\wedge
q_2\overset{a}{\rightarrow}\}
\end{aligned}\hspace{1cm}\right\}
      &&\phantom{neue}\mathrm{neue}~\mathrm{Errors}.\\
  \end{aligned}$
  \end{itemize}
  Dabei werden die synchronisierten Handlungen $Synch(S_1,
  S_2)=(I_1\cap O_2)\cup(O_1\cap I_2)$ nicht versteckt, sondern als Outputs der
  Komposition beibehalten.
\end{Def}

Nun werden wir drauf eingehen, dass eine Parallelkomposition nicht nur für
Automaten betrachtet werden kann, sondern auch über Transitionsfolgen. Ein
\emph{Trace} ist dann das Wort, das aus den Inputs und Outputs besteht, mit denen die
Übergängen beschriftet sind.

\begin{Def}[Parallelkomposition auf Traces]
  Gegeben zwei EIOs $S_1$ und $S_2,
  w_1\in\Sigma _1, w_2\in\Sigma _2, W_1\subseteq\Sigma _1^*, W_2\subseteq\Sigma
  _2^*$:
  \begin{itemize}
    \item $w_1\| w_2:=\{w\in (\Sigma _1\cup\Sigma _2)^*\mid w|_{\Sigma _1}=w_1\wedge
      w|_{\Sigma _2}=w_2\}$,
    \item $W_1\| W_2:=\bigcup\hspace{1pt}\{w_1\| w_2\mid w_1\in W_1\wedge w_2\in W_2\}$.
  \end{itemize}
\end{Def}

Die Semantik der späteren Kapitel basiert darauf die jeweiligen Zustände, die
zu Problemen führen, mit ihren Traces zu betrachten. Um dies besser umsetzten zu
können, definieren wir eine $prune$-Funktion, die alle Outputs am Ende
eines Traces entfernt. Zusätzlich werden Funktionen definiert, die
die Traces beliebig fortsetzen.

\begin{Def}[Pruning und Fortsetzungs Funktionen]
  Für einen EIO $S$ definieren wir:
  \begin{itemize}
    \item $prune:\Sigma ^*\rightarrow\Sigma ^*, w\mapsto u$, mit $w=uv,
      u=\varepsilon\vee u\in\Sigma ^*\cdot I$ und $v\in O^*$,
    \item $cont:\Sigma ^*\rightarrow\mathfrak{P}(\Sigma ^*),
      w\mapsto\{wu\mid u\in\Sigma ^*\}$,
    \item $cont:\mathfrak{P}(\Sigma ^*)\rightarrow\mathfrak{P}(\Sigma ^*),
      L\mapsto\bigcup\hspace{1pt}\{cont(w)\mid w\in L\}$.
  \end{itemize}
\end{Def}

Für zwei komponierbare EIOs $S_1$ und $S_2$ ist ein Ablauf ihrer
Parallelkomposition $S_{12}=S_1\| S_2$ eine Transitionsfolge der Form $(p_1,p_2)
\overset{w}{\Rightarrow} (q_1,q_2)$ für ein $w\in\Sigma_{12}^*$. So ein Ablauf
kann auf Abläufe von $S_1$ und $S_2$ projiziert werden. Diese Projektion
erfüllen $p_i \overset{w_i}{\Rightarrow} q_i$ mit $w|_{\Sigma
_i}=w_i$ für $i=1,2$. Umgekehrt sind zwei Abläufe von $S_1$ und $S_2$,
die wie oben aufgebaut sind, Projektionen von genau einem Ablauf
$S_{12}$, der ebenfalls wie oben aufgebaut ist. Daraus folgt das folgende Lemma.

\begin{lem}[Sprache der Parallelkomposition]
  \label{LemmaSprache}
  Für zwei komponierbare EIOs $S_1$ und $S_2$ gilt: \[L_{12} := L(S_1\|S_2) =
  L(S_1)\|L(S_2).\]
\end{lem}
