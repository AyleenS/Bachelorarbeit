\chapter{Definitionen und Notationen}

Die Definitionen dieses Kapitels sind größtenteils aus~\cite{Vogler2014EIO}
übernommen. Hierbei handelt es sich um die Grundlagen der Transitionssysteme, mit denen
hier gearbeitet werden soll. Jedoch wurde angepasst, dass für die
Parallelkomposition die Input-Aktionen der Error-IO-Transitionssysteme
(\EIO{}s) nicht disjunkt sein müssen. Dies wäre eine unnötige Einschränkung.
Die nicht mit den entsprechenden Outputs synchronisierten Inputs der zu komponierenden \EIO{}s werden als
Inputs der Parallelkomposition übernommen. Zusätzlich verzichten wir hier im
Gegensatz zu~\cite{Vogler2014EIO} auf
das verbergen der synchronisierten Aktionen. Die gleiche Betrachtungsweise
wurde bereits in~\cite{Schlosser2012BA} gewählt, deshalb stimmen die
Definitionen in diesem Kapitel mit denen aus~\cite{Schlosser2012BA} überein,
jedoch wurde diese Arbeit nicht als direkte Quelle verwendet. Auch das Kapitel
Verfeinerung über Errortraces wurde bereits in~\cite{Schlosser2012BA}
behandelt, jedoch wurden alle Beweise unabhängig davon neu geführt.\\
Dadurch das die synchronisierten Aktionen nicht verborgen werden, haben wir hier
ein Modell, mit dem nicht nur zwei Systeme miteinander kommunizieren können
sondern viele. Ein Output eines Systems ist somit ein Art Multicast. Jedes
System, dass diesen Output als Input haben kann, empfängt ihn somit auch, da
bei der Output bei jeder Komposition weitergeleitet wird an andere Systeme.
Systeme, die jedoch den Output nicht als Input haben, werden von dieser
Nachricht nicht beeinträchtigt.

\section{Error-IO-Transitionssystem}
Die hier betrachteten \EIO{}s sind Systeme, deren Übergänge mit Inputs und Outputs
beschriftet sind. Jeder Übergang ist dabei mit einem Input oder einem Output
versehen. Ebenfalls zulässig ist eine Transitionsbeschriftung mit $\tau$, einer
\emph{internen}, unbeobachtbaren \emph{Aktion}. Diese interne
Aktion lässt also keine Interaktion mit
der Umwelt zu. In~\cite{Vogler2014EIO} entsteht das $\tau$ in vielen Fällen
durch das Verbergen der Inputs und Outputs dieses Übergangs, da diese in einer
Komposition synchronisiert wurden. Hier werden diese Aktionen jedoch nicht
verborgen. Jedoch werden wir im weiteren Verlauf noch das Hiding betrachten, in
dem Outputs durch interne Aktionen ersetzt werden.

\begin{Def}[Error-IO-Transitionssystem]
  Ein \emph{Error-IO-Transitionssystem \linebreak (\EIO{})} ist
  ein Tupel $S=(Q,I,O,\delta, q_0, E)$, mit den Komponenten:
  \begin{itemize}
    \item $Q$ $-$ die Menge der Zustände,
    \item $I,O$ $-$ die disjunkten Mengen der (sichtbaren) Input- und
      Outputaktionen,
    \item $\delta\subseteq Q\times (I\cup O\cup\{\tau\})\times Q$ $-$ die
      Übergangsrelation,
    \item $q_0\in Q$ $-$ der Startzustand,
    \item $E\subseteq Q$ $-$ die Menge der Error-Zustände.
  \end{itemize}
\end{Def}

Die \emph{Aktionsmenge} eines \EIO{}s $S$ ist $\Sigma = I\cup O$ und die
\emph{Signatur} $Sig(S)=(I,O)$.\\
Um in graphischen Veranschaulichungen Inputs und Outputs zu unterscheiden wird
folgende Notation verwendet: $x?$ für den Input $x$ und $x!$ für den Output
$x$. Falls ein $x$ ohne $?$ oder $!$ verwendet wird, steht dies für eine
Aktion, bei der nicht festgelegt ist, ob sie ein Input oder ein Output ist.\\
Um die Komponenten der entsprechenden Struktur zuzuordnen, werden für
die Komponenten die gleichen Indizes wie für ihre zugehörigen Struktur
verwendet, z.B.\ schreiben wir $I_1$ für die Inputmenge des Transitionssystem $S_1$.
Diese Notation verwenden wir später analog für die Sprachen einer Struktur.\\
Die Elemente der Übergangsrelation $\delta$ werden wir wie folgt notieren:
\begin{itemize}
  \item $p\overset{\alpha}{\rightarrow} q$ für $(p,\alpha ,q)\in\delta$,
  \item $p\overset{\alpha}{\rightarrow}$ für $\exists q: (p,\alpha ,q)\in\delta$,
  \item $p\overset{w}{\rightarrow} q$ für $p \overset{\alpha _1}{\rightarrow}
    p_1 \overset{\alpha _2}{\rightarrow} p_2\dots \overset{\alpha
    _n}{\rightarrow} q$ mit $w\in (\Sigma\cup\{\tau\})^*, w=\alpha _1\alpha
    _2\dots \alpha _n$,
  \item $p\overset{w}{\rightarrow}$ für $p \overset{\alpha _1}{\rightarrow}
    \overset{\alpha _2}{\rightarrow} \dots \overset{\alpha _n}{\rightarrow}$
    mit $w\in (\Sigma\cup\{\tau\})^*, w=\alpha _1\alpha _2\dots \alpha _n$,
  \item $w|_B$ steht für die Zeichenfolge, die aus $w$ entsteht durch Löschen
    aller Zeichen, die nicht in $B\subseteq\Sigma$ enthalten sind, d.h.\ es
    bezeichnet die Projektion von $w$ auf die Menge $B$,
  \item $p\overset{w}{\Rightarrow} q$ für $w\in\Sigma^*$ mit $\exists
    w'\in(\Sigma\cup\{\tau\})^*:w'|_{\Sigma}=w\wedge p\overset{w'}{\rightarrow}
    q$,
  \item $p\overset{w}{\Rightarrow}$ für $\exists q:p\overset{w}{\Rightarrow}
    q$.
\end{itemize}
Die \emph{Sprache} von $S$ ist
$L(S)=\{w\in\Sigma^*\mid q_0\overset{w}{\Rightarrow}\}$.

\section{Parallelkomposition}
Zwei \EIO{}s sind komponierbar, wenn ihre Outputmengen disjunkt sind. Die
Error-Zustän\-de der Parallelkomposition setzten sich aus den Error-Zuständen der
beiden zusammengesetzten Komponenten (geerbte Errors) und den
unsynchronisierbaren Outputs (neue Errors) zusammen.\\
Die nächste Definition ist noch analog zu~\cite{Vogler2014EIO}, nur wird hier
darauf verzichtet die Inputmengen als disjunkt anzunehmen und der zweite
Definitionsteil über das verbergen der synchronisierten Aktionen wird
komplett weg gelassen. Das verzichten auf die analogen Definitionen für das
Verbergen wird sich auch bei den folgenden Definitionen zeigen. Zusätzlich
nehmen wir in der folgenden Definition eine Änderung an der Menge der
synchronisierten Aktionen vor, da nun nicht mehr $I_1\cap I_2 =\emptyset$
gelten muss, werden wir diese gemeinsamen Inputs synchronisieren. %TODO: lieber
%ein mal in Einleitung erwähnen und dann nicht mehr

\begin{Def}[Parallelkomposition]
  Zwei \EIO{}s $S_1, S_2$ sind \emph{komponierbar}, falls
  $O_1\cap O_2=\emptyset$ gilt. Die \emph{Parallelkomposition} ist
  $S_1\|S_2=(Q,I,O,\delta ,q_0,E)$ mit den Komponenten:
  \begin{itemize}
    \item $Q=Q_1\times Q_2$,
    \item $I=(I_1\backslash O_2)\cup(I_2\backslash O_1)$,
    \item $O=O_1\cup O_2$,
    \item $q_0=(q_{01},q_{02})$,
    \item $\begin{aligned}[t]
    \delta =&\{((q_1,q_2),\alpha ,(p_1,q_2))\mid (q_1,\alpha ,p_1)\in\delta
      _1,\alpha\in(\Sigma _1\cup\{\tau\})\backslash Synch(S_1,S_2)\}\\
      &\cup\{((q_1,q_2),\alpha ,(q_1,p_2))\mid (q_2,\alpha ,p_2)\in\delta
      _2,\alpha\in(\Sigma _2\cup\{\tau\})\backslash Synch(S_1,S_2)\}\\
      &\cup\{((q_1,q_2),\alpha ,(p_1,p_2))\mid (q_1,\alpha ,p_1)\in\delta
      _1, (q_2,\alpha ,p_2)\in\delta _2, \alpha\in Synch(S_1,S_2)\},
  \end{aligned}$
    \item $\begin{aligned}[t]
        E=&(Q_1\times E_2)\cup (E_1\times Q_2)
        &&\phantom{neue}\mathrm{geerbte}~\mathrm{Errors}\\
        &\left.\begin{aligned}
        &\cup\{(q_1,q_2)\mid \exists a\in O_1\cap I_2: q_1\overset{a}{\rightarrow}\wedge
      q_2\overset{a}{\not{\hspace{-0.1cm}\rightarrow}}\}\\
      &\cup\{(q_1,q_2)\mid \exists a\in I_1\cap O_2:
q_1\overset{a}{\not{\hspace{-0.1cm}\rightarrow}}\wedge
q_2\overset{a}{\rightarrow}\}
\end{aligned}\hspace{1cm}\right\}
      &&\phantom{neue}\mathrm{neue}~\mathrm{Errors}.\\
  \end{aligned}$
  \end{itemize}
  Dabei werden die \emph{synchronisierten Aktionen} $Synch(S_1,
  S_2)=(I_1\cap O_2)\cup(O_1\cap I_2)\cup (I_1\cap I_2)$ nicht versteckt,
  sondern als Outputs der Komposition beibehalten.\\
  Wir nennen $S_1$ einen \emph{Partner} von $S_2$, wenn ihre
  Parallelkomposition geschlossen ist, d.h.\ wenn sie duale Signaturen haben
  $Sig(S_1)=(I,O)$ und $Sig(S_2)=(O,I)$.
\end{Def}

Die Parallelkomposition kann nicht nur für Transitionssysteme betrachtet
werden, sondern auch über Aktionsfolgen. \emph{Traces} sind die möglichen Wege des
Systems, während ein bestimmtes Wort verarbeitet wird. Dieses Wort besteht aus
Inputs und Outputs, mit denen die Folge ab $q_0$ beschriftet ist.

\begin{Def}[Parallelkomposition auf Traces]
  Gegeben zwei \EIO{}s $S_1$ und $S_2$, sowie $w_1\in\Sigma _1, w_2\in\Sigma
  _2, W_1\subseteq\Sigma _1^*, W_2\subseteq\Sigma _2^*$:
  \begin{itemize}
    \item $w_1\| w_2:=\{w\in (\Sigma _1\cup\Sigma _2)^*\mid w|_{\Sigma _1}=w_1\wedge
      w|_{\Sigma _2}=w_2\}$,
    \item $W_1\| W_2:=\bigcup\hspace{1pt}\{w_1\| w_2\mid w_1\in W_1\wedge w_2\in W_2\}$.
  \end{itemize}
\end{Def}

Die Semantik der späteren Kapitel basiert darauf die jeweiligen Zustände, die
zu Problemen führen, mit den Traces zu betrachten, mit denen man zu diesem
Zuständen gelangen kann. Um dies besser umsetzten zu
können, definieren wir eine $\prune{}$-Funktion, die alle Outputs am Ende
eines Traces entfernt. Zusätzlich werden Funktionen definiert, die
die Traces beliebig fortsetzen.

\begin{Def}[Pruning- und Fortsetzungs-Funktion]
  Für ein \EIO{} $S$ definieren wir:
  \begin{itemize}
    \item $\prune{}:\Sigma ^*\rightarrow\Sigma ^*, w\mapsto u$, mit $w=uv,
      u=\varepsilon\vee u\in\Sigma ^*\cdot I$ und $v\in O^*$,
    \item $\cont{}:\Sigma ^*\rightarrow\mathfrak{P}(\Sigma ^*),
      w\mapsto\{wu\mid u\in\Sigma ^*\}$,
    \item $\cont{}:\mathfrak{P}(\Sigma ^*)\rightarrow\mathfrak{P}(\Sigma ^*),
      L\mapsto\bigcup\hspace{1pt}\{\cont{}(w)\mid w\in L\}$.
  \end{itemize}
\end{Def}

Für zwei komponierbare \EIO{}s $S_1$ und $S_2$ ist ein Ablauf ihrer
Parallelkomposition $S_{12}=S_1\| S_2$ eine Transitionsfolge der Form $(p_1,p_2)
\overset{w}{\Rightarrow} (q_1,q_2)$ für ein $w\in\Sigma_{12}^*$. So ein Ablauf
kann auf Abläufe von $S_1$ und $S_2$ projiziert werden. Diese Projektionen
erfüllen $p_i \overset{w_i}{\Rightarrow} q_i$ mit $w|_{\Sigma
_i}=w_i$ für $i=1,2$. Umgekehrt sind zwei Abläufe von $S_1$ und $S_2$ der Form
$p_i \overset{w_i}{\Rightarrow} q_i$ mit $w| _{\Sigma _i}= w_i$ für $i=1,2$,
Projektionen von genau einem Ablauf in $S_{12}$ der Form $(p_1,p_2)
\overset{w}{\Rightarrow} (q_1.q_2)$. Es ist dafür nötig, dass die Abläufe der
beiden Systeme und die Systeme selbst komponierbar sind. Dadurch, dass wir dann
ein $w$ wählen, dass projiziert auf die einzelnen Alphabete das jeweilige Wort
ergibt, können wir auch sagen, dass nur ein Ablauf möglich ist. Daraus folgt
das folgende Lemma.

\begin{lem}[Sprache der Parallelkomposition]
  \label{LemmaSprache}
  Für zwei komponierbare \EIO{}s $S_1$ und $S_2$ gilt: \[L_{12} := L(S_1\|S_2) =
  L(S_1)\|L(S_2).\]
\end{lem}

Dieses Lemma ist bereits in~\cite{Vogler2014EIO} enthalten. Hier wird jedoch
keine Sprache für die Parallelkomposition mit Verbergen erwähnt.
