\section{Hiding}
\begin{frame}
  \frametitle{Hiding}
  Ab hier wird nun nur noch Ruhe und nicht mehr Divergenz betrachtet.
  \begin{Def}[Internalisierungsoperator]
    Für ein \EIO{} $S=(Q,I,O,\delta ,q_0.E)$ ist $S/X$, mit
    dem \textbf{Internalisierungsoperator} $\cdot /\cdot$,
    definiert als $(Q,I,O',\delta ', q_0,E)$ mit:
    \begin{itemize}
      \item $\tau \notin X$,
      \item $X\subseteq O$,
      \item $O'=O\backslash X$,
      \item $\delta '=\left(\delta\cup\left\{(q,\tau ,q')\mid (q,x,q')\in\delta
        ,x\in X\right\}\right)\backslash \left\{(q,x,q')\mid x\in X\right\}$.
    \end{itemize}
  \end{Def}
  \uncover<2>{%
  \begin{prop}[Basisrelation bzgl. Internalisierung]
    Wenn $S_1\QBRel S_2$ gilt, dann folgt daraus, dass auch $S_1/X\QBRel S_2/X$
    gilt.
\end{prop}}
\end{frame}

\begin{frame}
  \begin{satz}[Präkongruenz bzgl. Internalisierung]
    Seien $S_1$ und $S_2$ zwei \EIO{}s für die $S_1\QRel S_2$ gilt, somit gilt
    auch $S_1/X\QRel S_2/X$. Es ist also \QRel{} eine Präkongruenz bezüglich
    $\cdot /\cdot$. Es gilt für die Sprachen und Traces:
    \begin{itemize}
      \item[(i)] $L(S/X) = \left\{w\in (\Sigma\backslash X){}^*\mid \exists w'\in L(S):
        w'|_{\Sigma\backslash X} = w\right\}$,
      \item[(ii)] $\ET{}(S/X) = \left\{w\in (\Sigma\backslash X){}^*\mid \exists
        w'\in \ET{}(S): w'|_{\Sigma\backslash X} = w\right\}$,
      \item[(iii)] $\EL{}(S/X) = \left\{w\in (\Sigma\backslash X){}^*\mid \exists w'\in
        \EL(S): w'|_{\Sigma\backslash X} = w\right\}$,
      \item[(iv)] $\QT{}(S/X) = \left\{w\in (\Sigma\backslash X){}^*\mid \exists w'\in
        \QT{}(S):w'|_{\Sigma\backslash X}=w\right\}$.
    \end{itemize}
  \end{satz}
\end{frame}

\begin{frame}
  \begin{Def}[Parallelkomposition mit Internalisierung]
    Seinen $S_1$ und $S_2$ komponierbare \EIO{}s, dann ist die
    Parallelkomposition mit Internalisierung definiert als $S_1|S_2 = S_{12}/(
    \Synch(S_1,S_2) \cap O_{12})$.
  \end{Def}
  \uncover<2>{%
  \begin{kor}[Präkongruenz mit Internalisierung]
    Die Relation \QRel{} ist eine Präkongruenz bezüglich der Parallelkomposition
    mit Internalisierung $\cdot |\cdot$.
\end{kor}}
\end{frame}
