\section{weitere Verfeinerungen}
\begin{frame}
  \frametitle{weitere Verfeinerungen}
  \begin{itemize}
    \item Verfeinerung für Error-Freiheit
    \item Verfeinerung für Error- und Ruhe-Freiheit
  \end{itemize}
\end{frame}

\section{Hiding}
\begin{frame}
  \frametitle{Hiding}
  \begin{Def}[Internalisierungsoperator]
    Für ein \EIO{} $S=(Q,I,O,\delta ,q_0.E)$ ist $S/X$, mit
    dem \textbf{Internalisierungsoperator} $\cdot /\cdot$,
    definiert als $(Q,I,O',\delta ', q_0,E)$ mit:
    \begin{itemize}
      \item $\tau \notin X$,
      \item $X\subseteq O$,
      \item $O'=O\backslash X$,
      \item $\delta '=\left(\delta\cup\left\{(q,\tau ,q')\mid (q,x,q')\in\delta
        ,x\in X\right\}\right)\backslash \left\{(q,x,q')\mid x\in X\right\}$.
    \end{itemize}
  \end{Def}
  \visible<2>{%
  \begin{Def}[Parallelkomposition mit Internalisierung]
    Seinen $S_1$ und $S_2$ komponierbare \EIO{}s, dann ist die
    Parallelkomposition mit Internalisierung definiert als $S_1|S_2 = S_{12}/(
    \Synch(S_1,S_2) \cap O_{12})$.
\end{Def}}
\end{frame}
