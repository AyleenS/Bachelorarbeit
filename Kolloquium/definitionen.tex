\section{Definitionen}
\begin{frame}
  \frametitle{Definitionen}
  \begin{Def}[Error-IO-Transitionssysteme]
    Ein \textbf{Error-IO-Transitionssysteme (\EIO{})} ist ein Tupel $S=(Q,I,O,\delta
    ,q_0,E)$ mit den Komponenten:
    \begin{itemize}
      \item $Q$ - die Menge der Zustände,
      \item $I,O$ - die disjunkte Menge der (sichtbaren) Input- und
        Output-Aktionen,
      \item $\delta\subseteq Q\times (I\cup O\cup \{\tau\})\times Q$ - die
        Transitionsrelation,
      \item $q_0\in Q$ - der Startzustand,
      \item $E\subseteq Q$ - die Menge der Error-Zustände.
    \end{itemize}
    Aktionsmenge von S: $\Sigma = I\cup O$\\
    Signatur: $\mathrm{Sig}(S)= (I,O)$
  \end{Def}
\end{frame}

\begin{frame}
  \begin{Def}[Parallelkomposition]
    Zwei \EIO{}s $S_1,S_2$ sind \textbf{komponierbar}, falls $O_1\cap
    O_2=\emptyset$ gilt. Die Parallelkomposition der \EIO{}s $S_1$ und $S_2$ ist
    $S_{12}:= S_1\|S_2= (Q,I,O,\delta ,q_0,E)$ mit den Komponenten:
    \begin{itemize}
      \item $Q=Q_1\times Q_2$,
      \item $I=(I_1\backslash O_2)\cup(I_2\backslash O_1)$,
      \item $O=O_1\cup O_2$,
      \item $q_0=(q_{01},q_{02})$,
      \only<1>{%
      \item $\begin{aligned}[t]
      \delta =&\left\{((q_1,q_2),\alpha ,(p_1,q_2))\mid (q_1,\alpha ,p_1)\in\delta
        _1,\right.\\ &\left.\alpha\in(\Sigma _1\cup\{\tau\})\backslash \Synch(S_1,S_2)\right\}\\
        &\cup\left\{((q_1,q_2),\alpha ,(q_1,p_2))\mid (q_2,\alpha ,p_2)\in\delta
        _2,\right.\\ &\left.\alpha\in(\Sigma _2\cup\{\tau\})\backslash \Synch(S_1,S_2)\right\}\\
        &\cup\left\{((q_1,q_2),\alpha ,(p_1,p_2))\mid (q_1,\alpha ,p_1)\in\delta
        _1, (q_2,\alpha ,p_2)\in\delta _2,\right.\\ &\left.\alpha\in \Synch(S_1,S_2)\right\},
      \end{aligned}$
      \item $E = \dots$.
      }
      \only<2>{%
      \item $\delta = \dots$,
      \item $\begin{aligned}[t]
          E=&(Q_1\times E_2)\cup (E_1\times Q_2)\\
          &\begin{aligned}
          &\cup\left\{(q_1,q_2)\mid \exists a\in O_1\cap I_2: q_1\overset{a}{\rightarrow}\wedge
        q_2\overset{a}{\not{\hspace{-0.1cm}\rightarrow}}\right\}\\
        &\cup\left\{(q_1,q_2)\mid \exists a\in I_1\cap O_2:
        q_1\overset{a}{\not{\hspace{-0.1cm}\rightarrow}}\wedge
        q_2\overset{a}{\rightarrow}\right\}.
      \end{aligned}\\
      \end{aligned}$
    }
    \end{itemize}
  \end{Def}
\end{frame}

\begin{frame}
  \emph{Traces} sind die möglichen Wege eines \EIO{}s, mit ihrer
  Transitionsbeschriftung.
  \visible<2>{%
  \begin{Def}[Pruning- und Fortsetzungs-Funktion]
    Für ein \EIO{} S wird definiert:
    \begin{itemize}
      \item $\prune :\Sigma ^*\rightarrow\Sigma ^*, w\mapsto u$, mit $w=uv,
        u=\varepsilon \wedge u\in\Sigma ^* \cdot I$ und $v\in O^*$,
      \item $\cont :\Sigma ^*\rightarrow \mathfrak{P}(\Sigma ^*), w\mapsto
        \{wu\mid u\in\Sigma ^*\}$,
      \item $\cont :\mathfrak{P}(\Sigma ^*)\rightarrow \mathfrak{P}(\Sigma ^*),
        L\mapsto \bigcup\,\{\cont (w)\mid w\in L\}$.
    \end{itemize}
\end{Def}}
\end{frame}

\begin{frame}
  \begin{Def}[Ruhe]
    Ein \textbf{Ruhe-Zustand} ist ein Zustand in einem \EIO{}, der keine
    Outputs und kein $\tau$ zulässt.\\
    Die Menge der Ruhe-Zustände in einem \EIO{} ist wie folgt formal definiert:
    $Qui := \left\{q\in Q\mid\forall\alpha\in (O\cap\{\tau\}):
  q\overset{\alpha}{\not{\hspace{-0.1cm}\rightarrow}}\right\}$.
  \end{Def}
  \visible<2>{%
  \begin{Def}[Divergenz]
    Ein \textbf{Divergenz-Zustand} ist ein Zustand in einem \EIO{}, der eine
    unendliche Folge von $\tau$s ausführen kann.\\
    Die Menge $Div(S)$ besteht aus all diesen divergenten Zuständen des \EIO{}s
    $S$.
\end{Def}}
\end{frame}
