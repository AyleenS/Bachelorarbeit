\documentclass[mathserif]{beamer}
\usepackage{ngerman,amsmath,amssymb,graphicx,tikz,xcolor}
\usepackage[utf8]{inputenc}
\usepackage[ngerman]{babel}

\usetikzlibrary{shapes,arrows}

\usetheme{AnnArbor}
\usecolortheme{crane}
\usefonttheme{structuresmallcapsserif}

\title{Kommunikationsfehler, Verklemmung und Divergenz bei Interface-Automaten}
\subtitle{Kolloquium zur Bachelorarbeit}
\author{Ayleen Schinko}
\date{\today}

\begin{document}
\maketitle
\section{Inhalt}
\begin{frame}
  \begin{itemize}
      \item Motivation
      \item Definitionen
      \item Verfeinerung bezüglich Kommunikationsfehler, Verklemmung und
        Divergenz
  \end{itemize}
\end{frame}

\section{Motivation}
\begin{frame}
  \begin{center}
    \Large{Motivation}
  \end{center}
  \begin{itemize}
    \item Modellierung von Systemen und deren Kommunikationsverhalten
      (Parallelkomposition)
    \item simulation parallel arbeitender Softwarekomponenten
    \item Kommunikationsfehler in Interface-Automaten nicht zulässig, deshalb
      Error-IO-Transitionssysteme als Abwandlung davon betrachtet
      \begin{itemize}
        \item Kommunikationsfehler zwischen Komponenten
        \item Verklemmung innerhalb einer Softwarekomponenten (keine Outputs
          mehr möglich)
        \item Divergenz einer Softwarekomponenten (unendliche viele intere
          Aktionen)
      \end{itemize}
  \end{itemize}
\end{frame}

\section{Definitionen}
\begin{frame}
\end{frame}
\end{document}
