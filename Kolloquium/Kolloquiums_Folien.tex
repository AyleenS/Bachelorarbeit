\documentclass[mathserif, xcolor=dvipsnames]{beamer}
\usepackage{ngerman,amsmath,amssymb,graphicx,tikz,xcolor,lastpage}
\usepackage[utf8]{inputenc}
\usepackage[ngerman]{babel}
\newtheorem{Def}{Definition}

\newcommand{\EIO}{EIO}
\newcommand{\prune}{\ensuremath{\mathrm{prune}}}
\newcommand{\cont}{\ensuremath{\mathrm{cont}}}
\newcommand{\Sig}{\ensuremath{\mathrm{Sig}}}
\newcommand{\Synch}{\ensuremath{\mathrm{Synch}}}

\usetikzlibrary{shapes,arrows}

\usetheme{AnnArbor}
\usecolortheme{crane}
\usefonttheme{structuresmallcapsserif}

\setbeamercolor{title}{bg=YellowOrange}
\setbeamercolor{frametitle}{bg=YellowOrange}
\setbeamertemplate{footline}{%
  \leavevmode%
  \hbox{%
  \begin{beamercolorbox}[wd=.333333\paperwidth,ht=2.25ex,dp=1ex,center]{author in head/foot}%
    \usebeamerfont{author in head/foot}\insertshortauthor
  \end{beamercolorbox}%
  \begin{beamercolorbox}[wd=.333333\paperwidth,ht=2.25ex,dp=1ex,center]{title in head/foot}%
    \usebeamerfont{date in head/foot}\insertshortdate{}\hspace*{2em}
  \end{beamercolorbox}%
  \begin{beamercolorbox}[wd=.333333\paperwidth,ht=2.25ex,dp=1ex,right]{date in head/foot}%
    \insertframenumber{} / \inserttotalframenumber\hspace*{2ex}
  \end{beamercolorbox}}%
}
\setbeamertemplate{navigation symbols}{}

\title{Kommunikationsfehler, Verklemmung und Divergenz bei Interface-Automaten}
\subtitle{Kolloquium zur Bachelorarbeit}
\author{Ayleen Schinko}
\date{\today}

\begin{document}
\begin{frame}[plain]
\maketitle
\end{frame}
\section{Inhalt}
\begin{frame}
  \frametitle{Inhalt}
  \begin{itemize}
      \item Motivation
      \item Definitionen
      \item Verfeinerung bezüglich Kommunikationsfehler, Verklemmung und
        Divergenz
  \end{itemize}
\end{frame}

\section{Motivation}
\begin{frame}
  \frametitle{Motivation}
  \begin{itemize}
    \item Modellierung von Systemen und deren Kommunikationsverhalten
      (Parallelkomposition)
    \item simulation parallel arbeitender Softwarekomponenten
    \item Kommunikationsfehler in Interface-Automaten nicht zulässig, deshalb
      Error-IO-Transitionssysteme als Abwandlung davon betrachtet
      \begin{itemize}
        \item Kommunikationsfehler zwischen Komponenten
        \item Verklemmung innerhalb einer Softwarekomponenten (keine Outputs
          mehr möglich)
        \item Divergenz einer Softwarekomponenten (unendliche viele intere
          Aktionen)
      \end{itemize}
  \end{itemize}
\end{frame}

\section{Definitionen}
\begin{frame}
  \frametitle{Definitionen}
  \begin{Def}[Error-IO-Transitionssysteme]
    Ein \textbf{Error-IO-Transitionssysteme (\EIO{})} ist ein Tupel $S=(Q,I,O,\delta
    ,q_0,E)$ mit den Komponenten:
    \begin{itemize}
      \item $Q$ - die Menge der Zustände,
      \item $I,O$ - die disjunkte Menge der (sichtbaren) Input- und
        Output-Aktionen,
      \item $\delta\subseteq Q\times (I\cup O\cup \{\tau\})\times Q$ - die
        Transitionsrelation,
      \item $q_0\in Q$ - der Startzustand,
      \item $E\subseteq Q$ - die Menge der Error-Zustände.
    \end{itemize}
  \end{Def}
  Aktionsmenge von S: $\Sigma = I\cup O$\\
  Signatur: $\mathrm{Sig}(S)= (I,O)$
\end{frame}

\begin{frame}
  \begin{Def}[Parallelkomposition]
    Zwei \EIO{}s $S_1,S_2$ sind \textbf{komponierbar}, falls $O_1\cap
    O_2=\emptyset$ gilt. Die Parallelkomposition der \EIO{}s $S_1$ und $S_2$ ist
    $S_{12}:= S_1\|S_2= (Q,I,O,\delta ,q_0,E)$ mit den Komponenten:
    \begin{itemize}
      \item $Q=Q_1\times Q_2$,
      \item $I=(I_1\backslash O_2)\cup(I_2\backslash O_1)$,
      \item $O=O_1\cup O_2$,
      \item $q_0=(q_{01},q_{02})$,
      \only<1>{
      \item $\begin{aligned}[t]
      \delta =&\left\{((q_1,q_2),\alpha ,(p_1,q_2))\mid (q_1,\alpha ,p_1)\in\delta
        _1,\right.\\ &\left.\alpha\in(\Sigma _1\cup\{\tau\})\backslash \Synch(S_1,S_2)\right\}\\
        &\cup\left\{((q_1,q_2),\alpha ,(q_1,p_2))\mid (q_2,\alpha ,p_2)\in\delta
        _2,\right.\\ &\left.\alpha\in(\Sigma _2\cup\{\tau\})\backslash \Synch(S_1,S_2)\right\}\\
        &\cup\left\{((q_1,q_2),\alpha ,(p_1,p_2))\mid (q_1,\alpha ,p_1)\in\delta
        _1, (q_2,\alpha ,p_2)\in\delta _2,\right.\\ &\left.\alpha\in \Synch(S_1,S_2)\right\},
      \end{aligned}$
      \item $E = \dots$.
      }
      \only<2>{
      \item $\delta = \dots$,
      \item $\begin{aligned}[t]
          E=&(Q_1\times E_2)\cup (E_1\times Q_2)\\
          &\begin{aligned}
          &\cup\left\{(q_1,q_2)\mid \exists a\in O_1\cap I_2: q_1\overset{a}{\rightarrow}\wedge
        q_2\overset{a}{\not{\hspace{-0.1cm}\rightarrow}}\right\}\\
        &\cup\left\{(q_1,q_2)\mid \exists a\in I_1\cap O_2:
        q_1\overset{a}{\not{\hspace{-0.1cm}\rightarrow}}\wedge
        q_2\overset{a}{\rightarrow}\right\}.
      \end{aligned}\\
      \end{aligned}$
    }
    \end{itemize}
  \end{Def}
\end{frame}

\begin{frame}
  \begin{Def}[Pruning- und Fortsetzungs-Funktion]
    Für ein \EIO{} S wird definiert:
    \begin{itemize}
      \item $\prune :\Sigma ^*\rightarrow\Sigma ^*, w\mapsto u$, mit $w=uv,
        u=\varepsilon \wedge u\in\Sigma ^* \cdot I$ und $v\in O^*$,
      \item $\cont :\Sigma ^*\rightarrow \mathfrak{P}(\Sigma ^*), w\mapsto
        \{wu\mid u\in\Sigma ^*\}$,
      \item $\cont :\mathfrak{P}(\Sigma ^*)\rightarrow \mathfrak{P}(\Sigma ^*),
        L\mapsto \bigcup\,\{\cont (w)\mid w\in L\}$.
    \end{itemize}
  \end{Def}
\end{frame}
\end{document}
