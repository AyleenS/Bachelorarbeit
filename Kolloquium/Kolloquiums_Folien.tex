\documentclass[mathserif, xcolor=dvipsnames]{beamer}
\usepackage{ngerman,amsmath,amssymb,graphicx,xcolor,lastpage}
\usepackage[utf8]{inputenc}
\usepackage[ngerman]{babel}

\usepackage{tikz}
\usetikzlibrary{arrows, automata, calc, matrix, shapes, positioning}

\newtheorem{Def}{Definition}
\newtheorem{satz}[Def]{Satz}
\newtheorem{bsp}[Def]{Beispiel}
\newtheorem{lem}[Def]{Lemma}
\newtheorem{kor}[Def]{Korollar}
\newtheorem{prop}[Def]{Proposition}

\newcommand{\EIO}{EIO}
\newcommand{\prune}{\ensuremath{\mathrm{prune}}}
\newcommand{\cont}{\ensuremath{\mathrm{cont}}}
\newcommand{\Sig}{\ensuremath{\mathrm{Sig}}}
\newcommand{\Synch}{\ensuremath{\mathrm{Synch}}}

\newcommand{\EBRel}{\ensuremath{\sqsubseteq _E^\mathrm{B}}}
\newcommand{\ECRel}{\ensuremath{\sqsubseteq _E^\mathrm{C}}}
\newcommand{\ERel}{\ensuremath{\sqsubseteq _E}}
\newcommand{\StET}{\ensuremath{StET}}
\newcommand{\PrET}{\ensuremath{PrET}}
\newcommand{\MIT}{\ensuremath{MIT}}
\newcommand{\ET}{\ensuremath{ET}}
\newcommand{\EL}{\ensuremath{EL}}

\newcommand{\QBRel}{\ensuremath{\sqsubseteq _{Qui}^\mathrm{B}}}
\newcommand{\QCRel}{\ensuremath{\sqsubseteq _{Qui}^\mathrm{C}}}
\newcommand{\QRel}{\ensuremath{\sqsubseteq _{Qui}}}
\newcommand{\StQT}{\ensuremath{StQT}}
\newcommand{\PrQT}{\ensuremath{PrQT}}
\newcommand{\QT}{\ensuremath{QET}}

\newcommand{\DBRel}{\ensuremath{\sqsubseteq _{Div}^\mathrm{B}}}
\newcommand{\DBaRel}{\ensuremath{\sqsubseteq _{Div_{\mathrm{alt}}}^\mathrm{B}}}
\newcommand{\DCRel}{\ensuremath{\sqsubseteq _{Div}^\mathrm{C}}}
\newcommand{\DCaRel}{\ensuremath{\sqsubseteq _{Div_{\mathrm{alt}}}^\mathrm{C}}}
\newcommand{\DRel}{\ensuremath{\sqsubseteq _{Div}}}
\newcommand{\DaRel}{\ensuremath{\sqsubseteq _{Div_{\mathrm{alt}}}}}
\newcommand{\StDT}{\ensuremath{StDT}}
\newcommand{\PrDT}{\ensuremath{PrDT}}
\newcommand{\EDT}{\ensuremath{EDT}}
\newcommand{\EDL}{\ensuremath{EDL}}
\newcommand{\QDT}{\ensuremath{QDT}}
\newcommand{\DT}{\ensuremath{DT}}

\usetheme{AnnArbor}
\usecolortheme{crane}
\usefonttheme{structuresmallcapsserif}

\setbeamercolor{title}{bg=YellowOrange}
\setbeamercolor{frametitle}{bg=YellowOrange}
\setbeamertemplate{footline}{%
  \leavevmode%
  \hbox{%
  \begin{beamercolorbox}[wd=.333333\paperwidth,ht=2.25ex,dp=1ex,center]{author in head/foot}%
    \usebeamerfont{author in head/foot}\insertshortauthor{}
  \end{beamercolorbox}%
  \begin{beamercolorbox}[wd=.333333\paperwidth,ht=2.25ex,dp=1ex,center]{title in head/foot}%
    \usebeamerfont{date in head/foot}\insertshortdate{}\hspace*{2em}
  \end{beamercolorbox}%
  \begin{beamercolorbox}[wd=.333333\paperwidth,ht=2.25ex,dp=1ex,right]{date in head/foot}%
    \insertframenumber{} / \inserttotalframenumber\hspace*{2ex}
  \end{beamercolorbox}}%
}
\setbeamertemplate{navigation symbols}{}

\title{Kommunikationsfehler, Verklemmung und Divergenz bei Interface-Automaten}
\subtitle{Kolloquium zur Bachelorarbeit}
\author{Ayleen Schinko}
\date{\today}

\begin{document}
\begin{frame}[plain]
\maketitle
\end{frame}

\begin{frame}
  \frametitle{Inhalt}
  \tableofcontents{}
\end{frame}

\section{Motivation}
\begin{frame}
  \frametitle{Motivation}
  \begin{itemize}
    \item Modellierung von Systemen und deren Kommunikationsverhalten
      (Parallelkomposition)
    \item simulation parallel arbeitender Softwarekomponenten
    \item Kommunikationsfehler in Interface-Automaten nicht zulässig, deshalb
      Error-IO-Transitionssysteme als Abwandlung davon betrachtet
      \begin{itemize}
        \item Kommunikationsfehler (bzw. Error) zwischen Komponenten
        \item Verklemmung (bzw. Ruhe) innerhalb einer Softwarekomponenten (keine
          Outputs mehr möglich)
        \item Divergenz einer Softwarekomponenten (unendliche viele intere
          Aktionen)
      \end{itemize}
  \end{itemize}
\end{frame}

\chapter{Definitionen und Notationen}

Die Definitionen dieses Kapitels sind größtenteils aus~\cite{Vogler2014EIO}
übernommen, mit den in der Einleitung erwähnten Abänderungen. In diesen
Definitionen werden die Grundlagen der Transitionssysteme, mit denen hier
gearbeitet werden soll, behandelt.

\section{Error-IO-Transitionssystem}
Die hier betrachteten \EIO{}s sind Systeme, deren Transitionen mit Inputs und
Outputs beschriftet sind. Jede Transition ist dabei mit einem Input oder einem
Output versehen. Ebenfalls zulässig ist eine Transitionsbeschriftung mit
$\tau$, einer \emph{internen}, unbeobachtbaren \emph{Aktion}. Diese interne
Aktion lässt also keine Interaktion mit der Umwelt, d.h.\ mit anderen Systemen,
zu. In~\cite{Vogler2014EIO} entsteht das $\tau$ in vielen Fällen durch das
Verbergen der Inputs und Outputs, die in einer Komposition synchronisiert
werden. Hier werden diese Aktionen hingegen nicht verborgen. Jedoch wird im
weiteren Verlauf noch das Hiding betrachten, in dem Outputs durch interne
Aktionen ersetzt werden.

\begin{Def}[Error-IO-Transitionssystem]
  Ein \emph{Error-IO-Transitionssystem \linebreak (\EIO{})} ist
  ein Tupel $S=(Q,I,O,\delta, q_0, E)$, mit den Komponenten:
  \begin{itemize}
    \item $Q$ $-$ die Menge der Zustände,
    \item $I,O$ $-$ die disjunkten Mengen der (sichtbaren) Input- und
      Output-Aktionen,
    \item $\delta\subseteq Q\times \left(I\cup O\cup\{\tau\}\right)\times Q$ $-$ die
      Transitionsrelation,
    \item $q_0\in Q$ $-$ der Startzustand,
    \item $E\subseteq Q$ $-$ die Menge der Error-Zustände.
  \end{itemize}
\end{Def}

Die \emph{Aktionsmenge} eines \EIO{}s $S$ ist $\Sigma = I\cup O$ und die
\emph{Signatur} $\Sig(S)=(I,O)$.\\
Um in graphischen Veranschaulichungen Inputs und Outputs zu unterscheiden, wird
folgende Notation verwendet: $x?$ für den Input $x$ und $x!$ für den Output
$x$. Falls ein $x$ ohne $?$ oder $!$ verwendet wird, steht dies für eine
Aktion, bei der nicht festgelegt ist, ob sie ein Input oder ein Output ist.\\
Um die Komponenten der entsprechenden Transitionssystem zuzuordnen, werden für
die Komponenten die gleichen Indizes wie für ihr zugehöriges System verwendet,
z.B.\ wird $I_1$ für die Inputmenge des Transitionssystems $S_1$ geschrieben. Diese
Notation wird später analog für die Sprachen, Traces und Zustandsmengen eines
Systems verwendet.\\
Die Elemente der Transitionsrelation $\delta$ werden wie folgt notieren:
\begin{itemize}
  \item $p\overset{\alpha}{\rightarrow} q$ für $(p,\alpha ,q)\in\delta$,
  \item $p\overset{\alpha}{\rightarrow}$ für $\exists q \in Q: (p,\alpha ,q)\in\delta$,
  \item $p\overset{w}{\rightarrow} q$ für $p \overset{\alpha _1}{\longrightarrow}
    p_1 \overset{\alpha _2}{\longrightarrow} p_2\dots \overset{\alpha
    _n}{\longrightarrow} q$ mit $w\in \left(\Sigma\cup\{\tau\}\right)^*, w=\alpha _1\alpha
    _2\dots \alpha _n$,
  \item $p\overset{w}{\rightarrow}$ für $p \overset{\alpha _1}{\longrightarrow}
    \overset{\alpha _2}{\longrightarrow} \dots \overset{\alpha _n}{\longrightarrow}$
    mit $w\in \left(\Sigma\cup\{\tau\}\right)^*, w=\alpha _1\alpha _2\dots \alpha _n$,
  \item $w|_B$ steht für die Zeichenfolge, die aus $w$ entsteht durch Löschen
    aller Zeichen, die nicht in $B\subseteq\Sigma$ enthalten sind, d.h.\ es
    bezeichnet die Projektion von $w$ auf die Menge $B$,
  \item $p\overset{w}{\Rightarrow} q$ für $w\in\Sigma^*$ mit $\exists
    w'\in\left(\Sigma\cup\{\tau\}\right)^*:w'|_{\Sigma}=w\wedge p\overset{w'}{\rightarrow}
    q$,
  \item $p\overset{w}{\Rightarrow}$ für $\exists q:p\overset{w}{\Rightarrow}
    q$.
\end{itemize}
Die \emph{Sprache} von $S$ ist
$L(S)=\left\{w\in\Sigma^*\mid q_0\overset{w}{\Rightarrow}\right\}$.

\section{Parallelkomposition}
Zwei \EIO{}s sind komponierbar, wenn ihre Output-Mengen disjunkt sind. Die
Error-Zustän\-de der Parallelkomposition setzen sich aus den Error-Zuständen der
beiden zusammengesetzten Komponenten (geerbte Errors) und den Zuständen, die
Outputs aus der Menge der synchronisierten Aktionen besitzen, für die im zu
komponierenden System jedoch kein passender Input vorhanden ist (neue Errors),
zusammen.\\
In der folgenden Definition muss eine Veränderung
gegenüber~\cite{Vogler2014EIO} an der Menge der synchronisierten
Aktionen vorgenommen werden. Da nicht mehr $I_1\cap I_2 =\emptyset$ gelten
muss, werden die gemeinsamen Inputs der Systeme synchronisiert. Somit handelt es sich in
der Parallelkomposition bei synchronisierten Aktionen nicht mehr nur um
Outputs, wie in~\cite{Vogler2014EIO}, sondern im Fall von $I_1\cap I_2$ auch um
Inputs. Falls es bei Inputs aus $I_1\cap I_2$ zu einem fehlenden Input für die
Synchronisation kommt, ist die Transition für die Parallelkomposition nicht
ausführbar, jedoch handelt es sich auch nicht um einen neuen Error, da es
zwischen den beiden Systemen dadurch nicht zu einem Kommunikationsfehler
kommt. Die beiden Transitionssysteme können über die beiden Inputs nicht
miteinander kommunizieren, sondern nur mit anderen Systemen.

\begin{Def}[Parallelkomposition]
\label{DefParallelkomposition}
  Zwei \EIO{}s $S_1, S_2$ sind \emph{komponierbar}, falls
  $O_1\cap O_2=\emptyset$ gilt. Die \emph{Parallelkomposition} der \EIO{}s
  $S_1$ und $S_2$ ist
  $S_{12}:=S_1\|S_2=(Q,I,O,\delta ,q_0,E)$ mit den Komponenten:
  \begin{itemize}
    \item $Q=Q_1\times Q_2$,
    \item $I=(I_1\backslash O_2)\cup(I_2\backslash O_1)$,
    \item $O=O_1\cup O_2$,
    \item $q_0=(q_{01},q_{02})$,
    \item $\begin{aligned}[t]
    \delta =&\left\{((q_1,q_2),\alpha ,(p_1,q_2))\mid (q_1,\alpha ,p_1)\in\delta
      _1,\alpha\in(\Sigma _1\cup\{\tau\})\backslash \Synch(S_1,S_2)\right\}\\
      &\cup\left\{((q_1,q_2),\alpha ,(q_1,p_2))\mid (q_2,\alpha ,p_2)\in\delta
      _2,\alpha\in(\Sigma _2\cup\{\tau\})\backslash \Synch(S_1,S_2)\right\}\\
      &\cup\left\{((q_1,q_2),\alpha ,(p_1,p_2))\mid (q_1,\alpha ,p_1)\in\delta
      _1, (q_2,\alpha ,p_2)\in\delta _2, \alpha\in \Synch(S_1,S_2)\right\},
  \end{aligned}$
    \item $\begin{aligned}[t]
        E=&(Q_1\times E_2)\cup (E_1\times Q_2)
        &&\phantom{neue}\mathrm{geerbte}~\mathrm{Errors}\\
        &\left.\begin{aligned}
        &\cup\left\{(q_1,q_2)\mid \exists a\in O_1\cap I_2: q_1\overset{a}{\rightarrow}\wedge
      q_2\overset{a}{\not{\hspace{-0.1cm}\rightarrow}}\right\}\\
      &\cup\left\{(q_1,q_2)\mid \exists a\in I_1\cap O_2:
q_1\overset{a}{\not{\hspace{-0.1cm}\rightarrow}}\wedge
q_2\overset{a}{\rightarrow}\right\}
\end{aligned}\hspace{1cm}\right\}
      &&\phantom{neue}\mathrm{neue}~\mathrm{Errors}.\\
  \end{aligned}$
  \end{itemize}
  Dabei werden die \emph{synchronisierten Aktionen} $\Synch(S_1,
  S_2)=(I_1\cap O_2)\cup(O_1\cap I_2)\cup (I_1\cap I_2)$ nicht versteckt,
  sondern als Outputs bzw.\ im Fall von $I_1\cap I_2$ als Inputs der
  Komposition beibehalten.\\
  Die oben definierte Notation $S_{12}=S_1\|S_2$ wird auch für andere
  Indizierungen der Systeme analog angewendet, so gilt also allgemein
  $S_{ij}:=S_i\|S_j$ für $i,j\in\mathbb{N}$.\\
  $S_1$ wird \emph{Partner} von $S_2$ genannt, wenn die Parallelkomposition von
  $S_1$ und $S_2$ geschlossen ist, d.h.\ wenn sie duale Signaturen
  $\Sig(S_1)=(I,O)$ und $\Sig(S_2)=(O,I)$ haben.
\end{Def}

Die Parallelkomposition kann nicht nur für Transitionssysteme betrachtet
werden, wie bisher in dieser Arbeit, sondern auch über Aktionsfolgen.
\emph{Traces} sind die möglichen Wege des Systems, mit ihrer
Transitionsbeschriftung. Diese Transitionsbeschriftung besteht aus Inputs und
Outputs, mit denen die Folge ab dem Startzustand $q_0$ beschriftet ist. Somit
kann ein Trace auch als das Wort aufgefasst werden, dass verarbeitet wird
während des Ablaufs des Systems.

\begin{Def}[Parallelkomposition auf Traces]
  Gegeben zwei \EIO{}s $S_1$ und $S_2$, sowie $w_1\in\Sigma _1, w_2\in\Sigma
  _2, W_1\subseteq\Sigma _1^*, W_2\subseteq\Sigma _2^*$:
  \begin{itemize}
    \item $w_1\| w_2:=\left\{w\in (\Sigma _1\cup\Sigma _2)^*\mid w|_{\Sigma _1}=w_1\wedge
      w|_{\Sigma _2}=w_2\right\}$,
    \item $W_1\| W_2:=\bigcup\hspace{1pt}\left\{w_1\| w_2\mid w_1\in W_1\wedge
      w_2\in W_2\right\}$.
  \end{itemize}
\end{Def}

Die Semantik der späteren Kapitel basiert darauf die jeweiligen Zustände, die
zu Problemen führen, mit den Traces zu betrachten, mit denen man diese
Zustände erreicht. Um dies besser umsetzen zu können, wird eine
$\prune{}$-Funktion definiert, die alle Outputs am Ende eines Traces entfernt.
Zusätzlich werden Funktionen definiert, die die Traces beliebig fortsetzen.

\begin{Def}[Pruning- und Fortsetzungs-Funktion]
  Für ein \EIO{} $S$ wird definiert:
  \begin{itemize}
    \item $\prune{}:\Sigma ^*\rightarrow\Sigma ^*, w\mapsto u$, mit $w=uv,
      u=\varepsilon\vee u\in\Sigma ^*\cdot I$ und $v\in O^*$,
    \item $\cont{}:\Sigma ^*\rightarrow\mathfrak{P}(\Sigma ^*),
      w\mapsto\left\{wu\mid u\in\Sigma ^*\right\}$,
    \item $\cont{}:\mathfrak{P}(\Sigma ^*)\rightarrow\mathfrak{P}(\Sigma ^*),
      L\mapsto\bigcup\hspace{1pt}\left\{\cont{}(w)\mid w\in L\right\}$.
  \end{itemize}
\end{Def}

Für zwei komponierbare \EIO{}s $S_1$ und $S_2$ ist ein Ablauf ihrer
Parallelkomposition $S_{12}$ eine Transitionsfolge der Form $(p_1,p_2)
\overset{w}{\Rightarrow} (q_1,q_2)$ für ein $w\in\Sigma_{12}^*$. So ein Ablauf
kann auf Abläufe von $S_1$ und $S_2$ projiziert werden. Diese Projektionen
erfüllen $p_i \overset{w_i}{\Rightarrow} q_i$ mit $w|_{\Sigma
_i}=w_i$ für $i=1,2$. Umgekehrt sind zwei Abläufe von $S_1$ und $S_2$ der Form
$p_i \overset{w_i}{\Rightarrow} q_i$ mit $w| _{\Sigma _i}= w_i$ für $i=1,2$,
Projektionen von einem Ablauf in $S_{12}$ der Form $(p_1,p_2)
\overset{w}{\Rightarrow} (q_1.q_2)$. Es ist dafür nötig, dass die Abläufe der
beiden Systeme $S_1$, $S_2$ und die Systeme selbst komponierbar sind. Das $w$ wurde so
gewählt, dass die Projektion auf die einzelnen Alphabete $\Sigma _1$ und $\Sigma
_2$ die jeweiligen Wörter $w_1$ und $w_2$ ergibt. Falls keine internen Aktionen
zugelassen wären, würde sogar nur genau ein Ablauf möglich sein in $S_{12}$. Da
jedoch auch interne Aktionen zulässig sind, sind mehrere Abläufe möglich, da
nicht klar ist, wann ein $\tau$ in einem Trace ausgeführt wird. Aus diesen
Feststellungen ergibt sich das folgende Lemma.

\begin{lem}[Sprache der Parallelkomposition]
\label{LemmaSprache}
  Für zwei komponierbare \EIO{}s $S_1$ und $S_2$ gilt: \[L_{12} := L(S_{12}) =
  L_1\|L_2.\]
\end{lem}

\section{Hiding}

Hiding wurde in dem hier verwendeten Kontext bereits in~\cite{Chilton2013} auf
Traces betrachtet. Da hier die Betrachtungsweise von Transitionssystemen aus
startet, wird auch Hiding aus der Sicht dieser Systeme definieren, wie
in~\cite{Schlosser2012BA}. Eine ähnliche Betrachtung für Hiding bei LTS mit
Inputs und Outputs wurde auch bereits in~\cite{Lynch1996} umgesetzt. Dort
werden nur Output-Aktionen internalisiert, jedoch gibt es eine Menge an
internen Aktionen und nicht nur eine. Das Hiding wird durch einen
Internalisierungsoperator umgesetzt. Es sollen dadurch Aktionen versteckt
werden können, d.h.\ durch $\tau$s ersetzt werden. In~\cite{Chilton2013} ist es
in der Definition des Hidings möglich Outputs und Inputs zu verstecken. Durch
das Verstecken von Outputs sind diese nach außen nicht mehr sichtbar. Werden
jedoch Inputs versteckt sind alle Traces, die diesen Input benötigen, nicht
mehr ausführbar. Sie sind dann ab dem versteckten Input nicht mehr
weiterführbar. Es handelt sich also um echte Einschränkungen des Systems. Die
Transitionen werden durch das Hiding von Inputs verboten, ähnlich wie bei der Anwendung
von Restriktionen in CSS, siehe dazu~\cite{Milner1989}. Diese Art der
Einschränkung der Transitionssysteme sollen hier jedoch nicht behandelt werden.
Somit wird in der folgenden Definition nur die Internalisierung von Outputs
erlaubt, entsprechend Quelle~\cite{Schlosser2012BA}.

\begin{Def}[Internalisierungsoperator]
  Für ein \EIO{} $S=(Q,I,O,\delta ,q_0.E)$ ist $S/X$, mit
  dem \emph{Internalisierungsoperator} $\cdot /\cdot$,
  definiert als $S'=(Q,I,O',\delta ', q_0,E)$ mit:
  \begin{itemize}
    \item $\tau \notin X$,
    \item $X\subseteq O$,
    \item $O'=O\backslash X$,
    \item $\delta '=\left(\delta\cup\left\{(q,\tau ,q')\mid (q,x,q')\in\delta
      ,x\in X\right\}\right)\backslash \left\{(q,x,q')\mid x\in X\right\}$.
  \end{itemize}
\end{Def}

Die Menge hinter dem Internalisierungsoperator ist in dieser Definition auf
Outputs beschränkt. Diese Einschränkung wurde vorgenommen, um die weitere
Betrachtung zu erleichtern. Jedoch kann es sinnvoll sein die Möglichkeit zu
haben dort weitere Aktionen aufnehmen zu können. Dies wird jedoch nicht mehr
Teil dieser Arbeit sein.\\
In~\cite{Vogler2014EIO} wird die Parallelkomposition nur mit Verbergen der
synchronisierten Aktionen betrachtet, die durch die Synchronisation von einem
Input mit einem Output entstehen. Diese Parallelkomposition wird nun mit dem
Internalisierungsoperator durch Hiding der synchronisierten Aktionen, die in
der Parallelkomposition zu Outputs werden, nachbildet. Da in dieser Arbeit
die Inputmengen der Systeme, die komponiert werden, nicht disjunkt sein müssen,
ergeben sich auch Inputs aus der Synchronisation von Aktionen. Diese können
jedoch mit der hier verwendeten Definition des Internalisierungsoperators nicht verborgen
werden. Dies wäre auch nicht sinnvoll, da diese Synchronisation von Inputs
keine Kommunikation zwischen den Systemen ist, sondern nur eine Zusammenfügung,
damit die Parallelkomposition über diesen Input mit weiteren Systemen
kommunizieren kann. Somit ergibt sich die folgende Definition, mit der die
Parallelkomposition aus~\cite{Vogler2014EIO} nachgebildet werden kann.

\begin{Def}[Parallelkomposition mit Internalisierung]
\label{defIntParal}
  Seinen $S_1$ und $S_2$ komponierbare \EIO{}s, dann ist
  $S_1|S_2=S_{12}/(\Synch(S_1,S_2)\cap O_{12})$.
\end{Def}


\chapter{Verfeinerung für Error-, Ruhe- und Divergenz-Freiheit}

\section{Präkongruenz für Divergenz}

In diesem Kapitel soll die Menge der betrachteten Zustände noch einmal
erweitert werden. Somit werden dann Errors, Ruhe-Zustände und
Divergente-Zustände betrachtet. Es eignet sich also~\cite{Chilton2013} als
Quelle, da nun auch noch die Divergenz betrachtet wird. Diese wurde dort
gleichzeitig mit der Ruhe eingeführt und betrachtet. Da es sich nur um eine
Erweiterung der Präkongruenzen aus den letzten beiden Kapiteln handelt, wird
dabei ähnlich vorgegangen wie in den letzten beiden Kapiteln.\\
Wie bereits oben und im letzten Kapitel erwähnt wurden in~\cite{Chilton2013}
auch noch divergente Zustände als Fehler-Zustände betrachtet. Um zu klären, was
darunter verstanden wird, wird nun noch eine Definition für Divergenz gegeben.

\begin{Def}[Divergenz]
  Ein \emph{Divergenz-Zustand} ist ein Zustand in einem \EIO{}, der eine
  unendliche Folge an $\tau$s ausführen kann.%\\
  % Somit ist die Menge der Divergenz-Zustände in einem \EIO{} wie folgt formal
  % definiert: $Div := \{q\in Q\mid \forall i\in \mathbb{N}\; \exists q_i\in Q: q
  % \overset{\tau}{\Rightarrow} q_i~\mathrm{und}~\forall j\in N\backslash \{i\}:
  % q_i\neq q_j\}$.
\end{Def}

Als Erreichbarkeitsbegriff wird wieder die lokale Erreichbarkeit verwendet.
Da das Divergieren eines Systems nicht mehr verhindert werden kann, sobald ein
divergenter Zustand lokal erreichbar ist, ist Divergenz als ähnlich
\glqq{}schlimm\grqq{} zu bewerten wie ein Error.

\begin{Def}[error-, ruhe- und divergenz-freie Kommunikation]
  Zwei \EIO{}s $S_1$ und $S_2$ kommunizieren \emph{error-, ruhe- und
  divergenz-frei}, wenn in ihre Parallelkomposition $S_{12}$ keine Errors,
  Ruhe-Zustände und Divergenz-Zustände lokal erreichbar sind.
\end{Def}

\begin{Def}[Divergenz-Verfeinerungs-Basisrelation]
\label{DefDivBasisrel}
Für \EIO{}s $S_1$ und $S_2$ mit der gleichen Signatur wird $S_1\DBRel S_2$
geschrieben, wenn ein Error, Ruhe-Zustand oder Divergenz-Zustand in $S_1$ nur
dann lokal erreichbar ist, wenn er auch in $S_2$ lokal erreichbar ist. Diese
\emph{Basisrelation} stellt eine \emph{Verfeinerung} bezüglich \emph{Error},
\emph{Ruhe} und \emph{Divergenz} dar.\\
\DCRel{} bezeichnet die \emph{vollständige abstrakte Präkongruenz} von \DBRel{}
bezüglich $\cdot\|\cdot$.
\end{Def}

Da nun die grundlegenden Definitionen für Divergenz festgehalten sind,
kann man sich nun einen Begriff für die Traces von divergenten Zuständen
bilden.\\

\begin{Def}[Divergenztraces]
  Sei $S$ ein \EIO{} und definiere:
  \begin{itemize}
    \item \emph{strikte Divergenztraces}: $\StDT{}(S) := \{w\in\Sigma ^*\mid
      q_0 \overset{w}{\Rightarrow} q\in Div\}$.
  \end{itemize}
\end{Def}

Da in~\cite{Chilton2013} bereits direkt Divergenz mit betrachtet wurde, wird
dort die Flutung der Traces so vorgenommen, dass $\ET\subseteq \DT\subseteq
\QT$ gilt. Dies soll auch hier in diesem Kapitel erreicht werden. Somit kann
zwar die Semantik aus dem Error-Kapitel übernehmen werden, jedoch wird für die
Ruhe eine andere Semantik benötigt, die sich von der im letzten Kapitel
unterscheidet. Die Inklusionskette der Fehlertraces scheint auch von der
Hierarchie her auf die Bewertung zu passen, als wie kritisch die einzelnen
Fehler zu bewerten sind.

\begin{Def}[Ruhe- und Divergenz-Semantik]
\label{DefRuheDivSemantik}
  Sei $S$ ein \EIO{}.
  \begin{itemize}
    \item Die Menge der \emph{error-gefluteten Divergenztraces} von $S$ ist
      $\DT{}(S) := \StDT{}(S)\cup \ET{}(S)$.
    \item Die Menge der \emph{divergenz-gefluteten Ruhetraces} von $S$ ist
      $\QT{}(S) := \StQT{}(S)\cup \DT{}(S)$.
  \end{itemize}
  Für zwei \EIO{}s $S_1, S_2$ mit der gleichen Signatur schreibt man $S_1\DRel
  S_2$, wenn $S_1\ERel S_2$, $\DT{}_1\subseteq \DT{}_2$ und
  $\QT{}_1\subseteq \QT{}_2$ gilt.
\end{Def}

In der letzten Definition wurde wieder durch das Fluten eine
Informationsvermischung vorgenommen. Im Fall von \DT{} mit den Errortraces und
im Fall von \QT{} mit den Divergenztraces. Jedoch entstehen hier wie im letzten
Kapitel keine neuen Traces, die nicht bereits in der error-gefluteten Sprache
\EL{} aus den Error-Kapitel enthalten wären. Somit kann diese Sprache ohne
weitere Flutung verwendet werden. Es folgt, dass die Relation \DRel{} ebenso
wie \QRel{} eine Einschränkung der Relation \ERel{} ist.\\
Ebenso wie in Satz~\ref{satzQuiSemantik} wird im nächsten Satz nur der
Vollständigkeit halber der erste und letzte Punkt erwähnt der Beweis dazu
findet sich in Satz~\ref{satzErrorSemanik}.

\begin{satz}[Error-, Ruhe- und Divergenz-Semantik für Parallelkompostionen]
\label{satzDivSemantik}
  Für zwei komponierbare \EIO{}s $S_1, S_2$ und ihre Komposition
  $S_{12}$ gilt:
  \begin{enumerate}
    \item $\ET{}_{12}=\cont (\prune ((\ET{}_1\|\EL{}_2)\cup
      (\EL{}_1\|\ET{}_2)))$,
    \item $\DT{}_{12}= (\DT{}_1\|\EL{}_2)\cup (\EL{}_1\|\DT{}_2)\cup
      \ET{}_{12}$,
    \item $\QT{}_{12}=(\QT{}_1\|\QT{}_2)\cup \DT{}_{12}$,
    \item $\EL{}_{12}=(\EL{}_1\|\EL{}_2)\cup \ET{}_{12}$.
  \end{enumerate}
\end{satz}

\begin{proof} Es wird hier nur der 2.\ und 3.\ Punkt bewiesen.

  2. \glqq{}$\subseteq$\grqq{}:\\
  Es muss hier unterschieden, ob $w\in \StDT{}_{12}\backslash \ET{}_{12}$ oder
  $w\in \ET{}_{12}$ betrachtet wird. Im zweiten Fall ist das $w$ in der rechten
  Seite der Gleichung enthalten. Deshalb wird im weiteren Verlauf dieses
  Beweises davon ausgegangen, dass $w\in \StDT{}_{12}\backslash \ET{}_{12}$
  gilt, und es wird versucht zu zeigen, dass dieses $w$ ebenfalls in der
  rechten Seite enthalten ist. Aus der Definition~\ref{DefRuheDivSemantik} weiß
  man, dass $(q_{01},q_{02}) \overset{w}{\Rightarrow} (q_1,q_2)$ mit
  $(q_1,q_2)\in Div_{12}$ gilt. Durch die Projektion auf die Transitionssysteme
  $S_1$ und $S_2$ erhält man $q_{01} \overset{w_1}{\Rightarrow} q_1$ und
  $q_{02} \overset{w_2}{\Rightarrow} q_2$ mit $w\in w_1\|w_2$. Aus
  $(q_1,q_2)\in Div_{12}$ folgt, dass \oBdA{} $q_1\in Div_1$ gilt, d.h.\
  $w_1\in \StDT{}_1$. Da $q_{02} \overset{w_2}{\Rightarrow}$ gilt, erhält man
  $w_2\in L_2\subseteq \EL{}_2$ gelten. Somit gilt insgesamt $w\in
  \DT{}_1\|\EL{}_2$ und $w$ ist in der rechten Seite der Gleichung enthalten.

  2. \glqq{}$\supseteq$\grqq{}:\\
  Falls $w\in\ET{}_{12}$ gilt, ist dieses $w$ auch in der linken Seite der
  Gleichung enthalten. Somit wird für den Rest des Beweises dieser Inklusion
  davon ausgegangen, dass $w\in(\DT{}_1\|\EL{}_2)\cup (\EL{}_1\|\DT{}_2)$ gilt.
  Es wird nun noch die Einschränkung vorausgesetzt, dass \oBdA{} $w\in
  \DT{}_1\|\EL{}_2$ gilt, d.h.\ es existieren $w_1\in\DT{}_1$ und
  $w_2\in\EL{}_2$ mit $w\in w_1\|w_2$.
  \begin{itemize}
    \item Fall 1 ($w_1\in \ET{}_1 \vee w_2\in \ET{}_2$): Dieser Fall läuft
      analog zum Fall 1 der selben Inklusionsrichtung vom Beweis zu
      Satz~\ref{satzQuiSemantik}. Es muss dazu nur \StQT{} durch \StDT{}
      ersetzt werden.
    \item Fall 2 ($w_1\in \StDT{}_1\backslash \ET{}_1 \wedge w_2\in
      \EL{}_2\backslash \ET{}_2$): Es gilt in diesem Fall also $q_{01}
      \overset{w_1}{\Rightarrow} q_1\in Div_1$ und $q_{02}
      \overset{w_2}{\Rightarrow} q_2$. Da $q_1$ ein unendliche Folge an $\tau$s
      ausführen kann, ist dies auch für den zusammengesetzten Zustand von $q_1$
      und $q_2$ in der Parallelkomposition möglich. Es gilt also für die
      Komposition $(q_{01},q_{02}) \overset{w}{\Rightarrow} (q_1,q_2)\in
      Div_{12}$ und somit ist $w$ in der linken Seite der Gleichung enthalten,
      da $w\in \StDT{}_{12}\subseteq \DT{}_{12}$ gilt.
  \end{itemize}

  3. \glqq{}$\subseteq$\grqq{}:\\
  Diese Inklusionsrichtung kann analog zum Beweis der selben Inklusionsrichtung
  von Satz~\ref{satzQuiSemantik} gezeigt werden. Es muss dabei nur in der
  Argumentation die Menge $\ET{}_{12}$ durch die Menge $\DT{}_{12}$ ersetzt
  werden. Dadurch kann ebenso gefolgert werden, dass der erreichte Zustand
  $(q_1,q_2)$ kein Error-Zustand sein kann, da $\ET{}_{12}\subseteq \DT{}_{12}$
  gilt.

  3. \glqq{}$\supseteq$\grqq{}:\\
  Es muss wieder danach unterschieden werden, aus welcher Menge das betrachtete
  Element stammt. Falls $w$ ein Element von $\ET{}_{12}$ ist, so folgt die
  Zugehörigkeit zur linken Seite der Gleichung direkt. Somit wird für den
  weiteren Punkt dieses Beweises davon ausgegangen, dass $w\in \QT{}_1\|\QT_2$
  gilt. Für dieses $w$ soll dann gezeigt werden, dass es auch in $\QT_{12}$
  enthalten ist. Da $\QT_i=\StQT{}_i\cup \DT{}_i$ gilt, existieren für $w_1$
  und $w_2$ mit $w\in w_1\|w_2$ unterschiedliche Möglichkeiten:
  \begin{itemize}
    \item Fall 1 ($w_1\in \DT{}_1 \vee w_2\in \DT{}_2$): \OBdA{} gilt $w_1\in
      \DT{}_1$. Es kann nun $w_2\in \StQT{}_2 \subseteq L_2$ gelten oder
      $w_2\in \DT{}_2$ und somit gilt auf jeden Fall $w_2\in \EL{}_2$. Daraus
      kann mit dem zweiten Punkt dieses Satzes gefolgert werden, dass $w\in
      \DT{}_{12}$ gilt und somit $w$ in der linken Seite der Gleichung
      enthalten ist.
    \item Fall 2 ($w_1\in \StQT{}_1\backslash \DT{}_1 \wedge w_2\in \StQT{}_2
      \backslash \DT{}_2$): Dieser Fall läuft analog zu Fall 2 der selben
      Inklusionsrichtung des Beweises von Satz~\ref{satzQuiSemantik}.
  \end{itemize}
\end{proof}

Analog wie in den beiden vorgegangenen Kapiteln, ergibt sich aus diesem Satz
als direkte Folgerung, dass es sich bei der Relation \DRel{} um eine
Präkongruenz handelt.

\begin{prop}[Divergenz-Präkongurenz]
  \DRel{} ist eine Präkongruenz bezüglich $\cdot\|\cdot$.
\end{prop}

\begin{proof}
  Um zu zeigen, dass es sich bei \DRel{} um eine Präkongruenz handelt, muss
  nachgewiesen werden, dass $S_{31}\DRel{} S_{32}$ für jedes komponierbare
  System $S_3$ gilt, wenn $S_1\DRel{} S_2$ erfüllt ist.  D.  h.\   es     ist
  zu zeigen, dass auch $S_1\ERel{} S_2$, $\DT{}_1\subseteq \DT{}_2$ und
  $\QT{}_1\subseteq \QT{}_2$ sowohl $S_{31}\ERel{} S_{32}$,
  $\DT{}_{31}\subseteq \DT_{32}$ als auch $\QT{}_{31}\subseteq \QT{}_{32}$
  folgt. Dies ergibt sich wie in den Beweisen zu den
  Propositionen~\ref{propPraekongruenz} und~\ref{propQuiPrae} aus der Monotonie
  von $\cdot\|\cdot$ auf Sprachen wie folgt:
  \begin{itemize}
    \item $\begin{aligned}[t]
        S_{31}
        \overset{\mathrm{Proposition}~\ref{propPraekongruenz}}{%
        \overset{\mathrm{und}}{\overset{S_1\ERel{} S_2}{\ERel{}}}} S_{32}
    \end{aligned}$
    \item $\begin{aligned}[t]
        \DT{}_{31} &\overset{\ref{satzDivSemantik}~2.}{=} (\DT{}_3\|\EL{}_1)
        \cup (\EL{}_3\|\DT{}_1) \cup \ET{}_{31}\\
        &\hspace{-0.5cm}\overset{\ET{}_{31}\subseteq
      \ET{}_{32},}{\overset{\EL{}_1\subseteq
      \EL{}_2}{\overset{\mathrm{und}}{\overset{\DT{}_1\subseteq
    \DT{}_2}{\subseteq}}}} (\DT{}_3\|\EL{}_2) \cup (\EL{}_3\|\DT{}_2) \cup
    \ET{}_{32}\\
      &\overset{\ref{satzDivSemantik}~2.}{=} \DT{}_{32},
    \end{aligned}$
    \item $\begin{aligned}[t]
        \QT{}_{31} &\overset{\ref{satzDivSemantik}~3.}{=} (\QT{}_3\|\QT{}_1)
        \cup \DT{}_{31}\\
        &\hspace{-0.5cm}\overset{\DT{}_{31}\subseteq
      \DT{}_{32},}{\overset{\mathrm{und}}{\overset{\QT{}_1\subseteq
      \QT{}_2}{\subseteq}}} (\QT{}_3\|\QT{}_2) \cup \DT{}_{32}\\
      &\overset{\ref{satzDivSemantik}~3.}{=} \QT{}_{32}.
    \end{aligned}$
  \end{itemize}
\end{proof}

Als nächstes soll nun eine Verfeinerungsrelation bezüglich guter Kommunikation
mit Partnern im Sinne von error-, ruhe- und divergenz-freier Kommunikation
betrachtet werden. Es muss in diesem Lemma eine Veränderung zu den analogen
Lemmata aus den vorangegangen Kapiteln vorgenommen. Die Einschränkung, dass $U$
ein Partner sein muss, kann nicht mehr beibehalten werden, da die Vermeidung
von Ruhe im Beweis aus dem letzten Kapitel hier zu Divergenz führen würde.
Somit werden dafür Aktionen außerhalb der Menge $Synch$ benötigt.

\begin{lem}[Vereinerung mit Divergenz-Zuständen]
\label{lemDivVerfeinerung}
  Gegeben sind zwei \EIO{}s $S_1$ und $S_2$ mit der gleichen Signatur. Wenn
  $U\|S_1\DBRel{} U\|S_2$ für komponierbaren $U$ gilt, dann folgt daraus
  $S_1\DRel{} S_2$.
\end{lem}

\begin{proof}
  %TODO
\end{proof}

Der folgenden Satz hält fest, dass \DRel{} die gröbste Präkongruenz bezüglich
$\cdot\|\cdot$ charakterisiert, die in \DBRel{} enthalten ist.

\begin{satz}[Full Abstractness für Divergenz-Semantik]
\label{satzDivFullAbst}
  Seinen $S_1$ und $S_2$ zwei \EIO{}s mit derselben Signatur. Dann gilt $S_1
  \DCRel{} S_2\Leftrightarrow S_1\DRel{} S_2$, insbesondere ist \DRel{} eine
  Präkongruenz.
\end{satz}

\begin{proof}
  %TODO
\end{proof}

Es wurde somit, wie in den letzten beiden Kapiteln, eine Kette an Folgerungen
gezeigt, die sich zu einem Ring schließen. Jedoch wird dafür ein Schritt
weniger benötigt, da in Lemma~\ref{lemDivVerfeinerung} bereits direkt $U$ nur
als komponierbar und nicht als Partner vorausgesetzt wurde. Diese
Folgerungskette ist in Abbildung~\ref{FolgerungsketteDiv} dargestellt.

\begin{figure}[h!tbp]
  \begin{center}
    \begin{tikzpicture}
      \matrix (m) [matrix of math nodes,row sep=2cm,column sep=4cm]{%
        S_1\DRel S_2 & S_1\DCRel S_2 \\
        & \substack{\forall~\mathrm{komponierbaren}~U:\\U\|S_1\DBRel U\|S_2}
      \\};
        \draw[-implies, double, double distance=1mm]
          (m-1-1) -- node [above] {\glqq{}$\Leftarrow$\grqq{} von
            Satz~\ref{satzDivFullAbst}} (m-1-2);
        \draw[-implies, double, double distance=1mm]
          (m-1-2) -- node [right] {Definition von \DCRel{}
          in~\ref{DefDivBasisrel}} (m-2-2);
        \draw[-implies, double, double distance=1mm]
          (m-2-2) -- node [below left]
          {Lemma~\ref{lemDivVerfeinerung}} (m-1-1);
    \end{tikzpicture}
    \caption{Folgerungskette}
\label{FolgerungsketteDiv}
  \end{center}
\end{figure}

Aus Satz~\ref{satzDivFullAbst} und Lemma~\ref{lemDivVerfeinerung} erhält man
das folgende Korollar. Angenommen man definiert, dass $S_1$ $S_2$ verfeinern
soll, genau dann wenn für alle Partner \EIO{}s $U$, für die $S_2$ error-, ruhe-
und divergenz-frei mit $U$ kommuniziert, folgt, dass $S_1$ ebenfalls error-,
ruhe- und divergenz-frei mit $U$ kommuniziert. Dann wird auch diese
Verfeinerung durch \DRel{} charakterisiert.

\begin{kor}
  Es gilt: $S_1\DRel{} S_2 \Leftrightarrow U\|S_1\DBRel{} U\|S_2$ für alle
  komponierbaren $U$.
\end{kor}


\end{document}
