\section{Verfeinerungen über Fehler-Freiheit}
\begin{frame}
  \frametitle{Verfeinerung}
  \begin{Def}[Basisrelation]
  Für \EIO{}s $S_1$ und $S_2$ mit der gleichen Signatur wird \dots
  \begin{columns}[]
    \begin{column}{.05\textwidth}
    \end{column}
    \begin{column}{.40\textwidth}
      \begin{block}{}
        \dots{} $S_1\QBRel S_2$ geschrieben, wenn ein \textbf{Error}- oder
        \textbf{Ruhe}-Zustand in $S_1$ nur dann lokal erreichbar ist, wenn er
        auch in $S_2$ lokal erreichbar ist.
      \end{block}
    \end{column}
    \begin{column}{.05\textwidth}
    \end{column}
    \begin{column}{.40\textwidth}
      \uncover<2->{%
      \begin{block}{}
        \dots{} $S_1\DBRel S_2$
        geschrieben, wenn ein \textbf{Error}-, \textbf{Ruhe}- oder
        \textbf{Divergenz}-Zustand in $S_1$ nur
        dann lokal erreichbar ist, wenn er auch in $S_2$ lokal erreichbar ist.
      \end{block}
    }
    \end{column}
    \begin{column}{.05\textwidth}
    \end{column}
  \end{columns}
    \vspace{1em}
    \uncover<3>{%
    \QCRel{} (bzw. \DCRel{}) bezeichnet die \textbf{vollständige abstrakte
    Präkongruenz} von \QBRel{} (bzw. \DBRel{}) bezüglich $\cdot\|\cdot$.
  }
  \end{Def}
\end{frame}

\begin{frame}
  \begin{Def}[Traces]
    Für ein \EIO{} $S$ wird definiert:
    \begin{itemize}
      \item \textbf{strikte Errortraces}: $\StET{}(S):=\left\{w\in\Sigma
        ^*\mid q_0\overset{w}{\Rightarrow}q\in E\right\}$,
      \item \textbf{gekürzte Errortraces}: $\PrET{}(S):=\bigcup\left\{\prune{}(w)\mid w\in
        \StET{}(S)\right\}$,
      \item \textbf{error-gefluteten Ruhetraces}: $\QT{}(S) := \StQT{}(S)\cup
        \ET{}(S)$,
      \item \textbf{Input-kritische Traces}: $\MIT{}(S):=\left\{wa\in\Sigma ^*\mid
        q_0\overset{w}{\Rightarrow}q\wedge a\in I\wedge
      q\overset{a}{\not{\hspace{-0.1cm}\rightarrow}}\right\}$,
      \item<2-> \textbf{strikte Ruhetraces}: $\StQT{}(S) := \left\{w\in\Sigma ^*\mid q_0
        \overset{w}{\Rightarrow} q\in Qui\right\}$,
      \item<3-> \textbf{strikte Divergenztraces}: $\StDT{}(S) := \left\{w\in\Sigma
          ^*\mid
        q_0 \overset{w}{\Rightarrow} q\in Div\right\}$,
      \item<3-> \textbf{gekürzte Divergenztraces}: $\PrDT{}(S) :=
        \bigcup\hspace{1pt}\left\{\prune{}(w)\mid w\in\StDT{}(S)\right\}$.
    \end{itemize}
  \end{Def}
\end{frame}

\begin{frame}
  \begin{Def}[Semantik]
    \small
    Für ein \EIO{} $S$ wird definiert:
  \begin{columns}[]
    \begin{column}{.05\textwidth}
    \end{column}
    \begin{column}{.42\textwidth}
      \begin{block}{}
      \begin{itemize}
        \item \textbf{Errortraces}: $\ET{}(S):=\cont{}(\PrET{}(S))\cup
          \cont{}(\MIT{}(S))$,
        \item \textbf{error-geflutete Ruhetraces}: $\QT{}(S) := \StQT{}(S)\cup
          \ET{}(S)$,
        \item \textbf{error-geflutete Sprache}: $\EL{}(S):=L(S)\cup \ET{}(S)$.
      \end{itemize}
      \end{block}
    \end{column}
    \begin{column}{.05\textwidth}
    \end{column}
    \begin{column}{.42\textwidth}
      \uncover<2->{%
      \begin{block}{}
      \begin{itemize}
        \item \textbf{Error-Divergenztraces}: $\EDT{}(S) :=
          \ET{}(S)\cup \cont{}(\PrDT{}(S))$,
        \item \textbf{error-divergenz-gefluteten Ruhetraces}: $\QDT{}(S) :=
          \StQT{}(S)\cup \EDT{}(S)$,
        \item \textbf{error-divergenz-gefluteten Sprache}:
          $\EDL{}(S) := L(S) \cup \EDT{}(S)$.
      \end{itemize}
      \end{block}
    }
    \end{column}
    \begin{column}{.05\textwidth}
    \end{column}
  \end{columns}
  \end{Def}
\end{frame}

\begin{frame}
  \begin{Def}[Semantik]
  Für zwei \EIO{}s $S_1, S_2$ mit der gleichen Signatur schreibt man \dots
  \begin{columns}[]
    \begin{column}{.05\textwidth}
    \end{column}
    \begin{column}{.40\textwidth}
      \begin{block}{}
        \dots{} $S_1\QRel S_2$, wenn:
        \begin{itemize}
          \item $\ET{}_1\subseteq \ET{}_2$,
          \item $\QT{}_1\subseteq \QT{}_2$ und
          \item $\EL{}_1\subseteq \EL{}_2$ gilt.
        \end{itemize}
      \end{block}
    \end{column}
    \begin{column}{.05\textwidth}
    \end{column}
    \begin{column}{.40\textwidth}
      \uncover<2->{%
      \begin{block}{}
        \dots{} $S_1\DRel S_2$, wenn:
        \begin{itemize}
          \item $\EDT{}_1\subseteq \EDT{}_2$,
          \item $\QDT{}_1\subseteq \QDT{}_2$ und
          \item $\EDL{}_1\subseteq \EDL{}_2$ gilt.
        \end{itemize}
      \end{block}
    }
    \end{column}
    \begin{column}{.05\textwidth}
    \end{column}
  \end{columns}
  \end{Def}
\end{frame}

\begin{frame}
  \small
  \begin{satz}[Semantik für Parallelkomposition]
    Für zwei komponierbare \EIO{}s $S_1, S_2$ und ihre Komposition
    $S_{12}$ gilt:
  \begin{columns}[]
    \begin{column}{.05\textwidth}
    \end{column}
    \begin{column}{.42\textwidth}
      \begin{block}{}
        \begin{enumerate}
          \item $\ET{}_{12}=\cont \left(\prune
              \left(\left(\ET{}_1\|\EL{}_2\right)\cup\right.\right.$
            $\left.\left.\left(\EL{}_1\|\ET{}_2\right)\right)\right)$,
          \item $\QT{}_{12}=(\QT{}_1\|\QT{}_2)\cup \ET{}_{12}$,
          \item $\EL{}_{12}=(\EL{}_1\|\EL{}_2)\cup \ET{}_{12}$.
        \end{enumerate}
      \end{block}
    \end{column}
    \begin{column}{.05\textwidth}
    \end{column}
    \begin{column}{.42\textwidth}
      \uncover<2->{%
      \begin{block}{}
        \begin{enumerate}
          \item $\EDT{}_{12}=\cont \left(\prune
              \left(\left(\EDT{}_1\|\EDL{}_2\right)\right.\right.$
                $\left.\left.\cup
            \left(\EDL{}_1\|\EDT{}_2\right)\right)\right)$,
          \item $\QDT{}_{12}=(\QDT{}_1\|\QDT{}_2)\cup \EDT{}_{12}$,
          \item $\EDL{}_{12}=(\EDL{}_1\|\EDL{}_2)\cup \EDT{}_{12}$.
        \end{enumerate}
      \end{block}
    }
    \end{column}
    \begin{column}{.05\textwidth}
    \end{column}
  \end{columns}
  \end{satz}
  \uncover<3>{%
  \begin{prop}[Präkongrunez]
    Die Relation \QRel{} (bzw. \DRel{}) ist eine Präkongruenz bezüglich $\cdot\|\cdot$.
  \end{prop}}
\end{frame}

\begin{frame}
  \begin{lem}[Verfeinerung]
    Gegeben sind zwei \EIO{}s $S_1$ und $S_2$ mit der gleichen Signatur.
  \begin{columns}[]
    \begin{column}{.05\textwidth}
    \end{column}
    \begin{column}{.42\textwidth}
      \begin{block}{}
        Wenn $U\|S_1\QBRel{} U\|S_2$ für alle Partner $U$ gilt, dann
        folgt daraus $S_1\QRel{} S_2$.
      \end{block}
    \end{column}
    \begin{column}{.05\textwidth}
    \end{column}
    \begin{column}{.42\textwidth}
      \uncover<2->{%
      \begin{block}{}
        Wenn $U\|S_1\DBRel{} U\|S_2$ für alle $\omega$-Partner $U$ gilt, dann
        folgt daraus $S_1\DRel{} S_2$.
      \end{block}
    }
    \end{column}
    \begin{column}{.05\textwidth}
    \end{column}
  \end{columns}
  \end{lem}
  \uncover<3->{%
  \begin{satz}[Vollstänige Abstraktheit]
    Seinen $S_1$ und $S_2$ zwei \EIO{}s mit derselben Signatur. Dann gilt:
  \begin{columns}[]
    \begin{column}{.05\textwidth}
    \end{column}
    \begin{column}{.42\textwidth}
      \uncover<3->{%
      \begin{block}{}
        $S_1 \QCRel{} S_2\Leftrightarrow S_1\QRel{} S_2$.
      \end{block}
    }
    \end{column}
    \begin{column}{.05\textwidth}
    \end{column}
    \begin{column}{.42\textwidth}
      \uncover<4->{%
      \begin{block}{}
        $S_1 \DCRel{} S_2\Leftrightarrow S_1\DRel{} S_2$.
      \end{block}
    }
    \end{column}
    \begin{column}{.05\textwidth}
    \end{column}
  \end{columns}
  \end{satz}
}
\end{frame}

% \begin{frame}
%   $x?$ bezeichnet den Input $x$ und $x!$ den Output $x$
%   \begin{figure} [h!tbp]
%   \begin{center}
%     \begin{tikzpicture}[->, >=latex',auto,node distance =3cm, semithick]
%       \node (0) {$q_0$};
%       \node (1) [right of=0] {$q_1$};
%       \node (dots) [right of=1] {$\dots$};
%       \node (n) [right of=dots] {$q_n$};
%       \node (n1) at ($(1)!0.5!(dots) + (0,-3)$) {$q_{n+1}$};

%       \path ($ (0) + (-1,0) $) edge (0)
%             (0) edge node {$x_1$} (1)
%                 edge [bend right] node [below, sloped] {$x?\neq x_1, \omega
%                 !$} (n1)
%             (1) edge node {$x_2$} (dots)
%                 edge node [below, sloped] {$x?\neq x_2, \omega !$} (n1)
%             (dots) edge node {$x_n$} (n)
%                    edge [dashed] (n1)
%             (n) edge node [above, sloped] {$x?\in I_U, \omega !$} (n1)
%                 edge [bend left] node [sloped] {$x_{n+1}$!} (n1)
%             (n1) edge [loop below] node {$x?\in I_U, \omega !$} (n1);
%     \end{tikzpicture}
%     \caption{$x?\neq x_i$ steht für alle $x\in I_U\backslash\{x_i\}$}
%   \end{center}
%   \end{figure}
% \end{frame}

% \begin{frame}
%   \begin{figure} [h!tbp]
%   \begin{center}
%     \begin{tikzpicture}[->, >=latex',auto,node distance =3cm, semithick]
%       \node (0) {$q_0$};
%       \node (1) [right of=0] {$q_1$};
%       \node (dots) [right of=1] {$\dots$};
%       \node (n) [right of=dots, rectangle, dotted, draw] {$q_n\in Qui_U$};
%       \node (q) at ($(1)!0.5!(dots) + (0,-3)$) {$q$};

%       \path ($ (0) + (-1,0) $) edge (0)
%             (0) edge node {$x_1$} (1)
%                 edge [bend right] node [below, sloped] {$x?\neq x_1, \omega
%                 !$} (q)
%             (1) edge node {$x_2$} (dots)
%                 edge [below, sloped] node {$x?\neq x_2, \omega !$} (q)
%             (dots) edge node {$x_n$} (n)
%                    edge [dashed] (q)
%             (n) edge [bend left] node [below,sloped] {$x?\in I_U$} (q)
%             (q) edge [loop below] node {$x?\in I_U, \omega !$} (q);
%     \end{tikzpicture}
%     \caption{$x?\neq x_i$ steht für alle $x\in I_U\backslash\{x_i\}$, $q_n$
%       ist der einzige Ruhe-Zustand}
%   \end{center}
%   \end{figure}
% \end{frame}

% \begin{frame}
%   \begin{figure} [h!tbp]
%   \begin{center}
%     \begin{tikzpicture}[->, >=latex',auto,node distance =2.5cm, semithick]

%       \node (0) {$q_0$};
%       \node (1) [right of=0] {$q_1$};
%       \node (dots) [right of=1] {$\dots$};
%       \node (n1) [right of=dots] {$q_{n-1}$};
%       \node (n) [right of=n1, rectangle, draw] {$q_n\in E_U$};
%       \node (q) at ($(dots) + (0,-3)$) {$q$};

%       \path ($ (0) + (-1,0) $) edge (0)
%             (0) edge node {$x_1$} (1)
%                 edge [bend right] node [below, sloped] {$x?\neq x_1, \omega
%                 !$} (q)
%             (1) edge node {$x_2$} (dots)
%                 edge node [below, sloped] {$x?\neq x_2, \omega !$} (q)
%             (dots) edge node {$x_{n-1}$} (n1)
%                    edge [dashed] (q)
%             (n1) edge node {$x_n$} (n)
%                  edge node [below, sloped] {$x?\neq x_n, \omega !$} (q)
%             (q) edge [loop below] node {$x?\in I_U, \omega !$} (q)
%             (n) edge [bend left] node {$\omega !$} (q);
%     \end{tikzpicture}
%     \caption{$x?\neq x_i$ steht für alle $x\in I_U\backslash\{x_i\}$, $q_n$
%       ist der einzige Error-Zustand}
%   \end{center}
%   \end{figure}
% \end{frame}

\begin{frame}
  \begin{figure} [h!tbp]
  \begin{center}
    \begin{tikzpicture}[auto]
      \node (m-1-1) {$S_1\QRel S_2$};
      \node[right = 5cm of m-1-1] (m-1-2) {$S_1\QCRel S_2$}
      edge[implies-, double, double distance=1mm] node [above]
      {\rmfamily\footnotesize\parbox{4cm}{\glqq{}$\Leftarrow$\grqq{} von Satz
      Vollständige Abstraktheit für Ruhe-Semantik}} (m-1-1);
      \node[below = 2cm of m-1-1] (m-2-1) {$\substack{\forall\;
      \mathrm{Partner}~U:\\ U\|S_1 \QBRel U\|S_2}$}
          edge[-implies, double, double distance=1mm] node [left]
          {\rmfamily\footnotesize\parbox{2.3cm}{Lemma Verfeinerung mit
          Ruhe-Zu\-stän\-den}} (m-1-1);
      \node[below = 2cm of m-1-2] (m-2-2) {$\substack{\forall\;
      \mathrm{komponierbaren}~U:\\ U\|S_1 \QBRel U\|S_2}$}
          edge[-implies, double, double distance=1mm] node [below]
          {$\substack{U~\mathrm{Partner}\\ \Downarrow\\
          U~\mathrm{komponierbar}}$} (m-2-1)
          edge[implies-, double, double distance=1mm] node [right]
          {\rmfamily\small\parbox{2cm}{Definition
          von \QCRel{}}} (m-1-2);
    \end{tikzpicture}
    \caption{Folgerungskette für Ruhe}
  \end{center}
  \end{figure}
\end{frame}

\begin{frame}
  \begin{figure} [h!tbp]
  \begin{center}
    \begin{tikzpicture}[auto]
      \node (m-1-1) {$S_1\DRel S_2$};
      \node[right = 5cm of m-1-1] (m-1-2) {$S_1\DCRel S_2$}
      edge[implies-, double, double distance=1mm] node [above]
      {\rmfamily\footnotesize\parbox{4cm}{\glqq{}$\Leftarrow$\grqq{} von Satz
      Vollständige Abstraktheit für Divergenz-Semantik}} (m-1-1);
      \node[below = 2cm of m-1-1] (m-2-1) {$\substack{\forall\;\omega
          -\mathrm{Partner}~U:\\ U\|S_1 \DBRel U\|S_2}$}
          edge[-implies, double, double distance=1mm] node [left]
          {\rmfamily\footnotesize\parbox{2.3cm}{Lemma Verfeinerung mit
          Div\-er\-genz-Zu\-stän\-den}} (m-1-1);
      \node[below = 2cm of m-1-2] (m-2-2) {$\substack{\forall\;
      \mathrm{komponierbaren}~U:\\ U\|S_1 \DBRel U\|S_2}$}
          edge[-implies, double, double distance=1mm] node [below]
          {$\substack{U~\omega\mathrm{-Partner}\\ \Downarrow\\
          U~\mathrm{komponierbar}}$} (m-2-1)
          edge[implies-, double, double distance=1mm] node [right]
          {\rmfamily\small\parbox{2cm}{Definition
          von \DCRel{}}} (m-1-2);
    \end{tikzpicture}
    \caption{Folgerungskette für Divergenz}
  \end{center}
  \end{figure}
\end{frame}

\begin{frame}
  \begin{kor}
    Es gilt:
  \begin{columns}[]
    \begin{column}{.05\textwidth}
    \end{column}
    \begin{column}{.40\textwidth}
      \begin{block}{}
        \begin{center}
          $S_1\QRel{} S_2$

          \vspace{0.5em}
          $\Updownarrow$
          \vspace{0.5em}

          $U\|S_1\QBRel{} U\|S_2$
        \end{center}
      \end{block}
    \end{column}
    \begin{column}{.05\textwidth}
    \end{column}
    \begin{column}{.40\textwidth}
      \uncover<2->{%
      \begin{block}{}
        \begin{center}
          $S_1\DRel{} S_2$

          \vspace{0.5em}
          $\Updownarrow$
          \vspace{0.5em}

          $U\|S_1\DBRel{} U\|S_2$
        \end{center}
      \end{block}
    }
    \end{column}
    \begin{column}{.05\textwidth}
    \end{column}
  \end{columns}
  \vspace{1em}
  für alle komponierbaren $U$.
  \end{kor}
\end{frame}
