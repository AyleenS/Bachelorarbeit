\chapter{Verfeinerung für Error-, Ruhe- und Divergenz-Freiheit}

\section{Präkongruenz für Divergenz}

In diesem Kapitel soll die Menge der betrachteten Zustände noch einmal
erweitert werden. Somit werden dann Errors, Ruhe-Zustände und
Divergente-Zustände betrachtet. Somit eignet sich~\cite{Chilton2013} hier als
Quelle, da nun auch noch die Divergenz betrachtet wird. Diese wurde dort
gleichzeitig mit der Ruhe eingeführt und betrachtet. Da es sich nur um eine
Erweiterung der Präkongruenzen aus den letzten beiden Kapiteln handelt, wird
dabei ähnlich vorgegangen wie in den letzten beiden Kapiteln.\\
Wie bereits im letzten Kapitel erwähnt wurden in~\cite{Chilton2013} auch noch
divergente Zustände als Fehler-Zustände betrachtet. Um zu klären, was darunter
verstanden wird, wird nun noch eine Definition für Divergenz gegeben.

\begin{Def}[Divergenz]
  Ein \emph{Divergenz-Zustand} ist ein Zustand in einem \EIO{}, der eine
  unendliche Folge an $\tau$s ausführen kann. %TODO formale Def möglicherweise
  %mit unendlicher indizierter Folge, da Q unendlich sein kann
\end{Def}

Als Erreichbarkeitsbegriff wird wieder die lokale Erreichbarkeit verwendet.
Da das divergieren eines Systems nicht mehr verhindert werden kann, sobald ein
divergenter Zustand lokal erreichbar ist, ist Divergenz als ähnlich
\glqq{}schlimm\grqq{} zu bewerten wie ein Error.

\begin{Def}[error-, ruhe- und divergenz-freie Kommunikation]
  Zwei \EIO{}s $S_1$ und $S_2$ kommunizieren \emph{error-,ruhe- und
  divergenz-frei}, wenn in ihre Parallelkomposition $S_{12}$ keine Errors,
  Ruhe-Zustände und Divergenz-Zustände lokal erreichbar sind.
\end{Def}

\begin{Def}[Divergenz-Verfeinerungs-Basisrelation]
Für \EIO{}s $S_1$ und $S_2$ mit der gleichen Signatur wird $S_1\DBRel S_2$
geschrieben, wenn ein Error, Ruhe-Zustand oder Divergenz-Zustand in $S_1$ nur
dann lokal erreichbar ist, wenn er auch in $S_2$ lokal erreichbar ist. Diese
\emph{Basisrelation} stellt eine \emph{Verfeinerung} bezüglich \emph{Errors},
\emph{Ruhe-Zuständen} und \emph{Divergenz-Zuständen} dar.\\
\DCRel{} bezeichnet die \emph{vollständige abstrakte Präkongruenz} von \DBRel{}
bezüglich $\cdot\|\cdot$.
\end{Def}

Da nun die grundlegenden Definitionen für Divergenz festgehalten sind,
kann man sich nun einen Begriff für die Traces von divergenten Zuständen
bilden.\\

\begin{Def}[Divergenztraces]
  Sei $S$ ein \EIO{} und definiere:
  \begin{itemize}
    \item strickte Divergenztraces: $\StDT{}(S) := \{w\in\Sigma ^*\mid q_0
      \overset{w}{\Rightarrow} q\in Div\}$,
    \item gekürzte Divergenztraces: $\PrDT{}(S) := \{\prune (w)\mid
      w\in\StDT{}(S)\}$.
  \end{itemize}
\end{Def}

Da in~\cite{Chilton2013} bereits direkt Divergenz mit betrachtet wurde, wird
dort die Flutung der Traces so vorgenommen, dass $\ET\sqsubseteq \DT\sqsubseteq
\QT$ gilt. Dies soll auch hier in diesem Kapitel erreicht werden. Somit kann
zwar die Semantik aus dem Error-Kapitel übernehmen werden, jedoch wird für die
Ruhe eine andere Semantik benötigt, die sich von der im letzten Kapitel
unterscheidet. Die Inklusionskette der Fehlertraces scheint auch von der
Hierarchie her auf die Bewertung zu passen, wie kritisch die einzelnen Fehler
sind.

\begin{Def}[Ruhe- und Divergenz-Semantik]
  \label{DefRuheDivSemantik}
  Sei $S$ ein \EIO{}.
  \begin{itemize}
    \item Die Menge der \emph{error-gefluteten Divergenztraces} von $S$ ist
      $\DT{}(S) := \cont{}(\PrDT{}(S))\cup \ET{}(S)$.
    \item Die Menge der \emph{divergenz-gefluteten Ruhetraces} von $S$ ist
      $\QT{}(S) := \PrQT{}(S)\cup \DT{}(S)$.
  \end{itemize}
  Für zwei \EIO{}s $S_1, S_2$ mit der gleichen Signatur schreibt man $S_1\DRel
  S_2$, wenn $S_1\ERel S_2$, $\DT{}_1\sqsubseteq \DT{}_2$ und
  $\QT{}_1\sqsubseteq \QT{}_2$ gilt.
\end{Def}

In der letzten Definition wurde wieder durch das Fluten eine
Informationsvermischung vorgenommen. Im Fall von \DT{} mit den Errortraces und
im Fall von \QT{} mit den Divergenztraces. Jedoch entstehen hier wie im letzten
Kapitel keine neuen Traces, die nicht bereits in \EL{} aus den Error-Kapitel
enthalten wären. Somit kann die error-geflutete Sprache ohne weitere
Flutung verwendet werden. Somit ist die Relation \DRel{} ebenso wie \QRel{}
eine Einschränkung der Relation \ERel{}.\\
Ebenso wie in Satz~\ref{satzQuiSemantik} wird im nächsten Satz nur der
Vollständigkeit halber der erste und letzte Punkt erwähnt der Beweis dazu
findet sich in Satz~\ref{satzErrorSemanik}.

\begin{satz}[Error-, Ruhe- und Divergenz-Semantik für Parallelkompostionen]
  Für zwei komponierbare \EIO{}s $S_1, S_2$ und ihre Komposition
  $S_{12}$ gilt:
  \begin{enumerate}
    \item $\ET{}_{12}=\cont (\prune ((\ET{}_1\|\EL{}_2)\cup
      (\EL{}_1\|\ET{}_2)))$,
    \item $\DT{}_{12}=\cont (\prune ((\DT{}_1\|\EL{}_2)\cup
      (\EL{}_1\|\DT{}_2)))\cup \ET{}_{12}$,
    \item $\QT{}_{12}=\prunenew (\QT{}_1\|\QT{}_2)\cup \DT{}_{12}$,
    \item $\EL{}_{12}=(\EL{}_1\|\EL{}_2)\cup \ET{}_{12}$.
  \end{enumerate}
\end{satz}

\begin{proof} Es wird hier nur der 2.\ und 3.\ Punkt bewiesen.

  2. ``$\subseteq$'':\\
  %TODO anpassen an neue Definitionen
  Da hier auf beiden Seiten der Gleichung Mengen stehen, die unter \cont{}
  abgeschlossen sind, reicht es nur präfix-minimale Elemente zu betrachten. Es
  muss hier unterschieden, ob ein präfix-minimales $w\in \cont{}(\PrDT{}_{12})$
  oder ein $w\in \ET{}_{12}$ betrachtet wird. Im zweiten Fall ist das $w$ in
  der rechten Seite der Gleichung enthalten. Deshalb wird im weiteren Verlauf
  dieses Beweises davon ausgegangen, dass $w\in \cont{}(\PrDT{}_{12})$ gilt,
  und es wird versucht zu zeigen, dass dieses $w$ ebenfalls in der rechten
  Seite enthalten ist. Aus der Definition~\ref{DefRuheDivSemantik} und der von
  \prune{} weiß man, dass es ein $v\in O^*_{12}$ gibt, so dass $(q_{01},q_{02})
  \overset{wv}{\Rightarrow} (q_1,q_2)$ mit $(q_1,q_2)\in Div_{12}$ gilt. Durch
  die Projektion auf die Transitionssysteme $S_1$ und $S_2$ erhält man $q_{01}
  \overset{w_1v_1}{\Rightarrow} q_1$ und $q_{02} \overset{w_2v_2}{\Rightarrow}
  q_2$ mit $w\in w_1\|w_2$ und $v\in v_1\|v_2$. Aus $(q_1,q_2)\in Div_{12}$
  folgt, dass \oBdA{} $q_1\in Div_1$ gilt. Damit der Zustand $(q_1,q_2)$ in der
  Parallelkomposition erreicht werden kann, muss dann $w_2v_2\in L_2\subseteq
  \ET{}_2$ gelten.

  2. ``$\supseteq$'':\\% TODO

  3. ``$\subseteq$'':\\% TODO

  3. ``$\supseteq$'':\\% TODO
\end{proof}
