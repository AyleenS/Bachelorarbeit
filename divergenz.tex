\chapter{Verfeinerung auf Error-, Ruhe- und Divergenztraces}

\section{Präkongruenz für Divergenz}

In diesem Kapitel wollen wir die Menge der betrachteten Zustände noch einmal
erweitern. Somit betrachten wir dann Errors, Ruhe-Zustände und
Divergente-Zustände. Somit eignet sich~\cite{Chilton2013} hier als Quelle, da
wir nun auch noch die Divergenz betrachten. Diese wurde dort gleichzeitig mit
der Ruhe eingeführt und betrachtet. Da es sich nur um eine Erweiterung der
Präkongruenzen aus den letzten beiden Kapiteln handelt, werden wir dabei
ähnlich vorgehen wie in den letzten beiden Kapiteln.\\
Wenn von einem Zustand aus eine unendliche Folge an $\tau$-Transitionen möglich
ist, nennen wir diesen Zustand divergent.

\begin{Def}[Divergenz]
  Ein \emph{Divergenz-Zustand} ist ein Zustand in einem \EIO{}, der eine
  unendliche Folge an $\tau$s ausführen kann.\\
  Somit ist die Menge der Divergenz-Zustände in einem \EIO{} wie folgt formal
  definiert: $Div := \{q\in Q\mid \exists q'\in Q': q
  \overset{\tau}{\Rightarrow} q' \wedge q'~\mathrm{ist}~\infty\mathrm{-oft}
  ~\mathrm{durch}~\tau ~\mathrm{erreichbar}\}$.
\end{Def}

Wir verwenden als Erreichbarkeitsbegriff wieder die lokale Erreichbarkeit.
Jedoch werden wir für Divergenz auch nicht die Methoden wählen, dass die Traces
beliebig fortsetzbar sind, sondern nur so wie im System bereits möglich. Auf
die gleiche Art zu Weise wie bei den Ruhe-Zuständen können wir hier auch dem
Fehler entkommen. Durch einen Input oder einen Output kann das Divergieren des
Systems verhindert werden und somit ist der Fehler auch als nicht so
\glqq{}schlimm\grqq{} zu bewerten wie ein Error.

\begin{Def}[error-, ruhe- und divergenz-freie Kommunikation]
  Zwei \EIO{}s $S_1$ und $S_2$ kommunizieren \emph{error-,ruhe- und
  divergenz-frei}, wenn in ihre Parallelkomposition $S_1\|S_2$ keine Errors,
  Ruhe-Zustände und Divergenz-Zustände lokal erreichbar sind.
\end{Def}

\begin{Def}[Divergenz-Verfeinerungs-Basisrelation]
Für \EIO{}s $S_1$ und $S_2$ mit der gleichen Signatur schreiben wir $S_1\DBRel
S_2$, wenn ein Error, Ruhe-Zustand oder Divergenz-Zustand in $S_1$ nur dann
lokal erreichbar ist, wenn er auch in $S_2$ lokal erreichbar ist. Diese
\emph{Basisrelation} stellt eine \emph{Verfeinerung} bezüglich \emph{Errors},
\emph{Ruhe-Zuständen} und \emph{Divergenz-Zuständen} dar.\\
\DCRel{} bezeichnet die \emph{vollständige abstrakte Präkongruenz} von \DBRel{}
bezüglich $\cdot\|\cdot$.
\end{Def}

Da wir nun die grundlegenden Definitionen für Divergenz festgehalten haben,
können wir uns nun einen Begriff für die Traces von divergenten Zuständen
bilden.\\
Wie oben bereits erwähnt, verwenden wir hier die gleiche Technik wie im letzten
Kapitel um die Traces zu kürzen und nur so fortzusetzen wie es das
Transitionssystem ermöglicht.

\begin{Def}[Divergenztraces]
  Sei $S$ ein \EIO{} und definiere:
  \begin{itemize}
    \item strickte Divergenztraces: $\StDT{}(S) := \{w\in\Sigma ^*\mid q_0
      \overset{w}{\Rightarrow} q\in Div\}$,
    \item Divergenztraces: $\PrDT{}(S) := \bigcup\{\prunenew (w)\mid
      w\in\StDT{}(S)\}$.
  \end{itemize}
\end{Def}

Da in~\cite{Chilton2013} bereits direkt Divergenz mit betrachtet wurde, wir dort
die Flutung der Traces so vorgenommen, dass $\ET\sqsubseteq \DT\sqsubseteq\QT$
gilt. Dies möchten wir auch gerne hier in diesem Kapitel erreichen. Somit
können wir zwar die Semantik aus dem Error-Kapitel übernehmen, benötigen jedoch
für die Ruhe eine andere Semantik wie im letzten Kapitel.

\begin{Def}[Ruhe- und Divergenz-Semantik]
  Sei $S$ ein \EIO{}.
  \begin{itemize}
    \item Die Menge der \emph{error-gefluteten Divergenztraces} von $S$ ist
      $\DT{}(S) := \PrDT{}(S)\cup \ET{}(S)$.
    \item Die Menge der \emph{divergenz-gefluteten Ruhetraces} von $S$ ist
      $\QT{}(S) := \PrQT{}(S)\cup \DT{}(S)$.
  \end{itemize}
  Für zwei \EIO{}s $S_1, S_2$ mit der gleichen Signatur schreiben wir $S_1\DRel
  S_2$, wenn $S_1\ERel S_2$, $\DT{}(S_1)\sqsubseteq \DT{}(S_2)$ und
  $\QT{}(S_1)\sqsubseteq \QT{}(S_2)$ gilt.
\end{Def}

\section{Hiding für Divergenz}
