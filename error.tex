\chapter{Verfeinerung für Error-Freiheit}

\section{Präkongruenz für Error}

Da es in dieser Arbeit vor allem um die Erreichbarkeit und die Kommunikation
zwischen \EIO{}s geht, wurden die nächsten beiden Definitionen explizit
getrennt und erweitert im Vergleich zu denen in~\cite{Vogler2014EIO}. Ebenfalls
wurde die Parallelkomposition geändert, wie in~\cite{Schlosser2012BA}.

\begin{Def}[error-freie Kommunikation]
  Ein Error-Zustand ist \emph{lokal erreichbar} in einem \EIO{} $S$, wenn ein
  $w\in O^*$ existiert mit $q_0 \overset{w}{\Rightarrow} q\in E$.\\
  Zwei \EIO{}s $S_1$ und $S_2$ \emph{kommunizieren error-frei}, wenn in ihrer
  Parallelkomposition $S_{12}$ keine Error-Zustände lokal erreicht werden können.
\end{Def}

Mittels der lokalen Erreichbarkeit von Error-Zuständen kann eine
Verfeinerungsrelation definiert werden. Zusätzlich wird bereits die
gröbste Präkongruenz definiert, die charakterisiert werden soll.

\begin{Def}[Error-Verfeinerungs-Basisrelation]
\label{DefErrorBasisrel}
  Für \EIO{}s $S_1$ und $S_2$ mit der gleichen Signatur wird
  $S_1\EBRel S_2$ geschrieben, wenn ein Error-Zustand in $S_1$ nur dann lokal erreichbar ist, wenn er
  auch in $S_2$ lokal erreichbar ist. Diese \emph{Basisrelation} stellt eine
  \emph{Verfeinerung} bezüglich \emph{Error} da.\\
  \ECRel{} bezeichnet die \emph{vollständig abstrakte Präkongruenz} von \EBRel{}
  bezüglich $\cdot\|\cdot$, d.h.\ die gröbste Präkongruenz bezüglich
  $\cdot\|\cdot$, die in \EBRel{} enthalten ist.
\end{Def}

Um sich näher mit den Präkongruenzen auseinandersetzen zu können, müssen bestimmte Traces
aus der Struktur hervorgehoben werden. Die strikten Errortraces entsprechen Wegen, die
direkt vom Startzustand zu einem Zustand in der Menge $E$ führen. Da Outputs
und $\tau$s Aktionen
sind, die von außen nicht verhindert werden können, wird auch noch die
Menge der Traces benötigt, die zu einem Zustand führen können, von dem aus mit lokalen Aktionen
ein Error-Zustand erreicht werden kann. Zusätzlich ist auch noch die Menge der Traces
interessant, für die es einen Input $a\in I$ gibt, durch den sie möglicherweise nicht
fortgesetzt werden können. Diese führen zwar nicht direkt zu einem
Error-Zustand,
jedoch in Komposition mit einem anderen Transitionssystem sind
dies gefährdete Stellen. Sie führen zu einem neuen Error, sobald dieser Input
für die Synchronisation fehlt.

\pagebreak

\begin{Def}[Errortraces]
\label{DefErrortraces}
  Für ein \EIO{} $S$ wird definiert:
  \begin{itemize}
    \item \emph{strikte Errortraces}: $\StET{}(S)=\left\{w\in\Sigma
      ^*\mid q_0\overset{w}{\Rightarrow}q\in E\right\}$,
    \item \emph{gekürzte Errortraces}: $\PrET{}(S)=\left\{\prune{}(w)\mid w\in
      \StET{}(S)\right\}$,
    \item \emph{Input-kritische Traces}: $\MIT{}(S)=\left\{wa\in\Sigma ^*\mid
      q_0\overset{w}{\Rightarrow}q\wedge a\in I\wedge
    q\overset{a}{\not{\hspace{-0.1cm}\rightarrow}}\right\}$.
  \end{itemize}
\end{Def}

In der folgenden Definition wird festgehalten, was als Errortrace aufgefasst
wird. Diese Menge ist dadurch, dass sie die fortgesetzten Traces aus $\PrET{}$
enthält, deutlich allgemeiner als die Menge $\StET{}$. Die fortgesetzten Traces
aus \MIT{} sind aufgrund der Möglichkeit für Kommunikationsfehler ebenfalls in
den Errortraces enthalten. Zusätzlich wird auch noch
die geflutete Sprache definiert, in der die Informationen aus der Sprache und den
Errortraces vereint werden und somit bei der Inklusion nicht mehr explizit
unterschieden werden.

\begin{Def}[Error-Semantik]
\label{DefETEL}
  Sei $S$ ein \EIO{}.
  \begin{itemize}
    \item Die Menge der \emph{Errortraces} von $S$ ist $\ET{}(S):=\cont{}(\PrET{}(S))\cup
      \cont{}(\MIT{}(S))$.
    \item Die \emph{error-geflutete Sprache} von $S$ ist $\EL{}(S):=L(S)\cup \ET{}(S)$.
  \end{itemize}
  Für zwei \EIO{}s $S_1, S_2$ mit der gleichen Signatur wird
  $S_1\ERel S_2$ geschrieben, wenn $\ET{}_1\subseteq \ET{}_2$ und
  $\EL{}_1\subseteq \EL{}_2$ gilt.
\end{Def}

Der folgende Satz wurde in~\cite{Vogler2014EIO} für die Parallelkomposition
mit verborgenen synchronisierten Aktionen formuliert, jedoch entspricht, der
hier aufgeführte Satz für die Parallelkomposition ohne verbergen der
synchronisierten Aktionen, dem
analogen Satz aus~\cite{Schlosser2012BA}. Da der Beweis ohne Beachtung
von~\cite{Schlosser2012BA} neu geführt wurde, wird hier eher auf die
Erwähnung der Unterschiede zu~\cite{Vogler2014EIO} Wert gelegt.

\begin{satz}[Error-Semanik für Parallelkompositionen]
\label{satzErrorSemanik}
  Für zwei komponierbare \EIO{}s $S_1, S_2$ und ihre Komposition
  $S_{12}$, gilt:
  \begin{enumerate}
    \item
      $\ET{}_{12}=\cont{}\left(\prune{}\left(\left(\ET{}_1\|\EL{}_2\right)\cup
      \left(\EL{}_1\|\ET{}_2\right)\right)\right)$,
    \item $\EL{}_{12}=(\EL{}_1\|\EL{}_2)\cup \ET{}_{12}$.
  \end{enumerate}
\end{satz}

\begin{proof}
  1. \glqq{}$\subseteq$\grqq{}:\\
  Da beide Seiten der Gleichung unter der Fortsetzung $\cont{}$ abgeschlossen sind, genügt es ein
  präfix-minimales Element $w$ von $\ET{}_{12}$ zu betrachten. Dieses Element ist
  aufgrund der Definition der Menge der Errortraces in $\MIT{}_{12}$ oder in
  $\PrET{}_{12}$ enthalten.
  \begin{itemize}
    \item Fall 1 ($w\in \MIT{}_{12}$): Aus der Definition von \MIT{} folgt, dass es eine
  Aufteilung $w=xa$ gibt mit $(q_{01},q_{02})
  \overset{x}{\Rightarrow}(q_1,q_2)\wedge a\in I_{12}\wedge (q_1,q_2)
  \overset{a}{\not{\hspace{-0.1cm}\rightarrow}}$. Da $I_{12}
  \overset{\ref{DefParallelkomposition}}{=}(I_1\backslash O_2)\cup
  (I_2\backslash O_1)=(I_1\cup I_2)\backslash (O_1\cup O_2)$ ist, folgt $a\in (I_1\cup
  I_2)$ und $a\notin (O_1\cup O_2)$. Es wird unterschieden, ob $a\in (I_1\cap I_2)$
  oder $a\in (I_1\cup I_2)\backslash (I_1\cap I_2)$ ist. Diese Unterscheidung
  ist in~\cite{Vogler2014EIO} nicht nötig, da dort $I_1\cap I_2=\emptyset$
  gilt, somit gibt es dort nur den Fall 1b).
  \begin{itemize}
    \item Fall 1a) ($a\in (I_1\cap I_2)$): Der Ablauf der Komposition kann nun
      auf die Transitionssysteme projiziert werden und man erhält dann \oBdA{}
      $q_{01}\overset{x_1}{\Rightarrow} q_1
      \overset{a}{\not{\hspace{-0.1cm}\rightarrow}}$ und
      $q_{02}\overset{x_2}{\Rightarrow} q_2
      \overset{a}{\not{\hspace{-0.1cm}\rightarrow}}$ oder
      $q_{02}\overset{x_2}{\Rightarrow} q_2 \overset{a}{\rightarrow}$ mit $x\in
      x_1\|x_2$. Daraus kann $x_1a\in \cont{}(\MIT{}_1)\subseteq \ET{}_1$ und
      $x_2a\in \EL{}_2$ ($x_2a\in \MIT{}_2$ oder $x_2a\in L_2$)
      gefolgert werden. Damit folgt $w\in (x_1\|x_2)\cdot\{a\}\subseteq
      (x_1a)\|(x_2a)\subseteq \ET{}_1\|\EL{}_2$, und somit ist $w$ in der
      rechten Seite der Gleichung enthalten.
  \item Fall 1b) ($a\in (I_1\cup I_2)\backslash(I_1\cap I_2)$): \OBdA{} gilt
      $a\in I_1$. Durch Projektion erhält man:
      $q_{01}\overset{x_1}{\Rightarrow} q_1
      \overset{a}{\not{\hspace{-0.1cm}\rightarrow}}$ und
      $q_{02}\overset{x_2}{\Rightarrow} q_2$ mit $x\in x_1\|x_2$. Daraus folgt
      $x_1a\in \cont{}(\MIT{}_1)\subseteq \ET{}_1$ und $x_2\in L_2\subseteq \EL{}_2$. Somit
      gilt $w\in (x_1\| x_2)\cdot\{a\}\subseteq (x_1a)\|x_2\subseteq \ET{}_1\|\EL{}_2$.
      Dies ist eine Teilmenge der rechten Seite der Gleichung.
  \end{itemize}
    \item Fall 2 ($w\in \PrET{}_{12}$): Aus den Definitionen von $\PrET{}$
      und $\prune{}$ folgt, dass es ein $v\in O_{12}^*$ gibt, so dass
      $(q_{01},q_{02}) \overset{w}{\Rightarrow} (q_1,q_2)
      \overset{v}{\Rightarrow} (q_1',q_2')$ gilt mit $(q_1',q_2')\in E_{12}$
      und $w=\prune{}(wv)$. Durch Projektion erhält man $q_{01}
      \overset{w_1}{\Rightarrow} q_1 \overset{v_1}{\Rightarrow} q_1'$ und
      $q_{02} \overset{w_2}{\Rightarrow} q_2 \overset{v_2}{\Rightarrow} q_2'$
      mit $w\in w_1\|w_2$ und $v\in v_1\|v_2$. Aus $(q_1',q_2')\in E_{12}$
      folgt, dass es sich entweder um einen geerbten oder einen neuen Error
      handelt. Bei einem geerbten wäre bereits einer der beiden Zustände $q_1$
      bzw. $q_2$ ein Error-Zustand gewesen. Ein neue Error hingegen wäre durch
      die fehlende Möglichkeit entstanden, eine synchronisierte Aktion
      auszuführen.
      \begin{itemize}
        \item Fall 2a) (geerbter Error): \OBdA{} gilt $q_1'\in E_1$. Daraus folgt
          $w_1v_1\in \StET{}_1\subseteq \cont{}(\PrET{}_1)\subseteq \ET{}_1$.
          Da $q_{02}\overset{w_2v_2}{\Longrightarrow}$ gilt, erhält man $w_2v_2\in
          L_2\subseteq \EL{}_2$. Dadurch ergibt sich $wv\in \ET{}_1\|\EL{}_2$ mit
          $w=\prune{}(wv)$ und somit ist $w$ in der rechten Seite der Gleichung
          enthalten.
        \item Fall 2b) (neuer Error): \OBdA{} gilt $a\in I_1\cap O_2$ mit
          $q_1'\overset{a}{\not{\hspace{-0.1cm}\rightarrow}} \wedge \; q_2'
          \overset{a}{\rightarrow}$. Daraus folgt $w_1v_1a\in \MIT{}_1\subseteq
          \ET{}_1$ und $w_2v_2a\in L_2\subseteq \EL{}_2$. Damit ergibt sich $wva\in
          \ET{}_1\|\EL{}_2$, da $a\in O_2\subseteq O_{12}$ gilt $w=\prune{}(wva)$ und
          somit ist $w$ in der rechten Seite der Gleichung enthalten.
      \end{itemize}
  \end{itemize}

  1. \glqq{}$\supseteq$\grqq{}:\\
  Wegen der Abgeschlossenheit beider Seiten der Gleichung gegenüber $\cont{}$
  wird auch in diesem Fall nur ein präfix-minimales Element $x\in
  \prune{}\left(\left(\ET{}_1\|\EL{}_2\right)\cup
    \left(\EL{}_1\|\ET{}_2\right)\right)$ betrachtet. Da $x$ durch
  die Anwendung der $\prune{}$-Funktion entstanden ist, existiert ein $y\in
  O_{12}^*$ mit $xy\in(\ET{}_1\|\EL{}_2)\cup (\EL{}_1\|\ET{}_2)$. \OBdA{} wird
  davon ausgegangen, dass
  $xy\in \ET{}_1\|\EL{}_2$ gilt, d.h.\ es gibt $w_1\in \ET{}_1$ und $w_2\in \EL{}_2$ mit
  $xy\in w_1\|w_2$. In dem Punkt, dass das präfix-minimale Element noch mit
  Outputs fortsetzt werden kann, unterscheidet sich dieser Beweis von dem
  in~\cite{Schlosser2012BA}. In dieser Quelle wird nicht weiter darauf eingegangen, dass
  die $\prune{}$-Funktion an dieser Stelle noch zur Anwendung kommt. Da jedoch später nur
  Präfixe von $x$ betrachtet werden, ist dieser Unterschied irrelevant.\\
  Im Folgenden wird für alle Fälle von $xy$ gezeigt, dass es ein $v\in
  \PrET{}_{12}\cup \MIT{}_{12}$ gibt, das ein Präfix von $xy$ ist und $v$
  entweder auf einen Input aus $I_{12}$ endet oder $v = \varepsilon$. Damit
  muss $v$ ein Präfix von $x$
  sein. $\varepsilon$ ist Präfix von jedem Wort und sobald $v$ mindestens einen
  Buchstaben enthält, muss das Ende von $v$ vor dem Anfang von $y\in O_{12}^*$
  liegen. Dadurch ist ein Präfix von $x$ in $\PrET{}_{12}\cup \MIT{}_{12}$
  enthalten und somit gilt $x\in \ET{}_{12}$, da \ET{} die Fortsetzung der
  Mengenvereinigung aus \PrET{} und \MIT{} ist.\\
  Sei $v_1$ das kürzeste Präfix von $w_1$ in $\PrET{}_1\cup \MIT{}_1$. Falls
  $w_2\in L_2$, so sei $v_2=w_2$, sonst soll $v_2$ das kürzeste Präfix von
  $w_2$ in $\PrET{}_2\cup \MIT{}_2$ sein. Jede Aktion in $v_1$ und $v_2$ hängt mit
  einer aus $xy$ zusammen. Es kann nun davon ausgegangen werden, dass entweder
  $v_2=w_2\in L_2$ gilt oder die letzte Aktion von $v_1$ vor oder
  gleichzeitig mit der letzten Aktion von $v_2$ statt findet. Ansonsten endet
  $v_2\in \PrET{}_2\cup \MIT{}_2$ vor $v_1$ und somit ist dieser Fall analog zu $v_1$
  endet vor $v_2$.
  \begin{itemize}
    \item Fall 1 ($v_1=\varepsilon$): Da $\varepsilon\in \PrET{}_1\cup
      \MIT{}_1$, ist bereits in $S_1$ ein Error-Zustand lokal erreichbar. $\varepsilon\in
      \MIT{}_1$ ist nicht möglich, da jedes Element aus $MIT$ nach Definition
      mindestens die Länge $1$ haben muss. Mit der Wahl
      $v_2'=v'=\varepsilon$ ist $v_2'$ ein Präfix von $v_2$.
    \item Fall 2 ($v_1\neq\varepsilon$): Aufgrund der Definitionen von $\PrET{}$
      und $MIT$ endet $v_1$ auf ein $a\in I_1$, d.h.\ $v_1=v_1’a$. $v'$ sei das
      Präfix von $xy$, das mit der letzten Aktion von $v_1$ endet, d.h.\ mit
      $a$ und $v_2'=v'|_{\Sigma _{2}}$. Falls $v_2 = w_2\in L_2$, dann ist
      $v_2'$ ein Präfix von $v_2$. Falls $v_2\in
      \PrET{}_2\cup \MIT{}_2$ gilt, dann ist durch die Annahme, dass $v_2$ nicht vor
      $v_1$ endet, $v_2'$ ein Präfix von $v_2$. Im Fall $v_2\in \MIT{}_2$ kann
      durch die gleiche Argumentation ebenfalls geschlossen, dass $v_2'$ ein
      Präfix von $v_2$ ist. Zusätzlich weiß man, dass $v_2$ auf $b\in
      I_2$ endet, jedoch muss nicht mehr wie in~\cite{Vogler2014EIO} $b\neq a$
      gelten. Es kann also keine Aussage mehr darüber getroffen, ob es sich um
      ein echtes Präfix handelt.
  \end{itemize}
  In allen Fällen erhält man: $v_2'=v'|_{\Sigma _2}$ ist ein Präfix von $v_2$
  und $v'\in v_1\| v_2'$ ist ein Präfix von $xy$. Da nicht mehr $b\neq a$
  gelten muss, kann nicht mehr für alle Fälle $q_{02}
  \overset{v_2'}{\Rightarrow}$ gefolgert werden, wie das
  in~\cite{Vogler2014EIO} möglich war, sondern nur wenn $a\notin I_2$ gilt.
  \begin{itemize}
    \item Fall I ($v_1\in \MIT{}_1$ und $v_1\neq\varepsilon$): Es gibt einen
      Ablauf der Form $q_{01} \overset{v_1'}{\Rightarrow}q_1
      \overset{a}{\not{\hspace{-0.1cm}\rightarrow}}$ und es gilt $v'=v''a$. Bei der
      folgenden Fallunterscheidung müssen im Gegensatz zu~\cite{Vogler2014EIO}
      zwei weitere Fälle (Ib) und Ic)) einfügt werden, da es zulässig ist, dass $a$
      sowohl in $I_1$ wie auch in $I_2$ enthalten ist.
      \begin{itemize}
        \item Fall Ia) ($a\notin\Sigma _2$): Es gilt $q_{02}
          \overset{v_2'}{\Rightarrow} q_2$ mit $v''\in v_1'\|v_2'$. Dadurch erhält man
          $(q_{01},q_{02}) \overset{v''}{\Rightarrow} (q_1,q_2)
          \overset{a}{\not{\hspace{-0.1cm}\rightarrow}}$ mit $a\in I_{12}$.
          Somit wird $v:=v''a=v'\in \MIT{}_{12}$ gewählt.
        \item Fall Ib) ($a\in I_2$ und $v_2'\in \MIT{}_2$): Es gilt $v_2'=v_2''a$
          mit $q_{02} \overset{v_2''}{\Rightarrow} q_2
          \overset{a}{\not{\hspace{-0.1cm}\rightarrow}}$ und $v''\in
          v_1'\|v_2''$. $a$ ist  für $S_2$, ebenso wie für $S_1$, ein fehlender
          Input. Daraus folgt, dass $(q_1,q_2)
          \overset{a}{\not{\hspace{-0.1cm}\rightarrow}}$ gilt. Es wird
          ebenfalls $v:=v''a=v'\in \MIT{}_{12}$ gewählt.
        \item Fall Ic) ($a\in I_2$ und $v_2'\in L_2$): Es gilt $q_{02}
          \overset{v_2''}{\Rightarrow} q_2 \overset{a}{\rightarrow}$ mit
          $v_2'=v_2''a$. Da jedoch die Menge der synchronisierten Aktionen
          bezüglich~\cite{Vogler2014EIO} erweitert wurde liegt $a$ in
          $\Synch(S_1,S_2)$, also folgt $(q_1,q_2)
          \overset{a}{\not{\hspace{-0.1cm}\rightarrow}}$ schon aus $q_1
          \overset{a}{\not{\hspace{-0.1cm}\rightarrow}}$. Somit kann hier
          nochmals $v:=v''a=v'\in \MIT{}_{12}$ gewählt werden.
        \item Fall Id) ($a\in O_2$): Es gilt $v_2'=v_2''a$ und $q_{02}
          \overset{v_2'}{\Rightarrow}$. Man erhält also $q_{02}
          \overset{v_2''}{\Rightarrow} q_2 \overset{a}{\rightarrow}$ mit
          $v''\in v_1'\|v_2''$. Daraus ergibt sich $(q_{01},q_{02})
          \overset{v''}{\Rightarrow} (q_1,q_2)$ mit $q_1
          \overset{a}{\not{\hspace{-0.1cm}\rightarrow}},a\in I_1, q_2
          \overset{a}{\rightarrow}$ und $a\in O_2$, somit gilt $(q_1,q_2)\in
          E_{12}$. Es wird $v:=\prune{}(v'')\in \PrET{}_{12}$ gewählt.
      \end{itemize}
  \item Fall II ($v_1\in \PrET{}_1$): $\exists u_1\in O_1^*:q_{01}
    \overset{v_1}{\Rightarrow} q_1 \overset{u_1}{\Rightarrow} q_1'$ mit
    $q_1'\in E_1$. Da es hier keine disjunkten Inputmengen wie
    in~\cite{Vogler2014EIO} gibt, kann das $a$, auf das $v_1$ im Fall $v_1\neq
    \varepsilon$ endet, ebenfalls der letzte Buchstabe von $v_2$
    sein. Im Fall von $v_2\in \MIT{}_2$ kann somit $a=b$ gelten, wodurch
    $v_2=v_2'$ gilt. Dieser Fall verläuft jedoch analog zu Fall Ic) und wird
    hier nicht weiter betrachtet. Es gilt für
    alle anderen Fälle $q_{02} \overset{v_2'}{\Rightarrow}q_2$ mit
    $(q_{01},q_{02}) \overset{v'}{\Rightarrow}(q_1,q_2)$.
    \begin{itemize}
      \item Fall IIa) \Big($u_2\in (O_1\cap I_2)^*, c\in (O_1\cap I_2)$, sodass
        $u_2c$ Präfix von $u_1|_{I_2}$ mit $q_2 \overset{u_2}{\Rightarrow} q_2'
        \overset{c}{\not{\hspace{-0.1cm}\rightarrow}}$\Big): Für das Präfix $u_1'c$
        von $u_1$ mit $u_1'c|_{I_2}=u_2c$ weiß man, dass $q_1
        \overset{u_1'}{\Rightarrow} q_1'' \overset{c}{\rightarrow}$. Somit gilt
        $u_1'\in u_1'\|u_2$ und $(q_1,q_2) \overset{u_1'}{\Rightarrow}
        (q_1'',q_2')\in E_{12}$, da für $S_2$ der entsprechende Input fehlt,
        der mit dem $c$ Output von $S_1$ zu koppeln wäre. Es handelt sich also
        um einen neuen Error. Es wird $v:=\prune{}(v'u_1')\in \PrET{}_{12}$ gewählt,
        dies ist ein Präfix von $v'$, da $u_1\in O_1^*$.
      \item Fall IIb) \big($q_2 \overset{u_2}{\Rightarrow} q_2'$ mit
        $u_2=u_1|_{I_2}$\big): Es gilt $u_1\in u_1\|u_2$ und $(q_1,q_2)
        \overset{u_1}{\Rightarrow} (q_1',q_2')\in E_{12}$, da $q_1'\in E_1$ und
        somit handelt es sich in $S_{12}$ um einen geerbten Error. Nun wird $v:=\prune{}
        (v'u_1)\in \PrET{}_{12}$ gewählt, das wiederum ein Präfix von $v'$ ist.
    \end{itemize}
  \end{itemize}

  2.:\\
  Der Beweis für diesen Punkt konnte bezüglich~\cite{Vogler2014EIO} fast
  unverändert übernommen werden.\\
  Es ist durch die Definition klar, dass $L_i\subseteq \EL{}_i$ und
  $\ET{}_i\subseteq \EL{}_i$ gilt. Die Argumentation wird von der rechten
  Seite der Gleichung aus begonnen:
  \begin{align*}
    (\EL{}_1\| \EL{}_2)\cup \ET{}_{12}&\overset{\ref{DefETEL}}{=}\left(\left(L_1\cup
  \ET{}_1\right)\|\left(L_2\cup \ET{}_2\right)\right)\cup \ET{}_{12}\\
    &=(L_1\|L_2) \cup \underset{\overset{1.}{\subseteq} \ET{}_{12}}{\underset{\subseteq
    (\EL{}_1\|\ET{}_2)}{\underbrace{(L_1\|\ET{}_2)}}} \cup
    \underset{\overset{1.}{\subseteq} \ET{}_{12}}{\underset{\subseteq
    (\ET{}_1\|\EL{}_2)}{\underbrace{(\ET{}_1\|L_2)}}} \cup
     \underset{\overset{1.}{\subseteq}
    \ET{}_{12}}{\underset{\subseteq (\EL{}_1\|\ET{}_2)}{\underbrace{(\ET{}_1\|\ET{}_2)}}} \cup
    \ET{}_{12}\\
    &=(L_1\|L_2) \cup \ET{}_{12}\\
    &\overset{\ref{LemmaSprache}}{=}L_{12}\cup \ET{}_{12}\\
    &\overset{\ref{DefETEL}}{=}\EL{}_{12}.
  \end{align*}
\end{proof}

Das folgende Korollar wurde hier noch explizit mit Beweis eingefügt im
Gegensatz zu den Ausführungen in~\cite{Vogler2014EIO}, in denen diese
Präkongruenz nur als Folgerung aus dem letzten Satz erwähnt wird. Die
Feststellung, dass es sich um eine Präkongruenz handelt, ist wichtig, da
dann die erste Eigenschaft erfüllt ist, um eine operationale Beschreibung der
vollständig abstrakten Präkongruenz \ECRel{} zu erhalten.

\begin{kor}[Error-Präkongruenz]
\label{propPraekongruenz}
  Die Relation \ERel{} ist eine Präkongruenz bezüglich $\cdot\|\cdot$.
\end{kor}

\begin{proof}
  Es muss gezeigt werden: Wenn $S_1\ERel S_2$ gilt, dann für jedes
  komponierbare $S_3$ auch $S_{31}\ERel S_{32}$. D.h.\ es ist zu zeigen,
  dass aus $\ET{}_1\subseteq \ET{}_2$ und $\EL{}_1\subseteq \EL{}_2$,
  $\ET{}_{31}\subseteq \ET{}_{32}$ und $\EL{}_{31}\subseteq
  \EL{}_{32}$ folgt. Dies ergibt sich aus der Monotonie von \cont{},
  \prune{} und $\cdot \|\cdot$ auf Sprachen wie folgt:\\
  \begin{itemize}
    \item $\begin{aligned}[t]
        \ET{}_{31} &\overset{\ref{satzErrorSemanik}~1.}{=}
      \cont{}\left(\prune{}\left(\left(\ET{}_3\|\EL{}_1\right)\cup
          \left(\EL{}_3\|\ET{}_1\right)\right)\right)\\
      &\hspace{-0.3cm}\overset{\ET{}_1\subseteq
    \ET{}_2}{\overset{\mathrm{und}}{\overset{\EL{}_1\subseteq \EL{}_2}{\subseteq}}}
    \cont{}\left(\prune{}\left(\left(\ET{}_3\|\EL{}_2\right)\cup
        \left(\EL{}_3\|\ET{}_2\right)\right)\right)\\
    &\overset{\ref{satzErrorSemanik}~1.}{=} \ET{}_{32},
    \end{aligned}$
    \item $\begin{aligned}[t]
        \EL{}_{31} &\overset{\ref{satzErrorSemanik}~2.}{=} (\EL{}_3\|\EL{}_1)\cup
        E_{31}\\
        &\hspace{-0.4cm}\overset{\EL{}_1\subseteq
      \EL{}_2}{\overset{\mathrm{und}}{\overset{\ET{}_{31}\subseteq
      \ET{}_{32}}{\subseteq}}} (\EL{}_3\|\EL{}_2)\cup \ET{}_{32}\\
      &\overset{\ref{satzErrorSemanik}~2.}{=} \EL{}_{32}.
    \end{aligned}$
  \end{itemize}
\end{proof}

In~\cite{Vogler2014EIO} wurde auch die Verfeinerung von \EIO{}s als Relation
mit Spezifikation und Implementierung betrachtet. Hier soll ebenfalls eine
Verfeinerungsrelation über \EIO{}s betrachtet werden, jedoch sollen die
synchronisierten Aktionen nicht verborgen werden. Dadurch ändern sich natürlich
auch Teile des folgenden Beweises, vor allem muss statt mit $\StET{}$ mit der Menge
$\PrET{}$ argumentiert werden. Dieses Lemma existiert in dieser Form nicht
in~\cite{Schlosser2012BA}, da es dort mit der Aussage von
Satz~\ref{satzFullAbstractness} kombiniert wurde. Jedoch ist die Aussage dieses
Lemmas trotzdem Teil dessen, was gezeigt wird und somit finden sich die Teile
des folgenden Beweises auch dort wieder.\\
Die Verfeinerungsrelation, die in dem nächsten Lemma betrachtet werden soll,
verfeinert über guter Kommunikation im Sinne der error-freien Kommunikation.

\begin{lem}[Verfeinerung mit Errors]
\label{lemVerfeinerung}
  Gegeben sind zwei \EIO{}s $S_1$ und $S_2$ mit der gleichen Signatur. Wenn
  $U\|S_1 \EBRel U\|S_2$ für alle Partner $U$ gilt, dann folgt daraus die
  Gültigkeit von $S_1\ERel S_2$.
\end{lem}

\begin{proof}
  Da $S_1$ und $S_2$ die gleichen Signaturen haben wird
  $I:=I_1=I_2$ und $O:=O_1=O_2$ definiert. Für jeden der Partner $U$ gilt $I_U=O$ und
  $O_U=I$.\\
  Um $S_1\ERel S_2$ zu zeigen, wird nachgeprüft, ob folgendes gilt:
  \begin{itemize}
    \item $\ET{}_1\subseteq \ET{}_2$,
    \item $\EL{}_1\subseteq \EL{}_2$.
  \end{itemize}
  Für ein gewähltes präfix-minimales Element $w\in \ET{}_1$ wird gezeigt,
  dass dieses $w$ oder eines seiner Präfixe in $\ET{}_2$ enthalten ist.
  Dies ist möglich, da die beide Mengen $\ET{}_1$ und $\ET{}_2$ durch $\cont{}$
  abgeschlossen sind.
  \begin{itemize}
    \item Fall 1 ($w=\varepsilon$): Es handelt sich um einen lokal erreichbaren
      Error-Zustand in $S_1$.
      Für $U$ wird ein Transitionssystem verwendet, das nur aus dem Startzustand und
      einer Schleife für alle Inputs $x\in I_U$ besteht. Somit kann $S_1$ die
      im Prinzip gleichen
      Error-Zustände lokal erreichen wie $U\|S_1$. Daraus folgt, dass auch
      $U\|S_2$ einen lokal erreichbaren Error-Zustand haben muss. Durch die
      Definition von $U$ kann dieser Error nur von $S_2$ geerbt sein. Es
      muss also in $S_2$ ein Error-Zustand durch interne Aktionen und Outputs
      erreichbar sein, d.h.\ es gilt $\varepsilon\in \PrET{}_2$.
    \item Fall 2 ($w=x_1\dots x_n x_{n+1}\in\Sigma ^+$ mit $n\geq 0$ und
      $x_{n+1}\in I = O_U$): Es wird der folgende Partner $U$ bedachtet (siehe auch
      Abbildung~\ref{UohneE}):
      \begin{itemize}
        \item $Q_U=\{q_0,q_1,\dots ,q_{n+1}\}$,
        \item $q_{0U}=q_0$,
        \item $E_U=\emptyset$,
        \item $\begin{aligned}[t]
            \delta _U=&\{(q_i,x_{i+1},q_{i+1})\mid  0\leq i\leq n\}\\
                      &\cup\{(q_i,x,q_{n+1})\mid  x\in I_U\backslash\{x_{i+1}\},
          0\leq i\leq n\}\\
          &\cup\{(q_{n+1},x,q_{n+1})\mid  x\in I_U\}.
        \end{aligned}$
      \end{itemize}
      \begin{figure} [h!tbp]
      \begin{center}
        \begin{tikzpicture}[->, >=latex',auto,node distance =3cm, semithick]

          \node (0) {$q_0$};
          \node (1) [right of=0] {$q_1$};
          \node (dots) [right of=1] {$\dots$};
          \node (n) [right of=dots] {$q_n$};
          \node (n1) at ($(1)!0.5!(dots) + (0,-3)$) {$q_{n+1}$};

          \path ($ (0) + (-1,0) $) edge (0)
                (0) edge node {$x_1$} (1)
                    edge [bend right] node [below, sloped] {$x?\neq x_1$} (n1)
                (1) edge node {$x_2$} (dots)
                    edge node [below, sloped] {$x?\neq x_2$} (n1)
                (dots) edge node {$x_n$} (n)
                       edge [dashed] (n1)
                (n) edge node [above, sloped] {$x?\in I_U$} (n1)
                    edge [bend left] node [sloped] {$x_{n+1}$!} (n1)
                (n1) edge [loop below] node {$x?\in I_U$} (n1);
        \end{tikzpicture}
        \caption{$x?\neq x_i$ steht für alle $x\in I_U\backslash\{x_i\}$}
\label{UohneE}
      \end{center}
      \end{figure}
      Für $w$ können nun zwei Fälle unterscheiden werden. Aus beiden wird
      folgen, dass $\varepsilon\in \PrET{}(U\|S_1)$. Dieses Vorgehen unterscheidet sich von dem
      in~\cite{Vogler2014EIO}, da hier die synchronisierten Aktionen als Outputs
      vorhanden bleiben und somit nicht $\varepsilon\in \StET{}(U\|S_1)$ gelten
      kann.
      \begin{itemize}
        \item Fall 2a) ($w\in \MIT{}_1$): In $U\|S_1$ erhält man
          $(q_0,q_{01}) \overset{x_1\dots x_n}{\xRightarrow{\hspace{1cm}}} (q_n,q')$ mit
          $q' \overset{x_{n+1}}{\not{\hspace{-0.2cm}\longrightarrow}}$ und $q_n
          \overset{x_{n+1}}{\longrightarrow}$. Deshalb gilt $(q_n,q')\in
          E_{U\|S_1}$ und $x_1\dots x_n\in \StET{}(U\|S_1)$. Da alle Aktionen aus
          $w$ bis auf $x_{n+1}$ synchronisiert werden und $I\cap I_U =
          \emptyset$, gilt $x_1,\dots ,x_n\in
          O_{U\|S_1}$. Daraus ergibt sich dann $\varepsilon\in \PrET{}(U\|S_1)$.
        \item Fall 2b) ($w\in \PrET{}_1$): In $U\|S_1$ erhält man
          $(q_0,q_{01}) \overset{w}{\Rightarrow} (q_{n+1},q'')
          \overset{u}{\Rightarrow} (q_{n+1},q')$ für $u\in O^*$ und $q'\in
          E_1$. Daraus folgt $(q_{n+1},q')\in E_{U\|S_1}$ und somit $wu\in
          \StET{}(U\|S_1)$. Da alle Aktionen aus $w$ synchronisiert werden und
          $I\cap I_U = \emptyset$, gilt
          $x_1,\dots ,x_n,$ $x_{n+1}\in O_{U\|S_1}$ und, da $u\in O^*$, folgt
          $u\in O_{U\|S_1}^*$. Somit ergibt sich $\varepsilon\in
          \PrET{}(U\|S_1)$.
      \end{itemize}
      Da $\varepsilon\in \PrET{}(U\|S_1)$ gilt, kann durch
      $U\|S_1\EBRel U\|S_2$ geschlossen werden, dass auch in $U\|S_2$ ein
      Error-Zustand lokal erreichbar sein muss.\\
      Dieser Error kann geerbt oder neu sein.
      \begin{itemize}
        \item Fall 2i) (neuer Error): Da jeder Zustand von $U$ alle Inputs $x\in
          O=I_U$ zulässt, muss ein lokal erreichbarer Error-Zustand der Form sein, dass
          ein Output $a\in O_U$ von $U$ möglich ist, der nicht mit einem
          passenden Input aus $S_2$ synchronisiert werden kann. Durch die
          Konstruktion von $U$ sind in $q_{n+1}$ keine Outputs möglich. Ein
          neuer Error muss also die Form $(q_i,q')$ haben mit $i\leq n, q'
          \overset{x_{i+1}}{\not{\hspace{-0.2cm}\longrightarrow}}$ und $x_{i+1}\in
          O_U=I$. Durch Projektion erhält man dann $q_{02} \overset{x_1\dots
          x_i}{\xRightarrow{\hspace{0.9cm}}} q'
          \overset{x_{i+1}}{\not{\hspace{-0.2cm}\longrightarrow}}$ und damit gilt
          $x_1\dots x_{i+1}\in \MIT{}_2\subseteq \ET{}_2$. Somit ist ein Präfix
          von $w$ in $\ET{}_2$ enthalten.
        \item Fall 2ii) (geerbter Error): $U$ hat $x_1\dots x_i u$
          mit $u\in I_U^*=O^*$ ausgeführt und ebenso hat $S_2$ diesen Wort
          abgearbeitet.
          Durch dies hat $S_2$ einen Zustand in $E_2$ erreicht, da von $U$
          keine Errors geerbt werden können. Es gilt dann $\prune{}(x_1\dots$
          $x_iu)=\prune{}(x_1\dots x_i)\in \PrET{}_2\subseteq \ET{}_2$. Da
          $x_1\dots x_i$ ein Präfix von $w$ ist, führt in diesem
          Fall eine Verlängerung um lokale Aktionen von einem
          Präfix von $w$ zu einem Error-Zustand. Da $\ET$ der Menge aller
          Verlängerungen von gekürzten Errortraces entspricht, ist $x_1\dots x_i$ in
          $\ET{}_2$ enthalten und somit ist ein Präfix von $w$ in $\ET{}_2$
          enthalten.
      \end{itemize}
  \end{itemize}
  Um die andere Inklusion zu beweisen, reicht es aufgrund der ersten
  Inklusion und der Definition von $\EL{}$ aus zu zeigen, dass
  $L_1\backslash \ET{}_1\subseteq \EL{}_2$ gilt.\\
  Es wird dafür ein beliebiges $w\in L_1\backslash \ET{}_1$ gewählt und
  gezeigt, dass es in $\EL{}_2$ enthalten ist.
  \begin{itemize}
    \item Fall 1 ($w=\varepsilon$): Da $\varepsilon$ immer in $\EL{}_2$
      enthalten ist, muss hier nichts gezeigt werden.
    \item Fall 2 ($w=x_1\dots x_n$ mit $n\geq 1$): Es wird ein
      Partner $U$ wie folgt konstruiert (siehe dazu auch Abbildung~\ref{UmitE}):
      \begin{itemize}
        \item $Q_U=\{q_0,q_1,\dots ,q_n,q\}$,
        \item $q_{0U}=q_0$,
        \item $E_U=\{q_n\}$,
        \item $\begin{aligned}[t]
            \delta _U=&\{(q_i,x_{i+1},q_{i+1})\mid 0\leq i< n\}\\
                      &\cup\{(q_i,x,q)\mid x\in I_U\backslash\{x_{i+1}\},0\leq
          i< n\}\\
          &\cup\{(q,x,q)\mid x\in I_U\}.
              \end{aligned}$
      \end{itemize}
      \begin{figure} [h!tbp]
      \begin{center}
        \begin{tikzpicture}[->, >=latex',auto,node distance =3cm, semithick]

          \node (0) {$q_0$};
          \node (1) [right of=0] {$q_1$};
          \node (dots) [right of=1] {$\dots$};
          \node (n1) [right of=dots] {$q_{n-1}$};
          \node (n) [right of=n1, rectangle, draw] {$q_n\in E_U$};
          \node (q) at ($(1)!0.5!(dots) + (0,-3)$) {$q$};

          \path ($ (0) + (-1,0) $) edge (0)
                (0) edge node {$x_1$} (1)
                    edge [bend right] node [below, sloped] {$x?\neq x_1$} (q)
                (1) edge node {$x_2$} (dots)
                    edge node [below, sloped] {$x?\neq x_2$} (q)
                (dots) edge node {$x_{n-1}$} (n1)
                       edge [dashed] (q)
                (n1) edge node {$x_n$} (n)
                edge [bend left] node [below, sloped] {$x?\neq x_n$} (q)
                (q) edge [loop below] node {$x?\in I_U$} (q);
        \end{tikzpicture}
        \caption{$x?\neq x_i$ steht für alle $x\in I_U\backslash\{x_i\}$, $q_n$
          ist der einzige Error-Zustand}
\label{UmitE}
      \end{center}
      \end{figure}
      Da $q_{01} \overset{w}{\Rightarrow} q'$ gilt, kann man schließen, dass $U\|S_1$
      einen lokal erreichbaren geerbten Error hat. Somit muss $U\|S_2$ ebenfalls einen
      lokal erreichbaren Error-Zustand haben.
      \begin{itemize}
        \item Fall 2a) (neuer Error aufgrund von $x_i\in O_U$ und $q_{02}
          \overset{x_1\dots x_{i-1}}{\xRightarrow{\hspace{1.3cm}}} q''
          \overset{x_i}{\not{\hspace{-0.1cm}\rightarrow}}$): Es gilt $x_1\dots
          x_i\in \MIT{}_2$ und somit $w\in \EL{}_2$. Anzumerken ist, dass nur
          auf diesem Weg Outputs von $U$ möglich sind, deshalb gibt es keine
          anderen Outputs von $U$, die zu einem neuen Error führen können.
        \item Fall 2b) (neuer Error aufgrund von $a\in O = I_U$): Der einzige
          Zustand, in dem $U$ nicht alle Inputs erlaubt sind, ist $q_n$, der
          bereits ein Error-Zustand ist. Da hier dieser Zustand in $U\|S_2$
          erreichbar ist, besitzt das komponierte \EIO{} einen geerbten Error und
          es gilt $w\in L_2\subseteq \EL{}_2$, wegen dem folgenden Fall 2c).
        \item Fall 2c) (geerbter Error von $U$): Da $q_n$ der einzige
          Error-Zustand in $U$ ist und alle Aktionen synchronisiert sind, ist dies nur
          möglich, wenn $q_{02} \overset{x_1\dots
          x_n}{\xRightarrow{\hspace{1cm}}}$ gilt. In
          diesem Fall gilt $w\in L_2\subseteq \EL{}_2$.
        \item Fall 2d) (geerbter Error von $S_2$): Es gilt dann $q_{02}
          \overset{x_1\dots x_iu}{\xRightarrow{\hspace{1.2cm}}} q'\in E_2$ für $i\geq 0$ und
          $u\in O^*$. Somit ist $x_1\dots x_iu\in \StET{}_2$ und damit
          $\prune{}(x_1\dots x_iu)=\prune{}(x_1\dots x_i)\in \PrET{}_2\subseteq
          \EL{}_2$. Somit gilt $w\in \EL{}_2$.
      \end{itemize}
  \end{itemize}
\end{proof}

Der folgende Satz sagt aus, dass \ERel{} die gröbste Präkongruenz ist, die
charakterisiert werden soll, also gleich der vollständig abstrakten
Präkongruenz \ECRel{}.

\begin{satz}[Vollständige Abstraktheit für Error-Semanik]
\label{satzFullAbstractness}
Für zwei \EIO{}s $S_1$ und $S_2$ mit derselben Signatur gilt
$S_1\ECRel S_2\Leftrightarrow S_1\ERel S_2$.
\end{satz}

\begin{proof}
  \glqq{}$\Leftarrow$\grqq{}: Nach Definition gilt, genau dann wenn
      $\varepsilon\in \ET{}(S)$, ist ein Error-Zustand lokal erreichbar in $S$.
      $S_1\ERel S_2$ impliziert, dass $\varepsilon\in
      \ET{}_2$ gilt, wenn $\varepsilon\in \ET{}_1$. Somit ist ein Error-Zustand
      in $S_1$ nur dann lokal erreichbar, wenn dieser auch in $S_2$ lokal
      erreichbar ist. Dadurch folgt, dass $S_1\EBRel S_2$ gilt, da \EBRel{} in
      Definition~\ref{DefErrorBasisrel} über die lokale Erreichbarkeit der
      Error-Zustände definiert wurde. Es ist also \ERel{} in \EBRel{} enthalten.
      Wie in Korollar~\ref{propPraekongruenz} gezeigt, ist \ERel{} eine
      Präkongruenz bezüglich $\cdot\|\cdot$. Da \ECRel{} die gröbste Präkongruenz bezüglich
      $\cdot\|\cdot$ ist, die in \EBRel{} enthalten ist, muss \ERel{} in
      \ECRel{} enthalten sein. Es folgt also aus $S_1\ERel S_2$, dass auch
      $S_1\ECRel{} S_2$ gilt.

      \glqq{}$\Rightarrow$\grqq{}: Durch die Definition von \ECRel{} als
      Präkongruenz in~\ref{DefErrorBasisrel} folgt aus $S_1\ECRel S_2$, dass
      $U\|S_1\ECRel U\|S_2$ für alle \EIO{}s $U$ gilt, die mit $S_1$ komponierbar
      sind. Da \ECRel{} nach Definition auch in \EBRel{} enthalten sein soll,
      folgt aus $U\|S_1\ECRel U\|S_2$ auch die Gültigkeit von $U\|S_1\EBRel
      U\|S_2$ für alle diese \EIO{}s $U$. Mit Lemma~\ref{lemVerfeinerung} folgt
      dann $S_1\ERel S_2$.
\end{proof}

Es wurde somit jetzt eine Kette an Folgerungen gezeigt, die sich zu einem
Ring schließt. Dies ist in Abbildung~\ref{Folgerungskette} dargestellt.

\begin{figure}[h!tbp]
  \begin{center}
    \begin{tikzpicture}[scale = 3]
      \matrix (m) [matrix of math nodes,row sep=2cm,column sep=4cm]{%
        S_1\ERel S_2 & S_1\ECRel S_2 \\
        \substack{\forall~\mathrm{Partner}~U:\\U\|S_1\EBRel U\|S_2} &
    \substack{\forall~\mathrm{komponierbaren}~U:\\U\|S_1\EBRel U\|S_2} \\};
        \draw[-implies, double, double distance=1mm]
          (m-1-1) -- node [above] {\glqq{}$\Leftarrow$\grqq{} von
            Satz~\ref{satzFullAbstractness}} (m-1-2);
        \draw[-implies, double, double distance=1mm]
          (m-1-2) -- node [right] {Definition von \ECRel{}
          in~\ref{DefErrorBasisrel}} (m-2-2);
        \draw[-implies, double, double distance=1mm]
          (m-2-1) -- node [left]
          {Lemma~\ref{lemVerfeinerung}} (m-1-1);
        \draw[-implies, double, double distance=1mm]
          (m-2-2) -- node [below]
          {$\substack{U~\mathrm{Partner}\\\Downarrow\\ U~\mathrm{komponierbar}}$} (m-2-1);
    \end{tikzpicture}
    \caption{Folgerungskette}
\label{Folgerungskette}
  \end{center}
\end{figure}

Angenommen man definiert, dass $S_1$ $S_2$ verfeinern
soll, genau dann wenn für alle Partner \EIO{}s $U$, für die $S_2$ error-frei
mit $U$ kommuniziert, folgt, dass $S_1$ ebenfalls error-frei mit $U$
kommuniziert. Dann wird auch diese Verfeinerung durch \ERel{}
charakterisiert.\\
Aus Satz~\ref{satzFullAbstractness} und Lemma~\ref{lemVerfeinerung} ergibt sich
das folgende Korollar.

\begin{kor}
  Es gilt: $S_1\ERel S_2 \Leftrightarrow U\|S_1\EBRel U\|S_2$ für alle Partner
  $U$.
\end{kor}

\section{Hiding und Error-Freiheit}

Es soll nun untersucht werden, was für Auswirkungen Hiding auf die
Verfeinerungsrelationen hat. Es werden also Outputs der betrachteten Systeme internalisiert.

\begin{prop}[Error-Basisrelation bzgl. Internalisierung]
\label{propErBaIn}
  Wenn $S_1\EBRel S_2$ gilt, dann folgt daraus, dass auch $S_1/X\EBRel S_2/X$
  gilt.
\end{prop}

\begin{proof}
  Da die Definition der lokalen Erreichbarkeit auf lokalen Aktionen beruht, die
  aus den Outputs und der internen Aktion besteht, ändert sich durch das
  Verbergen von Outputs nichts an der Error-Erreichbarkeit. Somit ist jeder
  Error-Zustand, der in $S_i$ lokal erreichbar ist über ein Trace, das Outputs aus $X$
  enthält, auch in $S_i/X$ erreichbar, jedoch enthält das Trace
  nicht mehr diese Outputs. Alle Traces, die keine Outputs aus der Menge hinter
  dem Internalisierungsoperator enthalten, bleiben unverändert erhalten. Es ist
  auch nicht möglich, dass durch das Verbergen von Outputs neue Errors
  entstehen. Auch in die umgekehrte Richtung kann durch das Ersetzten von
  $\tau$s durch Outputs nichts an der Erreichbarkeit oder der Menge der
  Error-Zustände geändert werden. Es ist also jeder Error-Zustand, der in $S_i/X$ lokal erreichbar
  ist, auch in $S_i$ erreichbar. Somit folgt die Behauptung.
\end{proof}

\begin{satz}[Error-Präkongurenz bzgl. Interalisierung]
\label{satzPraeInternalisierung}
  Seinen $S_1$ und $S_2$ zwei \EIO{}s für die $S_1\ERel S_2$ gilt, somit gilt
  auch $S_1/X\ERel S_2/X$. Daraus folgt insbesondere, dass \ERel{} eine
  Präkongruenz bezüglich $\cdot /\cdot$ ist.
  Es gilt für die Sprachen und Traces:
  \begin{compactenum}[(i)]
  \item $L(S/X) = \left\{w\in (\Sigma\backslash X){}^*\mid \exists w'\in L(S):
      w'|_{\Sigma\backslash X} = w\right\}$,
    \item $\ET{}(S/X) = \left\{w\in (\Sigma\backslash X){}^*\mid \exists
      w'\in \ET{}(S): w'|_{\Sigma\backslash X} = w\right\}$,
    \item $\EL{}(S/X) = \left\{w\in (\Sigma\backslash X){}^*\mid \exists w'\in
      \EL(S): w'|_{\Sigma\backslash X} = w\right\}$.
  \end{compactenum}
\end{satz}

\begin{proof}
  Zuerst wird hier die Richtigkeit der Aussagen (i) bis (iii) gezeigt. Daraus
  kann dann der Rest des Satzes gefolgert werden.

  (i)
  Für ein Wort aus der Sprache $L$ eines Transitionssystems $S$ gilt nach
  Definition $q_0 \overset{w'}{\Rightarrow} q$ mit $q\in Q$. Es gibt also zu
  jedem $w' = a_1a_2\dots a_n\in L(S)$ ein Ablauf $q_0
  \overset{a_1}{\Rightarrow} q_1 \overset{a_2}{\Rightarrow} \dots
  \overset{a_n}{\Rightarrow} q_n$ mit $q_n=q$. Hier ist wichtig zu beachten,
  dass die jeweiligen Zustände nicht exakt über eine Transition erreicht werden
  müssen. Es kann sich hier um eine Transitionsfolge aus beliebig vielen
  $\tau$s und dem jeweiligen $a_i\in\Sigma$ handeln. Dabei ist egal an welcher
  Stelle das $a_i$ auftaucht. Dies ist notwendig, da auf Trace-Ebene nicht mehr
  festgehalten wird, wann $\tau$-Transitionen auszuführen sind, um mit einer
  bestimmten Transition den Weg fortsetzen zu können. Dies ändert jedoch nichts
  an der Ausführungsreihenfolge der $a_i$s und auch nichts
  daran, dass alle $a_i$ atomare Aktionen darstellen.
  \begin{itemize}
    \item Fall 1 ($n=0$): Es gilt $w'=\varepsilon$. Somit enthält $w'$ keine
      Aktionen aus $X$. Es werden also durch die Anwendung des
      Internalisierungsoperators in diesem $w'$ keine Aktionen verborgen. Es
      gilt also $w'=w\in L(S/X)$. Somit ist für diesen Fall die Aussage über
      $L$ korrekt.
    \item Fall 2 ($n\geq 1$): Nach der Internalisierung bleiben von dem Ablauf
      nur noch die Aktionen übrig, die nicht Elemente aus $X$ sind. Der Ablauf
      reduziert sich also auf $q_0 \overset{\tau~\mathrm{falls}~a_1\in
      X}{\overset{\mathrm{sonst}~a_1}{\xRightarrow{\hspace{1.5cm}}}} q_1
      \overset{\tau~\mathrm{falls}~a_2\in
      X}{\overset{\mathrm{sonst}~a_2}{\xRightarrow{\hspace{1.5cm}}}} \dots
      \overset{\tau~\mathrm{falls}~a_n\in
      X}{\overset{\mathrm{sonst}~a_n}{\xRightarrow{\hspace{1.5cm}}}} q_n$. Dabei bleibt
      durch das Hiding von Aktionen aus $X$ in $w'$ nur noch
      $w:=w'|_{\Sigma\backslash X}$ erhalten. Diese Projektion des Wortes
      $w'$ auf die eingeschränkte Aktionenmenge ist dann in $L(S/X)$ enthalten,
      da immer noch derselbe Zustand durch das Worte erreicht wird. Es gilt
      also auch für diesen Fall die Aussage über die Sprache $L$.
  \end{itemize}
  Für ein Wort $w$ aus der Sprache $L$ des Transitionssystems $S/X$ existiert
  wie oben auch ein Ablauf. Hier ist es jedoch wichtig, dass auch
  $\tau$-Transitionen gemacht werden, um dieses Wort auszuführen. In dem ein
  Teil dieser $\tau$-Transitionen durch Transitionen mit Elementen aus $X$
  ersetzt werden erhält man ein Trace $w'$ aus der Sprache $L(S)$.

  (ii)
  Es wird ein Trace $w'=a_1a_2\dots a_n\in \ET{}(S)$ gewählt. Dieses Trace
  muss nicht wie bei Punkt (i) einem Ablauf in $S$ entsprechen. Jedoch kann ein
  Präfix von $w'$ gefunden werden, das besondere Eigenschaften erfüllen soll
  und für das es einen Ablauf gibt. Hierfür muss jedoch unterschieden werden
  aus welchem Grund das $w'$ in $\ET{}(S)$ enthalten ist.
  \begin{itemize}
    \item Fall 1 ($w'\in\cont{}(\PrET{}(S))$): In diesem Fall gibt es einen
      Ablauf für ein Präfix dieses $w'$s, der zu einem Ablauf ergänzt werden
      kann, der zu einem Zustand aus $E$ führt. Der Ablauf hat also die Form $q_0
      \overset{a_1}{\Rightarrow} q_1 \overset{a_2}{\Rightarrow} \dots
      \overset{a_m}{\Rightarrow} q_m \overset{b_1}{\Rightarrow} q_{m+1}
      \overset{b_2}{\Rightarrow} \dots \overset{b_l}{\Rightarrow} q_{m+l}$ mit
      $m\leq n$, $l\geq 0$ und $\forall i\in \{1,\dots ,l\}: b_i\in O(S)$. Es
      gilt dann $a_1a_2\dots a_mb_1b_2\dots b_l\in \StET{}(S)$. Analog wie im
      Beweisteil zu (i) wird dieser Ablauf durch die Internalisierung
      reduziert. Somit ist die Projektion von $a_1a_2\!\dots\!
      a_mb_1b_2\!\dots\! b_l$
      auf die eingeschränkte Aktionenmenge auf jeden Fall in $\StET{}(S/X)
      \subseteq \ET{}(S/X)$ enthalten. Da $b_1b_2\dots b_l\in O(S)^*$, gilt
      nach der Projektion auch $(b_1b_2\dots b_l)|_{\Sigma\backslash X}\in
      O(S/X)^*$ und somit $\prune{}((a_1a_2\dots a_m)|_{\Sigma\backslash
      X})=\prune{}((a_1a_2\dots a_mb_1b_2\dots b_l)|_{\Sigma\backslash X})$. Da
      \ET{} eine Menge ist, die nach Definition unter \cont{}
      abgeschlossen ist, sind alle Verlängerungen von $(a_1a_2\dots
      a_m)|_{\Sigma\backslash X}$ ebenfalls in $\ET{}(S/X)$ enthalten.
      Es gilt also speziell auch $w:=w'|_{\Sigma\backslash X}\in
      \ET{}(S/X)$. Da alle Elemente aus $\ET{}(S/X)$ nur Aktionen aus
      $\Sigma\backslash X$ enthalten, ist ausgeschlossen, dass eine
      Fortsetzung mit Aktionen außerhalb dieser Menge möglich ist.
    \item Fall 2 ($w'\in\cont{}(\MIT{}(S))$): In diesem Fall ist bereits für
      ein Präfix von $w'$ ein Ablauf zu einem Zustand möglich, der nicht für
      alle Inputs eine Transitionsmöglichkeit bietet. Der Ablauf hat also die
      Form $q_0 \overset{a_1}{\Rightarrow} q_1 \overset{a_2}{\Rightarrow} \dots
      \overset{a_{m-1}}{\Rightarrow} q_{m-1}
      \overset{a_m}{\not{\hspace{-0.1cm}\rightarrow}}$ mit $m\leq n$ und
      $a_1a_2\dots a_m\in \MIT{}(S)$. Analog zum letzten Fall und zum Teil (i)
      wird dieser Ablauf durch die Internalisierung reduziert. Somit ist die
      Projektion von $a_1a_2\dots a_m$ auf die eingeschränkte Aktionenmenge auf
      jeden Fall in $\MIT{}(S/X)\subseteq \ET{}(S/X)$ enthalten. Da \ET{} unter
      \cont{} abgeschlossen ist, sind alle Verlängerungen von $(a_1a_2\dots
      a_m)|_{\Sigma \backslash X}$ ebenfalls in $\ET{}(S/X)$ enthalten.
      Speziell gilt also auch $w:=w'|_{\Sigma\backslash X}\in \ET{}(S/X)$.
  \end{itemize}
  Ein Trace $w\in \ET{}(S/X)$ muss wie oben nicht im Transitionssystem
  enthalten sein. Es kann jedoch ein Präfix von $w$ wie oben gefunden werden,
  mit dem ein Error-Zustand über eine Verlängerung oder ein Zustand, für den
  nicht alle Input-Transitionen möglich sind, erreicht werden. Für dieses
  Präfix von $w$ bzw.\ für das Präfix von $w$ mit der entsprechenden
  Verlängerung existiert ein Ablauf im Transitionssystem $S/X$, der wie in (i)
  beschrieben, durch das Ersetzen von $\tau$-Transitionen durch Outputs aus der
  Menge $X$ auf einen Ablauf in
  $S$ erweitert werden kann. Dieser Ablauf kann dann analog zu den beiden
  Fällen oben verkürzt und aufgrund des Abschlusses gegenüber \cont{}
  verlängert werden, so dass ein $w'\in \ET{}(S)$ entsteht, das sich von $w$
  nur durch hinzugefügte Aktionen aus $X$ unterscheidet.

  (iii)
  Für ein Trace $w'=a_1a_2\dots a_n\in \EL{}(S)$ gilt $w'\in L(S)$ oder $w'\in
  \ET{}(S)$. Für beide Fälle wurde oben bereits gezeigt, dass dann
  $w:=w'|_{\Sigma\backslash X}$ in der entsprechenden Menge des
  Transitionssystems $S/X$ enthalten ist. Da \EL{} als Vereinigung aus den
  Mengen $L$ und \ET{} definiert wurde, ist dadurch auch gezeigt, dass
  $w\in \EL{}(S/X)$ gilt.\\
  Ebenfalls analog zu den beiden vorangegangenen Punkten kann auch argumentiert
  werden, dass zu jedem $w\in \EL{}(S/X)$ durch Hinzufügen von Aktionen aus $X$
  ein $w'\in \EL{}(S)$ gefunden werden kann.

  Da $S_1\ERel S_2$ gilt, weiß man, dass $\ET{}_1\subseteq \ET{}_2$ und
  $\EL{}_1\subseteq \EL{}_2$ gilt. Durch die Aussagen (i) bis (iii) kann draus
  direkt gefolgert werden, dass auch $\ET{}(S_1/X)\subseteq \ET{}(S_2/X)$ und
  $\EL{}(S_1/X)\subseteq \EL{}(S_2/X)$ gilt, da zu jedem Trace aus $\ET(S)$
  bzw.\ $\EL{}(S)$ ein entsprechendes Trace aus $\ET(S/X)$ bzw.\ $\EL{}(S/X)$
  gefunden werden kann und umgekehrt.\\
  Es folgt also insgesamt, dass die Relation \ERel{} trotz Hiding erhalten
  bleibt und somit diese Relation bezüglich des Internalisierungsoperator eine
  Präkongruenz darstellt.
\end{proof}

Aus Korollar~\ref{propPraekongruenz} ist bekannt, dass \ERel{} eine Präkongruenz
bezüglich $\cdot\|\cdot$ ist, und aus Satz~\ref{satzPraeInternalisierung}, dass
\ERel{} auch eine Präkongruenz bezüglich $\cdot/\cdot$ ist. Da sich nach
Definition~\ref{defIntParal} die Parallelkomposition mit Internalisierung nur
aus diesen Operatoren zusammensetzt, erhält man das folgende Korollar.

\begin{kor}[Error-Präkongurenz mit Internalisierung]
  Die Relation \ERel{} ist eine Präkongruenz bezüglich $\cdot|\cdot$.
\end{kor}
