\chapter{Verfeinerung über Errortraces}

\section{Präkongruenz für lokale Errors}
In diesem Kapitel wählen wir einen optimistischen Ansatz für die Fehler
Erreichbarkeit. Ein Error gilt hier somit als erreichbar, wenn er lokal
erreicht werden kann, d.h.\ durch lokale Aktionen. Die Menge bestehend aus der
internen Aktion $\tau$ und den Outputaktionen wir hier als lokale Aktionen
bezeichnet. Alle Elemente aus dieser Menge können ausgeführt werden ohne
weiteres Zutun von außen, somit kann auch nicht beeinflusst werden ob diese
Übergänge genommen werden oder nicht. Es besteht also die Möglichkeit, dass
der EIO einen Error-Zustand erreicht, sobald dieser lokal erreichbar ist. Diese
Art der Erreichbarkeit von Fehler wird in Kapitel 3 von~\cite{Vogler2014EIO}
dargestellt.

\begin{Def}[lokal Errorfreie Kommunikation]
  Ein Error ist lokal erreichbar in einem EIO $S$, wenn $\exists w\in O^*: q_0
  \overset{w}{\Rightarrow} q\in E$.\\
  Zwei EIOs $S_1, S_2$ kommunizieren gut, wenn keine lokalen Errors erreicht
  werden können in $S_1\| S_2$.
\end{Def}

Über der lokalen Erreichbarkeit von Fehler können wir uns eine
Verfeinerungsrelation definieren.

\begin{Def}[lokale Basisrelation]
  Für EIOs $S_1, S_2$ mit der gleichen Signatur schreiben wir $S_1\sqsubseteq
  _E^ B S_2$, wenn ein Error in $S_1$ nur dann lokal erreichbar ist, wenn er
  auch in $S_2$ lokal erreichbar ist.\\
  $\sqsubseteq _E^C$ bezeichnet die vollständig abstrakte Präkongruenz von
  $\sqsubseteq _E^B$ bezüglich $\|$.
\end{Def}

Um nun die gröbsten Präkongruenz finden zu können, müssen wir bestimmte Traces
aus unseren Automaten hervorheben. Die strikten Errortraces sind Wege, die
direkt vom Startzustand zu einem Zustand in $E$ führen. Da Outputs Aktionen
sind, die von außen nicht verhindert werden können, benötigen wir auch noch die
Menge der Traces, die zu einem Zustand führen, von dem aus mit lokalen Aktionen
ein Error erreicht werden kann. Zusätzlich ich auch noch die Menge der Traces
interessant, denen am Ende ein möglicher Input fehlt. Diese führen zwar nicht
direkt zu einem Fehler, jedoch in Komposition mit einem anderen Automaten sind
dies gefährdete Stellen für neue Errors, da für die Synchronisation dieser Input
fehlt.

\begin{Def}[Errortraces]
  Folgende Trace-Mengen sind definiert für einen EIO $S$:
  \begin{itemize}
    \item strikte Errortraces: $StT(S)=\{w\in\Sigma
      ^*\mid q_0\overset{w}{\Rightarrow}\in E\}$
    \item gekürzte Errortraces: $PrT(S)=\{prune(w)\mid w\in StT(S)\}$
    \item fehlende Input-Traces: $MIT(S)=\{wa\in\Sigma ^*\mid 
      q_0\overset{w}{\Rightarrow}\wedge a\in I\wedge
    q\overset{a}{\not{\hspace{-0.1cm}\rightarrow}}\}$
  \end{itemize}
\end{Def}

\begin{Def}[Lokale Error Semantik]
  \label{DefETEL}
  Sei $S$ ein EIO.
  \begin{itemize}
    \item Die Menge der Errortraces von $S$ ist $ET(S)=cont(Prt(S))\cup
      cont(MIT(S))$.
    \item Die geflutete Sprache von $S$ ist $EL(S)=L(S)\cup ET(S)$.
  \end{itemize}
  Für zwei EIOs $S_1, S_2$ mit der gleichen Signatur schreiben wir
  $S_1\sqsubseteq _E S_2$, wenn $ET(S_1)\subseteq ET(S_2)$ und
  $EL(S_1)\subseteq EL(S_2)$ gilt.
\end{Def}

\begin{satz}[Lokale Error Semanik für Parallelkomposition]
  \label{satzErrorSemanik}
  Für zwei komponierbare EIOs $S_1, S_2$ und $S_{12}=S_1\|S_2$, gilt:
  \begin{enumerate}
    \item $ET_{12}=cont(prune((ET_1\|EL_2)\cup(EL_1\|ET_2)))$
    \item $EL_{12}=(EL_1\|EL_2)\cup ET_{12}$
  \end{enumerate}
\end{satz}

\begin{proof}
  ~
  \begin{enumerate}
    \item ``$\subseteq$'':
  \end{enumerate}
  Da beide Seiten der Gleichung unter $cont$ abgeschlossen sind, genügt es ein
  Präfix-minimales Element $w$ von $ET_{12}$ zu betrachten. Dieses Element ist
  nach der Definition der Menge der Errortraces entweder in $MIT_{12}$ oder in
  $PrT_{12}$ enthalten.\\
  \begin{itemize}
    \item Fall 1 ($w\in MIT_{12}$): Aus der Definition von $MIT$ folgt, dass es eine
  Aufteilung $w=xa$ gibt mit
  $(q_{01},q_{02})\overset{x}{\Rightarrow}(q_1,q_2)\wedge a\in I_{12}\wedge
  (q_1,q_2)\overset{a}{\not{\hspace{-0.1cm}\rightarrow}}$. Da
  $I_{12}\overset{Def}{=}(I_1\backslash O_2)\cup (I_2\backslash O_1)=(I_1\cup
  I_2)\backslash (O_1\cup O_2)$ ist $a\in (I_1\cup I_2)$ und $a\notin (O_1\cup
  O_2)$. Somit müssen wir unterscheiden, ob $a\in (I_1\cap I_2)$ oder $a\in
  (I_1\cup I_2)\backslash (I_1\cap I_2)$ liegt.
  \begin{itemize}
    \item Fall 1a) ($a\in (I_1\cap I_2)$): Nun können wir den Ablauf der Komposition
  auf die Automaten projizieren und erhalten dann:
  $q_{01}\overset{x_1}{\Rightarrow} q_1
  \overset{a}{\not{\hspace{-0.1cm}\rightarrow}}$ und
  $q_{02}\overset{x_2}{\Rightarrow} q_2
  \overset{a}{\not{\hspace{-0.1cm}\rightarrow}}$ mit $x\in x_1\|x_2$, da sobald einer der Automaten
  einen Übergang für $a$ machen könnte, wäre dies auch für die Komposition der
  beiden möglich. Daraus können wir folgern $x_1a\in cont(MIT_1)\subseteq
  ET_1\subseteq EL_1$ und $x_2\in cont(MIT_2)\subseteq ET_2\subseteq EL_2$.
  Damit folgt $w\in (x_1\|x_2)\cdot\{a\}\subseteq x_1a\|x_2a\subseteq
  ET_1\|ET_2\subseteq (ET_1\|EL_2)\cup(EL_1\|ET_2)$, und somit ist $w$ in der
  rechten Seite der Gleichung enthalten.
\item Fall 1b) ($a\in (I_1\cup I_2)\backslash(I_1\cap I_2)$): \oBdA{} gilt
      $a\in I_1$. Durch Projektion erhalten wir:
      $q_{01}\overset{x_1}{\Rightarrow} q_1
      \overset{a}{\not{\hspace{-0.1cm}\rightarrow}}$ und
      $q_{02}\overset{x_2}{\Rightarrow} q_2$ mit $x\in x_1\|x_2$. Daraus folgt
      $x_1a\in cont(MIT_1)\subseteq ET_1$ und $x_2\in L_2\subseteq EL_2$. Somit
      gilt $w\in (x_1\| x_2)\cdot\{a\}\subseteq x_1a\|x_2\subseteq ET_1\|EL_2$,
      was eine Teilmenge der rechten Seite der Gleichung ist.
  \end{itemize}
    \item Fall 2 ($w\in PrT_{12}$): Durch die Definition von $PrT$ und $prune$
      wissen wir, dass es ein $v\in O_{12}^*$ gibt, so dass $(q_{01},q_{02})
      \overset{w}{\Rightarrow} (q_1,q_2) \overset{v}{\Rightarrow} (q_1',q_2')$
      gilt mit $(q_1',q_2')\in E_{12}$ und $w=prune(wv)$. Durch Projektion
      erhalten wir $q_{01} \overset{w_1}{\Rightarrow} q_1
      \overset{v_1}{\Rightarrow} q_1'$ und $q_{02} \overset{w_2}{\Rightarrow}
      q_2 \overset{v_2}{\Rightarrow} q_2'$ mit $w\in w_1\|w_2$ und $v\in
      v_1\|v_2$. Aus $(q_1',q_2')\in E_{12}$ folgt, dass entweder einer der
      beiden Zustände bereits ein Error-Zustand gewesen sein muss und der Fehler
      somit geerbt ist oder das der Error durch die fehlende Möglichkeit
      entstanden ist eine synchronisierte Handlung auszuführen, da der Input
      nicht möglich war und es sich somit um einen neuen Fehler handelt.
      \begin{itemize}
        \item Fall 2a) (geerbter Error): \oBdA{} $q_1'\in E_1$. Daraus folgt
          $w_1v_1\in StT_1\subseteq cont(PrT_1)\subseteq ET_1$. Da gilt
          $q_{02}\overset{w_2v_2}{\Rightarrow}$, erhalten wir $w_2v_2\in
          L_2\subseteq EL_2$. Dadurch ergibt sich $wv\in ET_1\|EL_2$ mit
          $w=prune(wu)$ und somit ist $w$ in der rechten Seite der Gleichung
          enthalten.
        \item Fall 2b) (neuer Error): \oBdA{} $a\in I_1\cap O_2$ mit
          $q_1'\overset{a}{\not{\hspace{-0.1cm}\rightarrow}} \wedge q_2'
          \overset{a}{\rightarrow}$. Daraus folgt $w_1v_1a\in MIT_1\subseteq
          ET_1$ und $w_2v_2a\in L_2\subseteq EL_2$. Damit ergibt sich $wva\in
          ET_1\|EL_2$, da $a\in O_2\subseteq O_{12}$ gilt $w=prune(wva)$ und
          somit ist $w$ in der rechten Seite der Gleichung enthalten.
          %letztes Argument???
      \end{itemize}
  \end{itemize}

  1. ``$\supseteq$'':\\
  Wegen der Abgeschlossenheit beider Seiten der Gleichung gegenüber $cont$
  betrachten wir auch in diesem Fall nur ein Präfix-minimales Element $x\in
  prune((ET_1\|EL_2)\cup (EL_1\|ET_2))$. Da $x$ durch die Anwenung der
  $prune$-Funktion entstanden ist existiert ein $y\in O_{12}$ mit
  $xy\in(ET_1\|EL_2)\cup (EL_1\|ET_2)$. \oBdA{} gehen wir davon aus, dass
  $xy\in ET_1\|EL_2$ gilt, d.h.\ es gibt $w_1\in ET_1$ und $w_2\in EL_2$ mit
  $xy\in w_1\|w_2$.\\
  Weiterführend werden wir für alle Fälle von $xy$ zeigen, dass es ein $v\in
  PrT(S_1\|S_2)\cup MIT(S_1\|S_2)$ gibt, das ein Präfix von $xy$ ist und $v$
  entweder auf ein Input aus $I_{12}$ endet oder $v = \varepsilon$. Da $v$
  entweder leer ist oder auf einen Input endet, muss $v$ ein Präfix von $x$
  sein. $\varepsilon$ ist Präfix von jedem Wort und sobald $x$ mindestens einen
  Buchstaben enthält, kann $y$ durch die Definition von $prune$ nur aus Outputs
  besteht und somit muss dass Ende von $v$ vor dem Anfang von $y$ liegen. $x$
  hat dadurch ein Präfix in $PrT(S_1\|S_2)\cup MIT(S_1\|S_2)$, dann ist $x$ in
  der Fortsetzung dieser Menge enthalten und somit gilt $x\in ET_{12}$.\\
  Sei $v_1$ das kürzestes Präfix von $w_1$ in $PrT_1\cup MIT_1$. Falls
  $w_2\in L_2$, so sei $v_2=w_2$, sonst soll $v_2$ das kürzestes Präfix von
  $w_2$ in $PrT_2\cup MIT_2$ sein. Jede Aktion in $v_1$ und $v_2$ hängt mit
  einer aus $xy$ zusammen. Wir gehen nun davon aus, dass entweder
  $v_2=w_2\in L_2$ gilt oder die letzte Aktion von $v_1$ findet vor oder
  gleichzeitig mit der letzten Aktion von $v_2$ statt. Ansonsten endet
  $v_2\in PrT_2\cup MIT_2$ vor $v_1$ und somit ist dieser Fall analog zu $v_1$
  endet vor $v_2$.
  \begin{itemize}
    \item Fall 1 ($v_1=\varepsilon$): Dadurch, dass $\varepsilon\in PrT_1\cup
      MIT_1$, ist bereits in $S_1$ ein Error lokal erreichbar. Wir wähle
      $v_2'=v'=\varepsilon$, somit ist $v_2'$ ein Präfix von $v_2$.
    \item Fall 2 ($v_1\neq\varepsilon$): Aufgrund der Definitionen von $PrT$
      und $MIT$ endet $v_1$ auf ein $a\in I_1$, d.h.\ $v_1=v_1’a$. $v'$ sei das
      Präfix von $xy$, das mit der letzten Aktion von $v_1$ endet, d.h.\ mit
      $a$, und $v_2'=v'|_{\Sigma _{2}}$. Falls $v_2\in L_2$, dann ist
      $v_2'$ ein Präfix von $v_2$, da kein Fehler möglich ist in der
      Parallelkomposition und somit maximal das gesamte Wort $v_2$ bereits in
      $v'$ enthalten sein kann durch Projektion auf $\Sigma _2$. Falls $v_2\in
      PrT_2\cup MIT_2$ gilt, dann ist durch die Annahme, dass $v_2$ nicht vor
      $v_1$ endet, $v_2'$ ein Präfix von $v_2$. Im Fall $v_2\in MIT_2$ können
      wir sogar schließen, dass $v_2'$ ein echtes Präfix von $v_2$ ist, da
      $v_2$ auf $b\in I_2$ endet und somit der letzte Input, $b$, noch nicht der
      fehlende sein kann.
  \end{itemize}
  In allen Fällen erhalten wir $q_{02}\overset{v_2'}{\Rightarrow}$ (*) und
  desweiteren ist $v_2'=v'|_{\Sigma _2}$ ein Präfix von $v_2$ und $v'\in v_1\|
  v_2'$ ist ein Präfix von $xy$.
  \begin{itemize}
    \item Fall 1 ($v_1\in MIT_1$ und $v_1\neq\varepsilon$): Es gibt $q_{01}
      \overset{v_1'}{\Rightarrow}q_1
      \overset{a}{\not{\hspace{-0.1cm}\rightarrow}}$ und sei $v'=v''a$.
      \begin{itemize}
        \item Fall 1a) ($a\notin Synch(S_1,S_2)$): Es folgt $a\notin
          \Sigma _2$ und durch (*) folgt $q_{02} \overset{v_2'}{\Rightarrow}
          q_2$ mit $v''\in v_1'\|v_2'$. Dadurch erhalten
          wir $(q_{01},q_{02}) \overset{v''}{\Rightarrow} (q_1,q_2)
          \overset{a}{\not{\hspace{-0.1cm}\rightarrow}}$ mit $a\in I_{12}$.
          Somit können wir wählen $v:=v''a=v'\in MIT(S_1\|S_2)$.
        \item Fall 1b) ($a\in\Sigma _2$): Es folgt $a\in O_2$ und
          $v_2'=v_2''a$. Durch (*) erhalten wir $q_{02}
          \overset{v_2''}{\Rightarrow} q_2 \overset{a}{\rightarrow}$ mit
          $v''\in v_1'\|v_2'$. Daraus ergibt sich $(q_{01},q_{02})
          \overset{v''}{\Rightarrow} (q_1,q_2)$ mit $q_1
          \overset{a}{\not{\hspace{-0.1cm}\rightarrow}},a\in I_1, q_2
          \overset{a}{\rightarrow},a\in O_2$, somit gilt $(q_1,q_2)\in
          E_{12}$. Wir wählen $v:=prune(v'')\in PrR(S_1\|S_2)$.
      \end{itemize}
  \item Fall 2 ($v_1\in PrT_1$): $\exists u_1\in O_1^*:q_{01}
    \overset{v_1}{\Rightarrow} q_1 \overset{u_1}{\Rightarrow} q_1'$ mit
    $q_1'\in E_1$. Es gilt $q_{02} \overset{v_2'}{\Rightarrow}q_2$ mit
    $(q_{01},q_{02}) \overset{v'}{\Rightarrow}(q_1,q_2)$.
    \begin{itemize}
      \item Fall 2a) ($u_2\in (O_1\cap I_2)*, c\in (O_1\cap I_2)$, sodass
        $u_2c$ Präfix von $u_1|_{I_2}$ mit $q_2 \overset{u_2}{\Rightarrow} q_2'
        \overset{c}{\not{\hspace{-0.1cm}\rightarrow}}$): Für das Präfix $u_1'c$
        von $u_1$ mit $u_1'c|_{I_2}=u_2c$ wissen wir, dass $q_1
        \overset{u_1'}{\Rightarrow} q_1'' \overset{c}{\rightarrow}$. Somit gilt
        $u_1'\in u_1'\|u_2$ und $(q_1,q_2) \overset{u_1'}{\Rightarrow}
        (q_1'',q_2')\in E_{12}$, da für $S_2$ der entsprechende Input fehlt,
        der mit dem $c$ Output von $S_1$ zu koppeln wäre, es haldelt sich also
        um einen neuen Error. Wir wäheln $v:=prune(v'u_1')\in PrT(S_1\| S_2)$,
        dies ist ein Präfix von $v'$, da $u_1\in O_1^*$.
      \item Fall 2b) ($q_{02} \overset{u_2}{\Rightarrow} q_2'$ mit
        $u_2=u_1|_{I_2}$): Somit ist $u_1\in u_1\|u_2$ und $(q_1,q_2)
        \overset{u_1}{\Rightarrow} (q_1',q_2')\in E_{12}$, da $q_1'\in E_1$ und
        somit handlt es sich um einen geerbten Error. Wir wählen nun $v:=prune
        (v'u_1)\in PrT(S_1\|S_2)$, dass wiederum ein Präfix von $v'$ ist.
    \end{itemize}
  \end{itemize}

  2.:\\
  Es ist durch die Definition klar, dass gilt $L_i\subseteq EL_i$ und
  $ET_i\subseteq EL_i$. Wir beginnen mit der Argumentation von der rechten
  Seite der Gleichung aus:
  \begin{align*}
    &(EL_1\| EL_2)\cup ET_{12}\overset{\ref{DefETEL}}{=}\\
    &(L_1\cup ET_1)\|(L_2\cup ET_2)\cup ET_{12}=\\
    &\underset{\overset{1.}{\subseteq} ET_{12}}{\underset{\subseteq
    (EL_1\|ET_2)}{\underbrace{(L_1\|ET_2)}}} \cup
    \underset{\overset{1.}{\subseteq} ET_{12}}{\underset{\subseteq
    (ET_1\|EL_2)}{\underbrace{(ET_1\|L_2)}}} \cup
    (L_1\|L_2) \cup \underset{\overset{1.}{\subseteq}
    ET_{12}}{\underset{\subseteq (EL_1\|ET_2)}{\underbrace{(ET_1\|ET_2)}}} \cup
    ET_{12}=\\
    &(L_1\|L_2) \cup ET_{12}\overset{\ref{LemmaSprache}}{=}\\
    &L_{12}\cup ET_{12}\overset{\ref{DefETEL}}{=}\\
    &EL_{12}
  \end{align*}
  %TODO
\end{proof}

In~\cite{Vogler2014EIO} wurde auch die Verfeinerung von EIOs als Relation betrachtet
mit Spezifikation und Implementierung. Hier soll ebenfalls eine
Verfeinerungsrelation über EIOs betrachtet werden, jedoch im Zusammenhang mit
den Definition von oben und dem Satz~\ref{satzErrorSemanik}.

\begin{lem}[Verfeinerung mit Errors]
  Gegeben sind zwei EIOs $S_1, S_2$ mit der gleichen Signatur. Wenn
  für alle EIO $U$ für die gilt, $S_2$ und $U$ kommunizieren gut, folgt $S_1$
  und $U$ kommunizieren gut, dann verfeinert $S_1$ $S_2$. Diese Verfeinerung
  entspricht der Relation $\sqsubseteq _E$ von oben, die hier dann wie folgt
  definiert ist: wenn $U\|S_1 \sqsubseteq _E^B U\|S_2$ für alle $U$, dann gilt
  $S_1\sqsubseteq _E S_2$.
\end{lem}

\begin{proof}
  Da $S_1$ und $S_2$ die gleichen Signaturen haben, definieren wir:
  $I:=I_1=I_2$ und $O:=O_1=O_2$. Für jeden der Partner $U$ gilt $I_U=O$ und
  $O_U=I$.\\
  Um zu zeigen, dass $S_1\sqsubseteq S_2$ müssen wir zeigen, dass gilt:
  \begin{itemize}
    \item $ET(S_1)\subseteq ET(S_2)$
    \item $EL(S_1)\subseteq EL(S_2)$
  \end{itemize}
  Wir wählen wein Präfix-minimales Element $w\in ET(S_1)$ und müssen dann
  zeigen, dass dieses $w$ oder eines seiner Präfixe in $ET(S_2)$ enthalten um
  die erste Inklusion zu zeigen.
  \begin{itemize}
    \item Fall 1 ($w=\varepsilon$): Es ist ein Error lokal erreichbar in $S_1$.
      Wir nehmen für $U$ einen Automaten, der nur aus dem Startzustand und
      einer Schleife für alle $x\in I_U$ besteht. Somit kann $S_1$ den gleichen
      Error-Zustände lokal erreichen wie $U\|S_1$. Daraus folgt, dass auch
      $U\|S_2$ einen lokal erreichbaren Error-Zustand haben muss. Durch unsere
      Definition von $U$ kann dieser Fehler nur geerbten werden von $S_2$. Es
      muss also in $S_2$ ein Error-Zustand durch interne Handlungen und Outputs
      erreichbar sein, d.h. $\varepsilon\in PrT(S_2)$.
    \item Fall 2 ($w=x_1\dots x_n x_{n+1})\in\Sigma ^+$ mit $n\geq 0$ und
      $x_{n+1}\in I$): Wir betrachten die folgenden Partner $U$, siehe auch
      Abbildung~\ref{UohneE}:
      \begin{itemize}
        \item $Q_U=\{q_0,q_1,\dots ,q_{n+1}\}$
        \item $q_{0U}=q_0$
        \item $E_U=\emptyset$
        \item $\begin{aligned}[t]
            \delta _U=&\{(q_i,x_{i+1},q_{i+1})\mid  0\leq i\leq n\}\cup\\
                      &\{(q_i,x,q_{n+1})\mid  x\in I_U\backslash\{x_{i+1}\},
          0\leq i\leq n\}\cup\\
          &\{(q_{n+1},x,q_{n+1})\mid  x\in I_U\}
        \end{aligned}$
      \end{itemize}
      \begin{figure} [h!tbp]
      \begin{center}
        \begin{tikzpicture}[->, >=latex',auto,node distance =3cm, semithick]

          \node (0) {$q_0$};
          \node (1) [right of=0] {$q_1$};
          \node (dots) [right of=1] {$\dots$};
          \node (n) [right of=dots] {$q_n$};
          \node (n1) at ($(1)!0.5!(dots) + (0,-3)$) {$q_{n+1}$};

          \path ($ (0) + (-1,0) $) edge (0)
                (0) edge node {$x_1$} (1)
                    edge [bend right] node [below, sloped] {$x?\neq x_1$} (n1)
                (1) edge node {$x_2$} (dots)
                    edge node [below, sloped] {$x?\neq x_2$} (n1)
                (dots) edge node {$x_n$} (n)
                       edge [dashed] (n1)
                (n) edge node [above, sloped] {$x?\in I_U$} (n1)
                    edge [bend left] node [sloped] {$x_{n+1}$!} (n1)
                (n1) edge [loop below] node {$x?\in I_U$} (n1);
        \end{tikzpicture}
        \caption{$x?\neq x_i$ steht für alle $x\in I_U\backslash\{x_i\}$}
        \label{UohneE}
      \end{center}
      \end{figure}
      Wir können für $w$ zwei Fälle unterscheiden. Beide führen zu
      $\varepsilon\in PrT(U\|S_1)$.
      \begin{itemize}
        \item Fall 2a) ($w\in MIT(S_1)$): In $U\|S_1$ erhalten wir
          $(q_0,q_{01}) \overset{x_1\dots x_n}{\Rightarrow} (q_n,q')$ mit
          $q' \overset{x_{n+1}}{\not{\hspace{-0.1cm}\rightarrow}}$ und $q_n
          \overset{x_{n+1}}{\rightarrow}$. Deshalb gilt $(q_n,q')\in
          E_{U\|S_1}$ und $x_1\dots x_n\in StT(U\|S_1)$. Da alle Aktionen aus
          $w$ bis auf $x_{n+1}$ synchronistiert werden gilt $x_1,\dots ,x_n\in
          O_{U\|S_1}$. Daraus ergibt sich dann $\varepsilon\in PrT(U\|S_1)$.
        \item Fall 2b) ($w\in PrT(S_1)$): In $U\|S_1$ erhalten wir
          $(q_0,q_{01}) \overset{w}{\Rightarrow} (q_{n+1},q'')
          \overset{u}{\Rightarrow} (q_{n+1},q')$ für $u\in O^*$ und $q'\in
          E_1$. Daraus folgt $(q_{n+1},q')\in E_{U\|S_1}$ und somit $wu\in
          StT(U\|S_1)$. Da alle Handlungen aus $w$ synchronisiert werden gilt
          $x_1,\dots ,x_n,x_{n+1}\in O_{U\|S_1}$ und da gilt $u\in O^*$ folgt
          daraus $u\in O_{U\|S_1}^*$. Somit ergibt sich $\varepsilon\in
          PrT(U\|S_1)$.
      \end{itemize}
      Da wir wissen, dass $\varepsilon\in StT(U\|S_1)$ gilt, können wir durch
      $U\|S_1\sqsubseteq _E^B U\|S_2$ schießen, dass auch in $U\|S_2$ ein Error
      lokal erreichbar sein muss.\\
      Dieser Error kann geerbt oder neu sein.\\
      \begin{itemize}
        \item Fall 2i) (neuer Error): Da jeder Zustand von $U$ alle Inputs $x\in
          O=I_U$ zulässt, muss ein lokal erreichbarer Error einer sein, bei dem
          ein Output $a\in O_U$ von $U$ möglich ist, der nicht mit einem
          passenden Input aus $S_2$ synchonisiert werden kann. Durch die
          Konstuktion von $U$ sind in $q_{n+1}$ keine Outputs möglich. Ein
          neuer Error muss also die Form $(q_i,q')$ haben mit $i\leq n, q'
          \overset{x_{i+1}}{\not{\hspace{-0.1cm}\rightarrow}}$ und $x_{i+1}\in
          O_U=I$. Durch Projektion erhalten wir dann $q_{02} \overset{x_1\dots
          x_i}{\Rightarrow} q'
          \overset{x_{i+1}}{\not{\hspace{-0.1cm}\rightarrow}}$ und damit gilt
          $x_1\dots x_{i+1}\in MIT(S_2)\subseteq ET(S_2)$. Somit ist ein Präfix
          von $w$ in $ET(S_2)$ enthalten.
        \item Fall 2ii) (geerbter Error): $U$ hat $x_1\dots x_i u$ ausgeführt
          mit $u\in I_U^*=O^*$ und ebenso hat $S_2$ diesen Weg ausgeführt.
          Durch dies hat $S_2$ einen Zustand in $E_2$ erreicht, da von $U$
          keine Fehler geerbt werdne können. Es gilt dann $prune(x_1\dots
          x_iu)=prune(x_1\dots x_i)\in PrT(S_2)\subseteq ET(S_2)$. Da $x_1\dots
          x_i$ ein Präfix von $w$ ist, für auch in diesem Fall ein Präfix von
          $w$ zu einem Error.
      \end{itemize}
  \end{itemize}
  Um uns von der zweiten Inklusion zu überzeugen reicht es zu zeigen, dass
  $L(S_1)\backslash ET(S_1)\subseteq EL(S_2)$ gilt, da wir die erste Inklusion
  bereits bewiesen haben und wegen der Definition von $EL$.\\
  Wir nehmen uns dafür ein beliebieges $w\in L(S_1)\backslash ET(S_1)$ und
  zeigen, dass es in $EL(S_2)$ enthalten ist.
  \begin{itemize}
    \item Fall 1 ($w=\varepsilon$): Da $\varepsilon$ immer in $EL(S_2)$
      enthalten ist, haben wir hier nicht zu zeigen.
    \item Fall 2 ($w=x_1\dots x_n$ mit $n\geq 1$): Wir konstuieren einen
      Partner $U$ wie folgt, siehe dazu auch Abbildung~\ref{UmitE}:
      \begin{itemize}
        \item $Q_U=\{q,q_0,q_1,\dots ,q_n\}$
        \item $q_{0U}=q_0$
        \item $E_U={q_n}$
        \item $\begin{aligned}[t]
            \delta _U=&\{(q_i,x_{i+1},q_{i+1})\mid 0\leq i< n\}\cup\\
                      &\{(q_i,x,q)\mid x\in I_U\backslash\{x_{i+1}\},0\leq
          i\leq n\}\cup\\
          &\{(q,x,q)\mid x\in I_U\}
              \end{aligned}$
      \end{itemize}
      \begin{figure} [h!tbp]
      \begin{center}
        \begin{tikzpicture}[->, >=latex',auto,node distance =3cm, semithick]

          \node (0) {$q_0$};
          \node (1) [right of=0] {$q_1$};
          \node (dots) [right of=1] {$\dots$};
          \node (n1) [right of=dots] {$q_{n-1}$};
          \node (n) [right of=n1, rectangle, draw] {$q_n\in E_U$};
          \node (q) at ($(1)!0.5!(dots) + (0,-3)$) {$q$};

          \path ($ (0) + (-1,0) $) edge (0)
                (0) edge node {$x_1$} (1)
                    edge [bend right] node [below, sloped] {$x?\neq x_1$} (q)
                (1) edge node {$x_2$} (dots)
                    edge node [below, sloped] {$x?\neq x_2$} (q)
                (dots) edge node {$x_{n-1}$} (n1)
                       edge [dashed] (q)
                (n1) edge node {$x_n$} (n)
                     edge node [sloped] {$x?\neq x_n$} (q)
                (q) edge [loop below] node {$x?\in I_U$} (q);
        \end{tikzpicture}
        \caption{$x?\neq x_i$ steht für alle $x\in I_U\backslash\{x_i\}$, $q_n$
          ist der einzige Error-Zustand}
        \label{UmitE}
      \end{center}
      \end{figure}
      Da $q_{01} \overset{w}{\Rightarrow} q$ gilt, wissen wir, dass $U|S_1$
      einen lokal erreichbaren Error hat. Somit muss $U|S_2$ ebenfalls einen
      lokal erreichbaren Error haben.
      \begin{itemize}
        \item Fall 2a) (neuer Error aufgrund von $x_i\in O_U$ und $q_{02}
          \overset{x_1\dots x_{i-1}}{\Rightarrow} q'
          \overset{x_i}{\not{\hspace{-0.1cm}\rightarrow}}$): Es gilt $x_1\dots
          x_i\in MIT(S_2)$ und somit $w\in EL(S_2)$. Anzumerken ist, dass nur
          auf diesem Weg Outputs von $U$ möglich sind, deshalb gibt es keine
          anderen Outputs von $U$, die zu einem neuen Fehler führen können.
        \item Fall 2b) (neuer Error aufgrund von $a\in O_2$): Der einzige
          Zustand, in dem $U$ nicht alle Inputs erlaubt sind ist $q_n$, der
          bereits ein Error-Zustand ist. Falls dieser Zustand erreichbar ist in
          $U\|S_2$, dann besitzt der komponierte EIO einen geerbten Error und
          es gilt $w\in L(S_2)\subseteq EL(S_2)$.
        \item Fall 2c) (geerbter Error von $U$): Da der einzige Zustand aus
          $E_U$ $q_n$ ist und alle Aktionen synchonisiert sind, ist dies nur
          möglich, wenn gilt $q_{02} \overset{x_1\dots x_n}{\Rightarrow}$. In
          diesem Fall gilt wie im letzten $w\in L(S_2)\subseteq EL(S_2)$.
        \item Fall 2d) (geerbter Error von $S_2$): Es gilt dann $q_{02}
          \overset{x_1\dots x_iu}{\Rightarrow} q'\in E_2$ für $i\geq 0$ und
          $u\in O^*$. Somit ist $x_1\dots x_iu\in StT(S_2)$ und damit
          $prune(x_1\dots x_iu)=prune(x_1\dots x_i)\in PrT(S_2)\subseteq
          EL(S_2)$. Damit gilt $w\in EL(S_2)$.
      \end{itemize}
  \end{itemize}
\end{proof}

\begin{satz}[Full Abstractness für lokale Error Semanik]
  Seien $S_1, S_2$ zwei EIOs mit der selben Signatur. Dann gilt $S_1\sqsubseteq
  _E^C S_2\Leftrightarrow S_1\sqsubseteq _E S_2$, insbesondere $\sqsubseteq _E$
  ist eine Präkongruenz.
\end{satz}

\begin{proof}
  %TODO
\end{proof}

