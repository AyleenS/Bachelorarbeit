\chapter{Verfeinerung über Errortraces}

\section{Präkongruenz für Error}

Da es in dieser Arbeit vor allem um die Erreichbarkeit und die Kommunikation
zwischen \EIO{}s geht, wurden die nächsten beiden Definitionen explizit getrennt
und erweitert zu denen in~\cite{Vogler2014EIO}. Ebenfalls wurde die
Parallelkomposition geändert, wie in~\cite{Schlosser2012BA}.

\begin{Def}[error-freie Kommunikation]
  Ein Error ist \emph{lokal erreichbar} in einem \EIO{} $S$, wenn $\exists w\in O^*: q_0
  \overset{w}{\Rightarrow} q\in E$.\\
  Zwei \EIO{}s $S_1$ und $S_2$ \emph{kommunizieren error-frei}, wenn in ihrer
  Parallelkomposition $S_1\| S_2$ keine lokalen Errors erreicht werden können.
\end{Def}

Mittels der lokalen Erreichbarkeit von Errors können wir eine
Verfeinerungsrelation definieren. Zusätzlich definieren wir uns bereits die
gröbste Präkongruenz, die wir suchen wollen. Somit können wir überprüfen, ob
eine andere Präkongruenz, die wir finden auch die Eigenschaft erfüllt die
gröbste zu sein.

\begin{Def}[Error-Verfeinerungs-Basisrelation]
  Für \EIO{}s $S_1$ und $S_2$ mit der gleichen Signatur schreiben wir
  $S_1\EBRel S_2$, wenn ein Error in $S_1$ nur dann lokal erreichbar ist, wenn er
  auch in $S_2$ lokal erreichbar ist. Es handelt sich dabei um eine
  \emph{Basisrelation} für die \emph{Verfeinerung} im Bezug auf \emph{Errors}.\\
  \ECRel{} bezeichnet die \emph{vollständig abstrakte Präkongruenz} von \EBRel{}
  bezüglich $\cdot\|\cdot$, d.h.\ die gröbste Präkongruenz bezüglich
  $\cdot\|\cdot$ die in \EBRel{} enthalten ist.
\end{Def}

Um uns nun näher mit den Präkongruenzen auseinandersetzen zu können, müssen wir bestimmte Traces
aus unser Struktur hervorheben. Die strikten Errortraces entsprechen Wege, die
direkt vom Startzustand zu einem Zustand in der Menge $E$ führen. Da Outputs Aktionen
sind, die von außen nicht verhindert werden können, benötigen wir auch noch die
Menge der Traces, die zu einem Zustand führen, von dem aus mit lokalen Aktionen
ein Error erreicht werden kann. Zusätzlich ist auch noch die Menge der Traces
interessant, für die es einen Input $a\in I$ gibt, durch den sie möglicherweise nicht
fortgesetzt werden können. Diese führen zwar nicht
direkt zu einem Error, jedoch in Komposition mit einem anderen
Transitionssystem sind
dies gefährdete Stellen. Sie führen zu einem neuen Error, sobald dieser Input
für die Synchronisation fehlt.

\begin{Def}[Errortraces]
  \label{DefErrortraces}
  Für ein \EIO{} $S$ definieren wir:
  \begin{itemize}
    \item \emph{strikte Errortraces}: $\StET{}(S)=\{w\in\Sigma
      ^*\mid q_0\overset{w}{\Rightarrow}q\in E\}$,
    \item \emph{gekürzte Errortraces}: $\PrET{}(S)=\{\prune{}(w)\mid w\in \StET{}(S)\}$,
    \item \emph{fehlende Input-Traces}: $\MIT{}(S)=\{wa\in\Sigma ^*\mid
      q_0\overset{w}{\Rightarrow}q\wedge a\in I\wedge
    q\overset{a}{\not{\hspace{-0.1cm}\rightarrow}}\}$.
  \end{itemize}
\end{Def}

In der folgenden Definition halten wir fest, was wir als Errortraces auffassen.
Diese Menge ist dadurch, dass sie die fortgesetzten Traces aus $\PrET{}$ enthält
deutlich allgemeiner wie die Menge $\StET{}$. Zusätzlich definieren wir auch noch
die geflutete Sprache, in der wir die Informationen aus der Sprache und den
Errortraces vereinen und somit bei der Inklusion dann nicht mehr explizit
unterscheiden.

\begin{Def}[Error-Semantik]
  \label{DefETEL}
  Sei $S$ ein \EIO{}.
  \begin{itemize}
    \item Die Menge der \emph{Errortraces} von $S$ ist $\ET{}(S):=\cont{}(\PrET{}(S))\cup
      \cont{}(\MIT{}(S))$.
    \item Die \emph{error-geflutete Sprache} von $S$ ist $\EL{}(S):=L(S)\cup \ET{}(S)$.
  \end{itemize}
  Für zwei \EIO{}s $S_1, S_2$ mit der gleichen Signatur schreiben wir
  $S_1\ERel S_2$, wenn $\ET{}(S_1)\subseteq \ET{}(S_2)$ und
  $\EL{}(S_1)\subseteq \EL{}(S_2)$ gilt.
\end{Def}

Der folgende Satz wurde in~\cite{Vogler2014EIO} nur für die Parallelkomposition
mit verborgenen synchronisierten Aktionen formuliert, jedoch entspricht er dem
analogen Satz aus~\cite{Schlosser2012BA}. Da der Beweis jedoch ohne Beachtung
von~\cite{Schlosser2012BA} neu geführt wurde, werden hier eher auf die
Erwähnung der Unterschiede zu~\cite{Vogler2014EIO} wert legen.

\begin{satz}[Error-Semanik für Parallelkompositionen]
  \label{satzErrorSemanik}
  Für zwei komponierbare \EIO{}s $S_1, S_2$ und $S_{12}=S_1\|S_2$, gilt:
  \begin{enumerate}
    \item $\ET{}_{12}=\cont{}(\prune{}((\ET{}_1\|\EL{}_2)\cup(\EL{}_1\|\ET{}_2)))$
    \item $\EL{}_{12}=(\EL{}_1\|\EL{}_2)\cup \ET{}_{12}$
  \end{enumerate}
\end{satz}

\begin{proof}
  ~
  \begin{enumerate}
    \item \glqq $\subseteq$\grqq :
  \end{enumerate}
  \vspace{-0.3cm}
  Da beide Seiten der Gleichung unter der Fortsetzung $\cont{}$ abgeschlossen sind, genügt es ein
  präfix-minimales Element $w$ von $\ET{}_{12}$ zu betrachten. Dieses Element ist
  aufgrund der Definition der Menge der Errortraces entweder in $\MIT{}_{12}$ oder in
  $\PrET{}_{12}$ enthalten.\\
  \begin{itemize}
    \item Fall 1 ($w\in \MIT{}_{12}$): Aus der Definition von $MIT$ folgt, dass es eine
  Aufteilung $w=xa$ gibt mit $(q_{01},q_{02})
  \overset{x}{\Rightarrow}(q_1,q_2)\wedge a\in I_{12}\wedge (q_1,q_2)
  \overset{a}{\not{\hspace{-0.1cm}\rightarrow}}$. Da $I_{12}
  \overset{\ref{DefParallelkomposition}}{=}(I_1\backslash O_2)\cup
  (I_2\backslash O_1)=(I_1\cup I_2)\backslash (O_1\cup O_2)$ ist $a\in (I_1\cup
  I_2)$ und $a\notin (O_1\cup O_2)$. Wir unterscheiden, ob $a\in (I_1\cap I_2)$
  oder $a\in (I_1\cup I_2)\backslash (I_1\cap I_2)$ ist. Diese Unterscheidung
  ist in~\cite{Vogler2014EIO} nicht nötig, da dort $I_1\cap I_2=\emptyset$
  gilt, somit gibt es dort nur den Fall 1b).
  \begin{itemize}
    \item Fall 1a) ($a\in (I_1\cap I_2)$): Nun können wir den Ablauf der
      Komposition auf die Transitionssysteme projizieren und erhalten dann \oBdA{}
      $q_{01}\overset{x_1}{\Rightarrow} q_1
      \overset{a}{\not{\hspace{-0.1cm}\rightarrow}}$ und
      $q_{02}\overset{x_2}{\Rightarrow} q_2
      \overset{a}{\not{\hspace{-0.1cm}\rightarrow}}$ oder
      $q_{02}\overset{x_2}{\Rightarrow} q_2 \overset{a}{\rightarrow}$ mit $x\in
      x_1\|x_2$. Daraus können wir $x_1a\in \cont{}(\MIT{}_1)\subseteq \ET{}_1\subseteq
      \EL{}_1$ und $x_2a\in \EL{}_2$ ($x_2a\in \MIT{}_2$ oder $x_2a\in L_2$)
      folgern. Damit folgt $w\in (x_1\|x_2)\cdot\{a\}\subseteq
      (x_1a)\|(x_2a)\subseteq \ET{}_1\|\EL{}_2$, und somit ist $w$ in der
      rechten Seite der Gleichung enthalten.
  \item Fall 1b) ($a\in (I_1\cup I_2)\backslash(I_1\cap I_2)$): \OBdA{} gilt
      $a\in I_1$. Durch Projektion erhalten wir:
      $q_{01}\overset{x_1}{\Rightarrow} q_1
      \overset{a}{\not{\hspace{-0.1cm}\rightarrow}}$ und
      $q_{02}\overset{x_2}{\Rightarrow} q_2$ mit $x\in x_1\|x_2$. Daraus folgt
      $x_1a\in \cont{}(\MIT{}_1)\subseteq \ET{}_1$ und $x_2\in L_2\subseteq \EL{}_2$. Somit
      gilt $w\in (x_1\| x_2)\cdot\{a\}\subseteq (x_1a)\|x_2\subseteq \ET{}_1\|\EL{}_2$.
      Dies ist eine Teilmenge der rechten Seite der Gleichung.
  \end{itemize}
    \item Fall 2 ($w\in \PrET{}_{12}$): Durch die Definitionen von $\PrET{}$
      und $\prune{}$ wissen wir, dass es ein $v\in O_{12}^*$ gibt, so dass
      $(q_{01},q_{02}) \overset{w}{\Rightarrow} (q_1,q_2)
      \overset{v}{\Rightarrow} (q_1',q_2')$ gilt mit $(q_1',q_2')\in E_{12}$
      und $w=\prune{}(wv)$. Durch Projektion erhalten wir $q_{01}
      \overset{w_1}{\Rightarrow} q_1 \overset{v_1}{\Rightarrow} q_1'$ und
      $q_{02} \overset{w_2}{\Rightarrow} q_2 \overset{v_2}{\Rightarrow} q_2'$
      mit $w\in w_1\|w_2$ und $v\in v_1\|v_2$. Aus $(q_1',q_2')\in E_{12}$
      folgt, dass es sich entweder um einen geerbten oder einen neuen Error
      handelt. Bei einem geerbten wäre bereits einer der beiden Zustände ein
      Error-Zustand gewesen. Der neue Error hingegen wäre durch die fehlende
      Möglichkeit entstanden eine synchronisierte Aktion auszuführen.
      \begin{itemize}
        \item Fall 2a) (geerbter Error): \OBdA{} $q_1'\in E_1$. Daraus folgt
          $w_1v_1\in \StET{}_1\subseteq \cont{}(\PrET{}_1)\subseteq \ET{}_1$. Da gilt
          $q_{02}\overset{w_2v_2}{\Rightarrow}$, erhalten wir $w_2v_2\in
          L_2\subseteq \EL{}_2$. Dadurch ergibt sich $wv\in \ET{}_1\|\EL{}_2$ mit
          $w=\prune{}(wv)$ und somit ist $w$ in der rechten Seite der Gleichung
          enthalten.
        \item Fall 2b) (neuer Error): \OBdA{} $a\in I_1\cap O_2$ mit
          $q_1'\overset{a}{\not{\hspace{-0.1cm}\rightarrow}} \wedge \; q_2'
          \overset{a}{\rightarrow}$. Daraus folgt $w_1v_1a\in \MIT{}_1\subseteq
          \ET{}_1$ und $w_2v_2a\in L_2\subseteq \EL{}_2$. Damit ergibt sich $wva\in
          \ET{}_1\|\EL{}_2$, da $a\in O_2\subseteq O_{12}$ gilt $w=\prune{}(wva)$ und
          somit ist $w$ in der rechten Seite der Gleichung enthalten.
      \end{itemize}
  \end{itemize}

  1. \glqq $\supseteq$\grqq :\\
  Wegen der Abgeschlossenheit beider Seiten der Gleichung gegenüber $\cont{}$
  betrachten wir auch in diesem Fall nur ein präfix-minimales Element $x\in
  \prune{}((\ET{}_1\|\EL{}_2)\cup (\EL{}_1\|\ET{}_2))$. Da $x$ durch die Anwendung der
  $\prune{}$-Funktion entstanden ist, existiert ein $y\in O_{12}^*$ mit
  $xy\in(\ET{}_1\|\EL{}_2)\cup (\EL{}_1\|\ET{}_2)$. \OBdA{} gehen wir davon aus, dass
  $xy\in \ET{}_1\|\EL{}_2$ gilt, d.h.\ es gibt $w_1\in \ET{}_1$ und $w_2\in \EL{}_2$ mit
  $xy\in w_1\|w_2$. In dem Punkt, dass wir das präfix-minimale Element noch mit
  Outputs fortsetzten können, unterscheidet sich dieser Beweis von dem
  in~\cite{Schlosser2012BA}. Dort wird nicht weiter darauf eingegangen, dass
  die $\prune{}$-Funktion hier noch zur Anwendung kommt, da wir jedoch später nur
  Präfixe von $x$ betrachten werden, ist dieser Unterschied irrelevant.\\
  Im Folgenden werden wir für alle Fälle von $xy$ zeigen, dass es ein $v\in
  \PrET{}(S_1\|S_2)\cup \MIT{}(S_1\|S_2)$ gibt, das ein Präfix von $xy$ ist und $v$
  entweder auf einen Input aus $I_{12}$ endet oder $v = \varepsilon$. Da $v$
  entweder leer ist oder auf einen Input endet, muss $v$ ein Präfix von $x$
  sein. $\varepsilon$ ist Präfix von jedem Wort und sobald $v$ mindestens einen
  Buchstaben enthält, muss das Ende von $v$ vor dem Anfang von $y\in O_{12}^*$
  liegen. Dadurch hat $x$ ein Präfix in $\PrET{}(S_1\|S_2)\cup
  \MIT{}(S_1\|S_2)$, damit ist $x$ in
  der Fortsetzung dieser Menge enthalten und somit gilt $x\in \ET{}_{12}$.\\
  Sei $v_1$ das kürzeste Präfix von $w_1$ in $\PrET{}_1\cup \MIT{}_1$. Falls
  $w_2\in L_2$, so sei $v_2=w_2$, sonst soll $v_2$ das kürzeste Präfix von
  $w_2$ in $\PrET{}_2\cup \MIT{}_2$ sein. Jede Aktion in $v_1$ und $v_2$ hängt mit
  einer aus $xy$ zusammen. Wir gehen nun davon aus, dass entweder
  $v_2=w_2\in L_2$ gilt oder die letzte Aktion von $v_1$ findet vor oder
  gleichzeitig mit der letzten Aktion von $v_2$ statt. Ansonsten endet
  $v_2\in \PrET{}_2\cup \MIT{}_2$ vor $v_1$ und somit ist dieser Fall analog zu $v_1$
  endet vor $v_2$.
  \begin{itemize}
    \item Fall 1 ($v_1=\varepsilon$): Dadurch dass $\varepsilon\in \PrET{}_1\cup
      \MIT{}_1$, ist bereits in $S_1$ ein Error lokal erreichbar. $\varepsilon\in
      \MIT{}_1$ ist nicht möglich, da jedes Element aus $MIT$ nach Definition
      mindestens die Länge $1$ haben muss. Wir wähle
      $v_2'=v'=\varepsilon$, somit ist $v_2'$ ein Präfix von $v_2$.
    \item Fall 2 ($v_1\neq\varepsilon$): Aufgrund der Definitionen von $\PrET{}$
      und $MIT$ endet $v_1$ auf ein $a\in I_1$, d.h.\ $v_1=v_1’a$. $v'$ sei das
      Präfix von $xy$, das mit der letzten Aktion von $v_1$ endet, d.h.\ mit
      $a$, und $v_2'=v'|_{\Sigma _{2}}$. Falls $v_2 = w_2\in L_2$, dann ist
      $v_2'$ ein Präfix von $v_2$. Falls $v_2\in
      \PrET{}_2\cup \MIT{}_2$ gilt, dann ist durch die Annahme, dass $v_2$ nicht vor
      $v_1$ endet, $v_2'$ ein Präfix von $v_2$. Im Fall $v_2\in \MIT{}_2$ können
      wir durch die gleiche Argumentation ebenfalls schließen, dass $v_2'$ ein
      Präfix von $v_2$ ist. Wir wissen zusätzlich, dass $v_2$ auf $b\in
      I_2$ endet, jedoch muss nicht mehr wie in~\cite{Vogler2014EIO} $b\neq a$
      gelten. Wir können also keine Aussage mehr darüber treffen, ob es sich um
      ein echtes Präfix handelt.
  \end{itemize}
  In allen Fällen erhalten wir $v_2'=v'|_{\Sigma _2}$ ist ein Präfix von $v_2$
  und $v'\in v_1\| v_2'$ ist ein Präfix von $xy$. Da nicht mehr $b\neq a$
  gelten muss, können wir nicht mehr für alle Fälle
  $q_{02}\overset{v_2'}{\Rightarrow}$ folgern, wie das in~\cite{Vogler2014EIO}
  möglich war.
  \begin{itemize}
    \item Fall I ($v_1\in \MIT{}_1$ und $v_1\neq\varepsilon$): Es gibt $q_{01}
      \overset{v_1'}{\Rightarrow}q_1
      \overset{a}{\not{\hspace{-0.1cm}\rightarrow}}$ und sei $v'=v''a$. Bei der
      folgenden Fallunterscheidung müssen wir bezüglich~\cite{Vogler2014EIO}
      zwei weitere Fälle (Ib) und Ic)) einfügen, da es zulässig ist, das $a$
      sowohl in $I_1$ wie auch in $I_2$ enthalten ist.
      \begin{itemize}
        \item Fall Ia) ($a\notin\Sigma _2$): Es gilt $q_{02}
          \overset{v_2'}{\Rightarrow} q_2$ mit $v''\in v_1'\|v_2'$, da $v_2'$
          und $v_2$ nicht auf $a$ enden können. Dadurch erhalten
          wir $(q_{01},q_{02}) \overset{v''}{\Rightarrow} (q_1,q_2)
          \overset{a}{\not{\hspace{-0.1cm}\rightarrow}}$ mit $a\in I_{12}$.
          Somit können wir wählen $v:=v''a=v'\in \MIT{}_{12}$.
        \item Fall Ib) ($a\in I_2$ und $v_2'\in \MIT{}_2$): Es gilt $v_2'=v_2''a$
          mit $q_{02} \overset{v_2''}{\Rightarrow} q_2
          \overset{a}{\not{\hspace{-0.1cm}\rightarrow}}$ und $v''\in
          v_1'\|v_2''$. $a$ ist  für $S_2$, ebenso wie für $S_1$, ein fehlender
          Input. Wir können somit schließen, dass $(q_1,q_2)
          \overset{a}{\not{\hspace{-0.1cm}\rightarrow}}$ gilt. Wir wählen
          $v:=v''a=v'\in \MIT{}_{12}$.
        \item Fall Ic) ($a\in I_2$ und $v_2'\in L_2$): Es gilt $q_{02}
          \overset{v_2''}{\Rightarrow} q_2 \overset{a}{\rightarrow}$ mit
          $v_2'=v_2''a$. Da jedoch $a$ in $Synch(S_1,S_2)$ liegt, da die Menge
          der synchronisierten Aktionen bezüglich~\cite{Vogler2014EIO}
          erweitert wurde, somit reicht schon, dass $q_1
          \overset{a}{\not{\hspace{-0.1cm}\rightarrow}}$ gilt, um folgendes
          schließen zu können $(q_1,q_2)
          \overset{a}{\not{\hspace{-0.1cm}\rightarrow}}$. Somit können wir hier
          $v:=v''a=v'\in \MIT{}_{12}$ wählen.
        \item Fall Id) ($a\in O_2$): Es gilt $v_2'=v_2''a$ und $q_{02}
          \overset{v_2'}{\Rightarrow}$, da der Input $b$ von $v_2$ hier nicht
          dem $a$ entsprechen kann. Wir erhalten also $q_{02}
          \overset{v_2''}{\Rightarrow} q_2 \overset{a}{\rightarrow}$ mit
          $v''\in v_1'\|v_2''$. Daraus ergibt sich $(q_{01},q_{02})
          \overset{v''}{\Rightarrow} (q_1,q_2)$ mit $q_1
          \overset{a}{\not{\hspace{-0.1cm}\rightarrow}},a\in I_1, q_2
          \overset{a}{\rightarrow},a\in O_2$, somit gilt $(q_1,q_2)\in
          E_{12}$. Wir wählen $v:=\prune{}(v'')\in \PrET{}_{12}$.
      \end{itemize}
  \item Fall II ($v_1\in \PrET{}_1$): $\exists u_1\in O_1^*:q_{01}
    \overset{v_1}{\Rightarrow} q_1 \overset{u_1}{\Rightarrow} q_1'$ mit
    $q_1'\in E_1$. Da wir hier keine disjunkten Inputmengen, wie
    in~\cite{Vogler2014EIO}, haben, kann
    das $a$, auf das $v_1$ endet ebenfalls der letzte Buchstabe von $v_2$
    sein. Im Fall von $v_2\in \MIT{}_2$ kann somit $a=b$ gelten und somit wäre
    $v_2=v_2'$. Dieser Fall verläuft jedoch analog zu Fall Ic) und wird somit
    hier nicht weiter betrachtet. Es gilt somit für
    alle anderen Fälle hier $q_{02} \overset{v_2'}{\Rightarrow}q_2$ mit
    $(q_{01},q_{02}) \overset{v'}{\Rightarrow}(q_1,q_2)$.
    \begin{itemize}
      \item Fall IIa) ($u_2\in (O_1\cap I_2)^*, c\in (O_1\cap I_2)$, sodass
        $u_2c$ Präfix von $u_1|_{I_2}$ mit $q_2 \overset{u_2}{\Rightarrow} q_2'
        \overset{c}{\not{\hspace{-0.1cm}\rightarrow}}$): Für das Präfix $u_1'c$
        von $u_1$ mit $u_1'c|_{I_2}=u_2c$ wissen wir, dass $q_1
        \overset{u_1'}{\Rightarrow} q_1'' \overset{c}{\rightarrow}$. Somit gilt
        $u_1'\in u_1'\|u_2$ und $(q_1,q_2) \overset{u_1'}{\Rightarrow}
        (q_1'',q_2')\in E_{12}$, da für $S_2$ der entsprechende Input fehlt,
        der mit dem $c$ Output von $S_1$ zu koppeln wäre. Es handelt sich also
        um einen neuen Error. Wir wählen $v:=\prune{}(v'u_1')\in \PrET{}(S_1\| S_2)$,
        dies ist ein Präfix von $v'$, da $u_1\in O_1^*$.
      \item Fall IIb) ($q_2 \overset{u_2}{\Rightarrow} q_2'$ mit
        $u_2=u_1|_{I_2}$): Somit ist $u_1\in u_1\|u_2$ und $(q_1,q_2)
        \overset{u_1}{\Rightarrow} (q_1',q_2')\in E_{12}$, da $q_1'\in E_1$ und
        somit handelt es sich um einen geerbten Error. Wir wählen nun $v:=\prune{}
        (v'u_1)\in \PrET{}(S_1\|S_2)$, das wiederum ein Präfix von $v'$ ist.
    \end{itemize}
  \end{itemize}

  2.:\\
  Der Beweis für diesen Punkt musste bezüglich~\cite{Vogler2014EIO} nicht
  verändert werden, bis auf die Ersetzung der Zeichen der Parallelkomposition.\\
  Es ist durch die Definition klar, dass gilt $L_i\subseteq \EL{}_i$ und
  $\ET{}_i\subseteq \EL{}_i$. Wir beginnen mit der Argumentation von der rechten
  Seite der Gleichung aus:
  \begin{align*}
    &(\EL{}_1\| \EL{}_2)\cup \ET{}_{12}\\
    &\overset{\ref{DefETEL}}{=}(L_1\cup \ET{}_1)\|(L_2\cup \ET{}_2)\cup \ET{}_{12}\\
    &=\underset{\overset{1.}{\subseteq} \ET{}_{12}}{\underset{\subseteq
    (\EL{}_1\|\ET{}_2)}{\underbrace{(L_1\|\ET{}_2)}}} \cup
    \underset{\overset{1.}{\subseteq} \ET{}_{12}}{\underset{\subseteq
    (\ET{}_1\|\EL{}_2)}{\underbrace{(\ET{}_1\|L_2)}}} \cup
    (L_1\|L_2) \cup \underset{\overset{1.}{\subseteq}
    \ET{}_{12}}{\underset{\subseteq (\EL{}_1\|\ET{}_2)}{\underbrace{(\ET{}_1\|\ET{}_2)}}} \cup
    \ET{}_{12}\\
    &=(L_1\|L_2) \cup \ET{}_{12}\\
    &\overset{\ref{LemmaSprache}}{=}L_{12}\cup \ET{}_{12}\\
    &\overset{\ref{DefETEL}}{=}\EL{}_{12}
  \end{align*}
\end{proof}

Die folgende Proposition wurde hier noch explizit mit Beweis eingefügt im
Gegensatz zu den Ausführungen in~\cite{Vogler2014EIO}, in denen diese
Präkongruenz nur als Folgerung aus dem letzten Satz erwähnt wird. Die
Feststellung, dass es sich um eine Präkongruenz handelt, ist wichtig, da wir
dann die erste Eigenschaft erfüllt haben um eine Präkongruenz zu finden, die
äquivalent ist zu der vollständig abstrakten Präkongruenz \ECRel{}.

\begin{prop}[Präkongruenz]
  \label{propPraekongruenz}
  \ERel{} ist eine Präkongruenz bezüglich $\cdot\|\cdot$.
\end{prop}

\begin{proof}
  Es muss gezeigt werden: Wenn $S_1\ERel S_2$ gilt, dann für jedes
  komponierbare $S_3$ auch $S_3\|S_1\ERel S_3\|S_1$. D.h.\ es ist zu zeigen,
  dass aus $\ET{}_1\subseteq \ET{}_2$ und $\EL{}_1\subseteq \EL{}_2$ folgt,
  $\ET{}(S_3\|S_1)\subseteq \ET{}(S_3\|S_2)$ und $\EL{}(S_3\|S_1)\subseteq
  \EL{}(S_3\|S_2)$. Dies ergibt sich aus der Monotonie von \cont{}, \prune{}
  und $\cdot \|\cdot$ auf Sprachen wie folgt:\\
  \begin{itemize}
    \item $\begin{aligned}[t]
        \ET{}(S_3\|S_1) &\overset{\ref{satzErrorSemanik}~1.}{=}
      \cont{}(\prune{}((\ET{}_3\|\EL{}_1)\cup (\EL{}_3\|\ET{}_1)))\\
      &\hspace{-0.3cm}\overset{\ET{}_1\subseteq
    \ET{}_2}{\overset{\mathrm{und}}{\overset{\EL{}_1\subseteq \EL{}_2}{\subseteq}}}
    \cont{}(\prune{}((\ET{}_3\|\EL{}_2)\cup (\EL{}_3\|\ET{}_2)))\\
      &\overset{\ref{satzErrorSemanik}~1.}{=} \ET{}(S_3\|S_2)
    \end{aligned}$
    \item $\begin{aligned}[t]
        \EL{}(S_3\|S_1) &\overset{\ref{satzErrorSemanik}~2.}{=} (\EL{}_3\|\EL{}_1)\cup
        E_{31}\\
        &\hspace{-0.4cm}\overset{\EL{}_1\subseteq
      \EL{}_2}{\overset{\mathrm{und}}{\overset{\ET{}_{31}\subseteq
      \ET{}_{32}}{\subseteq}}} (\EL{}_3\|\EL{}_2)\cup \ET{}_{32}\\
      &\overset{\ref{satzErrorSemanik}~2.}{=} \EL{}(S_3\|S_2)
    \end{aligned}$
  \end{itemize}
\end{proof}

In~\cite{Vogler2014EIO} wurde auch die Verfeinerung von \EIO{}s als Relation betrachtet
mit Spezifikation und Implementierung. Hier soll ebenfalls eine
Verfeinerungsrelation über \EIO{}s betrachtet werden, jedoch sollen die
synchronisierten Aktionen nicht verborgen werden. Dadurch ändern sich natürlich
auch Teile des Beweises, vor allem muss statt mit $\StET{}$ mit der Menge
$\PrET{}$ argumentiert werden. Dieses Lemma existiert in dieser Form nicht
in~\cite{Schlosser2012BA}, da es dort mit der Aussage von
Satz~\ref{satzFullAbstractness} kombiniert wurde. Jedoch ist die Aussage diese
Lemmas trotzdem Teil dessen, was gezeigt wird und somit finden sich die Teile
dieses Beweises auch dort wieder.

\begin{lem}[Verfeinerung mit Errors]
  \label{lemVerfeinerung}
  Gegeben sind zwei \EIO{}s $S_1$ und $S_2$ mit der gleichen Signatur. Wenn
  alle Partner \EIO{}s $U$, die mit $S_2$ gut kommunizieren, auch mit $S_1$ gut
  kommunizieren, dann verfeinert $S_1$ das \EIO{} $S_2$. Diese Verfeinerung
  entspricht der Relation \ERel{} von oben: Wenn $U\|S_1 \EBRel U\|S_2$ für
  alle Partner $U$, dann gilt $S_1\ERel S_2$.
\end{lem}

\begin{proof}
  Da $S_1$ und $S_2$ die gleichen Signaturen haben, definieren wir:
  $I:=I_1=I_2$ und $O:=O_1=O_2$. Für jeden der Partner $U$ gilt $I_U=O$ und
  $O_U=I$.\\
  Um $S_1\ERel S_2$ zu zeigen, wird nachgeprüft, dass gilt:
  \begin{itemize}
    \item $\ET{}(S_1)\subseteq \ET{}(S_2)$,
    \item $\EL{}(S_1)\subseteq \EL{}(S_2)$.
  \end{itemize}
  Wir wählen ein präfix-minimales Element $w\in \ET{}(S_1)$ und
  zeigen, dass dieses $w$ oder eines seiner Präfixe in $\ET{}(S_2)$ enthalten ist.
  Dies ist möglich, da beide Mengen durch $\cont{}$ abgeschlossen sind.
  \begin{itemize}
    \item Fall 1 ($w=\varepsilon$): Es handelt sich um einen lokal erreichbaren
      Error in $S_1$.
      Wir nehmen für $U$ ein Transitionssystem, das nur aus dem Startzustand und
      einer Schleife für alle Inputs $x\in I_U$ besteht. Somit kann $S_1$ die gleichen
      Error-Zustände lokal erreichen wie $U\|S_1$. Daraus folgt, dass auch
      $U\|S_2$ einen lokal erreichbaren Error-Zustand haben muss. Durch unsere
      Definition von $U$ kann dieser Error nur von $S_2$ geerbt sein. Es
      muss also in $S_2$ ein Error-Zustand durch interne Aktionen und Outputs
      erreichbar sein, d.h.\ es gilt $\varepsilon\in \PrET{}(S_2)$.
    \item Fall 2 ($w=x_1\dots x_n x_{n+1}\in\Sigma ^+$ mit $n\geq 0$ und
      $x_{n+1}\in I$): Wir betrachten den folgenden Partner $U$ (siehe auch
      Abbildung~\ref{UohneE}):
      \begin{itemize}
        \item $Q_U=\{q_0,q_1,\dots ,q_{n+1}\}$
        \item $q_{0U}=q_0$
        \item $E_U=\emptyset$
        \item $\begin{aligned}[t]
            \delta _U=&\{(q_i,x_{i+1},q_{i+1})\mid  0\leq i\leq n\}\\
                      &\cup\{(q_i,x,q_{n+1})\mid  x\in I_U\backslash\{x_{i+1}\},
          0\leq i\leq n\}\\
          &\cup\{(q_{n+1},x,q_{n+1})\mid  x\in I_U\}
        \end{aligned}$
      \end{itemize}
      \begin{figure} [h!tbp]
      \begin{center}
        \begin{tikzpicture}[->, >=latex',auto,node distance =3cm, semithick]

          \node (0) {$q_0$};
          \node (1) [right of=0] {$q_1$};
          \node (dots) [right of=1] {$\dots$};
          \node (n) [right of=dots] {$q_n$};
          \node (n1) at ($(1)!0.5!(dots) + (0,-3)$) {$q_{n+1}$};

          \path ($ (0) + (-1,0) $) edge (0)
                (0) edge node {$x_1$} (1)
                    edge [bend right] node [below, sloped] {$x?\neq x_1$} (n1)
                (1) edge node {$x_2$} (dots)
                    edge node [below, sloped] {$x?\neq x_2$} (n1)
                (dots) edge node {$x_n$} (n)
                       edge [dashed] (n1)
                (n) edge node [above, sloped] {$x?\in I_U$} (n1)
                    edge [bend left] node [sloped] {$x_{n+1}$!} (n1)
                (n1) edge [loop below] node {$x?\in I_U$} (n1);
        \end{tikzpicture}
        \caption{$x?\neq x_i$ steht für alle $x\in I_U\backslash\{x_i\}$}
        \label{UohneE}
      \end{center}
      \end{figure}
      Wir können für $w$ zwei Fälle unterscheiden. Beide führen zu
      $\varepsilon\in \PrET{}(U\|S_1)$. Dieses Resultat unterscheidet sich von dem
      in~\cite{Vogler2014EIO}, da hier die synchronisierten Aktionen als Outputs
      vorhanden bleiben und somit kann nicht $\varepsilon\in \StET{}(U\|S_1)$
      gelten.
      \begin{itemize}
        \item Fall 2a) ($w\in \MIT{}(S_1)$): In $U\|S_1$ erhalten wir
          $(q_0,q_{01}) \overset{x_1\dots x_n}{\Rightarrow} (q_n,q')$ mit
          $q' \overset{x_{n+1}}{\not{\hspace{-0.1cm}\rightarrow}}$ und $q_n
          \overset{x_{n+1}}{\rightarrow}$. Deshalb gilt $(q_n,q')\in
          E_{U\|S_1}$ und $x_1\dots x_n\in \StET{}(U\|S_1)$. Da alle Aktionen aus
          $w$ bis auf $x_{n+1}$ synchronisiert werden gilt $x_1,\dots ,x_n\in
          O_{U\|S_1}$. Daraus ergibt sich dann $\varepsilon\in \PrET{}(U\|S_1)$.
        \item Fall 2b) ($w\in \PrET{}(S_1)$): In $U\|S_1$ erhalten wir
          $(q_0,q_{01}) \overset{w}{\Rightarrow} (q_{n+1},q'')
          \overset{u}{\Rightarrow} (q_{n+1},q')$ für $u\in O^*$ und $q'\in
          E_1$. Daraus folgt $(q_{n+1},q')\in E_{U\|S_1}$ und somit $wu\in
          \StET{}(U\|S_1)$. Da alle Aktionen aus $w$ synchronisiert werden gilt
          $x_1,\dots ,x_n,x_{n+1}\in O_{U\|S_1}$ und da $u\in O^*$ folgt
          $u\in O_{U\|S_1}^*$. Somit ergibt sich $\varepsilon\in
          \PrET{}(U\|S_1)$.
      \end{itemize}
      Da wir wissen, dass $\varepsilon\in \PrET{}(U\|S_1)$ gilt, können wir durch
      $U\|S_1\EBRel U\|S_2$ schließen, dass auch in $U\|S_2$ ein Error
      lokal erreichbar sein muss.\\
      Dieser Error kann geerbt oder neu sein.
      \begin{itemize}
        \item Fall 2i) (neuer Error): Da jeder Zustand von $U$ alle Inputs $x\in
          O=I_U$ zulässt, muss ein lokal erreichbarer Error einer sein, bei dem
          ein Output $a\in O_U$ von $U$ möglich ist, der nicht mit einem
          passenden Input aus $S_2$ synchronisiert werden kann. Durch die
          Konstruktion von $U$ sind in $q_{n+1}$ keine Outputs möglich. Ein
          neuer Error muss also die Form $(q_i,q')$ haben mit $i\leq n, q'
          \overset{x_{i+1}}{\not{\hspace{-0.1cm}\rightarrow}}$ und $x_{i+1}\in
          O_U=I$. Durch Projektion erhalten wir dann $q_{02} \overset{x_1\dots
          x_i}{\Rightarrow} q'
          \overset{x_{i+1}}{\not{\hspace{-0.1cm}\rightarrow}}$ und damit gilt
          $x_1\dots x_{i+1}\in \MIT{}(S_2)\subseteq \ET{}(S_2)$. Somit ist ein Präfix
          von $w$ in $\ET{}(S_2)$ enthalten.
        \item Fall 2ii) (geerbter Error): $U$ hat $x_1\dots x_i u$ ausgeführt
          mit $u\in I_U^*=O^*$ und ebenso hat $S_2$ diesen Weg ausgeführt.
          Durch dies hat $S_2$ einen Zustand in $E_2$ erreicht, da von $U$
          keine Error geerbt werden können. Es gilt dann $\prune{}(x_1\dots
          x_iu)=\prune{}(x_1\dots x_i)\in \PrET{}(S_2)\subseteq \ET{}(S_2)$. Da $x_1\dots
          x_i$ ein Präfix von $w$ ist, führt auch in diesem Fall ein Präfix von
          $w$ zu einem Error.
      \end{itemize}
  \end{itemize}
  Um uns von der zweiten Inklusion zu überzeugen reicht es aufgrund der ersten
  Inklusion und der Definition von $\EL{}$, zu zeigen, dass
  $L(S_1)\backslash \ET{}(S_1)\subseteq \EL{}(S_2)$ gilt.\\
  Wir nehmen uns dafür ein beliebiges $w\in L(S_1)\backslash \ET{}(S_1)$ und
  zeigen, dass es in $\EL{}(S_2)$ enthalten ist.
  \begin{itemize}
    \item Fall 1 ($w=\varepsilon$): Da $\varepsilon$ immer in $\EL{}(S_2)$
      enthalten ist, haben wir hier nichts zu zeigen.
    \item Fall 2 ($w=x_1\dots x_n$ mit $n\geq 1$): Wir konstruieren einen
      Partner $U$ wie folgt (siehe dazu auch Abbildung~\ref{UmitE}):
      \begin{itemize}
        \item $Q_U=\{q,q_0,q_1,\dots ,q_n\}$
        \item $q_{0U}=q_0$
        \item $E_U={q_n}$
        \item $\begin{aligned}[t]
            \delta _U=&\{(q_i,x_{i+1},q_{i+1})\mid 0\leq i< n\}\\
                      &\cup\{(q_i,x,q)\mid x\in I_U\backslash\{x_{i+1}\},0\leq
          i\leq n\}\\
          &\cup\{(q,x,q)\mid x\in I_U\}
              \end{aligned}$
      \end{itemize}
      \begin{figure} [h!tbp]
      \begin{center}
        \begin{tikzpicture}[->, >=latex',auto,node distance =3cm, semithick]

          \node (0) {$q_0$};
          \node (1) [right of=0] {$q_1$};
          \node (dots) [right of=1] {$\dots$};
          \node (n1) [right of=dots] {$q_{n-1}$};
          \node (n) [right of=n1, rectangle, draw] {$q_n\in E_U$};
          \node (q) at ($(1)!0.5!(dots) + (0,-3)$) {$q$};

          \path ($ (0) + (-1,0) $) edge (0)
                (0) edge node {$x_1$} (1)
                    edge [bend right] node [below, sloped] {$x?\neq x_1$} (q)
                (1) edge node {$x_2$} (dots)
                    edge node [below, sloped] {$x?\neq x_2$} (q)
                (dots) edge node {$x_{n-1}$} (n1)
                       edge [dashed] (q)
                (n1) edge node {$x_n$} (n)
                edge [bend left] node [below, sloped] {$x?\neq x_n$} (q)
                (q) edge [loop below] node {$x?\in I_U$} (q);
        \end{tikzpicture}
        \caption{$x?\neq x_i$ steht für alle $x\in I_U\backslash\{x_i\}$, $q_n$
          ist der einzige Error-Zustand}
        \label{UmitE}
      \end{center}
      \end{figure}
      Da $q_{01} \overset{w}{\Rightarrow} q'$ gilt, wissen wir, dass $U\|S_1$
      einen lokal erreichbaren Error hat. Somit muss $U\|S_2$ ebenfalls einen
      lokal erreichbaren Error haben.
      \begin{itemize}
        \item Fall 2a) (neuer Error aufgrund von $x_i\in O_U$ und $q_{02}
          \overset{x_1\dots x_{i-1}}{\Rightarrow} q''
          \overset{x_i}{\not{\hspace{-0.1cm}\rightarrow}}$): Es gilt $x_1\dots
          x_i\in \MIT{}(S_2)$ und somit $w\in \EL{}(S_2)$. Anzumerken ist, dass nur
          auf diesem Weg Outputs von $U$ möglich sind, deshalb gibt es keine
          anderen Outputs von $U$, die zu einem neuen Error führen können.
        \item Fall 2b) (neuer Error aufgrund von $a\in O_2$): Der einzige
          Zustand, in dem $U$ nicht alle Inputs erlaubt sind, ist $q_n$, der
          bereits ein Error-Zustand ist. Falls dieser Zustand erreichbar ist in
          $U\|S_2$, dann besitzt der komponierte \EIO{} einen geerbten Error und
          es gilt $w\in L(S_2)\subseteq \EL{}(S_2)$.
        \item Fall 2c) (geerbter Error von $U$): Da der einzige Zustand aus
          $E_U$ $q_n$ ist und alle Aktionen synchronisiert sind, ist dies nur
          möglich, wenn gilt $q_{02} \overset{x_1\dots x_n}{\Rightarrow}$. In
          diesem Fall gilt, wie im letzten, $w\in L(S_2)\subseteq \EL{}(S_2)$.
        \item Fall 2d) (geerbter Error von $S_2$): Es gilt dann $q_{02}
          \overset{x_1\dots x_iu}{\Rightarrow} q'\in E_2$ für $i\geq 0$ und
          $u\in O^*$. Somit ist $x_1\dots x_iu\in \StET{}(S_2)$ und damit
          $\prune{}(x_1\dots x_iu)=\prune{}(x_1\dots x_i)\in \PrET{}(S_2)\subseteq
          \EL{}(S_2)$. Somit gilt $w\in \EL{}(S_2)$.
      \end{itemize}
  \end{itemize}
\end{proof}

Der folgende Satz sagt aus, dass \ERel{} die gröbste Präkongruenz ist, die wir
gesucht haben und somit mit äquivalent ist zur vollständig abstrakten
Präkongruenz \ECRel{}.

\begin{satz}[Full Abstractness für Error-Semanik]
  \label{satzFullAbstractness}
  Seien $S_1$ und $S_2$ zwei \EIO{}s mit derselben Signatur. Dann gilt
  $S_1\ECRel S_2\Leftrightarrow S_1\ERel S_2$, insbesondere ist \ERel{}
  eine Präkongruenz.
\end{satz}

\begin{proof}
  Wie bereits in Proposition~\ref{propPraekongruenz} festgehalten, ist \ERel{} eine
  Präkongruenz.

  \glqq $\Leftarrow$\grqq : Nach Definition gilt, wenn
      $\varepsilon\in \ET{}(S)$, ist ein Error lokal erreichbar in $S$.
      Somit impliziert $S_1\ERel S_2$, dass $\varepsilon\in
      \ET{}(S_2)$ gilt, wenn $\varepsilon\in \ET{}(S_1)$. Dadurch folgt ebenfalls,
      dass $S_1\EBRel S_2$ gilt. Somit folgt aus $S_1\ERel S_2$ der relationale
      Zusammenhang $S_1\ECRel S_2$.

      \glqq $\Rightarrow$\grqq : Durch die Definition von \ECRel{} folgt aus
  $S_1\ECRel S_2$, dass $U\|S_1\ECRel U\|S_2$ für alle \EIO{}s $U$, die mit
  $S_1$ komponierbar sind. Somit folgt auch die Gültigkeit von
  $U\|S_1\EBRel U\|S_2$ für alle diese \EIO{}s $U$. Mit
  Lemma~\ref{lemVerfeinerung} folgt dann $S_1\ERel S_2$.
\end{proof}

Wir haben somit jetzt eine Kette an Folgerungen gezeigt, die sich zu einem
Ring schließen. Dies ist in Abbildung~\ref{Folgerungskette} dargestellt.

\begin{figure}[h!tbp]
  \begin{center}
    \begin{tikzpicture}[scale = 3]
      \matrix (m) [matrix of math nodes,row sep=2cm,column sep=4cm]{
        S_1\ERel S_2 & S_1\ECRel S_2 \\
        \substack{\forall~\mathrm{Partner}~U:\\U\|S_1\EBRel U\|S_2} &
    \substack{\forall~\mathrm{komponierbaren}~U:\\U\|S_1\EBRel U\|S_2} \\};
        \draw[-implies, double, double distance=1mm]
          (m-1-1) -- node [above] {\glqq{}$\Leftarrow$\grqq{} von
            Satz~\ref{satzFullAbstractness}} (m-1-2);
        \draw[-implies, double, double distance=1mm]
          (m-1-2) -- node [right] {\glqq{}$\Rightarrow$\grqq{} von
            Satz~\ref{satzFullAbstractness}} (m-2-2);
        \draw[-implies, double, double distance=1mm]
          (m-2-1) -- node [left]
          {Lemma~\ref{lemVerfeinerung}} (m-1-1);
        \draw[-implies, double, double distance=1mm]
          (m-2-2) -- node [below]
          {$\substack{U~\mathrm{Partner}\\\Downarrow\\ U~\mathrm{komponierbar}}$} (m-2-1);
    \end{tikzpicture}
    \caption{Folgerungskette}
    \label{Folgerungskette}
  \end{center}
\end{figure}

Aus Satz~\ref{satzFullAbstractness} und Lemma~\ref{lemVerfeinerung} erhalten
wir das folgende Korollar.

\begin{kor}
  Ein \EIO{} $S_1$ verfeinert einen \EIO{} $S_2$ genau dann, wenn für alle \EIO{}s $U$
  für die $S_2$ gut mit $U$ kommuniziert folgt $S_1$ kommuniziert
  ebenfalls gut mit $U$.\\
  Dies lässt sich formal wie folgt ausdrücken: $S_1\ERel S_2
  \Leftrightarrow U\|S_1\EBRel U\|S_2$ für alle Partner $U$.
\end{kor}

\section{Hiding für Error}

Wir wollen nun untersuchen, was für Auswirkungen Hiding auf unsere
Verfeinerungsrelationen hat. Es werden also Outputs der Systeme internalisiert.

\begin{prop}[Error-Basisrelation bzgl. Internalisierung]
  \label{propErBaIn}
  Wenn $S_1\EBRel S_2$ gilt, dann folgt daraus, dass auch $(S_1/\{x_1,x_2,\dots
  ,x_n\})\EBRel (S_2/\{x_1,x_2,\dots ,x_n\})$ gilt.
\end{prop}

\begin{proof}
  Da die Definition der lokalen Erreichbarkeit auf lokalen Aktionen beruht, die
  aus den Outputs und der internen Aktion besteht, ändert sich durch das
  verbergen von Outputs nichts an der Error-Erreichbarkeit. Somit ist jeder
  Error, der in $S_i$ lokal Erreichbar ist über einen Traces, der einen
  Outputs aus $\{x_1,x_2,\dots ,x_n\}$ enthält auch in $S_i/\{x_1,x_2,\dots
  ,x_n\}$ erreichbar, jedoch enthält der Traces nicht mehr diesen Output. Alle
  Traces, die keinen Outputs aus der Menge hinter dem Internalisierungsoperator
  enthalten bleiben unverändert erhalten. Es ist auch nicht möglich, dass durch
  das Verbergen von Outputs neue Errors entstehen. Somit folgt die Behauptung.
\end{proof}

\begin{satz}[Präkongurenz bzgl. Interalisierung]
  \label{satzPraeInternalisierung}
  Seinen $S_1$ und $S_2$ zwei \EIO{}s für die $S_1\ERel S_2$ gilt, somit gilt
  auch $(S_1/\{x_1,x_2,\dots ,x_n\})\ERel (S_2/\{x_1,x_2,\dots ,x_n\})$. Somit
  ist also \ERel{} eine Präkongruenz bezüglich $\cdot /\cdot$.
\end{satz}

\begin{proof}
  Da $S_1\ERel S_2$ gilt, wissen wir, dass $\ET{}_1\subseteq \ET{}_2$ und
  $\EL{}_1\subseteq \EL{}_2$ gilt. Da für Elemente aus $X:=\{x_1,x_2,\dots
  ,x_n\}$ die nicht in $O$ enthalten sind nichts passiert bei der
  Internalisierung, gehen wir davon aus, das $X\subseteq O$ gilt. Wir wählen
  einen Traces $w=a_1a_2\dots a_m\in \ET{}_1$. Solang $w$ nicht nur aus Outputs
  besteht, gibt es also einen Ablauf $q_0 \overset{w'}{\Rightarrow}$ mit
  $w'=a_1a_2\dots a_k$, so dass $a_k\in I$, $k\leq m$ und $w'\in \MIT{}_1$ oder
  $w'\in \PrET{}_1$ gilt. Nach der Internalisierung bleiben von dem Ablauf nur
  noch die Aktionen übrig, die nicht gleichzeitig ein Output und ein Element
  aus $X$ sind. Der Traces reduziert sich also auf $w''=w|_{\Sigma\backslash
  (O\cap X)}$. Diese $w''$ hat einen zu $w'$ analoges Präfix, da der Input
  $a_k$ erhalten beleibt. Dieses Präfix von $w''$ ist dann in $\MIT{}(S_1/X)$
  oder in $\PrET{}(S_1/X)$ enthalten und somit ist $w''$ in $\ET{}(S_1/X)$
  enthalten. Diese Argumentation funktioniert jedoch nicht für $w$s, die nur
  aus Outputs bestehen. Da Elemente, aus $\cont{}(\MIT{}_1)$ mindestens einen
  Input enthalten müssen, muss so ein $w$ aus $\cont{}(\PrET{}_1)$ sein. Somit
  ergibt sich ohne die \cont{}-Funktion das Element $\varepsilon\in\PrET{}_1$.
  Da es in diesem Wort keine Elemente aus $X$ gibt, werden in diesem Wort auch
  keine Aktionen verborgen und somit gilt $\varepsilon\in\PrET{}(S_1/X)$.\\
  Nach Voraussetzung gilt $w\in\ET{}_2$. Nach analoger Argumentation gilt dann
  auch $w''\in\ET{}(S_2/X)$.\\
  Somit bleibt jetzt nur noch zu zeigen, dass $L(S_1/X)\backslash \ET{}(S_1/X)
  \subseteq EL(S_2/X)$ gilt. Die Argumentation, wieso nur diese Inklusion zu
  zeigen ist, kann dem Beweis zu Lemma~\ref{lemVerfeinerung} entnommen werden.
  Da wir bereits $L_1\backslash\ET{}_1\subseteq\EL{}_2$ wissen, können wir
  schließen, dass alle relevanten Traces bereits in $\EL{}_2$ enthalten sind
  und die Modifikation durch den Internalisierungsoperator wie oben keine
  Auswirkungen auf die Teilmengenbeziehung hat.
\end{proof}

Wir wissen aus~\ref{propPraekongruenz}, dass \ERel{} eine Präkongruenz
bezüglich $\cdot\|\cdot$ ist, und aus~\ref{satzPraeInternalisierung}, dass
\ERel{} auch eine Präkongruenz bezüglich $\cdot/\cdot$ ist. Da sich nach
Definition~\ref{defIntParal} die Parallelkomposition mit Internalisierung nur
aus diesen Operatoren zusammensetzt, erhalten wir das folgende Korollar.

\begin{kor}[Präkongurenz mit Internalisierung]
  \ERel{} ist eine Präkongruenz bezüglich $\cdot|\cdot$.
\end{kor}
