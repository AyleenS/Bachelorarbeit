\chapter{Verfeinerung über Error- und Quiescenttraces}

In diesem Kapitel werden wir und nicht nur um die Erreichbarkeit von
Error-Zuständen kümmern, sondern auch um die Erreichbarkeit von
Quiescent-Zuständen. Wir werden dabei ähnlich vorgehen wie im letzten Kapitel,
jedoch halten wir uns als Quelle an~\cite{Chilton2013}. Darin werden ähnliche Konzepte
beschrieben, jedoch aus Sicht der Traces.

\begin{Def}[Quiescent]
  Ein Quiescent-Zustand ist ein Zustand in einem EIO der keine Outputs besitzt
  oder ein Zustand, von dem aus über eine interne Handlung $\tau$ ein Zustand
  erreicht werden kann, der keine Outputs zulässt.\\
  Somit ist die Menge der Quiescent-Zustände in einem EIO wie folgt formal
  definiert: $Qui:=\{q\in Q\mid q\in K\vee \exists p\in Q:
  q\overset{\tau}{\Rightarrow} p\in K\}$ mit $K:=\{q\in Q\mid \forall a\in O:
  q\overset{a}{\not{\hspace{-0.1cm}\rightarrow}}\}$.
\end{Def}

Für die Erreichbarkeit verwenden wir wie im letzten Kapitel wieder den
optimistischen Ansatz der lokalen Erreichbarkeit.

\begin{Def}[lokal error- und quiescentfreie Kommunikation]
  Zwei EIOs $S_1$ und $S_2$ kommunizieren gut, wenn keine Errors und Quiescents
  lokal erreichbar sind in ihrer Parallelkomposition $S_1\| S_2$
\end{Def}

\begin{Def}[lokale Basisrelation]
  Für EIOs $S_1$ und $S_2$ mit der gleichen Signatur schreiben wir
  $S_1\sqsubseteq _{Qui}^B S_2$, wenn ein Error oder Quiescent in $S_1$ nur
  dann lokal erreichbar ist, wenn er auch in $S_2$ lokal erreichbar ist.\\
  $\sqsubseteq _{Qui}^C$ bezeichnet die vollständig abstrakte Präkongruenz von
  $\sqsubseteq  _{Qui}^ B$ bezüglich $\|$.
\end{Def}

\begin{Def}[Error und Quiescenttraces]
  \label{DefQuiescenttraces}
  Sei $S$ ein EIO und definiere:
  \begin{itemize}
    \item die Traces bezüglich Errors entsprechen denen
      aus~\ref{DefErrortraces},
    \item strickte Quiescenttraces: $StQT(S) := \{w\in\Sigma ^*\mid q_0
      \overset{w}{\Rightarrow} q\in Qui\}$.
  \end{itemize}
\end{Def}

\begin{Def}[Lokale Error und Quiescent Semantik]
  \label{DefQTQL}
  Sei $S$ ein EIO.
  \begin{itemize}
    \item Die Menge der Errortraces ist wie in~\ref{DefETEL} definiert.
    \item Die Menge der Quiescenttraces von $S$ ist $QT(S) := StQT(S)$.
    \item Die geflutete Sprache von $S$ ist $QL(S):=L(S)\cup ET(S)$
      und entspricht somit der gefluteten Sprache $EL(S)$
      in~\ref{DefETEL}.
  \end{itemize}
  Für zwei EIOs $S_1, S_2$ mit der gleichen Signatur schreiben wir
  $S_1\sqsubseteq _{Qui} S_2$, wenn $ET(S_1)\subseteq ET(S_2)$,
  $QT(S_1)\subseteq QT(S_2)$ und $QL(S_1)\subseteq QL(S_2)$ gilt.
\end{Def}

Da $QT$ nur die strickten Quiescenttraces enthält, also die Traces, die im
Automaten vorhanden sind und zu einem Quiescent-Zustand führen, gilt
$QT\subseteq L\subseteq QL$. Somit musste $QT$ nicht mehr explizit in die
Definition von $QL$ aufgenommen werden und trotzdem ist jeder Traces, der in
$QT$ möglich ist auch in der gefluteten Sprache enthalten.

\begin{satz}[Lokale Error und Quiescent Semantik für Parallelkompositonen]
  Für zwei komponierbare EIOs $S_1, S_2$ und $S_{12} = S_1\|S_2$ gilt:
  \begin{enumerate}
    \item $ET_{12} = cont(prune((ET_1\|QL_2)\cup (QL_1\|ET_2)))$,
    \item $QT_{12} = (QT_1\|QT_2)$,%TODO???
    \item $QL_{12} = (QL_1\|QL_2)\cup ET_{12}$.%TODO???
  \end{enumerate}
\end{satz}

\begin{proof}
  ~
  \begin{enumerate}
    \item \hspace{-0.2cm}:
  \end{enumerate}
  \vspace{-0.3cm}
  Der Beweis diese Punktes entspricht dem Beweis von Punkt 1.\ im Beweis von
  Satz~\ref{satzErrorSemanik}.

  2. ``$\subseteq$'':\\
  Wir betrachten $w\in StQT_{12}$ und versuchen dessen
  Zugehörigkeit zur rechten Menge zu zeigen. Aufgrund von
  Definition~\ref{DefQuiescenttraces} wissen wir es gibt gilt $(q_{01},q_{02})
  \overset{w}{\Rightarrow} (q_1,q_2)$ mit $(q_1,q_2)\in Qui_{12}$. Durch
  Projektion erhalten wir $q_{01} \overset{w_1}{\Rightarrow} q_1$ und $q_{02}
  \overset{w_2}{\Rightarrow} q_2$ mit $w\in w_1\|w_2$. Aus $(q_1,q_2)\in
  Qui_{12}$ können wir folgern, dass bereits $q_1\in Qui_1$ und $q_2\in
  Qui_2$ gilt. Somit gilt $w_1\in StQT_1\subseteq QT_1$ und $w_2\in
  StQT_2\subseteq QT_2$. Daraus folgt dann $w\in QT_1\|QT_2$ und somit ist $w$
  in der rechten Seiten der Gleichung enthalten.

  2. ``$\supseteq$'':\\
  Für diese Inklusionsrichtung betrachten wir ein Element $w\in QT_1\|QT_2$ und
  zeigen, dass es in der linken Menge enthalten ist. Da $QT_i = StQT_i$ gilt,
  existieren $w_1$ und $w_2$ für die gilt $q_{01} \overset{w_1}{\Rightarrow}
  q_1\in Qui_1$ und $q_{02} \overset{w_2}{\Rightarrow} q_2\in Qui_2$ mit $w\in
  w_1\| w_2$. Da für die Zustände $q_1$ und $q_2$ die Zugehörigkeit zur
  Quiescent Menge gilt, können wir folgern, dass der aus ihnen zusammengesetzte
  Zustand in der Parallelkomposition ebenfalls keine Outputs zulässt. Somit
  gilt also für die Komposition $(q_{01},q_{02}) \overset{w}{\Rightarrow}
  (q_1,q_2)\in Qui_{12}$ und dadurch ist $w$ in der linken Seite der Gleichung
  enthalten.

  3.:\\
  Der Beweis diese Punktes entspricht dem Beweis von Punkt 2.\ im Beweis von
  Satz~\ref{satzErrorSemanik}.
\end{proof}

$QT_{12}$ ist hier in $QL_{12}$ enthalten, da wie wir bereits oben festgestellt
haben $QT_i\subseteq QL_i$ für $i=1,2$ gilt. Somit gilt auch
$(QT_1\|QT_2)\subseteq (QL_1\|QL_2)$.
