\chapter{Verfeinerung für Error- und Ruhe-Freiheit}

\section{Präkongruenz für Ruhe}

In diesem Kapitel wird es nicht mehr nur um die Erreichbarkeit von
Error-Zuständen gehen, sondern auch um die Erreichbarkeit von
Ruhe-Zuständen. Es wird dabei eine analoge Herangehensweise zur der im letzten
Kapitel angewendet, jedoch wird~\cite{Chilton2013} als Quelle verwendet. Darin
werden ähnliche Konzepte beschrieben, jedoch aus Sicht der Traces. Es werden
dort zudem gleichzeitig auch noch Traces mit Divergenz betrachtet. Diese
Eigenschaft wird hier zunächst nicht betrachtet.\\
Zustände, die keine Outputs ohne einen Input ausführen können, werden als
in einer Art Verklemmung angesehen, da sie ohne Zutun von Außen den Zustand
nicht mehr verlassen können. So ein Zustand hat also keine
Transitionsmöglichkeiten für einen Output. Falls dieser Zustand
die Möglichkeit für eine interne Aktion hat, darf dadurch niemals ein Zustand
erreicht werden, von dem aus ein Output möglich ist. Ein Zustand, der keine
Outputs und $\tau$s ausführen kann, ist also ein Deadlock-Zustand, in denen das
System nichts mehr tun kann ohne einen Input. Wenn man eine Erweiterung um
$\tau$s zu Zuständen ohne Outputs zulässt, hat man zusätzliche noch
Verklemmungen der Art Livelock, da diese Zustände möglicherweise beliebig viele
interne Aktionen ausführen können, jedoch nie aus eigener Kraft einen
wirklichen Fortschritt in Form eines Outputs bewirken können. Die Menge der
Zustände, die sich in einer Verklemmung befinden, würde also durch $\left\{q\in
Q\mid \forall a\in O: q \overset{a}{\not{\hspace{-0.1cm}\Rightarrow}}\right\}$
beschrieben werden. Somit wären dies alle Zustände, die keine Möglichkeit haben
ohne einen Input von Außen je wieder einen Output machen zu können. Falls man
diese Definition verwenden würde, müsste man immer alle Zustände betrachten,
die durch $\tau$s erreichbar sind. Dies würde einige Betrachtungen deutlich
aufwendiger machen und soll deshalb hier nicht behandelt werden. Die Definition
für die betrachteten Verklemmungen, hier Ruhe genannt, beschränkt sich auf
Zustände, die keine Outputs und $\tau$s ausführen können.

\begin{Def}[Ruhe]
  Ein \emph{Ruhe-Zustand} ist ein Zustand in einem \EIO{}, der keine
  Outputs und kein $\tau$ zulässt.\\
  Somit ist die Menge der Ruhe-Zustände in einem \EIO{} wie folgt formal
  definiert: $Qui:=\big\{q\in Q\mid \forall \alpha\in (O\cup \{\tau\}): q
  \overset{\alpha}{\not{\hspace{-0.1cm}\rightarrow}}\big\}$.
\end{Def}

Für die Erreichbarkeit wird wie im letzten Kapitel wieder der
optimistische Ansatz der lokalen Erreichbarkeit für die Error-Zustände
verwendet. Ruhe ist kein unabwendbarer Fehler, sondern kann durch einen Input
repariert werden. Daraus ergibt sich, dass ein Ruhe-Zustand als nicht so
\glqq{}schlimmer Fehler\grqq{} anzusehen ist wie ein Error. Somit ist ein
Ruhe-Zustand ebenso wie ein Error-Zustand erreichbar, sobald er durch Outputs
und $\tau$s erreicht werden kann, jedoch ist nicht jede beliebige Fortsetzung
eines Traces, das durch lokale Aktionen zu einem Ruhe-Zustand führt, ein
Ruhetrace.

\begin{Def}[error- und ruhe-freie Kommunikation]
  Zwei \EIO{}s $S_1$ und $S_2$ \emph{kommunizieren error- und ruhe-frei}, wenn
  in ihrer Parallelkomposition $S_{12}$ keine Errors und keine Ruhe-Zustände
  lokal erreichbar sind.
\end{Def}

\begin{Def}[Ruhe-Verfeinerungs-Basisrelation]
\label{DefQuiBasisrel}
  Für \EIO{}s $S_1$ und $S_2$ mit der gleichen Signatur wird
  $S_1\QBRel S_2$ geschrieben, wenn ein Error oder Ruhe-Zustand in $S_1$ nur
  dann lokal erreichbar ist, wenn ein solcher auch in $S_2$ lokal erreichbar
  ist. Diese \emph{Basisrelation} stellt eine \emph{Verfeinerung} bezüglich
  \emph{Error} und \emph{Ruhe} dar.\\
  \QCRel{} bezeichnet die \emph{vollständig abstrakte Präkongruenz} von
  \QBRel{} bezüglich $\cdot\|\cdot$.
\end{Def}

Um eine genauere Auseinandersetzung mit den Präkongruenzen zu ermöglichen,
benötigt man wie im letzten Kapitel die Definition von Traces auf der Struktur.
Dadurch erhält man die Möglichkeit die gröbste Präkongruenz charakterisieren zu
können.\\
Wie bereits oben erwähnt, sind Ruhe-Zustände reparierbare Fehler im Gegensatz
zu Errors. Somit werden keine gekürzten Ruhetraces benötigt, bei denen die
\prune{}-Funktion zur Anwendung käme und auch keine beliebigen Verlängerungen.

\begin{Def}[Ruhetraces]
\label{DefRuhetraces}
  Sei $S$ ein \EIO{} und definiere:
  \begin{itemize}
    \item \emph{strikte Ruhetraces}: $\StQT{}(S) := \left\{w\in\Sigma ^*\mid q_0
      \overset{w}{\Rightarrow} q\in Qui\right\}$.
  \end{itemize}
\end{Def}

Es wird nur eine Semantik für die Ruhe definiert, die Error-Semantik wird aus
dem letzten Kapitel übernommen. Somit gelten für \ET{} und \EL{} die
Definitionen aus dem letzten Kapitel.

\begin{Def}[Ruhe-Semantik]
\label{DefQTQL}
  Sei $S$ ein \EIO{}.
  \begin{itemize}
    \item Die Menge der \emph{error-gefluteten Ruhetraces} von $S$ ist
      $\QT{}(S) := \StQT{}(S)\cup \ET{}(S)$.
  \end{itemize}
  Für zwei \EIO{}s $S_1, S_2$ mit der gleichen Signatur wird
  $S_1\QRel S_2$ geschrieben, wenn $S_1\ERel S_2$ und $\QT{}_1\subseteq
  \QT{}_2$ gilt.
\end{Def}

Für die Menge der error-gefluteten Ruhetraces \QT{} wurde eine Informationsvermischung
mit den Errortraces vorgenommen, wie beim Fluten der Sprache \EL{}. Da jedoch
durch die Ruhetraces keine neuen Traces entstehen, die nicht bereits in der
gefluteten Sprache \EL{} enthalten wären, würde eine neue Flutung von \EL{}
nichts ändern. Es wird also durch die Relation \QRel{} nur die
bereits existierende Präkongruenz \ERel{} eingeschränkt.\\
Das folgende Lemma soll explizit festhalten, wie Ruhezustände sich unter der
Parallelkomposition verhalten. Dies ist vor allem für den danach folgenden Satz
relevant.

\vspace{0.5cm}
\begin{lem}[Ruhe-Zustände unter Parallelkomposition]
\label{lemRuheParallelkomp}
  ~
  \begin{enumerate}
    \item Ein Zustand $(q_1,q_2)$ aus der Parallelkomposition $S_{12}$
      ist ruhig, wenn es auch die Zustände $q_1$ und $q_2$ in $S_1$ bzw.\ $S_2$
      sind.
    \item Wenn der Zustand $(q_1,q_2)$ ruhig ist und nicht in $E_{12}$
      enthalten ist, dann sind auch die auf die Teilsysteme projizierten
      Zustände $q_1$ und $q_2$ ruhig.
  \end{enumerate}
\end{lem}

\begin{proof}
  1.:\\
  Da $q_1\in Qui_1$ und $q_2\in Qui_2$ gilt, haben
  diese beiden Zustände jeweils höchstens die Möglichkeiten Transitionen
  auszuführen, die mit Inputs beschriftet sind, jedoch keine Möglichkeiten für
  Outputs oder $\tau$s.\\
  Angenommen der Zustand, der durch Parallelkomposition aus den Zuständen $q_1$
  und $q_2$ entsteht, ist nicht ruhig, d.h.\ er hat die Möglichkeit für eine
  Transition mit einem Output oder einem $\tau$.
  \begin{itemize}
    \item Fall 1 \big($(q_1,q_2) \overset{\tau}{\rightarrow}$\big): Ein $\tau$ ist eine
      interne Aktion und kann in dieser Arbeit nicht durch das Verbergen von
      Aktionen bei der Synchronisation entstehen. Ein $\tau$ in der
      Parallelkomposition ist also nur möglich, wenn dies bereits für einen der
      beiden Zustände ausführbar war, aus denen der Zustand zusammen gesetzt
      ist. Jedoch ist eine $\tau$-Transition für $q_1$ und $q_2$
      ausgeschlossen, deshalb kann auch $(q_1,q_2)$ keine solche Transition
      ausführen.
    \item Fall 2 \big($(q_1,q_2) \overset{a}{\rightarrow}$ mit $a\in O\backslash
      \Synch(S_1,S_2)$\big): Da es sich bei $a$ um einen Output handelt, der nicht
      in $\Synch(S_1,S_2)$ enthalten ist, kann dieser nicht aus der
      Synchronisation von zwei Aktionen entstanden sein, sondern muss bereits
      für $S_1$ oder $S_2$ ausführbar gewesen sein. Es gilt also \oBdA{} $q_1
      \overset{a}{\rightarrow}$ mit $a\in O_1$. Dies ist jedoch Aufgrund der
      Voraussetzungen nicht möglich. Somit kann die Parallelkomposition
      diese Transition für $(q_1,q_2)$ ebenfalls nicht ausführt werden.
    \item Fall 3 \big($(q_1,q_2) \overset{a}{\rightarrow}$ mit $a\in O\cap
      \Synch(S_1,S_2)$\big): Der Output $a$ ist in diesem Fall durch Synchronisation
      von einem Output mit einem Input entstanden. \OBdA{} gilt $a\in O_1\cap
      I_2$. Für die einzelnen Systeme muss also gelten, dass $q_1
      \overset{a}{\rightarrow}$ und $q_2 \overset{a}{\rightarrow}$. Die
      Transition für das System $S_1$ ist jedoch in der Voraussetzung
      ausgeschlossen worden. Somit ist es nicht möglich, dass $S_{12}$ diese
      in diesem Fall angenommene Transition für den Zustand $(q_1,q_2)$
      ausführen kann.
  \end{itemize}
  Da alle diese Fälle zu einem Widerspruch mit der Voraussetzung führen folgt,
  dass bereits die Annahme, dass der Zustand $(q_1,q_2)$ nicht ruhig ist,
  falsch war. Es folgt also, dass $(q_1,q_2)\in Qui_{12}$ gilt.

  2.:\\
  Es gilt $(q_1,q_2)\in Qui_{12}\backslash E_{12}$, somit hat
  dieser Zustand allenfalls die Möglichkeit Transitionen für Inputs
  auszuführen.\\
  Angenommen $q_1\notin Qui_1$, dann ist für $q_1$ entweder eine
  $\tau$-Transition oder eine Output-Transition möglich.
  \begin{itemize}
    \item Fall 1 \big($q_1 \overset{\tau}{\rightarrow}$\big): Da diese Transition für
      $S_1$ möglich ist, kann auch $S_{12}$ im Zustand $(q_1,q_2)$ diese
      Transition ausführen. Dies ist jedoch durch die Voraussetzung verboten.
      Somit gilt auch $q_1 \overset{\tau}{\not{\hspace{-0.1cm}\rightarrow}}$.
    \item Fall 2 \big($q_1 \overset{a}{\rightarrow}$ mit $a\in O_1\backslash
      \Synch(S_1,S_2)$\big): Da es sich bei $a$ um einen Output handelt, der nicht
      zu synchronisieren ist, wird dieser einfach in die Parallelkomposition
      übernommen. Es müsste also $(q_1,q_2) \overset{a}{\rightarrow}$ gelten,
      was jedoch verboten ist. Somit kann die Transition für $S_1$ in diesem
      Fall auch nicht möglich sein.
    \item Fall 3 \big($q_1 \overset{a}{\rightarrow}$ mit $a\in O_1\cap
      \Synch(S_1,S_2)$ und $q_2 \overset{a}{\rightarrow}$\big): In diesem Fall ist
      die Synchronisation des Outputs $a$ von $S_1$ mit dem Input $a$ von $S_2$
      möglich, so dass in der Parallelkomposition der Output $a$ als Transition
      für $(q_1,q_2)$ entsteht. Diese Transitionsmöglichkeit ist jedoch für
      $S_{12}$ nach Voraussetzung nicht erlaubt. Es folgt also auch, dass
      dieser Fall nicht eintreten kann.
    \item Fall 4 \Big($q_1 \overset{a}{\rightarrow}$ mit $a\in O_1\cap
      \Synch(S_1,S_2)$ und $q_2
      \overset{a}{\not{\hspace{-0.1cm}\rightarrow}}$\Big):
      Da $S_2$ diese Transition nicht ausführen kann, handelt es sich hier um
      einen neuen Error, da die Synchronisation des Outputs $a$ von $S_1$ mit
      einem Input $a$ von $S_2$ an dieser Stelle nicht möglich
      ist. Es würde also $(q_1,q_2)\in E_{12}$ gelten, dies wurde jedoch in der
      Voraussetzung bereits ausgeschlossen. Somit ist dieser Fall nicht
      möglich.
  \end{itemize}
  Alle aufgeführten Fälle führen zu einem Widerspruch mit der Voraussetzung,
  somit folgt, dass die Annahme bereits falsch war und $q_1\in Qui_1$ gelten
  muss. Analog kann für $q_2$ argumentiert werden, so dass dann auch $q_2\in
  Qui_2$ folgt.
\end{proof}

In dem folgenden Satz sind Punkt 1.\ und 3.\ nur zur Vollständigkeit aufgeführt.
Sie entsprechen Punkt 1.\ und 2.\ aus Satz~\ref{satzErrorSemanik}.

\begin{satz}[Error- und Ruhe-Semantik für Parallelkompositonen]
\label{satzQuiSemantik}
  Für zwei komponierbare \EIO{}s $S_1, S_2$ und ihre Komposition $S_{12}$ gilt:
  \begin{enumerate}
    \item $\ET{}_{12} = \cont{}\left(\prune{}\left(\left(\ET{}_1\|\EL{}_2\right)\cup
      \left(\EL{}_1\|\ET{}_2\right)\right)\right)$,
    \item $\QT{}_{12} = (\QT{}_1\|\QT{}_2)\cup \ET{}_{12}$,
    \item $\EL{}_{12} = (\EL{}_1\|\EL{}_2)\cup \ET{}_{12}$.
  \end{enumerate}
\end{satz}

\begin{proof}
  Es wird nur der 2. Punkt bewiesen.

  \glqq{}$\subseteq$\grqq{}:\\
  Hier muss unterschieden werden, ob ein $w\in\StQT{}_{12}\backslash \ET{}_{12}$
  oder ein $w\in \ET{}_{12}$ betrachtet wird. Im zweiten Fall ist das
  $w$ in der rechten Seite enthalten. Somit wird ab jetzt ein $w\in
  \StQT{}_{12}\backslash \ET{}_{12}$ betrachtet und es wird versucht dessen
  Zugehörigkeit zur rechten Menge zu zeigen. Aufgrund von
  Definition~\ref{DefRuhetraces} weiß man, dass $(q_{01},q_{02})
  \overset{w}{\Rightarrow} (q_1,q_2)$ gilt mit $(q_1,q_2)\in Qui_{12}\backslash
  E_{12}$. Durch Projektion erhält man $q_{01} \overset{w_1}{\Rightarrow} q_1$
  und $q_{02} \overset{w_2}{\Rightarrow} q_2$ mit $w\in w_1\|w_2$. Aus
  $(q_1,q_2)\in Qui_{12}\backslash E_{12}$ kann mit dem zweiten Punkt von
  Lemma~\ref{lemRuheParallelkomp} gefolgert werden, dass bereits $q_1\in Qui_1$
  und $q_2\in Qui_2$ gilt. Somit gilt $w_1\in \StQT{}_1\subseteq \QT{}_1$ und
  $w_2\in \StQT{}_2\subseteq \QT{}_2$. Daraus folgt dann $w\in
  \QT{}_1\|\QT{}_2$ und somit ist $w$ in der rechten Seiten der Gleichung
  enthalten.

  \glqq{}$\supseteq$\grqq{}:\\
  Es muss wieder danach unterschieden werden aus welcher Menge das betrachtete
  Element stammt. Falls $w\in \ET{}_{12}$ gilt, so kann die
  Zugehörigkeit zur linken Seite direkt gefolgert werden. Somit wird für den
  weiteren Beweis dieser Inklusionsrichtung ein Element $w\in \QT{}_1\|\QT{}_2$
  betrachtet und gezeigt, dass es in der linken Menge enthalten ist. Da
  $\QT{}_i = \StQT{}_i\cup \ET{}_i$ gilt, existieren für $w_1$ und $w_2$ mit
  $w\in w_1\| w_2$ unterschiedliche Möglichkeiten:
  \begin{itemize}
    \item Fall 1 ($w_1\in \ET{}_1\vee w_2\in \ET{}_2$): \OBdA{} gilt
      $w_1\in \ET{}_1$. Nun kann $w_2\in \StQT{}_2\subseteq L_2$ gelten
      oder $w_2\in \ET{}_2$ und somit gilt auf jeden Fall $w_2\in
      \EL{}_2$. Daraus kann dann mit dem ersten Punkt von
      Satz~\ref{satzErrorSemanik} gefolgert werden, dass $w\in \ET{}_{12}$ gilt
      und somit $w$ in der linken Seite der Gleichung enthalten ist.
    \item Fall 2 ($w_1\in \StQT{}_1\backslash\ET{}_1\wedge w_2\in
      \StQT{}_2\backslash\ET{}_2$): Es gilt in diesem Fall $q_{01}
      \overset{w_1}{\Rightarrow} q_1\in Qui_1$ und $q_{02}
      \overset{w_2}{\Rightarrow} q_2\in Qui_2$. Da $q_1$ und $q_2$ in der
      jeweiligen Ruhe-Menge enthalten sind, ist auch der Zustand, der aus ihnen
      zusammengesetzt ist, in der Parallelkomposition ruhig, wie bereits in
      ersten Punkt von
      Lemma~\ref{lemRuheParallelkomp} gezeigt. Es gilt also für die Komposition
      $(q_{01},q_{02}) \overset{w}{\Rightarrow} (q_1,q_2)\in Qui_{12}$ und
      dadurch ist $w$ in der linken Seite der Gleichung enthalten, da $w\in
      \StQT{}_{12}\subseteq \QT{}_{12}$ gilt.
  \end{itemize}
\end{proof}

Die folgende Proposition ist eine direkte Folgerung aus dem letzten Satz.
Jedoch ist es eine wichtige Feststellung um zu zeigen, dass es sich bei der
Relation \QRel{} um die gröbste Präkongruenz handeln könnte, die in diesem
Kapitel charakterisiert werden soll.

\begin{prop}[Ruhe-Präkongruenz]
\label{propQuiPrae}
  \QRel{} ist eine Präkongruenz bezüglich $\cdot\|\cdot$.
\end{prop}

\begin{proof}
  Es muss gezeigt werden: Wenn $S_1\QRel S_2$ gilt, so auch
  $S_{31}\QRel S_{32}$ für jedes komponierbare System $S_3$. D.h.\ es ist zu zeigen, dass aus
  $S_1\ERel S_2$ und $\QT{}_1\subseteq \QT{}_2$ sowohl $S_{31}\ERel S_{32}$ als
  auch $\QT{}_{31}\subseteq \QT{}_{32}$ folgt. Dies ergibt sich, wie im Beweis
  zu Proposition~\ref{propPraekongruenz}, aus der Monotonie von $\cdot\|\cdot$
  auf Sprachen wie folgt:
  \begin{itemize}
    \item $\begin{aligned}[t]
        S_{31} \overset{\mathrm{Proposition}
        ~\ref{propPraekongruenz}}{\overset{\mathrm{und}}{\overset{S_1\ERel
    S_2}{\ERel}}} S_{32},
    \end{aligned}$
    \item $\begin{aligned}[t]
        \QT{}_{31} &\overset{\ref{satzQuiSemantik}~2.}{=}
        (\QT{}_3\|\QT{}_1)\cup \ET{}_{31}\\
        &\hspace{-0.5cm}\overset{\ET{}_{31}\subseteq
      \ET{}_{32}}{\overset{\mathrm{und}}{\overset{\QT{}_1\subseteq
      \QT{}_2}{\subseteq}}} (\QT{}_3\|\QT{}_2) \cup \ET{}_{32}\\
        &\overset{\ref{satzQuiSemantik}~2.}{=} \QT{}_{32}.
    \end{aligned}$
  \end{itemize}
\end{proof}

Im nächsten Lemma soll eine Verfeinerungsrelation bezüglich guter Kommunikation
mit Partnern im Sinne von error- und ruhe-freier Kommunikation betrachtet
werden.

\begin{lem}[Verfeinerung mit Ruhe-Zuständen]
\label{lemQuiVerfeinerung}
  Gegeben sind zwei \EIO{}s $S_1$ und $S_2$ mit der gleichen Signatur. Wenn
  $U\|S_1\QBRel U\|S_2$ für alle Partner $U$ gilt, dann folgt daraus $S_1\QRel
  S_2$.
\end{lem}

\begin{proof}
  Da davon ausgegangen wird, dass $S_1$ und $S_2$ die gleiche Signatur haben,
  definiert man $I:=I_1=I_2$ und $O:=O_1=O_2$. Für jeden Partner $U$ gilt
  $I_U=O$ und $O_U=I$.\\
  Um zu zeigen, dass die Relation $S_1\QRel S_2$ gilt, müssen die
  folgenden Punkte nachgewiesen werden:
  \begin{itemize}
    \item $S_1\ERel S_2$,
    \item $\QT{}_1\subseteq \QT{}_2$.
  \end{itemize}
  In Lemma~\ref{lemVerfeinerung} wurde bereits etwas Ähnliches gezeigt, jedoch
  wurde dort als Voraussetzung $U\|S_1\EBRel U\|S_2$ für alle Partner $U$
  verwendet und hier dieselbe Aussage mit der Basisrelation der Ruhe. Dadurch,
  dass die hier verwendete Basisrelation nichts über die Art des erreichbaren
  Fehlers in den Komponenten aussagt, kann der Beweis aus Lemma~\ref{lemVerfeinerung}
  nicht verwendet werden. Es kann also aus der lokalen Erreichbarkeit eines
  Errors in $S_1'$ und dem relationalen Zusammenhang von $S_1'\QBRel S_2'$ nur
  geschlossen werden, dass in $S_2'$ auch ein Fehler lokal erreichbar ist,
  jedoch kann dieser Fehler Error oder Ruhe sein. Analog
  verhält es sich, wenn in $S_1'$ ein Ruhe-Zustand lokal erreichbar ist.\\
  Es muss also für den ersten Punkt noch folgendes nachgewiesen werden:
  \begin{itemize}
    \item $\ET{}_1\subseteq \ET{}_2$,
    \item $\EL{}_1\subseteq \EL{}_2$.
  \end{itemize}
  Es wird nun damit begonnen, den ersten Unterpunkt des ersten Beweispunktes zu
  zeigen, d.h.\ es wird unter der Voraussetzung $U\|S_1\QBRel U\|S_2$ gezeigt,
  dass $\ET{}_1\subseteq \ET{}_2$ gilt. Da beide \ET{}-Mengen
  unter \cont{} abgeschlossen sind, reicht es ein präfix-minimales Element
  $w\in\ET{}_1$ zu betrachten und zu zeigen, dass dieses $w$ oder eines seiner
  Präfixe in $\ET{}_2$ enthalten ist.
  \begin{itemize}
    \item Fall 1 ($w=\varepsilon$): Es handelt sich um einen lokal erreichbaren
      Error in $S_1$. Für $U$ wird ein Transitionssystem verwendet, das nur aus
      dem Startzustand, einer Schleife für alle Inputs $x\in I_U$ und einer
      Schlinge für $\tau$ besteht. Somit kann $S_1$ die im Prinzip gleichen
      Error-Zustände lokal erreichen wie $U\|S_1$. Es folgt also, dass in $U\|S_2$ ein Fehler lokal
      erreichbar ist. Es kann sich bei dem Fehler nur um einen Error handeln,
      da es in der Komposition mit $U$ keine Ruhe-Zustände geben kann. Da $U$
      keinen Error-Zustand und auch keine fehlenden Input-Möglichkeiten
      enthält, kann der Error nur von $S_2$ geerbt sein. Somit muss in $S_2$
      ein Error-Zustand lokal erreichbar sein, d.h.\ es gilt $\varepsilon\in
      \PrET{}_2\subseteq \ET{}_2$.
    \item Fall 2 ($w=x_1\dots x_nx_{n+1}\in\Sigma{} ^+$ mit $n\geq 0$ und
      $x_{n+1}\in I$): Es wird der folgende Partner $U$ betrachtet (siehe auch
      Abbildung~\ref{UohneEmitTau}):
      \begin{itemize}
        \item $Q_U=\{q_0,q_1,\dots ,q_{n+1}\}$,
        \item $q_{0U}=q_0$,
        \item $E_U=\emptyset$,
        \item $\begin{aligned}[t]
            \delta _U=&\{(q_i,x_{i+1},q_{i+1})\mid  0\leq i\leq n\}\\
                      &\cup\{(q_i,x,q_{n+1})\mid  x\in I_U\backslash\{x_{i+1}\},
          0\leq i\leq n\}\\
          &\cup\{(q_{n+1},x,q_{n+1})\mid  x\in I_U\}\\
          &\cup\{(q_i,\tau,q_i)\mid 0\leq i\leq n+1\}.
        \end{aligned}$
      \end{itemize}
      \begin{figure} [h!tbp]
      \begin{center}
        \begin{tikzpicture}[->, >=latex',auto,node distance =3cm, semithick]
          \node (0) {$q_0$};
          \node (1) [right of=0] {$q_1$};
          \node (dots) [right of=1] {$\dots$};
          \node (n) [right of=dots] {$q_n$};
          \node (n1) at ($(1)!0.5!(dots) + (0,-3)$) {$q_{n+1}$};

          \path ($ (0) + (-1,0) $) edge (0)
                (0) edge node {$x_1$} (1)
                    edge [bend right] node [below, sloped] {$x?\neq x_1$} (n1)
                    edge [loop above] node {$\tau$} (0)
                (1) edge node {$x_2$} (dots)
                    edge node [below, sloped] {$x?\neq x_2$} (n1)
                    edge [loop above] node {$\tau$} (1)
                (dots) edge node {$x_n$} (n)
                       edge [dashed] (n1)
                (n) edge node [above, sloped] {$x?\in I_U$} (n1)
                    edge [bend left] node [sloped] {$x_{n+1}$!} (n1)
                    edge [loop above] node {$\tau$} (n)
                (n1) edge [loop below] node {$x?\in I_U, \tau$} (n1);
        \end{tikzpicture}
        \caption{$x?\neq x_i$ steht für alle $x\in I_U\backslash\{x_i\}$}
\label{UohneEmitTau}
      \end{center}
      \end{figure}
      Die Menge der Ruhe-Zustände des hier betrachteten $U$s ist leer. Da im
      Vergleiche zum Transitionssystem in Abbildung~\ref{UohneE} nur die
      $\tau$-Schlingen ergänzt wurden, ändert sich nichts an den Fällen 2a) und
      2b). Die Begründungen, wieso in den beiden Fällen $\varepsilon\in
      \PrET{}(U\|S_1)$ gilt, bleibt also analog zum Beweis  des ersten Punktes von
      Lemma~\ref{lemVerfeinerung}. Durch die $\tau$-Schlingen wurde, genau wie
      im letzten Fall, nur erreicht, dass in einer Parallelkomposition mit $U$
      keine Ruhe-Zustände möglich sind. Es kann also auch hier aus der lokalen
      Erreichbarkeit eines Error-Zustandes in $U\|S_1$ auf die lokale
      Erreichbarkeit eines Errors in $U\|S_2$ geschlossen werden. Die weitere
      Argumentation verläuft dann analog zu Fall 2, derselben Inklusion im
      Beweis zu Lemma~\ref{lemVerfeinerung}. Da $\tau$s nur interne Aktionen
      eines einzelnen Systems sind, verändert sich auch nichts an den Traces
      über die argumentiert wird. Es können zwar möglicherweise
      $\tau$-Transitionen ausgeführt werden, diese können jedoch weder zu einem
      Fehler führen noch beeinflussen, dass ein anderes Trace nicht ausgeführt
      werden kann.
  \end{itemize}

  Nun wird mit dem zweiten Unterpunkt des ersten Beweispunktes begonnen. Genau
  wie im Beweis zu~\ref{lemVerfeinerung} ist hier jedoch aufgrund des bereits
  geführten Beweisteils nur noch $L_1\backslash \ET{}_1\subseteq
  \EL{}_2$ zu zeigen. Es wird also für ein beliebig gewähltes $w\in
  L_1\backslash \ET{}_1$ gezeigt, dass dieses auch in $\EL{}_2$ enthalten
  ist.
  \begin{itemize}
    \item Fall 1 ($w=\varepsilon$): Ebenso wie in~\ref{lemVerfeinerung} gilt
      auch hier, dass $\varepsilon$ immer in $\EL{}_2$ enthalten ist.
    \item Fall 2 ($w=x_1\dots x_n$ mit $n\geq 1$): Die Konstruktion des
      Partners $U$ weicht wie im letzten Beweisteil nur durch die
      $\tau$-Schleifen an den Zuständen des Transitionssystems vom Beweis es
      zweiten Punktes aus
      Lemma~\ref{lemVerfeinerung} ab. Somit ist der Partner $U$ dann wie folgt
      definiert (siehe dazu auch Abbildung~\ref{UmitEundTau}):
      \begin{itemize}
        \item $Q_U=\{q_0,q_1,\dots ,q_n,q\}$,
        \item $q_{0U}=q_0$,
        \item $E_U=\{q_n\}$,
        \item $\begin{aligned}[t]
            \delta _U=&\{(q_i,x_{i+1},q_{i+1})\mid 0\leq i< n\}\\
                      &\cup\{(q_i,x,q)\mid x\in I_U\backslash\{x_{i+1}\},0\leq
          i < n\}\\
          &\cup\{(q_i,\tau ,q_i)\mid 0\leq i\leq n\}\\
          &\cup\{(q,\alpha ,q)\mid \alpha\in I_U\cup \{\tau\}\}.
              \end{aligned}$
      \end{itemize}
      \begin{figure} [h!tbp]
      \begin{center}
        \begin{tikzpicture}[->, >=latex',auto,node distance =3cm, semithick]

          \node (0) {$q_0$};
          \node (1) [right of=0] {$q_1$};
          \node (dots) [right of=1] {$\dots$};
          \node (n1) [right of=dots] {$q_{n-1}$};
          \node (n) [right of=n1, rectangle, draw] {$q_n\in E_U$};
          \node (q) at ($(1)!0.5!(dots) + (0,-3)$) {$q$};

          \path ($ (0) + (-1,0) $) edge (0)
                (0) edge node {$x_1$} (1)
                    edge [bend right] node [below, sloped] {$x?\neq x_1$} (q)
                    edge [loop above] node {$\tau$} (0)
                (1) edge node {$x_2$} (dots)
                    edge node [below, sloped] {$x?\neq x_2$} (q)
                    edge [loop above] node {$\tau$} (1)
                (dots) edge node {$x_{n-1}$} (n1)
                       edge [dashed] (q)
                (n1) edge node {$x_n$} (n)
                     edge [bend left] node [below, sloped] {$x?\neq x_n$} (q)
                     edge [loop above] node {$\tau$} (n1)
                (q) edge [loop below] node {$x?\in I_U, \tau$} (q)
                (n) edge [loop above] node {$\tau$} (n);
        \end{tikzpicture}
        \caption{$x?\neq x_i$ steht für alle $x\in I_U\backslash\{x_i\}$, $q_n$
          ist der einzige Error-Zustand}
\label{UmitEundTau}
      \end{center}
      \end{figure}
      Da durch die $\tau$-Schlingen an den Zuständen wie oben vermieden wird,
      dass es in einer Komposition mit $U$ und auch in $U$ selbst Ruhe-Zustände
      gibt, verläuft der Rest des Beweises dieses Punktes analog zum Beweis der
      selben Inklusionsrichtung von
      Lemma~\ref{lemVerfeinerung}. Und somit gilt für alle Fälle (2a) bis 2d)),
      dass $w$ in $\EL{}_2$ enthalten ist.
  \end{itemize}

  So bleibt nur noch der letzte Beweispunkt zu zeigen, d.h.\ die Inklusion
  $\QT{}_1\subseteq \QT{}_2$. Diese Inklusion kann jedoch, anlog zum Beweis
  der Inklusion der error-gefluteten Sprachen, noch weiter eingeschränkt werden.
  Da bereits bekannt ist, dass $\ET{}_1\subseteq\ET{}_2$ gilt, muss nur
  noch $\StQT{}_1\backslash \ET{}_1\subseteq \QT{}_2$ gezeigt werden.\\
  Es wird ein $w\in \StQT{}_1\backslash\ET{}_1$ gewählt und gezeigt, dass
  es auch in $\QT{}_2$ enthalten ist.\\
  Durch die Wahl des $w$s wird vom Startzustand von $S_1$ durch das Wort $w$
  ein ruhiger Zustand erreichbar. Dies hat nur Auswirkungen auf die
  Parallelkomposition $U\|S_1$, wenn in $U$ ebenfalls ein Ruhe-Zustand durch
  $w$ erreichbar ist.\\
  Das betrachtete $w$ hat also die Form $w=x_1\dots x_n\in \Sigma ^*$ mit
  $n\geq 0$. Es wird der folgende Partner $U$ betrachtet (siehe auch
  Abbildung~\ref{UohneEmitI}):
  \begin{itemize}
    \item $Q_U=\{q_0,q_1,\dots ,q_n, q\}$,
    \item $q_{0U}=q_0$,
    \item $E_U=\emptyset$,
    \item $\begin{aligned}[t]
        \delta _U=&\{(q_i,x_{i+1},q_{i+1})\mid  0\leq i< n\}\\
                  &\cup\{(q_i,x,q)\mid  x\in I_U\backslash\{x_{i+1}\}, 0\leq i< n\}\\
                  &\cup\{(q_i,\tau,q_i)\mid 0\leq i< n\}\\
                  &\cup\{(q_n,x,q)\mid x\in I_U\}\\
                  &\cup\{(q,\alpha,q)\mid \alpha\in I_U\cup\{\tau\}\}.
    \end{aligned}$
  \end{itemize}
  \begin{figure} [h!tbp]
  \begin{center}
    \begin{tikzpicture}[->, >=latex',auto,node distance =3cm, semithick]
      \node (0) {$q_0$};
      \node (1) [right of=0] {$q_1$};
      \node (dots) [right of=1] {$\dots$};
      \node (n) [right of=dots, rectangle, dashed, draw] {$q_n\in Qui_U$};
      \node (q) at ($(1)!0.5!(dots) + (0,-3)$) {$q$};

      \path ($ (0) + (-1,0) $) edge (0)
            (0) edge node {$x_1$} (1)
                edge [loop above] node {$\tau$} (0)
                edge [bend right] node [below, sloped] {$x?\neq x_1$} (q)
            (1) edge node {$x_2$} (dots)
                edge [loop above] node {$\tau$} (1)
                edge [below, sloped] node {$x?\neq x_2$} (q)
            (dots) edge node {$x_n$} (n)
                   edge [dashed] (q)
            (n) edge [bend left] node [below,sloped] {$x?\in I_U$} (q)
            (q) edge [loop below] node {$x?\in I_U, \tau$} (q);
    \end{tikzpicture}
    \caption{$x?\neq x_i$ steht für alle $x\in I_U\backslash\{x_i\}$, $q_n$
    ist der einzige Ruhe-Zustand}
\label{UohneEmitI}
  \end{center}
  \end{figure}
  Falls für das betrachtete $w =\varepsilon$ gilt, reduziert sich der Partner
  $U$ auf den Zustand $q_n =q_0$ und den Zustand $q$. Es ist also in diesem
  Fall der Startzustand gleich dem ruhigen Zustand.\\
  Allgemein ist der Zustand $q_n$ aus $U$ ist der einzige ruhige Zustand in $U$. Es gilt
  wegen des ersten Punktes von Lemma~\ref{lemRuheParallelkomp}, dass auch in
  der Parallelkomposition $U\|S_1$ ein Ruhe-Zustand mit $w$ erreicht wird. Da
  es sich bei allen in $w$ befindlichen Aktionen um synchronisierte
  Aktionen handelt und $I_U\cap I=\emptyset$, folgt $w\in O_{U\|S_1}^*$ und
  $w\in \StQT{}(U\|S_1)$. Es kann also in der Parallelkomposition durch $w$
  ein Ruhe-Zustand lokal erreicht werden. Da $w\notin \ET{}_1$ gilt, kann
  auf dem Weg, der mit $w$ im Transitionssystem $S_1$ zurückgelegt wird,
  kein Error lokal erreicht werden. Es kann also weder von $S_1$ noch von
  $U$ ein Error auf diesem Weg geerbt werden und durch den Aufbau von $U$
  kann auch kein neuer Error in der Parallelkomposition beider Systeme
  entstehen. Da ein Ruhe-Zustand in $U\|S_1$
  lokal erreichbar ist, muss auch ein Fehler in $U\|S_2$ lokal erreichbar
  sein. Hier kann jedoch zunächst keine Aussage darüber getroffen werden,
  ob das $w$ ausführbar ist und ob es sich bei dem Fehler um Ruhe oder
  Error handelt.
  \begin{itemize}
    \item Fall a) ($\varepsilon\in \ET{}(U\|S_2)$): Es handelt sich bei
      dem lokal erreichbaren Fehler um einen Error. Es ist somit nicht
      relevant, ob das $w$ ausführbar ist. Der Error kann sowohl von $S_2$
      geerbt sein, wie auch durch fehlende Synchronizationsmöglichkeiten
      als neuer Error in der Parallelkomposition entstanden sein. Es gilt
      also, dass bereits in $S_2$ ein Präfix von $w$ in $\ET{}_2$ enthalten
      ist, wegen des Beweises des ersten Punktes aus
      Lemma~\ref{lemVerfeinerung} und da $U$ nur Synchronisationsfehler
      auf dem Trace $w$ zulässt. Da die Menge \ET{} unter \cont{}
      abgeschlossen ist, gilt also auch $w\in \ET{}_2\subseteq \QT{}_2$.
    \item Fall b) (Ruhe-Zustand lokal erreichbar in $U\|S_2$ und
      $\varepsilon\notin \ET{}(U\|S_2)$): Da in $U$
      nur durch $w$ ein ruhiger Zustand erreicht werden kann, muss es sich
      bei dem lokal erreichbaren Ruhe-Zustand in $U\|S_2$ um einen handeln,
      der mit $w$ erreicht werden kann. Mit Lemma~\ref{lemRuheParallelkomp}
      kann somit gefolgert werden, dass auch in $S_2$ ein Ruhe-Zustand mit
      $w$ erreichbar sein muss. Es gilt also $w\in \StQT{}_2\subseteq
      \QT{}_2$.
  \end{itemize}
\end{proof}

Mit dem folgenden Satz wird festgehalten, dass mit \QRel{} die gröbste
Präkongruenz charakterisiert wurde bezüglich $\cdot\|\cdot$, die in \QBRel{}
enthalten ist.

\begin{satz}[Vollständige Abstraktheit für Ruhe-Semantik]
\label{satzQuiFullAbst}
  Seien $S_1$ und $S_2$ zwei \EIO{}s mit derselben Signatur. Dann gilt
  $S_1\QCRel S_2\Leftrightarrow S_1\QRel S_2$.
\end{satz}

\begin{proof}
  \glqq{}$\Leftarrow$\grqq{}: Nach Definition gilt $w\in\QT{}(S)$ mit $w\in
  O^*$ genau dann, wenn in $S$ ein Ruhe-Zustand oder ein Error-Zustand lokal
  erreichbar ist. $S_1\QRel S_2$ impliziert, dass $w\in\QT{}_2$ gilt, wenn
  $w\in\QT{}_1$ gilt. Somit ist ein Ruhe-Zustand oder ein Error-Zustand nur dann in
  $S_1$ lokal erreichbar, wenn auch ein solcher in $S_2$ lokal erreichbar ist.
  Daraus folgt, dass $S_1\QBRel S_2$ gilt. Es ist also \QRel{} in \QBRel{}
  enthalten. In Proposition~\ref{propQuiPrae} wurde festgestellt, dass \QRel{}
  eine Präkongruenz ist. Da jedoch \QCRel{} nach Definition~\ref{DefQuiBasisrel} die gröbste
  Präkongruenz bezüglich $\cdot\|\cdot$ ist, die in \QBRel{} enthalten ist,
  muss \QRel{} in \QCRel{} enthalten sein. Es folgt also aus $S_1\QRel S_2$,
  dass auch $S_1\QCRel S_2$ gilt.

  \glqq{}$\Rightarrow$\grqq{}: Durch die Definition von \QCRel{} als
  Präkongruenz in~\ref{DefQuiBasisrel} folgt aus $S_1\QCRel{} S_2$, dass
  $U\|S_1\QCRel U\|S_2$ für alle \EIO{}s $U$ gilt, die mit $S_1$ komponierbar sind.
  Da \QCRel{} nach Definition in \QBRel{} enthalte ist, folgt auch die
  Gültigkeit von $U\|S_1\QBRel U\|S_2$ für alle diese \EIO{}s
  $U$. Mit Lemma~\ref{lemQuiVerfeinerung} folgt dann $S_1\QRel{} S_2$.
\end{proof}

Es wurde somit, wie im letzten Kapitel, eine Kette an Folgerungen gezeigt, die
sich zu einem Ring schließen. Dies ist in Abbildung~\ref{FolgerungsketteQui}
dargestellt.

\begin{figure}[h!tbp]
  \begin{center}
    \begin{tikzpicture}
      \matrix (m) [matrix of math nodes,row sep=2cm,column sep=4cm]{%
        S_1\QRel S_2 & S_1\QCRel S_2 \\
        \substack{\forall~\mathrm{Partner}~U:\\U\|S_1\QBRel U\|S_2} &
      \substack{\forall~\mathrm{komponierbaren}~U:\\U\|S_1\QBRel U\|S_2} \\};
        \draw[-implies, double, double distance=1mm]
          (m-1-1) -- node [above] {\glqq{}$\Leftarrow$\grqq{} von
            Satz~\ref{satzQuiFullAbst}} (m-1-2);
        \draw[-implies, double, double distance=1mm]
          (m-1-2) -- node [right] {Definition von \QCRel{}
          in~\ref{DefQuiBasisrel}} (m-2-2);
        \draw[-implies, double, double distance=1mm]
          (m-2-1) -- node [left]
          {Lemma~\ref{lemQuiVerfeinerung}} (m-1-1);
        \draw[-implies, double, double distance=1mm]
        (m-2-2) -- node [below]
        {$\substack{U~\mathrm{Partner}\\\Downarrow\\U~\mathrm{komponierbar}}$} (m-2-1);
    \end{tikzpicture}
    \caption{Folgerungskette}
\label{FolgerungsketteQui}
  \end{center}
\end{figure}

Aus Satz~\ref{satzQuiFullAbst} und Lemma~\ref{lemQuiVerfeinerung} erhält man
das folgende Korollar. Angenommen man definiert, dass $S_1$ $S_2$ verfeinern
soll genau dann, wenn für alle Partner \EIO{}s $U$ für die $S_2$ error- und
ruhe-frei mit $U$ kommuniziert, folgt, dass $S_1$ ebenfalls error- und
ruhe-frei mit $U$ kommuniziert. Dann wird auch diese Verfeinerung durch \QRel{}
charakterisiert.

\begin{kor}
  Es gilt: $S_1\QRel S_2\Leftrightarrow U\|S_1\QBRel U\|S_2$ für alle Partner
  $U$.
\end{kor}

\section{Hiding und Ruhe-Freiheit}

Es soll nun auch hier die Auswirkungen der Internalisierung von Aktionen auf
die Verfeinerungsrelationen untersucht werden. Es werden Outputs in interne
Aktionen umgewandelt. Da jedoch bei den Ruhe-Zuständen auch
$\tau$-Transitionen verboten wurden, verändert sich nichts an der Menge der
ruhigen Zustände. Da die Erreichbarkeit von Ruhe-Zuständen mittels lokaler
Aktionen betrachtet wurde, kann sich auch nichts an der Erreichbarkeit der
Ruhe-Zustände ändern. Somit kann eine analoge Proposition zu~\ref{propErBaIn}
formuliert werden.

\begin{prop}[Ruhe-Basisrelation bzgl. Internalisierung]
  Wenn $S_1\QBRel S_2$ gilt, dann folgt daraus, dass auch $S_1/X\QBRel S_2/X$
  gilt.
\end{prop}

\begin{proof}
  Dass die Error-Erreichbarkeit unverändert bleibt unter Hiding, wurde bereits
  im Beweis zu Proposition~\ref{propErBaIn} gezeigt. Mit der analogen
  Argumentation folgt auch, dass sich nichts an der Erreichbarkeit der
  Ruhe-Zustände ändert. Es können durch Hiding nur Outputs verborgen
  werden, die bereits in der Menge der lokalen Aktionen enthalten sind. Die
  Menge der Ruhe-Zustände kann sich durch das Internalisieren nicht
  vergrößern oder verkleinern, wie oben bereits festgestellt. Also gilt: Wenn
  $S_i$ einen Error oder Ruhe-Zustand lokal erreichen kann, dann kann das auch
  $S_i/X$ und umgekehrt, da dabei nur $\tau$s durch Outputs ersetzt werden. Es
  folgt also aus der Relation $S_1 \QBRel S_2$ auch der Zusammenhang $S_1/X
  \QBRel S_2/X$.
\end{proof}

\begin{satz}[Ruhe-Präkongruenz bzgl. Internalisierung]
\label{satzPraeInterQui}
  Seien $S_1$ und $S_2$ zwei \EIO{}s für die $S_1\QRel S_2$ gilt, somit gilt
  auch $S_1/X\QRel S_2/X$. Es ist also \QRel{} eine Präkongruenz bezüglich
  $\cdot /\cdot$. Es gilt für die Sprachen und Traces:
  \begin{compactenum}[(i)]
  \item $L(S/X) = \left\{w\in (\Sigma\backslash X){}^*\mid \exists w'\in L(S):
      w'|_{\Sigma\backslash X} = w\right\}$,
    \item $\ET{}(S/X) = \left\{w\in (\Sigma\backslash X){}^*\mid \exists
      w'\in \ET{}(S): w'|_{\Sigma\backslash X} = w\right\}$,
    \item $\EL{}(S/X) = \left\{w\in (\Sigma\backslash X){}^*\mid \exists w'\in
      \EL(S): w'|_{\Sigma\backslash X} = w\right\}$,
    \item $\QT{}(S/X) = \left\{w\in (\Sigma\backslash X){}^*\mid \exists w'\in
      \QT{}(S):w'|_{\Sigma\backslash X}=w\right\}$.
  \end{compactenum}
\end{satz}

\begin{proof}
  Als Erstes sollte man sich von der Richtigkeit der Aussagen (i) bis (iv)
  überzeugen. Da jedoch im Beweis zu Satz~\ref{satzPraeInternalisierung}
  bereits die ersten drei Punkte gezeigt wurden, muss hier nur noch (iv)
  betrachtet werden.

  (iv)
  Die Korrektheit dieser Aussage kann analog zum Beweis der Punkte (i) bis
  (iii) im Satz~\ref{satzPraeInternalisierung} gezeigt werden. Für ein $w'\in
  \QT{}(S)$ gilt $w'\in\ET{}(S)$ oder $w'\in\StQT{}(S)\subseteq L(S)$.
  Es gilt also mit Punkt (i) und (ii), dass $w:=w'|_{\Sigma\backslash X}$ in
  $\ET{}(S/X)$ oder in $\StQT{}(S/X)$ enthalten ist. Für ein $w'\in\StQT{}(S)$
  ist jedoch anzumerken, dass im Beweis zu Punkt (i) durch die Einschränkung
  des Traces immer noch der gleiche Zustand erreicht wird. Es folgt also
  insgesamt, dass $w:=w'_{\Sigma\backslash X}\in \QT{}(S/X)$ gilt.\\
  Für ein $w\in\QT{}(S/X)$ kann, ebenfalls wie im Beweis zu Punkt (i) und
  (ii),
  durch Ersetzen von $\tau$s im jeweiligen Ablauf ein Ablauf gefunden werden,
  der in $S$ enthalten ist. Es gibt also eine Erweiterung $w'$ von $w$ um
  Aktionen aus $X$, sodass diese in $\QT{}(S)$ enthalten ist.

  Da $S_1\QRel S_2$ gilt, kann geschlossen werden, dass $S_1\ERel S_2$ und
  $\QT{}_1\subseteq \QT{}_2$ gilt. Aufgrund von
  Satz~\ref{satzPraeInternalisierung} ist bekannt, dass daraus $S_1/X\ERel
  S_2/X$ folgt. Mit der Argumentation für den Punkt (iv) von oben, kann aus der
  Voraussetzung $\QT{}_1\subseteq \QT{}_2$ ebenfalls $\QT{}(S_1/X)\subseteq
  \QT{}(S_2/X)$ gefolgert werden.\\
  Es folgt also insgesamt, dass die Relation \QRel{} trotz Hiding erhalten
  bleibt und somit bezüglich des Hiding diese Relation eine Präkongruenz ist.
\end{proof}

In Definition~\ref{defIntParal} wurde mit Hilfe des Internalisierungsoperator
aus der Parallelkomposition ohne Verbergen die Parallelkomposition mit
Verbergen der synchronisierten Aktionen nachgebildet. Es kann deren
Eigenschaft als Präkongruenz aus den Präkongruenz\-eigenschaften von
$\cdot\|\cdot$ und $\cdot /\cdot$ bezüglich \QRel{} aus der
Proposition~\ref{propQuiPrae} und dem Satz~\ref{satzPraeInterQui} geschlossen
werden.

\begin{kor}[Ruhe-Präkongruenz mit Internalisierung]
  \QRel{} ist eine Präkongruenz bezüglich $\cdot |\cdot$.
\end{kor}
