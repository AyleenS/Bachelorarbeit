\chapter{Verfeinerung über Error- und Ruhetraces}

\section{Präkongruenz für Ruhe}

In diesem Kapitel werden wir uns nicht nur um die Erreichbarkeit von
Error-Zuständen kümmern, sondern auch um die Erreichbarkeit von
Ruhe-Zuständen. Wir werden dabei ähnlich vorgehen wie im letzten Kapitel,
jedoch halten wir uns als Quelle an~\cite{Chilton2013}. Darin werden ähnliche
Konzepte beschrieben, jedoch aus Sicht der Traces. Es werden dort zudem
gleichzeitig auch noch Traces mit Divergenz betrachtet. Diese Zustands Art
wollen wir hier zunächst nicht betrachten.\\
Wir sehen die Zustände, die keine Outputs und keine Transitionsmöglichkeit für
eine interne Aktion haben als ruhig an.

\begin{Def}[Ruhe]
  Ein \emph{Ruhe-Zustand} ist ein Zustand in einem \EIO{} der keine
  Outputs zulässt und keine Transitions mit $\tau$ besitzt.\\
  Somit ist die Menge der Ruhe-Zustände in einem \EIO{} wie folgt formal
  definiert: $Qui:=\big\{q\in Q\mid \forall \alpha\in (O\cup \{\tau\}): q
  \overset{\alpha}{\not{\hspace{-0.1cm}\rightarrow}}\big\}$.
\end{Def}

Für die Erreichbarkeit verwenden wir wie im letzten Kapitel wieder den
optimistischen Ansatz der lokalen Erreichbarkeit für die Error-Zustände
verwenden. Ruhe ist kein unabwendbaren Fehler, sondern kann durch einen Input
reparierbar werden. Somit ist ein Ruhe-Zustand als nicht so \glqq{}schlimmer
Fehler\grqq{} anzusehen wie ein Error. Somit ist für uns ein Ruhe-Zustand im
Gegensatz zu ein Error-Zustand erst dann erreichbar wenn er durch $\tau$s
erreicht werden kann.

\begin{Def}[error- und ruhe-freie Kommunikation]
  Zwei \EIO{}s $S_1$ und $S_2$ \emph{kommunizieren error- und ruhe-frei}, wenn
  in ihrer Parallelkomposition $S_1\| S_2$ keine Errors lokal und keine
  Ruhe-Zustände durch interne Aktionen erreichbar sind.
\end{Def}

\begin{Def}[Ruhe-Verfeinerungs-Basisrelation]
  Für \EIO{}s $S_1$ und $S_2$ mit der gleichen Signatur schreiben wir
  $S_1\QBRel S_2$, wenn ein Error oder Ruhe-Zustand in $S_1$ nur dann lokal
  bzw.\ durch eine interne Aktion erreichbar ist, wenn er auch in $S_2$ lokal
  bzw.\ durch eine interne erreichbar ist. Diese \emph{Basisrelation} stellt
  eine \emph{Verfeinerung} bezüglich \emph{Errors} und \emph{Ruhe-Zustände}
  dar.\\
  \QCRel{} bezeichnet die \emph{vollständig abstrakte Präkongruenz} von
  \QBRel{} bezüglich $\cdot\|\cdot$.
\end{Def}

Um uns genauer mit den Präkongruenzen auseinandersetzten zu können, brauchen
wir wie im letzten Kapitel die Definition von Traces auf unserer Struktur.
Dadurch erhalten wir die Möglichkeit die gröbste Präkongruenz finden und
definieren zu können.\\
Wie bereits oben erwähnt, werden wir die Erreichbarkeit der Ruhe-Zustände nicht
auf die gleiche Weise umsetzten, wie bei Errortraces, somit benötigen wir keine
gekürzten Ruhetraces bei der die \prune{}-Funktion zur Anwendung käme.

\begin{Def}[Ruhetraces]
  \label{DefRuhetraces}
  Sei $S$ ein \EIO{} und definiere:
  \begin{itemize}
    \item \emph{strickte Ruhetraces}: $\StQT{}(S) := \{w\in\Sigma ^*\mid q_0
      \overset{w}{\Rightarrow} q\in Qui\}$.
  \end{itemize}
\end{Def}

Wir definieren nur für die Ruhe eine neue Semantik, die Error-Semantik wird aus
dem letzten Kapitel übernommen. Somit gelten für \ET{} und \EL{} die
Definitionen aus dem letzten Kapitel.

\begin{Def}[Ruhe-Semantik]
  \label{DefQTQL}
  Sei $S$ ein \EIO{}.
  \begin{itemize}
    \item Die Menge der \emph{error-gefluteten Ruhetraces} von $S$ ist
      $\QT{}(S) := \StQT{}(S)\cup \ET{}(S)$.
  \end{itemize}
  Für zwei \EIO{}s $S_1, S_2$ mit der gleichen Signatur schreiben wir
  $S_1\QRel S_2$, wenn $S_1\ERel S_2$ und $\QT{}(S_1)\subseteq \QT{}(S_2)$ gilt.
\end{Def}

Für die Menge der error-gefluteten Ruhetraces \QT{} haben wir eine Informationsvermischung
mit den Errortraces vorgenommen wie beim fluten der Sprache \EL{}. Da jedoch
durch die Ruhetraces keine neuen Traces entstehen, die nicht bereits in der
gefluteten Sprache \EL{} enthalten wären, müssen wir hier keine neue Flutung
vornehmen. Wir schränken also durch die Relation \QRel{} nur die
bereits existierende Präkongruenz \ERel{} ein.\\
Das folgende Lemma soll explizit festhalten, wie Ruhezustände sich unter der
Parallelkomposition verhalten. Dies ist vor allem in dem danach folgenden Satz
relevant.

\begin{lem}[Ruhe-Zustände unter Parallelkomposition]
  \label{lemRuheParallelkomp}
  \begin{enumerate}
    \item Ein Zustand $(q_1,q_2)$ aus der Parallelkomposition $S_{12}=S_1\|S_2$
      ist ruhig, wenn es auch die Zustände $q_1$ und $q_2$ in $S_1$ bzw.\ $S_2$
      sind.
    \item Wenn der Zustand $(q_1,q_2)$ ruhig ist und nicht in $\ET{}_{12}$
      enthalten ist, dann sind auch die auf die Teilsysteme projizierten
      Zustände $q_1$ und $q_2$ ruhig.
  \end{enumerate}
\end{lem}

\begin{proof}
  ~
  \begin{enumerate}
    \item \hspace{-0.2cm}:
  \end{enumerate}
  \vspace{-0.3cm}
  Da $q_1\in Qui_1$ und $q_2\in Qui_2$ gilt, haben
  diese beiden Zustände jeweils höchstens die Möglichkeiten Transitionen
  auszuführen, die mit Inputs beschriftet sind. Jedoch keine Möglichkeiten für
  Outputs oder $\tau$s. Der Zustand, der durch Parallelkomposition aus diesen
  beiden Zuständen entsteht hat die Transitionsmöglichkeiten dieser Zustände
  die parallel ausgeführt werden. Es gibt somit drei unterschiedliche
  Möglichkeiten:
  \begin{itemize}
    \item Fall 1 ($a\in I_i\backslash Synch(S_1,S_2)\wedge q_i
      \overset{a}{\rightarrow}$ für ein $i\in \{1,2\}$): Da $a$ ein
      unsynchronisierter Input ist, wird dieser unabhängig ausgeführt, somit hat
      $(q_1,q_2)$ ebenfalls die Möglichkeit eine Transition mit $a$ als Input
      auszuführen.
    \item Fall 2 ($a\in I_1\cap I_2\wedge q_i \overset{a}{\rightarrow}$ für
      beide $i\in \{1,2\}$): Da beide Zustände den gleichen Input ausführen
      können, der zu synchronisieren ist, gilt somit $(q_1,q_2)
      \overset{a}{\rightarrow}$. Wobei $a\in I_{12}$ gilt.
    \item Fall 3 ($a\in I_i\cap Synch(S_1,S_2)\wedge q_i
      \overset{a}{\rightarrow}$ für ein $i\in \{1,2\}$): Hier handelt es sich
      um eine Transition, die für $(q_1,q_2)$ nicht ausführbar ist, da es keine
      passende Transition für das andere System gibt, mit dem diese Aktion
      synchronisiert werden könnte. Da wir hier jedoch den Input der zu
      synchronisierenden Aktion haben, handelt es sich hier auch nicht um einen
      neu entstanden Fehler sondern einfach um eine Transition, die nicht
      genommen werden kann, weil der passende Output nicht vorhanden ist.
  \end{itemize}
  Dies sind alle Fälle, die auftreten können für die Parallelkomposition der
  Transitionsmöglichkeiten. Bei keinen dieser Möglichkeiten ist ein Output oder
  ein $\tau$ entstanden und somit hat der Zustand $(q_1,q_2)$ auch keine
  Möglichkeiten solche Transitionen auszuführen. Daraus folgt also, dass
  $(q_1,q_2)\in Qui_{12}$ gilt.

  2.:\\
  Es gilt $(q_1,q_2)\in Qui_{12}\backslash E_{12}$, somit hat
  dieser Zustand maximal die Möglichkeit Transitionen für Inputs auszuführen.
  Diese Transitionen für Inputs können nur entstehen aus Inputs, die nicht
  in der Menge der synchronisierten Handlungen enthalten sind oder aus der
  Synchronisation von zwei Inputs aus der Menge $I_1\cup I_2$. Durch die
  Parallelkomposition von Outputs mit anderen Aktionen können keine Inputs
  entstehen und in unserer Definition der Parallelkomposition auch keine
  $\tau$s. Outputs, die nicht in der Menge $Synch(S_1,S_2)$ enthalten sind,
  würden als Outputs des zusammengesetzten Zustandes übernommen werden und
  können somit weder bei $q_1$ noch bei $q_2$ vorhanden sein. Falls wir jedoch
  $(q_1,q_2)\in E_{12}$ zulassen würden, wäre es möglich, dass einer der beiden
  Zustände $q_i$ eine Transitionsmöglichkeit für einen Output hat, der aufgrund
  eines fehlenden Inputs des anderen Zustandes nicht synchronisiert werden kann
  und somit ein neuer Error entsteht. Dieser Fall wird jedoch durch die
  zusätzliche Einschränkung ausgeschlossen. Falls einer der beiden Zustände
  eine Transitionsmöglichkeit für ein $\tau$ gehabt hätte, müsste dies auch der
  zusammen gesetzte Zustand haben und könnte somit nach unserer Definition
  nicht ruhig sein. Es folgt also, dass $q_1$ und $q_2$ ebenso maximal
  Transitionen mit Inputs ausführen können. Es gilt also $q_1\in Qui_1$ und
  $q_2\in Qui_2$.
\end{proof}

In dem folgenden Satz sind Punkt 1.\ und 3.\ nur zur Vollständigkeit aufgeführt.
Sie entsprechen Punkt 1.\ und 2.\ aus Satz~\ref{satzErrorSemanik}.

\begin{satz}[Error- und Ruhe-Semantik für Parallelkompositonen]
  \label{satzQuiSemantik}
  Für zwei komponierbare \EIO{}s $S_1, S_2$ und ihre Komposition $S_{12} =
  S_1\|S_2$ gilt:
  \begin{enumerate}
    \item $\ET{}_{12} = \cont{}(\prune{}((\ET{}_1\|\EL{}_2)\cup (\EL{}_1\|\ET{}_2)))$,
    \item $\QT{}_{12} = (\QT{}_1\|\QT{}_2)\cup \ET{}_{12}$,
    \item $\EL{}_{12} = (\EL{}_1\|\EL{}_2)\cup \ET{}_{12}$.
  \end{enumerate}
\end{satz}

\begin{proof}
  ~
  \begin{enumerate}
    \item \hspace{-0.2cm}:
  \end{enumerate}
  \vspace{-0.3cm}
  Der Beweis diese Punktes entspricht dem Beweis von Punkt 1.\ im Beweis von
  Satz~\ref{satzErrorSemanik}.

  2. \glqq $\subseteq$\grqq :\\
  Hier müssen wir unterscheiden ob wir ein $w\in\StQT{}_{12}$ betrachten oder
  ein $w\in \ET{}_{12}$. Im zweiten Fall ist das $w$ in der rechten Seite
  enthalten. Somit betrachten wir ab jetzt ein
  $w\in \StQT{}_{12}$ und versuchen dessen Zugehörigkeit zur rechten Menge zu
  zeigen. Aufgrund von Definition~\ref{DefRuhetraces} wissen wir, dass
  $(q_{01},q_{02}) \overset{w}{\Rightarrow} (q_1,q_2)$ gilt mit $(q_1,q_2)\in
  Qui_{12}$. Durch Projektion erhalten wir $q_{01} \overset{w_1}{\Rightarrow}
  q_1$ und $q_{02} \overset{w_2}{\Rightarrow} q_2$ mit $w\in w_1\|w_2$. Aus
  $(q_1,q_2)\in Qui_{12}$ können wir folgern, dass bereits $q_1\in Qui_1$ und
  $q_2\in Qui_2$ gilt. Somit gilt $w_1\in \StQT{}_1\subseteq \QT{}_1$ und
  $w_2\in \StQT{}_2\subseteq \QT{}_2$. Daraus folgt dann $w\in
  \QT{}_1\|\QT{}_2$ und somit ist $w$ in der rechten Seiten der Gleichung
  enthalten.

  2. \glqq $\supseteq$\grqq :\\
  Wir müssen nun wieder unterscheiden, nach dem aus welcher Menge unser
  betrachtetes Element stammt. Falls $w\in \ET{}_{12}$ gilt, so können wir die
  Zugehörigkeit zur linken Seite direkt folgern. Deshalb betrachten wir für den
  weiteren Beweis dieser Inklusionsrichtung ein Element $w\in
  (\QT{}_1\|\QT{}_2)$ und zeigen, dass es in der linken Menge enthalten ist. Da
  $\QT{}_i = \StQT{}_i\cup \ET{}_i$ gilt, existieren für $w_1$ und $w_2$ mit
  $w\in w_1\| w_2$ unterschiedliche Möglichkeiten:
  \begin{itemize}
    \item Fall 1 ($w_1\in \StQT{}_1\wedge w_2\in \StQT{}_2$): Es gilt in
      diesem Fall $q_{01} \overset{w_1}{\Rightarrow} q_1\in Qui_1$ und $q_{02}
      \overset{w_2}{\Rightarrow} q_2\in Qui_2$. Da $q_1$ und $q_2$ in der
      Ruhe-Menge enthalten sind, ist auch der Zustand, der aus ihnen
      zusammengesetzt ist in der Parallelkomposition ruhig und lässt keine
      $\tau$-Transitionen zu. Es gilt also für die Komposition $(q_{01},q_{02})
      \overset{w}{\Rightarrow} (q_1,q_2)\in Qui_{12}$ und dadurch ist $w$ in
      der linken Seite der Gleichung enthalten, da $w\in \StQT{}_{12}\subseteq
      \QT{}_{12}$ gilt.
    \item Fall 2 ($w_1\in \ET{}_1\vee w_2\in \ET{}_2$): \OBdA{} gilt
      $w_1\in \ET{}_1$. Nun kann $w_2\in \StQT{}_2\subseteq L_2$ gelten
      oder $w_2\in \ET{}_2$ und somit gilt auf jeden Fall $w_2\in
      \EL{}_2$. Daraus können wir dann mit dem ersten Punkt von
      Satz~\ref{satzErrorSemanik} bzw.\ aus dem ersten Punkt dieses Satzes
      folgern, dass $w\in \ET{}_{12}$ gilt und somit in der linken Seite der
      Gleichung enthalten ist.
  \end{itemize}

  3.:\\
  Der Beweis diese Punktes entspricht dem Beweis von Punkt 2.\ im Beweis von
  Satz~\ref{satzErrorSemanik}.
\end{proof}

Die folgende Proposition ist eine direkte Folgerung aus dem letzten Satz.
Jedoch ist es eine wichtige Feststellung für die weiteren Verlauf die gröbste
Präkongruenz finden zu wollen.

\begin{prop}[Präkongruenz]
  \label{propQuiPrae}
  \QRel{} ist eine Präkongruenz bezüglich $\cdot\|\cdot$.
\end{prop}

\begin{proof}
  Es muss gezeigt werden, wenn $S_1\QRel S_2$ gilt, für jedes $S_3$ auch
  $S_3\|S_1\QRel S_3\|S_2$ gilt. D.h.\ es ist zu zeigen, dass aus $S_1\ERel
  S_2$ und $\QT{}_1\subseteq \QT{}_2$ folgt, $S_{31}\ERel S_{32}$ und
  $\QT{}_{31}\subseteq \QT{}_{32}$. Dies ergibt sich wie im Beweis zu
  Proposition~\ref{propPraekongruenz} aus der Monotonie von \cont{}, \prune{}
  und $\cdot\|\cdot$ auf Sprachen wie folgt:
  \begin{itemize}
    \item $\begin{aligned}[t]
        S_{31} \overset{\mathrm{Beweis}~\ref{propPraekongruenz}}{\ERel} S_{32}
    \end{aligned}$
    \item $\begin{aligned}[t]
        \QT{}_{31} &\overset{\ref{satzQuiSemantik}~2.}{=}
        (\QT{}_3\|\QT{}_1)\cup \ET{}_{31}\\
        &\hspace{-0.5cm}\overset{\ET{}_{31}\subseteq
      \ET{}_{32}}{\overset{\mathrm{und}}{\overset{\QT{}_1\subseteq
      \QT{}_2}{\subseteq}}} (\QT{}_3\|\QT{}_2) \cup \ET{}_{32}\\
        &\overset{\ref{satzQuiSemantik}~2.}{=} \QT{}_{32}
    \end{aligned}$
  \end{itemize}
\end{proof}

\begin{lem}[Verfeinerung mit Ruhe-Zuständen]
  \label{lemQuiVerfeinerung}
  Gegeben sind zwei \EIO{}s $S_1$ und $S_2$ mit der gleichen Signatur. Wenn
  alle Partner \EIO{}s $U$, die mit $S_2$  gut kommunizieren, auch mit $S_1$
  gut kommunizieren, dann verfeinert $S_1$ das \EIO{} $S_2$. Diese Verfeinerung
  entspricht der Relation \QRel{} von oben: Wenn $U\|S_1\QBRel U\|S_2$ für alle
  Partner $U$, dann gilt $S_1\QRel S_2$.
\end{lem}

\begin{proof}
  Da wir davon ausgehen, dass $S_1$ und $S_2$ die gleiche Signatur haben,
  definieren wir $I:=I_1=I_2$ und $O:=O_1=O_2$. Für jeden Partner $U$ gilt
  $I_U=O$ und $O_U=I$.\\
  Um zu zeigen, dass die Relation $S_1\QRel S_2$ gilt, müssen wir die
  folgenden Punkte nachweisen:
  \begin{itemize}
    \item $S_1\ERel S_2$,
    \item $\QT{}(S_1)\subseteq \QT{}(S_2)$.
  \end{itemize}
  Der erste Punkt wurde bereits in Lemma~\ref{lemVerfeinerung}
  gezeigt. So bleibt uns nur noch die Inklusion $\QT{}_1\subseteq \QT{}_2$ zu
  zeigen. Diese Inklusion können wir jedoch noch anlog zum Beweis der Inklusion
  der gefluteten Sprachen in Lemma~\ref{lemVerfeinerung} weiter einschränken.
  Da wir bereits wissen, dass $\ET{}_1\subseteq\ET{}_2$ gilt, müssen wir nur
  noch $\StQT{}_1\subseteq \QT{}_2$ zeigen.\\
  Wir wählen ein $w\in \StQT{}(S_1)$ und zeigen, dass es auch in $\QT{}(S_2)$
  enthalten ist.\\
  Es ist vom Startzustand von $S_1$ durch das Wort $w$ ein ruhiger Zustand
  erreichbar. Dies hat keine Auswirkungen auf die Parallelkomposition $U\|S_1$,
  wenn in $U$ kein Ruhe-Zustand durch $w$ erreicht wird. Somit betrachten wir
  den Partner $U$ bei dem durch $w$ ebenfalls ein Ruhe-Zustand erreicht wird.
  Daraus folgt dann, dass in $U\|S_1$ ein Ruhe-Zustand mit $w$ erreicht wird
  und durch die gegebene Relation folgt dann auch, dass in $U\|S_2$ auch ein
  Ruhe-Zustand durch $w$ erreicht werden muss. Dies kann nur der Fall sein,
  wenn in $S_2$ bereits ein Ruhe-Zustand durch $w$ erreicht wird. Somit gilt
  also $w\in \StQT{}_2$.
\end{proof}

Mit dem folgenden Satz halten wir fest, dass wir mit \QRel{} die gröbste
Präkongruenz gefunden haben bezüglich $\cdot\|\cdot$ die in \QBRel{} enthalten
ist.

\begin{satz}[Full Abstractness für Ruhe-Semantik]
  \label{satzQuiFullAbst}
  Seien $S_1$ und $S_2$ zwei \EIO{}s mit derselben Signatur. Dann gilt
  $S_1\QCRel S_2\Leftrightarrow S_1\QRel S_2$, insbesondere ist \QRel{} eine
  Präkongruenz.
\end{satz}

\begin{proof}
  Wie bereits in Proposition~\ref{propQuiPrae} festgehalten, ist \QRel{} eine
  Präkongruenz.

  \glqq{}$\Leftarrow$\grqq{}: Nach Definition gilt, wenn
  $\varepsilon\in\QT{}(S)$, ist in $S$ ein Ruhe-Zustand durch interne Aktionen
  oder ein Error-Zustand lokal erreichbar. Somit impliziert $S_1\QRel S_2$,
  dass $\varepsilon\in\QT{}_2$ gilt, wenn $\varepsilon\in\QT{}_1$. Daraus folgt
  ebenfalls, dass $S_1\QBRel S_2$ gilt. Somit folgt aus $S_1\QRel S_2$ der
  relationale Zusammenhang $S_1\QCRel S_2$.

  \glqq{}$\Rightarrow$\grqq{}: Durch die Definition von \QCRel{} folgt aus
  $S_1\QCRel{} S_2$, dass $U\|S_1\QCRel U\|S_2$ für alle EIOs $U$ , die mit
  $S_1$ komponierbar sind. Somit folgt auch die Gültigkeit von $U\|S_1\EBRel
  U\|S_2$ für alle diese EIOs $U$. Mit Lemma~\ref{lemQuiVerfeinerung} folgt
  dann $S_1\QRel{} S_2$.
\end{proof}

Wir haben somit, wie im letzten Kapitel, eine Kette an Folgerungen gezeigt, die
sich zu einem Ring schließen. Dies ist in Abbildung~\ref{FolgerungsketteQui}
dargestellt.

\begin{figure}[h!tbp]
  \begin{center}
    \begin{tikzpicture}
      \matrix (m) [matrix of math nodes,row sep=2cm,column sep=4cm]{
        S_1\QRel S_2 & S_1\QCRel S_2 \\
        \substack{\forall~\mathrm{Partner}~U:\\U\|S_1\QBRel U\|S_2} &
      \substack{\forall~\mathrm{komponierbaren}~U:\\U\|S_1\QBRel U\|S_2} \\};
        \draw[-implies, double, double distance=1mm]
          (m-1-1) -- node [above] {\glqq{}$\Leftarrow$\grqq{} von
            Satz~\ref{satzQuiFullAbst}} (m-1-2);
        \draw[-implies, double, double distance=1mm]
          (m-1-2) -- node [right] {\glqq{}$\Rightarrow$\grqq{} von
            Satz~\ref{satzQuiFullAbst}} (m-2-2);
        \draw[-implies, double, double distance=1mm]
          (m-2-1) -- node [left]
          {Lemma~\ref{lemQuiVerfeinerung}} (m-1-1);
        \draw[-implies, double, double distance=1mm]
        (m-2-2) -- node [below]
        {$\substack{U~\mathrm{Partner}\\\Downarrow\\U~\mathrm{komponierbar}}$} (m-2-1);
    \end{tikzpicture}
    \caption{Folgerungskette}
    \label{FolgerungsketteQui}
  \end{center}
\end{figure}

Aus Satz~\ref{satzQuiFullAbst} und Lemma~\ref{lemQuiVerfeinerung} erhalten wir
das folgende Korollar.

\begin{kor}
  Ein \EIO{} $S_1$ verfeinert einen \EIO{} $S_2$ genau dann, wenn für alle
  \EIO{}s $U$ für die $S_2$ gut mit $U$ kommuniziert folgt $S_1$ kommuniziert
  ebenfalls gut mit $U$.\\
  Dies lässt sich formal wie folgt ausdrücken: $S_1\QRel S_2\Leftrightarrow
  U\|S_1\QBRel U\|S_2$ für alle Partner $U$.
\end{kor}

\section{Hiding für Ruhe}

Wir wollen nun auch hier die Auswirkungen der Internalisierung von Aktionen auf
unsere Verfeinerungsrelationen untersuchen. Es werden Outputs in interne
Aktionen umgewandelt. Da wir jedoch bei unseren Ruhe-Zuständen auch
$\tau$-Transitionen verboten haben, verändert sich nichts an der Menge der ruhigen
Zustände. Da wir die Erreichbarkeit von Ruhe-Zuständen mittels interner Aktionen
betrachten, können durch das verbergen von Outputs neue erreichbare
Ruhe-Zustände hinzu kommen. Somit ist es nicht möglich hier eine analoge
Proposition zu~\ref{propErBaIn} zu formulieren. Wir können zwar daraus
schließen, dass alle Ruhe-Zustände, die vor dem Hiding zu erreichen waren mit
$\tau$s auch danach noch erreichbar sind, jedoch können durch die
Internalisierung neue erreicht werden, die möglicherweise in dem anderen
Transitionssystem nicht erreicht werden können. Da wir jedoch für die
Präkongruenz \QRel{} noch Wissen über die Teilmengenbeziehung der Traces haben,
können wir einen Satz analog zu~\ref{satzPraeInternalisierung} formulieren.

\begin{satz}[Präkongurenz bzgl. Internalisierung]
  \label{satzPraeInterQui}
  Seien $S_1$ und $S_2$ zwei \EIO{}s für die $S_1\QRel S_2$ gilt, somit gilt
  auch $(S_1/\{x_1,x_x,\dots ,x_n\})\QRel (S_2/\{x_1,x_x,\dots ,x_n\})$. Es
  ist also \QRel{} eine Präkongruenz bezüglich $\cdot /\cdot$.
\end{satz}

\begin{proof}
  Da $S_1\QRel S_2$ gilt, wissen wir, dass $S_1\ERel S_2$ und $\QT{}_1\subseteq
  \QT{}_2$ gilt. Wir wissen bereits aufgrund von
  Satz~\ref{satzPraeInternalisierung}, dass daraus $(S_1/\{x_1,x_x,\dots
  ,x_n\})\ERel (S_2/\{x_1,x_x,\dots ,x_n\})$ folgt. Ebenso wie im Beweis zu
  Satz~\ref{satzPraeInternalisierung} können wir hier davon ausgehen, dass
  $X\subseteq O$ gilt. Für jeden Trace aus $\QT{}_1$ können wir einen passenden
  in $\QT{}(S_1/X)$ finden und ebenso für das Transitionssystem $S_2$. Die
  Argumentation läuft hierfür analog zum Beweis von
  Satz~\ref{satzPraeInternalisierung}. Wir haben hier auch kein Problem, dass
  neue Zustände hinzu kommen, von denen aus durch $\tau$s Ruhe-Zustände
  erreichbar sind, da diese $\tau$s aus Outputs entstanden sind, die verborgen
  wurden. Es war also bereits ein Trace in $\QT{}_i$ vorhanden, der nur aus
  Outputs aus $X$ bestanden hat und der dann durch das Hiding zu einem $\tau$
  wurde, was aber der passende Trace ist, den man zu dem ursprünglichen in
  $\QT{}(S_i/X)$ findet. Somit gilt
  also auch $\QT{}(S_1/X)\subseteq \QT{}(S_2/X)$. Daraus folgt dann, dass die
  Relation \QRel{} trotz Hiding erhalten bleibt und somit das Hiding bezüglich
  dieser Relation eine Präkongruenz ist.
\end{proof}

In Definition~\ref{defIntParal} wurde mit Hilfe des Internalisierungsoperator
aus der Parallelkomposition ohne Verbergen die Parallelkomposition mit
Verbergen der synchronisierten Aktionen nachgebildet. Wir können deren
Eigenschaft als Präkongruenz aus den Präkongruenz-Eigenschaften von
$\cdot\|\cdot$ und $\cdot /\cdot$ bezüglich \QRel{} aus der
Proposition~\ref{propQuiPrae} und dem Satz~\ref{satzPraeInterQui} schließen.

\begin{kor}[Präkongruenz mit Internalisierung]
  \QRel{} ist eine Präkongruenz bezüglich $\cdot |\cdot$.
\end{kor}

\section{Diskussion für Veränderungen an Semantik oder anderen Definitionen}

Es währe hier auch denkbar, dass man auch für die Ruhe-Zustände einen lokalen
Erreichbarkeitsbegriff verwendet. Es könnte dadurch sogar möglich sein eine
noch gröbere Präkongruenz zu erhalten. Dies wurde hier jedoch nicht gemacht, da
es bei den denkbaren Umsetzung eines solchen Erreichbarkeitsbegriffes zu
Problemen kommt, so dass wir nicht mehr alle Inklusionen erreichen können, die
wir in unseren Sätzen gefordert haben.\\
Die Umsetzung des lokalen Erreichbarkeitsbegriffes wie im letzten Kapitel, mit
abschneiden der Outputs und beliebigem Fortsetzten der Traces, würde zu einem
Problem in Satz~\ref{satzQuiSemantik} in Punkt 2.\ führen, dass analog zu dem
Problem währe in der Semantik, die als nächstes beschrieben wird. Zusätzlich könnte
man nicht die \EL{} als geflutete Sprache beibehalten sondern müsste noch
zusätzlich mit den Ruhetraces fluten, da sonst nicht mehr alle
Informationen, die in Ruhetraces enthalten auch in der gefluteten Sprache
enthalten wäre. Durch diese Abänderung käme es zusätzlich noch zu einem Problem
in Punkt 3.\ bei Satz~\ref{satzQuiSemantik}. Durch die Anwendung der Definition
der gefluteten Sprachen würden Terme entstehen, in denen Ruhetraces und Traces
aus der Sprache $L$ in Parallelkomposition gesetzt würden. Da die Ruhetraces
nicht mehr Teil der Sprache sind, kann nicht mehr gelten, dass die Teil von
$L_1\|L_2$ ist. Es kann auch nicht Teil der Ruhetraces sein, da Ruhetraces nur
aus Kombination von zwei Ruhetraces entstehen können oder in dem sie
Errortraces sind, was beides hier nicht der Fall ist.\\
Man muss jedoch nicht an dieser Umsetzung für die lokale Erreichbarkeit
festhalten. Man könnte sich auch eine Umsetzung denken, in dem man die
Ruhetraces zwar um die Outputs kürzt, jedoch dann nur Fortsetzungen zulässt,
die bereits im Transitionssystem möglich sind. Somit benötigt man keine weitere
Flutung der gefluteten Sprache \EL{}, da weiterhin alle Ruhetraces in der
Sprache $L$ enthalten sind. Jedoch stoßen wir hier wieder bei Punkt 2.\ von
Satz~\ref{satzQuiSemantik} an Probleme. Hierzu müssen wir davon ausgehen, dass
wir uns eine neue \prune{}-Funktion definiert haben bzw.\ eine neue
\cont{}-Funktion, wie z.B. $\prunenew{}: \Sigma ^*\rightarrow
\mathfrak{P}(\Sigma ^*), w\mapsto \{u\mid \exists v\in O^*:w=uv\}$. Somit würde
sich unsere Definition der Ruhetraces entsprechend abändern. So das statt
\StQT{} ein entsprechendes \PrQT{} enthalten wäre, dass durch die Anwendung
unserer \prunenew{} Funktion entstanden wäre. Somit wäre Punkte 2.\ von
Satz~\ref{satzQuiSemantik} entsprechend auch leicht abgewandelt und müsste
$\QT{}_{12}=\prunenew (\QT{}_1\|\QT{}_2)\cup \ET{}_{12}$ lauten. Das Problem
würde nun auftauchen wenn man die Inklusionsrichtung $\supseteq$ zu beweisen
versucht. Hierzu gibt es jedoch folgendes Gegenbeispiel, dass die Behauptung
gar nicht stimmen kann.\\
Für $S_1$ verwenden wir das folgende Transitionssystem (siehe dazu auch
Abbildung~\ref{S1}):
\begin{itemize}
  \item $Q_1=\{q_{01}, q_{11}\}$
  \item $I_1=\emptyset$
  \item $O_1=\{a\}$
  \item $\delta _1 = \{(q_{01},a,q_{11})\}$
  \item $E_1=\emptyset$
  \item $Qui_1=\{q_{11}\}$
\end{itemize}
\begin{figure} [h!tbp]
\begin{center}
  \begin{tikzpicture}[->, >=latex',auto,node distance =3cm, semithick]

    \node (0) {$q_{01}$};
    \node (1) [right of=0, rectangle, draw] {$q_{11}\in Qui_1$};

    \path (0) edge node {$a!$} (1);
  \end{tikzpicture}
  \caption{$S_1$}
  \label{S1}
\end{center}
\end{figure}
Für $S_2$ verwenden wir das folgende Transitionssystem (siehe dazu auch
Abbildung~\ref{S2}):
\begin{itemize}
  \item $Q_2=\{q_{02}, q_{12}, q_{22}\}$
  \item $I_2=\{a\}$
  \item $O_2=\{b\}$
  \item $\delta _2 = \{(q_{02},a,q_{22}), (q_{02},b,q_{12}), (q_{12},a,q_{22}),
    (q_{22},b,q_{22})\}$
  \item $E_2=\emptyset$
  \item $Qui_2=\{q_{12}\}$
\end{itemize}
\begin{figure} [h!tbp]
\begin{center}
  \begin{tikzpicture}[->, >=latex',auto,node distance =3cm, semithick]

    \node (0) {$q_{02}$};
    \node (1) [right of=0, rectangle, draw] {$q_{12}\in Qui_2$};
    \node (2) at ($(0)!0.5!(1) + (0,-2)$){$q_{22}$};

    \path (0) edge node {$b!$} (1)
              edge [below left] node {$a?$} (2)
          (1) edge node {$a?$} (2)
          (2) edge [loop below] node {$b!$} (2);
  \end{tikzpicture}
  \caption{$S_2$}
  \label{S2}
\end{center}
\end{figure}
Somit ergibt sich für die Parallelkomposition $S_{12} = S_1\| S_2$ das folgende
Transitionssystem, bei dem bereits unerreichbare Zustände weggelassen sind
(siehe dazu auch Abbildung~\ref{S12}):
\begin{itemize}
  \item $Q_{12}=\{(q_{01},q_{02}), (q_{01},q_{12}), (q_{11},q_{22})\}$
  \item $I_{12}=\emptyset$
  \item $O_{12}=\{a,b\}$
  \item $\begin{aligned}[t]
      \delta _{12} = \{&((q_{01},q_{02)},a,(q_{11},q_{22})),
      ((q_{01},q_{12}),a,(q_{11},q_{22})),
    ((q_{01},q_{02}),b,(q_{01},q_{12})),\\
  &((q_{11},q_{22}),b,(q_{11},q_{22}))\}
        \end{aligned}$
  \item $E_{12}=\emptyset$
  \item $Qui_{12}=\emptyset$
\end{itemize}
\begin{figure} [h!tbp]
\begin{center}
  \begin{tikzpicture}[->, >=latex',auto,node distance =3cm, semithick]

    \node (0) {$(q_{01},q_{02})$};
    \node (1) [right of=0] {$(q_{01},q_{12})$};
    \node (2) at ($(0)!0.5!(1) + (0,-2)$){$(q_{11},q_{22})$};

    \path (0) edge node {$b!$} (1)
              edge [below left] node {$a!$} (2)
          (1) edge node {$a!$} (2)
          (2) edge [loop below] node {$b!$} (2);
  \end{tikzpicture}
  \caption{$S_{12}$}
  \label{S12}
\end{center}
\end{figure}

Es wäre möglich gewesen, die Menge der Ruhe-Zustände bereits anders zu
definieren. Dadurch, dass wir die $\tau$-Transitionen verboten haben, sind wir
einer Fallunterscheidung entgangen. Da jedoch Quiescents als Deadlock-Zustände
anzusehen sind, aus denen das System ohne Hilfe nicht mehr heraus kommt,
müssten wir hier eigentlich darauf achten, wohin die $\tau$-Transitionen führen
und dann möglicherweise die Zustände trotz dieser Transitionen in unsere Menge
aufnehmen. Ein Zustand, der keine Outputs machen kann und als einzige
$\tau$-Transition eine Schlinge auf sich selbst hat, kann ebenfalls nicht ohne
Inputs sich aus diesem Zustand heraus bewegen. Dieser Zustand wir jedoch nach
unserer Definition nicht als ruhig angesehen. Jedoch ist so ein Zustand
divergent, da er eine unendliche Folge an internen Aktionen ausführen kann.
Dies wurde in~\cite{Chilton2013} mit untersucht. Auch bei anderen
Divergenz-Möglichkeiten eines Zustandes, bei denen nicht die Möglichkeit
besteht von einem durch $\tau$ erreichbaren Zustand aus einen Output zu machen,
sollte der Zustand als ruhig angesehen werden. Die passende Definition würde
dann wie folgt lauten.

\begin{Def}[Ruhe Alternative]
  Ein Ruhe-Zustand ist ein Zustand in einem \EIO{}, der keine Möglichkeit hat
  ohne einen Input von außen je wieder einen Output zu machen.\\
  Somit ist die Menge der Ruhe-Zustände in einem \EIO{} wie folgt formal
  definiert: $Qui:=\left\{q\in Q\mid \forall a\in O: q
  \overset{a}{\not{\hspace{-0.1cm}\Rightarrow}}\right\}$.
\end{Def}

Diese Art der Definition hätte jedoch die Betrachtung deutlich aufwendiger
gemacht und soll deshalb hier nicht behandelt werden.
