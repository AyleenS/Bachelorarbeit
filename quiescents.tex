\chapter{Verfeinerung über Error- und Quiescenttraces}

In diesem Kapitel werden wir und nicht nur um die Erreichbarkeit von
Error-Zuständen kümmern, sondern auch um die Erreichbarkeit von
Quiescent-Zuständen. Wir werden dabei ähnlich vorgehen wie im letzten Kapitel,
jedoch halten wir uns als Quelle an~\cite{Chilton2013}. Darin werden ähnliche Konzepte
beschrieben, jedoch aus Sicht der Traces.

\begin{Def}[Quiescent]
  Ein Quiescent-Zustand ist ein Zustand in einem EIO der keine Outputs besitzt
  oder ein Zustand, von dem aus über eine interne Handlung $\tau$ ein Zustand
  erreicht werden kann, der keine Outputs zulässt.\\
  Somit ist die Menge der Quiescent-Zustände in einem EIO wie folgt formal
  definiert: $Qui=\{q\in Q\mid q\in K\vee \exists p\in Q:
  q\overset{\tau}{\Rightarrow} p\in K\}$ mit $K=\{q\in Q\mid \forall a\in O:
  q\overset{a}{\not{\hspace{-0.1cm}\rightarrow}}\}$.
\end{Def}

Für die Erreichbarkeit verwenden wir wie im letzten Kapitel wieder den
optimistischen Ansatz der lokalen Erreichbarkeit.

\begin{Def}[lokal error- und quiescentfreie Kommunikation]
  Zwei EIOs $S_1$ und $S_2$ kommunizieren gut, wenn keine Errors und Quiescents
  lokal erreichbar sind in ihrer Parallelkomposition $S_1\| S_2$
\end{Def}

\begin{Def}[lokale Basisrelation]
  Für EIOs $S_1$ und $S_2$ mit der gleichen Signatur schreiben wir
  $S_1\sqsubseteq _{Qui}^B S_2$, wenn ein Error oder Quiescent in $S_1$ nur
  dann lokal erreichbar ist, wenn er auch in $S_2$ lokal erreichbar ist.\\
  $\sqsubseteq _{Qui}^C$ bezeichnet die vollständig abstrakte Präkongruenz von
  $\sqsubseteq  _{Qui}^ B$ bezüglich $\|$.
\end{Def}

\begin{Def}[Error und Quiescenttraces]
  \label{DefQuiescenttraces}
  Sei $S$ ein EIO und definiere:
  \begin{itemize}
    \item die Traces bezüglich Errors entsprechen denen
      aus~\ref{DefErrortraces},
    \item strickte Quiescenttraces: $StQT(S) = \{w\in\Sigma ^*\mid q_0
      \overset{w}{\Rightarrow} q\in Qui\}$,
    \item gekürzte Quiescenttraces: $PrQT(S) = \{prune(w)\mid w\in StQT(S)\}$.
  \end{itemize}
\end{Def}

\begin{Def}[Lokale Error und Quiescent Semantik]
  \label{DefQTQL}
  Sei $S$ ein EIO.
  \begin{itemize}
    \item Die Menge der Errortraces ist wie in~\ref{DefETEL} definiert.
    \item Die Menge der Quiescenttraces von $S$ ist $QT(S) = cont(PrQT(S))$.
    \item Die geflutete Sprache von $S$ ist $QL(S)=L(S)\cup ET(S)\cup QT(S)$
      und unterscheidet sich somit von der gefluteten Sprache $EL(S)$
      in~\ref{DefETEL}.
  \end{itemize}
  Für zwei EIOs $S_1, S_2$ mit der gleichen Signatur schreiben wir
  $S_1\sqsubseteq _{Qui} S_2$, wenn $ET(S_1)\subseteq ET(S_2)$,
  $QT(S_1)\subseteq QT(S_2)$ und $QL(S_1)\subseteq QL(S_2)$ gilt.
\end{Def}

\begin{satz}[Lokale Error und Quiescent Semantik für Parallelkompositonen]
  Für zwei komponierbare EIOs $S_1, S_2$ und $S_{12} = S_1\|S_2$ gilt:
  \begin{enumerate}
    \item $ET_{12} = cont(prune((ET_1\|QL_2)\cup (QL_1\|ET_2)))$
    \item $QT_{12} = cont(prune((QT_1\|QL_2)\cup (QL_1\|QT_2)))$%TODO???
    \item $QL_{12} = (QL_1\|QL_2)\cup ET_{12}\cup QT_{12}$%TODO???
  \end{enumerate}
\end{satz}

\begin{proof}
  ~
  \begin{enumerate}
    \item \hspace{-0.2cm}:
  \end{enumerate}
  \vspace{-0.3cm}
  Der Beweis diese Punktes entspricht dem Beweis von Punkt 1.\ im Beweis von
  Satz~\ref{satzErrorSemanik}.

  2. ``$\subseteq$'':\\
  Da hier beide Seiten unter $cont$ abgeschlossen sind, genügt es ein
  präfix-minimales Element $w\in PrQT_{12}$ zu betrachten und dessen
  Zugehörigkeit zur rechten Menge zu zeigen. Aufgrund von
  Definition~\ref{DefQuiescenttraces} wissen wir es gibt ein $v\in O_{12}^*$,
  sodass $(q_{01},q_{02}) \overset{w}{\Rightarrow} (q_1,q_2)
  \overset{v}{\Rightarrow} (q_1',q_2')$ mit $(q_1',q_2')\in Qui_{12}$ und
  $w=prune(wv)$ gilt. Durch Projektion erhalten wir $q_{01}
  \overset{w_1}{\Rightarrow} q_1 \overset{v_1}{\Rightarrow} q_1'$ und $q_{02}
  \overset{w_2}{\Rightarrow} q_2 \overset{v_2}{\Rightarrow} q_2'$ mit $w\in
  w_1\|w_2$ und $v\in v_1\|v_2$. Aus $(q_1',q_2')\in Qui_{12}$ können wir
  folgern, dass bereits $q_1'\in Qui_1$ und $q_2'\in Qui_2$ gilt. Somit gilt
  $w_1v_1\in StQT_1\subseteq QT_1\subseteq QL_1$ und $w_2v_2\in StQT_2\subseteq
  QT_2\subseteq QL_2$. Daraus folgt dann $wv\in QT_1\|QL_2$ mit $w\in
  prune(wv)$ und somit ist $w$ in der rechten Seiten der Gleichung enthalten.

  2. ``$\supseteq$'':\\
  Wegen der Abgeschlossenheit beider Seiten der Gleichung gegenüber $cont$
  betrachten wir auch in diesem Fall nur ein präfix-minimales Element $x\in
  prune((QT_1\|QL_2)\cup (QL_1\|QT_2))$. Somit können wir schließen, dass es
  ein $y\in Q_{12}^*$ gibt, sodass $xy\in (QT_1\|QL_2)\cup (QL_1\|QT_2)$.
  \OBdA{} gilt $xy\in QT_1\|QL_2$, d.h.\ es existieren $w_1\in QT_1$ und
  $w_2\in QL_2$ mit $xy\in w_1\| w_2$.\\
  Im weiteren werden wir für alle Fälle von $xy$ zeigen, dass es ein $v\in
  PrQT_{12}$ gibt, das ein Präfix von $xy$ ist und $v$ entweder auf einen Input
  aus $I_{12}$ endet oder $v=\varepsilon$. Da $v$ entweder leer ist oder auf
  einen Input endet, muss $v$ ein Präfix von $x$ sein mit der gleichen
  Begründung wie im Beweis zu Satz~\ref{satzErrorSemanik}. Somit gilt durch die
  Existenz dieses Präfixes von $x$, dass $x\in QT_{12}$ gilt. Sei $v_1$ das
  kürzeste Präfix von $w_1$ in $PrQT_1$. Falls $w_2\in L_2$, so sei $v_2=w_2$,
  sonst soll $v_2$ das kürzeste Präfix von $w_2$ in $PrQT_2$ sein. Jede Aktion
  in $v_1$ und $v_2$ hängt mit einer aus $xy$ zusammen. Wir gehen nun davon
  aus, dass entweder $v_2=w_2\in L_2$ gilt oder die letzte Aktion von $v_1$
  findet vor oder gleichzeitig mit der letzten Aktion von $v_2$ statt.
  Ansonsten endet $v_2\in PrQT_2$ vor $v_1$ und somit ist dieser Fall analog zu
  $v_1$ endet vor $v_2$.
  \begin{itemize}
    \item Fall 1 ($v_1=\varepsilon$): Somit gilt $\varepsilon\in PrQT_1$ und es
      ist deshalb bereits ein Quiescent-Zustand in $S_1$ erreichbar. Wir wählen
      $v_2'=v'=\varepsilon$, somit ist $v_2'$ ein Präfix von $v_2$.
    \item Fall 2 ($v_1\neq \varepsilon$): Aufgrund der Definition von $PrQT$
      endet $v_1$ auf ein $a\in I_1$, d.h.\ $v_1=v_1'a$. $v'$ sei das Präfix
      von $xy$, das mit der letzten Aktion von $v_1$ endet, d.h.\ mit $a$, und
      $v_2'=v'|_{\Sigma _2}$.\\
      \emph{Fehler, ab hier geht es nicht weiter}
  \end{itemize}
  %TODO Beweis!!!

  3.:\\
  Es ist durch die Definition klar, dass gilt $L_i\subseteq QL_i$,
  $ET_i\subseteq QL_i$ und $QT_i\subseteq QL_i$. Wir beginnen mit der
  Argumentation von der rechten Seite der Gleichung aus:
  \begin{align*}
    &(QL_1\| QL_2)\cup ET_{12}\cup QT_{12}\\
    &\overset{\ref{DefQTQL}}{=}(L_1\cup ET_1\cup QT_1)\|(L_2\cup ET_2\cup
    QT_2)\cup ET_{12}\cup QT_{12}\\
    &=(L_1\|L_2)\cup
    \underset{\overset{1.}{\subseteq} ET_{12}}{\underset{\subseteq
    (QL_1\|ET_2)}{\underbrace{(L_1\|ET_2)}}} \cup
    \underset{\overset{2.}{\subseteq} QT_{12}}{\underset{\subseteq
    (QL_1\|QT_2)}{\underbrace{(L_1\|QT_2)}}} \cup
    \underset{\overset{1.}{\subseteq} ET_{12}}{\underset{\subseteq
    (ET_1\|QL_2)}{\underbrace{(ET_1\|L_2)}}} \cup
    \underset{\overset{1.}{\subseteq} ET_{12}}{\underset{\subseteq
    (QL_1\|ET_2)}{\underbrace{(ET_1\|ET_2)}}} \cup
    \underset{\overset{2.}{\subseteq} QT_{12}}{\underset{\subseteq
    (QL_1\|QT_2)}{\underbrace{(ET_1\|QT_2)}}}\\
    &\phantom{=}\;\cup\underset{\overset{2.}{\subseteq} QT_{12}}{\underset{\subseteq
    (QT_1\|QL_2)}{\underbrace{(QT_1\|L_2)}}} \cup
    \underset{\overset{2.}{\subseteq} QT_{12}}{\underset{\subseteq
    (QT_1\|QL_2)}{\underbrace{(QT_1\|ET_2)}}} \cup
    \underset{\overset{2.}{\subseteq} QT_{12}}{\underset{\subseteq
    (QL_1\|QT_2)}{\underbrace{(QT_1\|QT_2)}}} \cup
    ET_{12} \cup QT_{12}\\
    &=(L_1\|L_2) \cup ET_{12} \cup QT_{12}\\
    &\overset{\ref{LemmaSprache}}{=}L_{12}\cup ET_{12}\cup QT_{12}\\
    &\overset{\ref{DefQTQL}}{=}QL_{12}
  \end{align*}
  %TODO anpassen
\end{proof}
