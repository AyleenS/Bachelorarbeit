\chapter{Verfeinerung über Error- und Quiescenttraces}

In diesem Kapitel werden wir und nicht nur um die Erreichbarkeit von
Error-Zuständen kümmern, sondern auch um die Erreichbarkeit von
Quiescent-Zuständen. Wir werden dabei ähnlich vorgehen wie im letzten Kapitel,
jedoch halten wir uns als Quelle an~\cite{Chilton2013}. Darin werden ähnliche Konzepte
beschrieben, jedoch aus Sicht der Traces.

\begin{Def}[Quiescent]
  Ein Quiescent-Zustand ist ein Zustand in einem EIO der keine Outputs besitzt
  oder ein Zustand, von dem aus über eine interne Handlung $\tau$ ein Zustand
  erreicht werden kann, der keine Outputs zulässt.\\
  Somit ist die Menge der Quiescent-Zustände in einem EIO wie folgt formal
  definiert: $Qui:=\{q\in Q\mid q\in K\vee \exists p\in Q:
  q\overset{\tau}{\Rightarrow} p\in K\}$ mit $K:=\{q\in Q\mid \forall a\in O:
  q\overset{a}{\not{\hspace{-0.1cm}\rightarrow}}\}$.
\end{Def}

Für die Erreichbarkeit verwenden wir wie im letzten Kapitel wieder den
optimistischen Ansatz der lokalen Erreichbarkeit.

\begin{Def}[lokal error- und quiescentfreie Kommunikation]
  Zwei EIOs $S_1$ und $S_2$ kommunizieren gut, wenn keine Errors und Quiescents
  lokal erreichbar sind in ihrer Parallelkomposition $S_1\| S_2$
\end{Def}

\begin{Def}[lokale Basisrelation]
  Für EIOs $S_1$ und $S_2$ mit der gleichen Signatur schreiben wir
  $S_1\sqsubseteq _{Qui}^B S_2$, wenn ein Error oder Quiescent in $S_1$ nur
  dann lokal erreichbar ist, wenn er auch in $S_2$ lokal erreichbar ist.\\
  $\sqsubseteq _{Qui}^C$ bezeichnet die vollständig abstrakte Präkongruenz von
  $\sqsubseteq  _{Qui}^ B$ bezüglich $\|$.
\end{Def}

\begin{Def}[Error und Quiescenttraces]
  \label{DefQuiescenttraces}
  Sei $S$ ein EIO und definiere:
  \begin{itemize}
    \item die Traces bezüglich Errors entsprechen denen
      aus~\ref{DefErrortraces},
    \item strickte Quiescenttraces: $StQT(S) := \{w\in\Sigma ^*\mid q_0
      \overset{w}{\Rightarrow} q\in Qui\}$,
    \item gekürzte Quiescenttraces: $PrQT(S) :=\{prune(w)\mid w\in StQT(S)\}$.
  \end{itemize}
\end{Def}

\begin{Def}[Lokale Error und Quiescent Semantik]
  \label{DefQTQL}
  Sei $S$ ein EIO.
  \begin{itemize}
    \item Die Menge der Errortraces ist wie in~\ref{DefETEL} definiert.
    \item Die Menge der Quiescenttraces von $S$ ist $QT(S) := cont(PrQT(S))$.
    \item Die geflutete Sprache von $S$ ist $QL(S):=L(S)\cup ET(S)\cup QT(S)$
      und unterscheidet sich somit von der gefluteten Sprache $EL(S)$
      in~\ref{DefETEL}.
  \end{itemize}
  Für zwei EIOs $S_1, S_2$ mit der gleichen Signatur schreiben wir
  $S_1\sqsubseteq _{Qui} S_2$, wenn $ET(S_1)\subseteq ET(S_2)$,
  $QT(S_1)\subseteq QT(S_2)$ und $QL(S_1)\subseteq QL(S_2)$ gilt.
\end{Def}

\begin{satz}[Lokale Error und Quiescent Semantik für Parallelkompositonen]
  \label{satzQuiSemantik}
  Für zwei komponierbare EIOs $S_1, S_2$ und $S_{12} = S_1\|S_2$ gilt:
  \begin{enumerate}
    \item $ET_{12} = cont(prune((ET_1\|QL_2)\cup (QL_1\|ET_2)))$,
    \item $QT_{12} = cont(prune(QT_1\|QT_2))$,%TODO???
    \item $QL_{12} = (QL_1\|QL_2)\cup ET_{12}\cup QT_{12}$.%TODO???
  \end{enumerate}
\end{satz}

\begin{proof}
  ~
  \begin{enumerate}
    \item \hspace{-0.2cm}:
  \end{enumerate}
  \vspace{-0.3cm}
  Der Beweis diese Punktes entspricht dem Beweis von Punkt 1.\ im Beweis von
  Satz~\ref{satzErrorSemanik}.

  2. ``$\subseteq$'':\\
  Wegen der Abgeschlossenheit beider Seiten der Gleichung gegenüber $cont$
  betrachten wir ein präfix-minimales Element $w\in PrQT_{12}$ und versuchen dessen
  Zugehörigkeit zur rechten Menge zu zeigen. Aufgrund von
  Definition~\ref{DefQuiescenttraces} wissen wir es gibt $v\in Q_{12}^*$,
  sodass gilt $(q_{01},q_{02}) \overset{w}{\Rightarrow} (q_1,q_2)
  \overset{v}{\Rightarrow} (q_1'q_2')$ mit $(q_1',q_2')\in Qui_{12}$. Durch
  Projektion erhalten wir $q_{01} \overset{w_1}{\Rightarrow} q_1
  \overset{v_1}{\Rightarrow} q_1'$ und $q_{02} \overset{w_2}{\Rightarrow} q_2
  \overset{v_2} q_2'$ mit $w\in w_1\|w_2$ und $v\in v_1\|v_2$. Aus
  $(q_1',q_2')\in Qui_{12}$ können wir folgern, dass bereits $q_1'\in Qui_1$
  und $q_2'\in Qui_2$ gilt. Somit gilt, wegen $w_1=prune(w_1v_1)$ und
  $w_2=prune(w_2v_2)$, $w_1\in PrQT_1\subseteq QT_1$ und $w_2\in
  PrQT_2\subseteq QT_2$. Daraus folgt dann $w\in QT_1\|QT_2$ und somit ist $w$
  in der rechten Seiten der Gleichung enthalten.

  2. ``$\supseteq$'':\\
  Für diese Inklusionsrichtung betrachten wir, wegen der Abgeschlossenheit
  gegenüber $cont$, ein präfix-minimales Element $x\in prune(QT_1\|QT_2)$ und
  zeigen, dass es in der linken Menge enthalten ist. Da $x$ durch die
  $prune$-Funktion entstanden ist, existiert ein $x\in O_{12}^*$ mit $xy\in
  QT_1\|QT_2$. Somit existieren $w_1\in QT_1$ und $w_2\in QT_2$ mit $xy\in
  w_1\|w_2$. Es gilt also $q_{01} \overset{w_1}{\Rightarrow}
  q_1\in Qui_1$ und $q_{02} \overset{w_2}{\Rightarrow} q_2\in Qui_2$. Da für
  die Zustände $q_1$ und $q_2$ die Zugehörigkeit zur Quiescent Menge gilt,
  können wir folgern, dass der aus ihnen zusammengesetzte Zustand in der
  Parallelkomposition ebenfalls keine Outputs zulässt. Somit gilt also für die
  Komposition $(q_{01},q_{02}) \overset{xy}{\Rightarrow} (q_1,q_2)\in Qui_{12}$
  und dadurch ist $x$ in der linken Seite der Gleichung enthalten, da
  $x=prune(xý)$ ist.

  3.:\\
  Es ist durch die Definition klar, dass gilt $L_i\subseteq QL_i$,
  $ET_i\subseteq QL_i$ und $QT_i\subseteq QL_i$. Wir beginnen mit der
  Argumentation von der rechten Seite der Gleichung aus:
  \begin{align*}
    &(QL_1\| QL_2)\cup ET_{12}\cup QT_{12}\\
    &\overset{\ref{DefQTQL}}{=}(L_1\cup ET_1\cup QT_1)\|(L_2\cup ET_2\cup
    QT_2)\cup ET_{12}\cup QT_{12}\\
    &=(L_1\|L_2) \cup
    \underset{\overset{1.}{\subseteq} ET_{12}}{\underset{\subseteq
    (EL_1\|ET_2)}{\underbrace{(L_1\|ET_2)}}} \cup
    \underset{?}{\underbrace{(L_1\|QT_2)}} \cup
    \underset{\overset{1.}{\subseteq} ET_{12}}{\underset{\subseteq
    (ET_1\|EL_2)}{\underbrace{(ET_1\|L_2)}}} \cup
    \underset{\overset{1.}{\subseteq}
    ET_{12}}{\underset{\subseteq (EL_1\|ET_2)}{\underbrace{(ET_1\|ET_2)}}} \cup
    \underset{?}{\underbrace{(ET_1\|QT_2)}}\\
    &\hspace{0.6cm}\cup\underset{?}{\underbrace{(QT_1\|L_2)}} \cup
    \underset{?}{\underbrace{(QT_1\|ET_2)}} \cup
    \underset{\overset{2.}{\subseteq}
    QT_{12}}{\underbrace{(QT_1\|QT_2)}} \cup
    ET_{12}\cup QT_{12}\\
    &=(L_1\|L_2) \cup ET_{12}\cup QT_{12}\\
    &\overset{\ref{LemmaSprache}}{=}L_{12}\cup ET_{12}\cup QT_{12}\\
    &\overset{\ref{DefQTQL}}{=}QL_{12}
  \end{align*}
\end{proof}

\begin{prop}[Präkongruenz]
  $\sqsubseteq _{Qui}$ ist eine Präkongruenz.
\end{prop}

\begin{proof}
  Es muss gezeigt werden, wenn $S_1\sqsubseteq _{Qui} S_2$ gilt,  für   jedes $
  S_3$ auch $S_3\|S_1\sqsubseteq _{Qui} S_3\|S_2$ gilt. D.h.\ es ist zu zeigen,
  dass aus $ET_1\subseteq ET_2$, $QT_1\subseteq QT_2$ und $QL_1\subseteq QL_2$
  folgt, $ET_{31}\subseteq ET_{32}$, $QT_{31}\subseteq QT_{32}$ und
  $QL_{31}\subseteq QL_{32}$.
  \begin{itemize}
    \item $\begin{aligned}[t]
        ET_{31} \overset{\mathrm{Beweis}~\ref{korPraekongruenz}
        ~\mathrm{Punkt}~1}{\subseteq} ET_{32}
    \end{aligned}$
    \item $\begin{aligned}[t]
        QT_{31} &\overset{\ref{satzQuiSemantik}~2.}{=} cont(prune(QT_3\|QT_1))\\
                &\hspace{-0.3cm}\overset{QT_1\subseteq QT_2}{\subseteq}
        cont(prune(QT_3\|QT_2))\\
                &\overset{\ref{satzQuiSemantik}~2.}{=} QT_{32}
    \end{aligned}$
    \item $\begin{aligned}[t]
        QL_{31} &\overset{\ref{satzQuiSemantik}~3.}{=} (QL_3\|QL_1)\cup EL_{31}
          \cup QL_{31}\\
        &\overset{QL_1 \subseteq QL_2,}{\overset{EL_{31}\subseteq
          EL_{32}}{\overset{\mathrm{und}}{\overset{QT_{31}\subseteq
          QT_{32}}{\subseteq}}}} (QL_3\|QL_2)\cup EL_{32}\cup QT_{32}
        &\overset{\ref{satzQuiSemantik}~3.}{=} QL_{32}
    \end{aligned}$
  \end{itemize}
\end{proof}

\begin{lem}[Verfeinerung mit Quiescents]
  Gegeben sind zwei EIOs $S_1$ und $S_2$ mit der gleichen Signatur. Wenn alle
  EIOs $U$ für die $S_2$ und $U$ gut kommunizieren auch $S_1$ und $U$ gut
  kommunizieren, dann verfeinert $S_1$ den EIO $S_2$. Diese Verfeinerung
  entspricht der Relation $\sqsubseteq _{Qui}$ von oben: Wenn
  $U\|S_1\sqsubseteq _{Qui}^B U\|S_2$ für alle $U$, dann gilt $S_1\sqsubseteq
  _{Qui} S_2$.
\end{lem}

\begin{proof}
  Da wir davon ausgehen, dass $S_1$ und $S_2$ die gleiche Signatur haben,
  definieren wir $I:=I_1=I_2$ und $O:=O_1=O_2$. Für jeden Partner $U$ gilt
  $I_U=O$ und $O_U=I$.\\
  Um zu zeigen, dass die Relation $S_1\sqsubseteq _{Qui}$ gilt, müssen wir die
  folgenden Punkte nachweisen:
  \begin{itemize}
    \item $ET(S_1)\subseteq ET(S_2)$,
    \item $QT(S_1)\subseteq QT(S_2)$,
    \item $QL(S_1)\subseteq QL(S_2)$.
  \end{itemize}
  Der erste wurde bereits in Lemma~\ref{lemVerfeinerung} gezeigt.
\end{proof}
