\chapter{Verfeinerung über Error- und Ruhetraces}

\section{Präkongruenz für Ruhe}

In diesem Kapitel werden wir uns nicht nur um die Erreichbarkeit von
Error-Zuständen kümmern, sondern auch um die Erreichbarkeit von
Ruhe-Zuständen. Wir werden dabei ähnlich vorgehen wie im letzten Kapitel,
jedoch halten wir uns als Quelle an~\cite{Chilton2013}. Darin werden ähnliche
Konzepte beschrieben, jedoch aus Sicht der Traces.\\
Wir sehen die Zustände, die keine Outputs und keine Transitionsmöglichkeit für
eine interne Aktion haben als ruhig an.

\begin{Def}[Ruhe]
  Ein \emph{Ruhe-Zustand} ist ein Zustand in einem \EIO{} der keine
  Outputs zulässt und keine Transitions mit $\tau$ besitzt.\\
  Somit ist die Menge der Ruhe-Zustände in einem \EIO{} wie folgt formal
  definiert: $Qui:=\big\{q\in Q\mid \forall \alpha\in (O\cup \{\tau\}): q
  \overset{\alpha}{\not{\hspace{-0.1cm}\rightarrow}}\big\}$.
\end{Def}

Für die Erreichbarkeit verwenden wir wie im letzten Kapitel wieder den
optimistischen Ansatz der lokalen Erreichbarkeit. Ruhe ist kein
unabwendbaren Fehler, sondern kann durch einen Input reparierbar werden. Somit
ist ein Ruhe-Zustand als nicht so \glqq{}schlimmer\grqq{} Fehler anzusehen wie
ein Error. Somit ist für uns ein Ruhe-Zustand zwar ebenso wie ein Error-Zustand
erreichbar sobald er durch Outputs und $\tau$s erreicht werden kann, jedoch ist
nicht jede beliebige Fortsetzung eines Traces, der diese Eigenschaft erfüllt
ein Ruhe-Trace.

\begin{Def}[error- und ruhefreie Kommunikation]
  Zwei \EIO{}s $S_1$ und $S_2$ \emph{kommunizieren gut}, wenn in ihrer
  Parallelkomposition $S_1\| S_2$ keine Errors und keine Ruhe-Zustände lokal
  erreichbar sind.
\end{Def}

\begin{Def}[Ruhe-Verfeinerungs-Basisrelation]
  Für \EIO{}s $S_1$ und $S_2$ mit der gleichen Signatur schreiben wir
  $S_1\QBRel S_2$, wenn ein Error oder Ruhe-Zustand in $S_1$ nur dann lokal
  erreichbar ist, wenn er auch in $S_2$ lokal erreichbar ist. Diese
  \emph{Basisrelation} stellt eine \emph{Verfeinerung} bezüglich \emph{Errors}
  und \emph{Ruhe-Zustände} dar.\\
  \QCRel{} bezeichnet die \emph{vollständig abstrakte Präkongruenz} von
  \QBRel{} bezüglich $\cdot\|\cdot$.
\end{Def}

Um uns genauer mit den Präkongruenzen auseinandersetzten zu können, brauchen
wir wie im letzten Kapitel die Definition von Traces auf unserer Struktur.
Dadurch erhalten wir die Möglichkeit die gröbste Präkongruenz finden und
definieren zu können.

\begin{Def}[Ruhetraces]
  \label{DefRuhetraces}
  Sei $S$ ein \EIO{} und definiere:
  \begin{itemize}
    \item \emph{strickte Ruhetraces}: $\StQT{}(S) := \{w\in\Sigma ^*\mid q_0
      \overset{w}{\Rightarrow} q\in Qui\}$,
    \item \emph{gekürzte Ruhetraces}: $\PrQT{}(S) := \bigcup\{\prunenew(w)\mid
      w\in \StQT{}(S)\}$.
  \end{itemize}
\end{Def}

\begin{Def}[Ruhe-Semantik]
  \label{DefQTQL}
  Sei $S$ ein \EIO{}.
  \begin{itemize}
    \item Die Menge der \emph{Ruhetraces} von $S$ ist $\QT{}(S) :=
      \PrQT{}(S)\cup \ET{}(S)$.
  \end{itemize}
  Für zwei \EIO{}s $S_1, S_2$ mit der gleichen Signatur schreiben wir
  $S_1\QRel S_2$, wenn $S_1\ERel S_2$ und $\QT{}(S_1)\subseteq \QT{}(S_2)$ gilt.
\end{Def}

Für die Menge der Ruhetraces \QT{} haben wir auch eine Informationsvermischung
mit den Errortraces vorgenommen wie beim fluten der Sprache \EL{}. Da jedoch
durch die Ruhetraces keine neuen Traces entstehen, die nicht bereits in der
gefluteten Sprache \EL{} enthalten wären, müssen wir hier keine neue Flutung
dafür vornehmen. Wir schränken also durch die Relation \QRel{} nur die
bereits existierende Präkongruenz \ERel{} ein.

\begin{satz}[Error- und Ruhe-Semantik für Parallelkompositonen]
  \label{satzQuiSemantik}
  Für zwei komponierbare \EIO{}s $S_1, S_2$ und $S_{12} = S_1\|S_2$ gilt:
  \begin{enumerate}
    \item $\ET{}_{12} = \cont{}(\prune{}((\ET{}_1\|\EL{}_2)\cup (\EL{}_1\|\ET{}_2)))$,
    \item $\QT{}_{12} = \prunenew(\QT{}_1\|\QT{}_2)\cup \ET{}_{12}$,
    \item $\EL{}_{12} = (\EL{}_1\|\EL{}_2)\cup \ET{}_{12}$.
  \end{enumerate}
\end{satz}

\begin{proof}
  ~
  \begin{enumerate}
    \item \hspace{-0.2cm}:
  \end{enumerate}
  \vspace{-0.3cm}
  Der Beweis diese Punktes entspricht dem Beweis von Punkt 1.\ im Beweis von
  Satz~\ref{satzErrorSemanik}.

  2. ``$\subseteq$'':\\
  Hier müssen wir unterscheiden ob wir ein $w\in\PrQT{}_{12}$ betrachten oder
  ein $w\in \ET{}_{12}$. Im zweiten Fall ist das $w$ in der rechten Seite
  enthalten. Somit betrachten wir ab jetzt ein
  $w\in \PrQT{}_{12}$ und versuchen dessen Zugehörigkeit zur rechten Menge zu
  zeigen. Aufgrund von Definition~\ref{DefRuhetraces} wissen wir es gibt ein
  $v\in O_{12}^*$, so dass gilt
  $(q_{01},q_{02}) \overset{wv}{\Rightarrow} (q_1,q_2)$ mit $(q_1,q_2)\in
  Qui_{12}$. Durch Projektion erhalten wir $q_{01} \overset{w_1v_1}{\Rightarrow}
  q_1$ und $q_{02} \overset{w_2v_2}{\Rightarrow} q_2$ mit $w\in w_1\|w_2$ und
  $v\in v_1\|v_2$. Aus
  $(q_1,q_2)\in Qui_{12}$ können wir folgern, dass bereits $q_1\in Qui_1$ und
  $q_2\in Qui_2$ gilt. Somit gilt $w_1v_1\in \StQT{}_1\subseteq \QT{}_1$ und
  $w_2v_2\in \StQT{}_2\subseteq \QT{}_2$. Jedoch können wir nicht $w_i\in
  \PrQT{}_i$ für $i\in\{1,2\}$ folgern, da die Outputs in $v$ erst durch die
  Parallelkomposition entstehen können und somit nicht $v_i\in O_i$ gelten
  muss. Wir können jedoch trotzdem folgern, dass $wv\in \QT{}_1\|\QT{}_2$ gilt
  und somit $w\in\prunenew (\QT{}_1\|\QT{}_2)$. Es ist also $w$ in der rechten
  Seiten der Gleichung enthalten.

  2. ``$\supseteq$'':\\
  Wir müssen nun wieder unterscheiden, nach dem aus welcher Menge unser
  betrachtetes Element stammt. Falls $w\in \ET{}_{12}$ gilt, so können wir die
  Zugehörigkeit zur linken Seite direkt folgern. Deshalb betrachten wir für den
  weiteren Beweis dieser
  Inklusionsrichtung ein Element $w\in \prunenew (\QT{}_1\|\QT{}_2)$ und
  zeigen, dass es in der linken Menge enthalten ist. Durch die Anwendung der
  \prunenew{}-Funktion wissen wir, dass ein $v\in O_{12}$ existiert, so dass
  $wv\in \QT{}_1\|\QT{}_2$ gilt. Da $\QT{}_i = \PrQT{}_i\cup \ET{}_i$ gilt,
  existieren für $w_1v_1$ und $w_2v_2$ mit $w\in w_1\| w_2$ und $v\in v_1\|
  v_2$ unterschiedliche Möglichkeiten:
  \begin{itemize}
    \item Fall 1 ($w_1v_1\in \StQT{}_1\wedge w_2v_2\in \StQT{}_2$): Es gilt in diesem
      Fall $q_{01} \overset{w_1v_1}{\Rightarrow} q_1\in Qui_1$ und $q_{02}
      \overset{w_2v_2}{\Rightarrow} q_2\in Qui_2$. Da $q_1$ und $q_2$ in der
      Ruhe-Menge enthalten sind, ist auch der Zustand, der aus ihnen
      zusammengesetzt ist in der Parallelkomposition ruhig und lässt keine
      Outputs und $\tau$-Transitionen zu. Es gilt also für die Komposition
      $(q_{01},q_{02}) \overset{wv}{\Rightarrow} (q_1,q_2)\in Qui_{12}$ und
      dadurch ist $w$ in der linken Seite der Gleichung enthalten, da $w\in
      \PrQT{}_{12}\subseteq \QT{}_{12}$ gilt.
    \item Fall 2 ($w_1v_1\in \ET{}_1\vee w_2v_2\in \ET{}_2$): \OBdA{} gilt
      $w_1v_1\in \ET{}_1$. Nun kann $w_2v_2\in \StQT{}_2\subseteq L_2$ gelten
      oder $w_2v_2\in \ET{}_2$ und somit gilt auf jeden Fall $w_2v_2\in
      \EL{}_2$. Daraus können wir dann mit dem ersten Punkt von
      Satz~\ref{satzErrorSemanik} folgern, dass $w\in \ET{}_{12}$ gilt und
      somit in der linken Seite der Gleichung enthalten ist.
  \end{itemize}

  3.:\\
  Der Beweis diese Punktes entspricht dem Beweis von Punkt 2.\ im Beweis von
  Satz~\ref{satzErrorSemanik}.
\end{proof}

Die folgende Proposition ist eine direkte Folgerung aus dem letzten Satz.
Jedoch ist es eine wichtige Feststellung für die weiteren Verlauf die gröbste
Präkongruenz finden zu wollen.

\begin{prop}[Präkongruenz]
  \label{propQuiPrae}
  \QRel{} ist eine Präkongruenz bezüglich $\cdot\|\cdot$.
\end{prop}

\begin{proof}
  Es muss gezeigt werden, wenn $S_1\QRel S_2$ gilt, für jedes $S_3$ auch
  $S_3\|S_1\QRel S_3\|S_2$ gilt. D.h.\ es ist zu zeigen, dass aus $S_1\ERel
  S_2$ und $\QT{}_1\subseteq \QT{}_2$ folgt, $S_{31}\ERel S_{32}$ und
  $\QT{}_{31}\subseteq \QT{}_{32}$. Dies ergibt sich wie im Beweis zu
  Proposition~\ref{propPraekongruenz} aus der Monotonie von \cont{}, \prune{}
  und $\cdot\|\cdot$ auf Sprachen wie folgt:
  \begin{itemize}
    \item $\begin{aligned}[t]
        S_{31} \overset{\mathrm{Beweis}~\ref{propPraekongruenz}}{\ERel} S_{32}
    \end{aligned}$
    \item $\begin{aligned}[t]
        \QT{}_{31} &\overset{\ref{satzQuiSemantik}~2.}{=}
        (\QT{}_3\|\QT{}_1)\cup \ET{}_{31}\\
        &\hspace{-0.5cm}\overset{\ET{}_{31}\subseteq
      \ET{}_{32}}{\overset{\mathrm{und}}{\overset{\QT{}_1\subseteq
      \QT{}_2}{\subseteq}}} (\QT{}_3\|\QT{}_2) \cup \ET{}_{32}\\
        &\overset{\ref{satzQuiSemantik}~2.}{=} \QT{}_{32}
    \end{aligned}$
  \end{itemize}
\end{proof}

\begin{lem}[Verfeinerung mit Ruhe-Zuständen]
  \label{lemQuiVerfeinerung}
  Gegeben sind zwei \EIO{}s $S_1$ und $S_2$ mit der gleichen Signatur. Wenn
  alle Partner \EIO{}s $U$, die mit $S_2$  gut kommunizieren, auch mit $S_1$
  gut kommunizieren, dann verfeinert $S_1$ das \EIO{} $S_2$. Diese Verfeinerung
  entspricht der Relation \QRel{} von oben: Wenn $U\|S_1\QBRel U\|S_2$ für alle
  Partner $U$, dann gilt $S_1\QRel S_2$.
\end{lem}

\begin{proof}
  Da wir davon ausgehen, dass $S_1$ und $S_2$ die gleiche Signatur haben,
  definieren wir $I:=I_1=I_2$ und $O:=O_1=O_2$. Für jeden Partner $U$ gilt
  $I_U=O$ und $O_U=I$.\\
  Um zu zeigen, dass die Relation $S_1\QRel S_2$ gilt, müssen wir die
  folgenden Punkte nachweisen:
  \begin{itemize}
    \item $S_1\ERel S_2$,
    \item $\QT{}(S_1)\subseteq \QT{}(S_2)$.
  \end{itemize}
  Der erste Punkt wurde bereits in Lemma~\ref{lemVerfeinerung}
  gezeigt. So bleibt uns nur noch die Inklusion $\QT{}_1\subseteq \QT{}_2$ zu
  zeigen. Diese Inklusion können wir jedoch noch anlog zum Beweis der Inklusion
  der geflutteten Sprachen in Lemma~\ref{lemVerfeinerung} weiter einschränken.
  Da wir bereits wissen, dass $\ET{}_1\subseteq\ET{}_2$ gilt, müssen wir nur
  noch $\PrQT{}_1\subseteq \QT{}_2$ zeigen.\\
  Wir wählen ein $w\in \PrQT{}(S_1)$ und zeigen, dass es auch in $\QT{}(S_2)$
  enthalten ist.\\
  Wir finden ein $v\in O_1$, so dass $wv\in \StQT{}_1$ gilt. Es ist also vom
  Startzustand von $S_1$ durch das Wort $wv$ ein ruhiger Zustand erreichbar.
  Dies hat keine Auswirkungen auf die Parallelkomposition $U\|S_1$, wenn in $U$
  kein Ruhe-Zustand durch $wv$ erreicht wird. Somit betrachten wir den Partner
  $U$ bei dem durch $wv$ ebenfalls ein Ruhe-Zustand erreicht wird. Daraus folgt
  dann, dass in $U\|S_1$ ein Ruhe-Zustand mit $wv$ erreicht wird und durch die
  gegebene Relation folgt dann auch, dass in $U\|S_2$ auch ein Ruhe-Zustand
  durch $wv$ erreicht werden muss. Dies kann nur der Fall sein, wenn in $S_2$
  bereits ein Ruhe-Zustand durch $wv$ erreicht wird. Somit gilt also $wv\in
  \StQT{}_2$ und daraus können wir mit $v\in O_1=O_2$ folgern, dass $w\in
  \PrQT{}_2\subseteq \QT{}_2$ gilt.
\end{proof}

\begin{satz}[Full Abstractness für Ruhe-Semantik]
  \label{satzQuiFullAbst}
  Seien $S_1$ und $S_2$ zwei \EIO{}s mit derselben Signatur. Dann gilt
  $S_1\QCRel S_2\Leftrightarrow S_1\QRel S_2$, insbesondere ist \QRel{} eine
  Präkongruenz.
\end{satz}

\begin{proof}
  Wie bereits in Proposition~\ref{propQuiPrae} festgehalten, ist \QRel{} eine
  Präkongruenz.

  \glqq{}$\Leftarrow$\grqq{}: Nach Definition gilt, wenn
  $\varepsilon\in\QT{}(S)$, ist in $S$ ein Ruhe-Zustand oder ein Error-Zustand
  lokal erreichbar. Somit impliziert $S_1\QRel S_2$, dass
  $\varepsilon\in\QT{}_2$ gilt, wenn $\varepsilon\in\QT{}_1$. Daraus folgt
  ebenfalls, dass $S_1\QBRel S_2$ gilt. Somit folgt aus $S_1\QRel S_2$ der
  relationale Zusammenhang $S_1\QCRel S_2$.

  \glqq{}$\Rightarrow$\grqq{}: Durch die Definition von \QCRel{} folgt aus
  $S_1\QCRel{} S_2$, dass $U\|S_1\QCRel U\|S_2$ für alle EIOs $U$ , die mit
  $S_1$ komponierbar sind. Somit folgt auch die Gültigkeit von $U\|S_1\EBRel
  U\|S_2$ für alle diese EIOs $U$. Mit Lemma~\ref{lemQuiVerfeinerung} folgt
  dann $S_1\QRel{} S_2$.
\end{proof}

Wir haben somit, wie im letzten Kapitel, eine Kette an Folgerungen gezeigt, die
sich zu einem Ring schließen. Dies ist in Abbildung~\ref{FolgerungsketteQui}
dargestellt.

\begin{figure}[h!tbp]
  \begin{center}
    \begin{tikzpicture}
      \matrix (m) [matrix of math nodes,row sep=2cm,column sep=4cm]{
        S_1\QRel S_2 & S_1\QCRel S_2 \\
        \substack{\forall~\mathrm{Partner}~U:\\U\|S_1\QBRel U\|S_2} &
      \substack{\forall~\mathrm{komponierbaren}~U:\\U\|S_1\QBRel U\|S_2} \\};
        \draw[-implies, double, double distance=1mm]
          (m-1-1) -- node [above] {\glqq{}$\Leftarrow$\grqq{} von
            Satz~\ref{satzQuiFullAbst}} (m-1-2);
        \draw[-implies, double, double distance=1mm]
          (m-1-2) -- node [right] {\glqq{}$\Rightarrow$\grqq{} von
            Satz~\ref{satzQuiFullAbst}} (m-2-2);
        \draw[-implies, double, double distance=1mm]
          (m-2-1) -- node [left]
          {Lemma~\ref{lemQuiVerfeinerung}} (m-1-1);
        \draw[-implies, double, double distance=1mm]
        (m-2-2) -- node [below]
        {$\substack{U~\mathrm{Partner}\\\Downarrow\\U~\mathrm{komponierbar}}$} (m-2-1);
    \end{tikzpicture}
    \caption{Folgerungskette}
    \label{FolgerungsketteQui}
  \end{center}
\end{figure}

Aus Satz~\ref{satzQuiFullAbst} und Lemma~\ref{lemQuiVerfeinerung} erhalten wir
das folgende Korollar.

\begin{kor}
  Ein \EIO{} $S_1$ verfeinert einen \EIO{} $S_2$ genau dann, wenn für alle
  \EIO{}s $U$ für die $S_2$ gut mit $U$ kommuniziert folgt $S_1$ kommuniziert
  ebenfalls gut mit $U$.\\
  Dies lässt sich formal wie folgt ausdrücken: $S_1\QRel S_2\Leftrightarrow
  U\|S_1\QBRel U\|S_2$ für alle Partner $U$.
\end{kor}

\section{Hiding für Ruhe}

Wir wollen nun auch hier die Auswirkungen der Internalisierung von Aktionen auf
unsere Verfeinerungsrelationen untersuchen. Es werden Outputs in interne
Handlungen umgewandelt. Da wir jedoch bei unseren Ruhe-Zuständen auch $\tau$
Übergänge verboten haben, verändert sich nichts an der Menge der ruhigen
Zustände. Da wir die Erreichbarkeit von Ruhe-Zuständen mittels lokaler Aktionen
betrachten, kann sich auch nichts an der Erreichbarkeit der Ruhe-Zustände
ändern. Somit können wir eine analoge Proposition zu~\ref{propErBaIn}
formulieren.

\begin{prop}[Ruhe-Basisrelation bzgl. Interlaisierung]
  Wenn $S_1\QBRel S_2$ gilt, dann folgt daraus, dass auch
  $(S_1/\{x_1,x_2,\dots, x_n\})\QBRel (S_2/\{x_1,x_2,\dots, x_n\})$ gilt.
\end{prop}

\begin{proof}
  Dass die Error-Erreichbarkeit unverändert bleibt unter Hiding wurde bereits
  im Beweis zur Proposition~\ref{propErBaIn} gezeigt. Mit der analogen
  Argumentation folgt auch, dass sich nicht an der Erreichbarkeit der
  Ruhe-Zustände ändert. Es können durch Hiding nämlich nur Outputs verborgen
  werden, die bereits in der Menge der lokalen Handlungen enthalten sind.
\end{proof}

\begin{satz}[Präkongurenz bzgl. Internalisierung]
  \label{satzPraeInterQui}
  Seien $S_1$ und $S_2$ zwei \EIO{}s für die $S_1\QRel S_2$ gilt, somit gilt
  auch $(S_1/\{x_1,x_x,\dots ,x_n\})\QRel (S_2/\{x_1,x_x,\dots ,x_n\})$. Somit
  ist also \QRel{} eine Präkongruenz bezüglich $\cdot /\cdot$.
\end{satz}

\begin{proof}
  Da $S_1\QRel S_2$ gilt, wissen wir, dass $S_1\ERel S_2$ und $\QT{}_1\subseteq
  \QT{}_2$ gilt. Wir wissen bereits aufgrund von
  Satz~\ref{satzPraeInternalisierung}, dass daraus $(S_1/\{x_1,x_x,\dots
  ,x_n\})\ERel (S_2/\{x_1,x_x,\dots ,x_n\})$ folgt. Ebenso wie im Beweis zu
  Satz~\ref{satzPraeInternalisierung} gehen wir hier davon aus, dass
  $X\subseteq O$ gilt. Für jeden Trace aus $\QT{}_1$ können wir einen passenden in
  $\QT{}(S_1/X)$ finden und ebenso für das Transitionssystem $S_2$. Die
  Argumentation läuft hierfür analog zum Beweis von
  Satz~\ref{satzPraeInternalisierung}. Somit gilt
  also auch $\QT{}(S_1/X)\subseteq \QT{}(S_2/X)$. Daraus folgt dann, dass die
  Relation \QRel{} trotz Hiding erhalten ist und somit das Hiding bezüglich
  dieser Relation eine Präkongruenz ist.
\end{proof}

In Definition~\ref{defIntParal} wurde mit Hilfe des Internalisierungsoperator
aus der Parallelkomposition ohne Verbergen die Parallelkomposition mit
Verbergen der synchronisierten Aktionen nachgebildet. Wir können deren
Eigenschaft als Präkongruenz aus den Präkongruenz-Eigenschaften von
$\cdot\|\cdot$ und $\cdot /\cdot$ bezüglich \QRel{} aus der
Proposition~\ref{propQuiPrae} und dem Satz~\ref{satzPraeInterQui} schließen.

\begin{kor}[Präkongruenz mit Internalisierung]
  \QRel{} ist eine Präkongruenz bezüglich $\cdot |\cdot$.
\end{kor}
