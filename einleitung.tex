\chapter{Einleitung}

Wie schon aus dem Titel hervor geht, sollen in dieser Arbeit Interface
Automaten und Fehler, die durch die Kommunikation mehrere solcher enstehen
betrachtet werden. Eine entsprechende Definition dieser Interface Automaten
findet sich in~\cite{Alfaro2004}. Es handelt sich also um Systeme, die via
Inputs und Outputs mit anderen Systemen kommunizieren können. Jedoch wir in
ausgeschlossen, dass es Fehler in einem Interface Automaten gibt. Es müsste
also jedes mal, wenn ein Fehler durch eine Kommunikation, die nicht so möglich
ist wie gewollt, eine Veränderung an der Komposition von zwei Systemen
vorgenommen werden. Es müssten also alle Wege, die nicht beeinflusst werden
können und die zu einem solchen Fehler-Zustand führen aus dem Automaten
entfernt werden. Dies ist ein sehr umständliches Vorgehen, wenn man
ausschließlich solche Systeme betrachten möchte. Deshalb wurden
Transitionssysteme eingeführt, die Fehler-Zustände zulassen. Die sich jedoch
trotzdem noch entsprechend den Interface Automaten verhalten. Es kann dann nach
der Betrachtung trotzdem die Fehlerfreiheit eines Transitionssystems mit Fehler
hergestellt werden, in dem die entsprechenden Wege entfernt werden.\\
Die Betrachtung der Transitionssysteme mit Fehler-Zuständen hat auch den
Vorteil, dass die Inputs nicht als deterministisch vorausgesetzt werden müssen
um sicher zu stellen, dass nach dem entfernen eines Weges zu einem Fehler, der
gleiche Input nicht noch zu einem anderen Zustand führt.\\
Um die Begrifflichkeiten hier eindeutiger zu machen, wird im weiteren Verlauf
das Wort Error für Kommunikationsfehler verwendet und für Verklemmung das Wort
Ruhe. Als Fehler werden im weiteren Kommunikationsfehler, Verklemmung und
Divergenz bezeichnet.\\
Der Anfang dieser Arbeit orientiert sich sehr stark an~\cite{Vogler2014EIO}.
Jedoch wird hier darauf verzichtet die Input-Mengen der
Error-IO-Transitionssysteme (\EIO{}s) als disjunkt anzunehmen und alle
Definitionen und Sätze werden erst einmal ohne das Verbergen der
synchronisierten Aktionen betrachtet.\\
Dadurch dass die synchronisierten Aktionen nicht verborgen werden, wird hier
ein Modell betrachtet, mit dem nicht nur zwei Systeme miteinander kommunizieren können,
sondern beliebig viele. Ein Output eines Systems ist somit eine Art Multicast.
Jedes System, das diesen Output als Input verarbeiten kann, empfängt ihn somit auch,
da bei jeder Komposition der Output weitergeleitet wird an andere Systeme.
Kann jedoch ein System den Output nicht als Input aufnehmen, wird dieses System von
der Nachricht nicht beeinträchtigt.\\
Anschießend werden die Auswirkung von Hiding auf diese Struktur
untersucht und somit das Verbergen in der Parallelkomposition nachgebildet.
Durch das Hiding können Outputs durch interne Aktionen ersetzt werden.\\
Diese Art der Betrachtung der
\EIO{}s wurde auch bereits in~\cite{Schlosser2012BA} gewählt, jedoch wurde
diese Arbeit nicht als direkte Quelle genutzt, bis auf den Abschnitt des
Hidings. Die Feststellungen im Definitionskapitel und dem Kapitel über
Errors stimmen mit dieser Quelle überein, jedoch wurden alle Beweise davon unabhängig neu
geführt.\\
In dieser Arbeit wird ein optimistischer Ansatz für die Erreichbarkeit
der jeweils betrachteten Zuständen verwendet. Ein Zustand gilt nach der Definition in dieser
Arbeit als erreichbar, wenn er lokal erreicht
werden kann, d.h.\ durch lokale Aktionen. Die Menge bestehend aus der internen
Aktion $\tau$ und den Output-Aktionen wird hier als Menge der lokale Aktionen
bezeichnet.
Alle Elemente aus dieser Menge können ohne weiteres Zutun von außen ausgeführt
werden. Somit kann nicht beeinflusst werden, ob diese Transitionen genutzt
werden oder nicht. Es besteht also die Möglichkeit, dass das \EIO{} in einen
der betrachteten Zustände übergeht, sobald dieser lokal erreichbar ist. Diese Art der
Erreichbarkeit von Zuständen wird auch in Kapitel 3 von~\cite{Vogler2014EIO}
für Error-Zustände behandelt.\\
Neben dem hier betrachteten optimistischen Ansatz gibt es noch zwei weitere
Ansätze in~\cite{Vogler2014EIO} für die Erreichbarkeit von Error-Zuständen:
einen hyper-optimistischen Ansatz, bei dem ein Error als erreichbar gilt, wenn
er durch interne Aktionen erreicht werden kann, und einen pessimistischen
Ansatz, bei dem ein Error als erreichbar gilt, sobald es eine Folge an Inputs
und Outputs gibt, mit denen der Error-Zustand vom Startzustand aus erreicht
werden kann.\\
Es wird versucht bei allen betrachteten Zustandsmengen die gröbste Präkongruenz zu
finden, die in der jeweiligen Basisrelation enthalten ist und die eine
Präkongruenz bezüglich der Parallelkomposition ist.\\
Es werden im Verlauf dieser Arbeit Ruhe-Zustände betrachtet, die keine Outputs
und keine $\tau$s zulassen. Somit befindet sich das betrachtete
Transitionssystem in einer Art Verklemmung, wenn es in einem Ruhe-Zustand ist.
Das System ist dann auf einen Input von Außen angewiesen um sich wieder aus
diesem Zustand befreien zu können. Es kann ohne diesen Input keinen Fortschritt
mehr geben, in Form von Outputs. Da aber auch die $\tau$-Transitionen verboten
sind, kann das System auch keine interne Aktion zu einem anderen Zustand
ausführen.\\
Eine andere Betrachtungsweise zeigt dann die Hinzunahme von unendlichen Traces,
die mit der Eigenschaft der Divergenz genauer betrachtet werden sollen. Hierbei
kann ein System unendliche viele $\tau$-Transitionen ausführen.

\scriptsize\textcolor{lgray}{TODO: erweitern/umformulieren (bis jetzt nur Teile
aus anderen Kapitel in Einleitung verschoben)}

\normalsize
