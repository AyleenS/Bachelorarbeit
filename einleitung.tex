\chapter{Einleitung}

Der Anfang dieser Arbeit orientiert sich sehr stark an~\cite{Vogler2014EIO}.
Jedoch wird hier darauf verzichtet die Input-Mengen der
Error-IO-Transitionssysteme (\EIO{}s) als disjunkt anzunehmen und alle
Definitionen und Sätze werden erst einmal ohne das Verbergen der
synchronisierten Aktionen betrachtet.\\
Dadurch dass die synchronisierten Aktionen nicht verborgen werden, wird hier
ein Modell betrachtet, mit dem nicht nur zwei Systeme miteinander kommunizieren können,
sondern beliebig viele. Ein Output eines Systems ist somit eine Art Multicast.
Jedes System, das diesen Output als Input verarbeiten kann, empfängt ihn somit auch,
da bei jeder Komposition der Output weitergeleitet wird an andere Systeme.
Kann jedoch ein System den Output nicht als Input aufnehmen, wird dieses System von
der Nachricht nicht beeinträchtigt.\\
Anschießend werden die Auswirkung von Hiding auf diese Struktur
untersucht und somit das Verbergen in der Parallelkomposition nachgebildet.
Durch das Hiding können Outputs durch interne Aktionen ersetzt werden.\\
Diese Art der Betrachtung der
\EIO{}s wurde auch bereits in~\cite{Schlosser2012BA} gewählt, jedoch wurde
diese Arbeit nicht als direkte Quelle genutzt, bis auf den Abschnitt des
Hidings. Die Feststellungen im Definitionskapitel und dem Kapitel über
Errors stimmen mit dieser Quelle überein, jedoch wurden alle Beweise davon unabhängig neu
geführt.\\
In dieser Arbeit wird ein optimistischer Ansatz für die Erreichbarkeit
der Error-Zustände verwendet. Ein Error gilt ist nach der Definition in dieser
Arbeit erreichbar, wenn er lokal erreicht
werden kann, d.h.\ durch lokale Aktionen. Die Menge bestehend aus der internen
Aktion $\tau$ und den Output-Aktionen wird hier als Menge der lokale Aktionen
bezeichnet.
Alle Elemente aus dieser Menge können ohne weiteres Zutun von außen ausgeführt
werden. Somit kann nicht beeinflusst werden, ob diese Transitionen genutzt
werden oder nicht. Es besteht also die Möglichkeit, dass das \EIO{} in einen
Error-Zustand übergeht, sobald dieser lokal erreichbar ist. Diese Art der
Erreichbarkeit von Zuständen wird auch in Kapitel 3 von~\cite{Vogler2014EIO}
behandelt.\\
Neben dem hier betrachteten optimistischen Ansatz gibt es noch zwei weitere
Ansätze in~\cite{Vogler2014EIO} für die Erreichbarkeit von Error-Zuständen:
einen hyper-optimistischen Ansatz, bei dem ein Error als erreichbar gilt, wenn
er durch interne Aktionen erreicht werden kann, und einen pessimistischen
Ansatz, bei dem ein Error als erreichbar gilt, sobald es eine Folge an Inputs
und Outputs gibt, mit denen der Error-Zustand vom Startzustand aus erreicht
werden kann.\\
Im Kapitel über die Ruhe-Zuständen wird für die Erreichbarkeit einen
hyper-optimistischen Ansatz wählen, da man den Ruhe-Zuständen durch einen Input
entrinnen kann und sie somit als nicht so \glqq{}schlimm\grqq{} anzusehen sind
wie Errors.\\
Es wird versucht bei allen betrachteten Zustandsmengen die gröbste Präkongruenz zu
finden, die in der jeweiligen Basisrelation enthalten ist und die eine
Präkongruenz bezüglich der Parallelkomposition ist.

\scriptsize\textcolor{lgray}{TODO: erweitern/umformulieren (bis jetzt nur Teile
aus anderen Kapitel in Einleitung verschoben)}

%TODO Verklemmung diskutieren (Quiescents)

\normalsize
