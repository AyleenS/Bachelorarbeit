\chapter{Einleitung}

Wir wählen in dieser Arbeit einen optimistischen Ansatz für die
Erreichbarkeit von Zuständen, die wir in den jeweiligen Kapiteln betrachten
werden. Ein Zustand gilt hier als erreichbar, wenn er lokal
erreicht werden kann, d.h.\ durch lokale Aktionen. Die Menge, bestehend aus der
internen Aktion $\tau$ und den Outputaktionen, bezeichnen wir hier als lokale
Aktionen. Alle Elemente aus dieser Menge können ohne
weiteres Zutun von außen aufgeführt werden. Somit kann nicht beeinflusst werden ob diese
Übergänge genommen werden oder nicht. Es besteht also die Möglichkeit, dass
das \EIO{} in einen dieser bestimmten Zustände übergeht, sobald dieser lokal erreichbar ist. Diese
Art der Erreichbarkeit von Zuständen wird auch in Kapitel 3 von~\cite{Vogler2014EIO}
behandelt.\\
Neben dem hier betrachteten optimistischen Ansatz gibt es noch zwei weitere
Ansätze in~\cite{Vogler2014EIO}. Einen hyper-optimistischen Ansatz, bei dem ein
Zustand als erreichbar gilt, wenn er durch interne Aktionen erreicht werden
kann, und einen pessimistischen Ansatz, bei dem ein Zustand als erreichbar gilt,
sobald es eine Folge an Inputs und Outputs gibt, mit denen einer dieser
bestimmten Zustände vom Startzustand aus erreicht werden kann.\\
Wir versuche bei allen Ansätzen eine gröbste Präkongruenz zu finden, die in der
jeweiligen Basisrelation enthalten ist und die eine Präkongruenz bezüglich der
Parallelkomposition ist.
