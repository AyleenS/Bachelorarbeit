\chapter{Einleitung}

Der Anfang dieser Arbeit orientiert sich sehr stark an~\cite{Vogler2014EIO}.
Jedoch wird hier darauf verzichtet die Inputmengen Der
Error-IO-Transitionssysteme (\EIO{}s) als disjunkt anzunehmen und alle
Definitionen und Sätze werden erst einmal ohne das verbergen der
synchronisierten Aktionen betrachtet. Anschießend, wird jedoch die Auswirkung
von Hiding auf unsere Struktur untersucht und somit das Verbergen nachgebildet.
Diese Art der Betrachtung der \EIO{}s wurde auch bereits
in~\cite{Schlosser2012BA} gewählt, jedoch wurde dieser Arbeit nicht als direkte
Quelle genutzt, bis auf den Abschnitt des Hidings. Die Feststellungen in dem
Definitions-Kapitel und der Kapitel über Errors stimmen somit überein, jedoch
wurden alle Beweise unabhängig davon neu geführt.\\
Wir wählen in dieser Arbeit einen optimistischen Ansatz für die
Erreichbarkeit von Error-Zuständen. Ein Error gilt hier als erreichbar, wenn er
lokal erreicht werden kann, d.h.\ durch lokale Aktionen. Die Menge, bestehend
aus der internen Aktion $\tau$ und den Outputaktionen, bezeichnen wir hier als
lokale Aktionen. Alle Elemente aus dieser Menge können ohne weiteres Zutun von
außen ausgeführt werden. Somit kann nicht beeinflusst werden ob diese Übergänge
genommen werden oder nicht. Es besteht also die Möglichkeit, dass das
\EIO{} in einen Error-Zustände übergeht, sobald dieser lokal erreichbar ist.
Diese Art der Erreichbarkeit von Zuständen wird auch in Kapitel 3
von~\cite{Vogler2014EIO} behandelt.\\
Neben dem hier betrachteten optimistischen Ansatz gibt es noch zwei weitere
Ansätze in~\cite{Vogler2014EIO}. Einen hyper-optimistischen Ansatz, bei dem ein
Zustand als erreichbar gilt, wenn er durch interne Aktionen erreicht werden
kann, und einen pessimistischen Ansatz, bei dem ein Zustand als erreichbar gilt,
sobald es eine Folge an Inputs und Outputs gibt, mit denen der Zustand vom
Startzustand aus erreicht werden kann. Der hyper-optimistische Ansatz wird in
dieser Arbeit für die Erreichbarkeit von Quiescents verwendet.\\
Wir versuche bei allen Ansätzen eine gröbste Präkongruenz zu finden, die in der
jeweiligen Basisrelation enthalten ist und die eine Präkongruenz bezüglich der
Parallelkomposition ist.

\scriptsize\textcolor{lgray}{TODO: erweitern/umformulieren (bis jetzt nur Teile aus anderen
Kapitel in Einleitung verschoben)}

\normalsize
