\chapter{Einleitung}

Interface-Automaten werden im Allgemeinen dafür eingesetzt, um Verhalten vom
System oder Teilen davon zu modellieren und zu
untersuchen. Sie beschreiben abstrakt des Kommunikationsverhaltens
eines Systems oder einer Komponente durch Inputs und Outputs. Es können durch
Interface-Automaten
parallel arbeitende Softwarekomponenten simuliert werden. Die Kommunikation
zwischen  Systemen wird durch eine Parallelkomposition der beteiligten Komponenten
erreicht. Grundsätzlich wird hier ein Output bzw.\ eine
interne Aktion als vom jeweiligen System
kontrolliert angesehen und ein Input als von der Umwelt kontrolliert. Falls in
einer Parallelkomposition ein Output von einem System ausgeführt wird, muss
dieser von allen anderen beteiligten System, die diese Aktion in ihrer
Inputmenge haben, durch einen
Input synchronisiert werden. Falls dies nicht möglich ist, tritt ein
Kommunikationsfehler ein, der verhindert werden sollte, da das
Verhalten des Systems dann unvorhersehbar wird. In~\cite{Lynch1996} wird die
Abwesenheit dieser Kommunikationsfehler
für die darin betrachteten I/O-Automaten garantiert, in dem für alle Zustände alle Inputs möglich
sind. In Interface-Automaten ist ein fehlender Input jedoch durch die
Voraussetzung zu begründen, dass die Umwelt diesen Input in diesem Zustand nicht senden
darf. Es ist also nicht sinnvoll, für die Vermeidung von Kommunikationsfehler
in Interface-Automaten, alle Inputs für alle Zustände zuzulassen, da dann nicht
mehr alle zulässigen Voraussetzungen erfüllt werden könnten.\\
So wie die Interface-Automaten in~\cite{Alfaro2004} definiert sind, kann es
in der Parallelkomposition von zwei dieser Automaten zu Kommunikationsfehlern
kommen, in dem eine Komponente einen Output macht, den die andere nicht als Input
aufnehmen kann. Diese Fehler-Zustände, die daraus entstehen, sind jedoch nicht
zulässig und müssen inklusive aller Zustände, durch die man mit lokalen
Aktionen einen dieser Fehler-Zustände erreichen kann, mit einer speziell definieren
Operation gestutzt werden.\\
Um dem Problem des aufwendigen Stutzens entgegenzuwirken, soll hier eine
andere Modellierung betrachtet werden. Die Komponenten sollen als beschriftetes
Transitionssystem (LTS) mit disjunkten Input- und Output-Aktionen und einer
internen Aktion $\tau$ modelliert werden. Diese Modelle werden dann
Error-IO-Transitionssysteme (\EIO{}s) genannt. Für die Systeme sollen explizite
Error-Zustände möglich sein. Durch die Parallelkomposition können dann wie oben
beschrieben neue Error-Zustände entstehen und es werden zusätzlich auch noch die
Error-Zustände der einzelnen Komponenten geerbt.\\
Darüber sind dann Verfeinerungsrelationen möglich, diese sollen die
Voraussetzung erfüllen, dass eine fehlerfrei Spezifikation nur durch ein
fehlerfreies System verfeinert werden kann. Dies wird hier als Basisrelation
verstanden, die durch die explizite Definition der jeweiligen Fehler-Art parametrisiert
ist. Die Verfeinerungsrelation kann auch als Implementierung aufgefasst
werden.\\
Da Modularität bzw.\ Austauschbarkeit der Komponenten gewünscht ist, sollte die
Verfeinerungsrelationen eine Präkongruenz sein. Falls eine Komponente einer
Parallelkomposition durch eine Verfeinerung ersetzt wird, soll die Komposition
selbst auch verfeinert werden. Da die Basisrelation keine Präkongruenz ist,
wird eine größte Präkongruenz bezüglich der Parallelkomposition
charakterisiert, die in der Basisrelation enthalten ist.\\
In~\cite{Vogler2014EIO} wurde für Kommunikationsfehler-Freiheit nachgewiesen, dass jedes \EIO{}
äquivalent ist zu einem ohne Error-Zustände unter Berücksichtigung der
Präkongruenz. Diesen erhält man durch Stutzen wie in~\cite{Alfaro2004}. Es ist
also möglich die \EIO{}s nur für die Untersuchung zu verwenden und die
Fehler-Zustände am Ende zu entfernen, um wieder Interface-Automaten zu
erhalten.\\
Die Betrachtung der Transitionssysteme mit Fehler-Zuständen hat auch den
Vorteil, dass die Inputs nicht als deterministisch vorausgesetzt werden müssen
um sicher zu stellen, dass nach dem Entfernen eines Weges zu einem Fehler, der
gleiche Input nicht noch zu einem anderen zulässigen Zustand führt. Jedoch muss
dann vor der Umwandlung in einen Interface-Automaten durch das Entfernen der
Fehler-Zustände auf diese Inputs geachtet werden.\\
Um die Begrifflichkeiten hier eindeutiger und intuitiver, bzgl.\ dessen was
wirklich in dem System passiert, zu machen, wird im weiteren Verlauf
das Wort Error für Kommunikationsfehler verwendet und für Verklemmung das Wort
Ruhe. Als Fehler werden im weiteren Kommunikationsfehler, Verklemmung und
Divergenz bezeichnet.\\
Der Anfang dieser Arbeit orientiert sich sehr stark an~\cite{Vogler2014EIO}.
Jedoch wird hier darauf verzichtet die Input-Mengen der \EIO{}s als disjunkt
anzunehmen und alle Definitionen und Sätze werden erst einmal ohne das
Verbergen der synchronisierten Aktionen betrachtet. Da die Input-Mengen einen
nicht leeren Schnitt haben können, muss festgelegt werden, dass die Komposition
aus zwei Systemen einen Input aus diesem Schnitt nur ausführen kann, wenn beide
Systeme die Möglichkeit für diesen Input haben.\\
Dadurch dass die synchronisierten Aktionen nicht verborgen werden, wird hier
ein Modell betrachtet, mit dem nicht nur zwei Systeme miteinander kommunizieren können,
sondern beliebig viele. Ein Output eines Systems ist somit eine Art Multicast.
Jedes System, das diesen Output als Input verarbeiten kann, empfängt ihn auch,
da bei jeder Komposition der Output weitergeleitet wird an andere Systeme.
Kann jedoch ein System den Output nicht als Input aufnehmen, wird dieses System von
der Nachricht nicht beeinträchtigt.\\
Anschießend werden die Auswirkung von Hiding auf diese Struktur
untersucht und somit das Verbergen in der Parallelkomposition nachgebildet.
Durch das Hiding können Outputs durch interne Aktionen ersetzt werden. Es wird
also die Kommunikation durch diesen Output mit anderen Systemen verboten, bzw.\
der Kommunikationskanal wird geschlossen, da die Aktion internalisiert wird.\\
Diese Art der Betrachtung der
\EIO{}s wurde auch bereits in~\cite{Schlosser2012BA} gewählt, jedoch wurde
diese Arbeit nicht als direkte Quelle genutzt, bis auf den Abschnitt des
Hidings. Die Feststellungen im Definitionskapitel und dem Kapitel über
Errors stimmen mit dieser Quelle überein, jedoch wurden alle Beweise davon unabhängig neu
geführt.\\
In dieser Arbeit wird ein optimistischer Ansatz für die Erreichbarkeit
der jeweils betrachteten Zustände verwendet. Ein Zustand gilt nach der Definition in dieser
Arbeit als erreichbar, wenn er lokal erreicht
werden kann, d.h.\ durch lokale Aktionen. Die Menge, bestehend aus der internen
Aktion $\tau$ und den Output-Aktionen, wird hier als Menge der lokalen Aktionen
bezeichnet.
Alle Elemente aus dieser Menge können ohne weiteres Zutun von außen ausgeführt
werden. Somit kann nicht beeinflusst werden, ob diese Transitionen genutzt
werden oder nicht. Es besteht also die Möglichkeit, dass das \EIO{} in einen
der betrachteten Zustände übergeht, sobald dieser lokal erreichbar ist. Diese Art der
Erreichbarkeit von Zuständen wird auch in Kapitel 3 von~\cite{Vogler2014EIO}
für Error-Zustände behandelt.\\
Allgemein werden hier nur optimistische Ansätze betrachtet, d.h.\ es wird davon
ausgegangen, dass ein Fehler kein Problem ist, solang er in einer hilfreichen
Umgebung nicht erreicht wird. Jedoch ist auch die pessimistische Ansicht weit
verbreitet, dass ein System mit einem Fehler niemals zulässig sein kann. Da man
nicht immer von einer fehlerfreien Spezifikation ausgehen kann, scheint der
optimistische Ansatz näher an der Realität zu liegen.\\
Neben dem hier betrachteten optimistischen Ansatz gibt es noch zwei weitere
Ansätze in~\cite{Vogler2014EIO} für die Erreichbarkeit von Error-Zuständen:
einen hyper-optimistischen Ansatz, bei dem ein Error als erreichbar gilt, wenn
er durch interne Aktionen erreicht werden kann, und einen pessimistischen
Ansatz, bei dem ein Error als erreichbar gilt, sobald es eine Folge an Inputs
und Outputs gibt, mit denen der Error-Zustand vom Startzustand aus erreicht
werden kann.\\
Es wird versucht bei allen betrachteten Zustandsmengen die gröbste Präkongruenz zu
finden, die in der jeweiligen Basisrelation enthalten ist und die eine
Präkongruenz bezüglich der Parallelkomposition ist.\\
Es werden im Verlauf dieser Arbeit Ruhe-Zustände betrachtet, die keine Outputs
und keine $\tau$s zulassen. Somit befindet sich das betrachtete
Transitionssystem in einer Art Verklemmung, wenn es in einem Ruhe-Zustand ist.
Das System ist dann auf einen Input von Außen angewiesen um sich wieder aus
diesem Zustand befreien zu können. Es kann ohne diesen Input keinen Fortschritt
mehr geben in Form von Outputs. Da aber auch die $\tau$-Transitionen verboten
sind, kann das System auch keine interne Aktion zu einem anderen Zustand
ausführen. Somit wurden die Ruhe-Zustände als Art Deadlocks umgesetzt,
jedoch falls man die $\tau$s nicht komplett verbieten würde, könnten auch
Livelocks damit dargestellt werden.\\
Eine andere Betrachtungsweise zeigt dann die Hinzunahme von unendlichen Traces,
die mit der Eigenschaft der Divergenz genauer betrachtet werden sollen. Hierbei
kann ein System unendlich viele $\tau$-Transitionen ausführen. Es kann dann
auch relevant sein, dass die hier betrachteten Transitionssysteme nicht endlich sein
müssen.

Kapitel 2 erläutert die grundlegenden Definitionen. In Kapitel 3 bis 5 werden
nacheinander die Fehler-Mengen, die betrachtet werden, immer weiter vergrößert.
Zuerst werden in Kapitel 3 nur Errors betrachtet, in Kapitel 4 auch noch
zusätzlich Ruhe. Im letzten Kapitel werden dann Error, Ruhe und Divergenz
betrachtet.
