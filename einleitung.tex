\chapter{Einleitung}

Der Anfang dieser Arbeit orientiert sich sehr stark an~\cite{Vogler2014EIO}.
Jedoch wird hier darauf verzichtet die Inputmengen Der
Error-IO-Transitionssysteme (\EIO{}s) als disjunkt anzunehmen und alle
Definitionen und Sätze werden erst einmal ohne das verbergen der
synchronisierten Aktionen betrachtet.\\
Dadurch das die synchronisierten Aktionen nicht verborgen werden, haben wir hier
ein Modell, mit dem nicht nur zwei Systeme miteinander kommunizieren können
sondern beliebig viele. Ein Output eines Systems ist somit ein Art Multicast.
Jedes System, dass diesen Output als Input haben kann, empfängt ihn somit auch,
da bei jeder Komposition der Output weitergeleitet wird an andere Systeme.
Kann jedoch ein System den Output nicht als Input haben, wird dieses System von
der Nachricht nicht beeinträchtigt.\\
Anschießend, wird die Auswirkung von Hiding auf unsere Struktur
untersucht und somit das Verbergen nachgebildet.\\
Diese Art der Betrachtung der
\EIO{}s wurde auch bereits in~\cite{Schlosser2012BA} gewählt, jedoch wurde
dieser Arbeit nicht als direkte Quelle genutzt, bis auf den Abschnitt des
Hidings. Die Feststellungen in dem Definitions-Kapitel und dem Kapitel über
Errors stimmen somit überein, jedoch wurden alle Beweise unabhängig davon neu
geführt.\\
Wir wählen in dieser Arbeit einen optimistischen Ansatz für die Erreichbarkeit
der jeweils betrachteten Zustände. Ein Zustand gilt hier als erreichbar, wenn
er lokal erreicht werden kann, d.h.\ durch lokale Aktionen. Die Menge,
bestehend aus der internen Aktion $\tau$ und den Outputaktionen, bezeichnen wir
hier als lokale Aktionen. Alle Elemente aus dieser Menge können ohne weiteres
Zutun von außen ausgeführt werden. Somit kann nicht beeinflusst werden ob diese
Übergänge genommen werden oder nicht. Es besteht also die Möglichkeit, dass das
\EIO{} in einen der betrachteten Zustände übergeht, sobald dieser lokal
erreichbar ist. Diese Art der Erreichbarkeit von Zuständen wird auch in Kapitel
3 von~\cite{Vogler2014EIO} behandelt.\\
Neben dem hier betrachteten optimistischen Ansatz gibt es noch zwei weitere
Ansätze in~\cite{Vogler2014EIO}. Einen hyper-optimistischen Ansatz, bei dem ein
Zustand als erreichbar gilt, wenn er durch interne Aktionen erreicht werden
kann, und einen pessimistischen Ansatz, bei dem ein Zustand als erreichbar gilt,
sobald es eine Folge an Inputs und Outputs gibt, mit denen der Zustand vom
Startzustand aus erreicht werden kann.\\
Wir versuche bei allen betrachteten Zustandsmengen eine gröbste Präkongruenz zu
finden, die in der jeweiligen Basisrelation enthalten ist und die eine
Präkongruenz bezüglich der Parallelkomposition ist.

\scriptsize\textcolor{lgray}{TODO: erweitern/umformulieren (bis jetzt nur Teile
aus anderen Kapitel in Einleitung verschoben)}

\normalsize
